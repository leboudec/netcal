\subsection{The Integrated Services Model for Internet Routers}

The reservation phase assumes that all routers can export their
characteristics using a very simple model. The model is based on
the view that an integrated services router implements a practical
approximation of GPS, such as PGPS, or more generally, a GR node.
We have shown in \sref{sec-gr} that the service curve offered to a
flow by a router implementing GR is a rate-latency function, with
rate $R$ and latency $T$ connected by the relationship
\begin{equation}\mylabel{eq-rtltc}
  T = \frac{C}{R} + D
\end{equation}
with $C=$ the maximum packet size for the flow and
$D=\frac{L}{c}$, where $L$ is the maximum packet size in the
router across all flows, and $c$ the total rate of the scheduler.
This is the model defined for an Internet node \cite{SV96}.

\begin{fact}
The Integrated Services model for a router is that the service
curve offered to a flow is always a rate-latency function, with
parameters related by a relation of the form~(\ref{eq-rtltc}).
\end{fact}

The values of $C$ and $D$ depend on the specific implementation of
a router, see \coref{cor-gr-sercur} in the case of GR nodes. Note
that a router does not necessarily implement a scheduling method
that approximates GPS. In fact, we discuss in
Section~\ref{sec-sced} a family of schedulers that has many
advantages above GPS. If a router implements a method that largely
differs from GPS, then we must find a service curve that
lower-bounds the best service curve guarantee offered by the
router. In some cases, this may mean loosing important information
about the router. For example, it is \emph{not} possible to
implement a network offering constant delay to flows by means of a
system like SCED+, discussed in Section~\ref{sec-sced+}, with the
Integrated Services router model.
% a VBR scheduler is another example
\subsection{Reservation Setup with RSVP}

Consider a flow defined by TSPEC $(M, p, r, b)$, that traverses
nodes $1, \ldots, N$. Usually, nodes 1 and $N$ are end-systems
while nodes $n$ for $1<n<N$ are routers. The Integrated Services
model assumes that node $n$ on the path of the flow offers a rate
latency service curve $\beta_{R_n,T_n}$, and further assumes that
$T_{n}$ has the form
 $$T_{n}=\frac{C_{n}}{R} +D_{n}$$
where $C_{n}$ and $D_{n}$ are constants that depend on the
characteristics of node $n$.

The reservation is actually put in place by means of a flow setup
procedure such as the resource reservation protocol (RSVP)%
\index{RSVP}. At the end of the procedure, node $n$ on the path
has allocated to the flow a value $ R_{n}\geq r$. This is
equivalent to allocating a service curve $\beta_{R_n,T_n}$. From
Theorem~\ref{theo-netSerCur} on page~\pageref{theo-netSerCur}, the
end-to-end service curve offered to the flow is the rate-latency
function with rate $R$ and latency $T$ given by
$$
\bracket{
 R = \min_{n=1...N}R_{n} \\
 T = \sum_{n=1}^N \left(\frac{C_n}{R_n} + D_n\right)
 }
$$
 Let
$C_{\mbox{tot}}=\sum_{n=1}^{N} C_n$ and
$D_{\mbox{tot}}=\sum_{n=1}^{N} D_n$. We can re-write the last
equation as
\begin{equation}\mylabel{eq-rtlcnete1}
 T = \frac{C_{\mbox{tot}}}{R} + D_{\mbox{tot}} -
 \sum_{n=1}^N S_n
\end{equation}
with
\begin{equation}\mylabel{eq-rtlcnete2}
 S_n =C_n \left(\frac{1}{R}- \frac{1}{R_n}\right)
\end{equation}
The term $S_n$ is called the ``local slack" term at node $n$.


From Proposition~\ref{theo-vbr-ratlat} we deduce immediately:
\begin{proposition}
    If $R \geq r$, the bound on the end-to-end delay,
    under the conditions described above is
    \begin{equation}
        \frac{b-M}{R} \left(
                  \frac{p-R}{p-r}
                  \right)^{+}
    + \frac{M + C_{tot}}{R}  +  D_{tot}  - \sum_{n=1}^N S_n
        \mylabel{eq-tot-del-1}
    \end{equation}
\end{proposition}

We can now describe the reservation setup with RSVP. Some details
of flow setup with RSVP are illustrated on
Figure~\ref{fig-rsvp-1}. It shows that two RSVP flows are
involved: an advertisement (\texttt{PATH}) flow and a reservation
(\texttt{RESV}) flow. We describe first the point-to-point case.
\begin{itemize}
   \item A \texttt{PATH} message is sent by the source; it
   contains
    the T-SPEC of the flow (source T-SPEC), which is not modified in
    transit, and another field, the ADSPEC, which is accumulated along
    the path.  At a destination, the ADSPEC field contains, among
    others,  the values of
   $C_{\mbox{tot}}, D_{\mbox{tot}}$ used in Equation \ref{eq-tot-del-1}.
    \texttt{PATH} messages do not cause any reservation to be made.
    \item  \texttt{RESV} messages are sent by the destination and cause
    the actual reservations to be made. They follow the reverse path
    marked by
    PATH messages. The \texttt{RESV} message contains a value, $R'$, (as part of the
    so-called R-SPEC), which is a lower bound on the rate parameters $R_n$ that routers
    along the path will have to reserve. The value of $R'$ is
    determined by the destination based on the end-to-end delay objective
    $d_{\mbox{obj}}$, following the procedure described below. It
    is normally not changed by the intermediate nodes.
\end{itemize}

\begin{figure}[!htbp]
\insfig{D2RSVP1}{0.8}
    \mycaption{Setup of Reservations, showing the PATH and RESV flows}
    \mylabel{fig-rsvp-1}
\end{figure}

Define function $f$ by
    $$
    f(R') :=
    \frac{b-M}{R'} \left(
                  \frac{p-R'}{p-r}
                  \right)^{+}
    + \frac{M + C_{tot}}{R'}  +  D_{tot}
    $$
In other words, $f$ is the function that defines the end-to-end
delay bound, assuming all nodes along the path would reserve $R_n=
R'$. The destination computes $R'$ as the smallest value $\geq r$
for which $f(R') \leq d_{\mbox{obj}}$. Such a value exists only if
$D_{tot} < d_{\mbox{obj}}$.

In the figure, the destination requires a delay variation
objective of 600 ms, which imposes a minimum value of $R'=$622
kb/s.  The value of $R'$ is sent to the next upstream node in the
R-SPEC field of the \texttt{PATH} message.  The intermediate nodes
do not know the complete values $C_{\mbox{tot}}$ and
$D_{\mbox{tot}}$, nor do they know the total delay variation
objective. Consider the simple case where all intermediate nodes
are true PGPS schedulers. Node $n$ simply checks whether it is
able to reserve $R_n=R'$ to the flow; this involves verifying that
the sum of reserved rates is less than the scheduler total rate,
and that there is enough buffer available (see below). If so, it
passes the RESV message upstream, up to the destination if all
intermediate nodes accept the reservation. If the reservation is
rejected, then the node discards it and normally informs the
source. In this simple case, all nodes should set their rate to
$R_n=R'$ thus $R=R'$, and Equation~(\ref{eq-tot-del-1}) guarantees
that the end-to-end delay bound is guaranteed.

In practice, there is a small additional element (use of the slack
term), due to the fact that the designers of RSVP also wanted to
support other schedulers. It works as follows.

\begin{figure}[!htbp]
\insfig{D2RSVP2}{0.8}
    \mycaption{Use of the slack term}
    \mylabel{fig-rsvp-2}
\end{figure}

There is another term in the R-SPEC, called the \emph{slack} term.
Its use is illustrated on Figure \ref{fig-rsvp-2}. In the figure,
we see that the end-to-end delay variation requirement, set by the
destination, is 1000 ms. In that case, the destination reserves
the minimum rate, namely, 512~kb/s. Even so, the delay variation
objective $D_{obj}$ is larger than the bound $D_{max}$ given by
Formula (\ref{eq-tot-del-1}). The difference $  D_{obj} - D_{max}$
is written in the slack term $S$ and passed to the upstream node
in the RESV message. The upstream node is not able to compute
Formula (\ref{eq-tot-del-1}) because it does not have the value of
the end-to-end parameters. However, it can use the slack term to
increase its internal delay objective, on top of what it had
advertised. For example, a guaranteed rate node may increase its
value of $v$ (Theorem \ref{theo-GR}) and thus reduce the internal
resources required to perform the reservation. The figure shows
that R1 reduces the slack term by 100 ms. This is equivalent to
increasing the $D_{tot}$ parameter by $100ms$, but without
modifying the advertised $D_{tot}$.

The delays considered here are the total (fixed plus variable)
delays. RSVP also contains a field used for advertising the fixed
delay part, which can be used to compute the end-to-end fixed
delay. The variable part of the delay (called delay jitter) is
then obtained by subtraction.

\subsection{A Flow Setup Algorithm}

There are many different ways for nodes to decide which parameter
they should allocate. We present here one possible algorithm. A
destination computes the worst case delay variation, obtained if
all nodes reserve the sustainable rate $r$. If the resulting delay
variation is acceptable, then the destination sets $R=r$ and the
resulting slack may be used by intermediate nodes to add a local
delay on top of their advertised delay variation defined by $C$
and $D$. Otherwise, the destination sets $R$ to the minimum value
$R_{min}$ that supports the end-to-end delay variation objective
and sets the slack to $0$. As a result, all nodes along the path
have to reserve $R_{min}$. As in the previous cases, nodes may
allocate a rate larger than the value of $R$ they pass upstream,
as a means to reduce their buffer requirement.

\begin{definition}[A Flow Setup Algorithm]

\begin{itemize}
    \item At a destination system $I$, compute
    $$
    D_{max}= f_{T}(r) +\frac{C_{tot}}{r}+ D_{tot}
    $$

    If $D_{obj}> D_{max}$ then assign to the flow a rate $R_{I}= r$
    and an additional delay variation $d_{I} \leq D_{obj}-D_{max}$; set
    $S_{I}= D_{obj}-D_{max} -d_{I}$ and
    send reservation request $R_{I}, S_{I}$ to
    station $I-1$.

    Else ($D_{obj}\leq D_{max}$) find the minimum $R_{min}$ such that
    $f_{T}(R_{min}) +\frac{C_{tot}}{R_{min}} \leq D_{obj} - D_{tot}$,
    if it exists.  Send reservation request $R_{I}=R_{min}, S_{I}=0$
    to station $I-1$.  If $R_{min}$ does not exist, reject the
    reservation or increase the delay variation objective $D_{obj}$.

    \item At an intermediate system $i$: receive from $i+1$ a
    reservation request $R_{i+1}, S_{i+1}$.

    If $S_{i}=0$, then perform reservation for rate $R_{i+1}$ and if
    successful, send reservation request $R_{i}=R_{i+1}, S_{i}=0$
    to station $i-1$.

    Else ($S_{i}>0$), perform a reservation for rate $R_{i+1}$ with
    some additional delay variation $d_{i}\leq S_{i+1}$.  if
    successful, send reservation request $R_{i}=R_{i+1},
    S_{i}=S_{i+1}-d_{i}$ to station $i-1$.
\end{itemize}
\mylabel{def-flowsetup}
\end{definition}

The algorithm ensures a constant reservation rate. It is easy to
check that the end to end delay variation is bounded by $D_{obj}$.

\subsection{Multicast Flows}

Consider now a multicast situation. A source $S$ sends to a number
of destinations, along a multicast tree. \texttt{PATH} messages
are forwarded along the tree, they are duplicated at splitting
points; at the same points, \texttt{RESV} messages are merged.
Consider such a point, call it node $i$, and assume it receives
reservation requests for the same T-SPEC but with respective
parameters $R'_{in}, S'_{in}$ and $R''_{in}, S''_{in}$. The node
performs reservations internally, using the algorithm in Definition~\ref{def-flowsetup}. Then it has to merge the reservation requests it will send to
node $i-1$. Merging uses the following rules:

\paragraph{R-SPEC Merging Rules} The merged reservation $R,S$ is
given by
$$R = \max(R', R'')$$
$$S= \min(S', S'')$$

Let us consider now a tree where the algorithm in Definition~\ref{def-flowsetup} is applied. We want
to show that the end-to-end delay bounds at all destinations are
respected.

The rate along the path from a destination to a source cannot
decrease with this algorithm. Thus the minimum rate along the tree
towards the destination is the rate set at the destination, which
proves the result.

A few more features of RSVP are:
\begin{itemize}
    \item  states in nodes need to be refreshed; if they are not
    refreshed, the reservation is released (``soft states'').

    \item  routing is not coordinated with the reservation of the flow

\end{itemize}
We have so far looked only at the delay constraints. Buffer
requirements can be computed using the values in
Proposition~\ref{theo-vbr-ratlat}.
%\pointex{ex-bufreqd-1} \pointex{ex-bufreqd-2}
\subsection{Flow Setup with ATM}
With ATM, there are the following differences:
\begin{itemize}
    \item  The path is determined at the flow setup time only.
    Different connections may follow different routes depending on their
    requirements, and once setup, a connection always uses the same path.

    \item  With standard ATM signaling, connection setup is initiated at
    the source and is confirmed by the destination and all intermediate
    systems.
\end{itemize}

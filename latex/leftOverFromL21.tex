
%leftover from old chpater2, to be put into chapter aggregate scheduling
\section{Application to Differentiated Services} \mylabel{sec-ds}
\subsection{Differentiated Services}
\mylabel{sec-dsgen}

\subsection{A Bounding Method for Aggregate Scheduling}
\mylabel{sec-timestoppingm}
Consider a node $m$. If all traffic
entering the node is access traffic only (no transit traffic) then
an arrival curve for the aggregate traffic entering the node is
$\sum_{i \in F_m} \alpha_i$, where $F_m$ is the set of flows that
enter the network at this node. The aggregate flow is guaranteed a
service curve $S_m$, thus, from \sref{3bornes}, the total required
buffer is bounded by $v(\sum_{i \in F_m} \alpha_i, S_m)$, and the
delay by $v(\sum_{i \in F_m} \alpha_i, S_m)$.

In order to use \sref{3bornes} we need now to find an arrival
curve for the traffic exiting node $m$. This in general requires
more assumptions on the aggregate scheduler. We examine in
\cref{L24} the properties of FIFO schedulers and will find there
some strong, but complex results. We have a simple result if the
service curves are \emph{strict}, as defined in \sref{def-sercur};
this applies to many practical cases, in particular to all
variable capacity nodes and priority schedulers.
\begin{theorem}
Consider a node serving two flows in an aggregate manner. Assume
the aggregate is guaranteed a \emph{strict} service curve $\beta$.
Assume also that the second flow is $\alpha_2$-smooth. Then the
first flow is guaranteed a minimum service curve $\beta^*_1$ given
by
 $$
\beta^*_1(t)= [\beta(t) - \alpha_2(t)]^+
 $$
 \mylabel{theo-wcl}
\end{theorem}
\pr
Call $R_i$ and $R'_i$ the input and output, for $i=1,2$. Fix some
arbitrary $t$ and call $s$ the beginning of the busy period (for
the aggregate). Thus
$$
R'_1(t) + R'_2(t) \geq R_1(s) + R_2(s) + \beta(t-s)
$$
Now
$$
R'_2 (t) \leq R_2(t) \leq R_2(s) + \alpha_2(t-s)
$$
thus
\begin{equation}\mylabel{eq-diffserv-1}
  R'_1(t)  \geq R_1(s) + \beta(t-s) - \alpha_2(t-s)
\end{equation}
Since $s$ is the beginning of the busy period, we have
$R'_1(s)+R'_2(s)=R_1(s)+R_2(s)$ and therefore $R'_1(s)=R_1(s)$.
Thus
$$
 R'_1(t)  \geq R'_1(s) =R_1(s)
$$
combining this with \eref{eq-diffserv-1}, we find
$$
R'_1(t)  \geq R_1(s) + [\beta(t-s) - \alpha_2(t-s)]^+
$$
which proves the service curve property. \qed

If the arrival curves are affine, then the following corollary of
\thref{theo-wcl} expresses the burstiness increase due to
multiplexing.
\begin{corollary}
Consider a node serving two flows in an aggregate manner. Assume
the aggregate is guaranteed a \emph{strict} service curve
$\beta_{R,T}$. Assume also that flow $i$ is constrained by one
leaky bucket with parameters $(r_i,b_i)$. The output of the first
flow is constrained by a leaky bucket with parameters $(r_1,
b^*_1)$ with
$$
b^*_1 = b_1 + r_1 T + r_1 \frac{b_2+ r_2 T}{R-r_2}
$$
 \mylabel{coro-wcl}
\end{corollary}
Note that the burstiness increase contains a term $r_1 T$ that is
found even if there is no multiplexing; the second term $r_1
\frac{b_2+ r_2 T}{R-r_2}$ comes from multiplexing with flow 2.
Note also that if we further assume that the node is FIFO, then we
have a better bound (\cref{L24}).
\pr
From \thref{theo-wcl}, the first flow is guaranteed a service
curve $\beta_{R', T'}$ with $R'=R-r_2$ and $T'= \frac{b_2 +T
r_2}{R-r_2}$. The result follows from a direct application of
\thref{theo-output}. \qed

We could think of using \coref{coro-wcl} to analyze the general
network in \fref{fig-diffservmod}. However, this would work only
in a \emph{feed-forward} network, namely, a network where we can
number nodes such that the sequence of node numbers followed by
any flow is always increasing. Equivalently, this means that there
is no cycle for the relation between nodes defined by $m
\rightarrow m'$ iff among the flows entering $m'$, at least one
exits $m$. In a feed-forward, we can recursively apply
\coref{coro-wcl} and find arrival curve constraints at all nodes,
thus backlog and delay bounds.

However, feed-forward topologies are very restrictive, and most
networks are not feedforward. We show now an analysis method that
can be applied to non-feedforward networks.

\paragraph{Assumptions and notation}

We consider in the rest of this subsection the case where all
fresh arrival curves $\alpha_i$ have the form
$$
\alpha_i=\gamma_{r_i, b_i}$$ In other words, the constraints are
expressed by a single leaky bucket. To simplify the notation, we
consider that the nodes are are constant rate, work conserving
servers, with rate $C_m$ for node $m$. We assume that there is no
loop on the path followed by any flow. We use the notation $m \in
i$ or $i\ni m$ to express that node $m$ is on the path of flow
$i$.

We call ``utilization factor"%
\index{utilization factor}%
 at link $l$ the ratio
$$\nu_l= \frac{\sum_{i \ni m} r_i}{C_l} $$
We also assume that the network is ``sub-critical", that is, for
all node $m$:
\begin{equation}\mylabel{eq-ds-condrates}
  \nu_l < 1
\end{equation}

We will illustrate the method on a specific example, shown on
\fref{fig-dsnet1}.


In a non-feedforward network, we cannot directly apply
\coref{coro-wcl}. However, we can use the following trick, called
the ``time stopping method", which was introduced in
\cite{cru91b}. The method has two steps. First, we assume that
there is a finite burstiness bound for all flows;  using
\coref{coro-wcl} we obtain some equations for computing these
bounds. Second, we use the same equations to show that, under some
conditions, finite bounds exist.

\begin{figure}[!htbp]
  \insfig{dsnet1}{0.7}
  \mycaption{A simple example with aggregate scheduling, used to
   illustrate the bounding method. There are three nodes numbered
   $0,1,2$ and six flows, numbered $0,...,5$.
   For $i=0,1,2$, the path of flow $i$ is $i, (i+1) \mod 3, (i+2) \mod
   3$ and the path of flow $i+3$ is $i, (i+2) \mod 3, (i+1) \mod
   3$. The fresh arrival curve is the same
for all flows, and is given by $\alpha_i=\gamma_{\rho,\sigma}$.
All nodes are constant rate, work conserving servers, with rate
$C$. The utilization factor at all nodes is $6\frac{\rho}{C}$.}
  \mylabel{fig-dsnet1}
\end{figure}

\paragraph{First step: inequations for the bounds}


For any flow $i$ and any node $m \in i$, define $b_i^m$ as the
maximum backlog that this flow would generate in a constant rate
server with rate $r_i$. By convention, the fresh inputs are
considered as the outputs of a virtual node numbered $-1$. In this
first step, we assume that $b_i^m$ is finite for all $i$ and $m
\in i$.

From \lref{lem-buf-simple} on \pgref{lem-buf-simple}, $b_i^m$ is
the smallest value such that the output of node $m$ is constrained
by the leaky bucket with parameters $(r_i, b_i^m) $. By applying
\coref{coro-wcl} we find that for all $i$ and $m \in i$:
\begin{equation}\mylabel{eq-ds-lin}
\bracket{
 b_i^0 \leq b_i \\
b_i^m = b_i^{\mbox{pred}_i(m)} + r_i \frac{
 \sum_{j \ni m, j \neq i}b_j^{\mbox{pred}_j(m)}
 }
 {C-\sum_{j \ni m, j \neq i} r_j}
 }
\end{equation}
where $\mbox{pred}_i(m)$ is the predecessor of node $m$. If $m$ is
the first node on the path of flow $i$, we set by convention
$\mbox{pred}_i(m)=-1$ and $b_i^{-1}=b_i$.

Now put all the $b_i^m$, for all $(i,m)$ such that $m \in i$, into
a vector $\vec{x}$ with one column and $n$ rows, for some
appropriate $n$. We can re-write \eref{eq-ds-lin} as
\begin{equation}\mylabel{eq-ds-lin2}
\vec{x} \leq A \vec{x} + \vec{a}
\end{equation}
where $A$ is an $n \times n$, non-negative matrix and $\vec{a}$ is
a non-negative vector depending only on the known quantities
$b_i$. The method now consists in assuming that the spectral
radius of matrix $A$ is less than $1$. In that case the power
series $I + A + A^2 + A^3 + ...$ converges and is equal to
$(I-A)^{-1}$, where $I$ is the $n \times n$ identity matrix. Since
$A$ is non-negative, $(I-A)^{-1}$ is also non-negative; we can
thus multiply \eref{eq-ds-lin} to the left by $(I-A)^{-1}$ and
obtain:
\begin{equation}\mylabel{eq-ds-lin3}
\vec{x} \leq (I-A)^{-1} \vec{a}
\end{equation}
which is the required result, since $\vec{x}$ describes the
burstiness of all flows at all nodes. From there we can obtain
bounds on delays and backlogs.

Let us apply this step to our network example. By symmetry, we
have only two unknowns $x$ and $y$, defined as the burstiness
after one and two hops:
$$\bracket{
x=b_0^0=b^1_1=b_2^2=b_3^0=b_4^2=b_5^1\\
y=b_0^1=b^2_1=b_2^0=b_3^2=b_4^1=b_5^0
 }
$$
\eref{eq-ds-lin} becomes
$$
\bracket{
 x \leq \sigma + \frac{\rho}{C -5 \rho} ( \sigma + 2x + 2y)\\
 y \leq x + \frac{\rho}{C -5 \rho} ( 2 \sigma +  x + 2y)
  }
$$
Define $\eta=\frac{\rho}{C-5 \rho}$; from \eref{eq-ds-condrates}
we conclude that $0 \leq \eta <1$. We can now write
\eref{eq-ds-lin2} with
$$
 \vec{x}=\left(
  \begin{array}{c}
    x\\y
  \end{array}\right)
  ,\;
 A= \left(\begin{array}{c c}
    2 \eta &  2 \eta\\1 + \eta & 2 \eta
  \end{array}\right)
  ,\;
   \vec{a}=\left(
  \begin{array}{c}
    \sigma (1 + \eta)\\ 2 \sigma \eta
  \end{array}\right)
$$
Some remnant from linear algebra, or a symbolic computation
software, tells us that
$$
(I-A)^{-1}=\left(\begin{array}{c c}
    \frac{1-2  \eta}{1-6  \eta+2
{\eta^2}} & \frac{2  \eta}{1-6  \eta+2  {\eta^2}}  \\
   \frac{1+\eta}{1-6  \eta+2
{\eta^2}}& \frac{1-2  \eta}{1-6  \eta+2  {\eta^2}}
  \end{array}\right)
$$
If $\eta < \frac{1}{2} (3 -\sqrt{7}) \approx 0.177$ then
$(I-A)^{-1}$ is positive. This is the condition for the spectral
radius of $A$ to be less than 1. The corresponding condition on
the utilization factor $\nu= \frac{6\rho}{C}$ is
\begin{equation}\mylabel{eq-example-ds-1}
 \nu <2 \frac{8-\sqrt{7}}{19} \approx 0.564
\end{equation}
Thus, for this specific example, if \eref{eq-example-ds-1} holds,
and if the burstiness terms $x$ and $y$ are finite, then they are
bounded as given in \eref{eq-ds-lin3}, with $(I-A)^{-1}$ and
$\vec{a}$ given above.

\paragraph{Second Step: time stopping}

We now prove that there is a finite bound if the spectral radius
of $A$ is less than 1. For any time $\tau>0$, consider the virtual
system made of the original network, where all sources are stopped
at time $\tau$. For this network the total number of bits in
finite, thus we can apply the conclusion of step 1, and the
burstiness terms are bounded by \eref{eq-ds-lin3}. Since the
right-handside \eref{eq-ds-lin3} is independent of $\tau$, letting
$\tau$ tend to $+\infty $ shows the following.
\begin{proposition}
With the notation in this section, if the spectral radius of $A$
is less than $1$, then the burstiness terms $b^m_i$ are bounded by
the corresponding terms in \eref{eq-ds-lin3}.
\mylabel{prop-stoppingTimeMethod}
\end{proposition}
Back to the example of \fref{fig-dsnet1}, we find that if the
utilization factor $\nu$ is less than $0.564$, then the burstiness
terms $x$ and $y$ are bounded by
$$
\bracket{
 x \leq 2 \sigma \frac{18 -33 \nu + 16 \nu^2}
 {36 -96 \nu + 57 \nu^2 }\\
 y \leq  2 \sigma \frac{18 - 18  \nu + \nu^2}
 {36 -96 \nu + 57 \nu^2}
 }
$$
The aggregate traffic at any of the three nodes is $\gamma_{6\rho,
b}$-smooth with $b=2(\sigma + x + y)$. Thus a bound on delay is
(see also \fref{fig-L21-ds}):
$$
 d=\frac{b}{C}= 2 \frac{\sigma}{C}
 \frac{108 - 198 \nu + 91 \nu^2}{36 - 96 \nu + 57 \nu^2}
$$

\begin{figure}[!htbp]
  \insfig{L21-ds}{0.5}
  \mycaption{The bound $d$ on delay at any node obtained by the method
  presented here for the network of
  \fref{fig-dsnet1}  (thin line).
  The graph shows $d$ normalized by $\frac{\sigma}{C}$
  (namely, $\frac{d C}{\sigma}$),
  plotted as a function of
  the utilization factor. The thick line is a delay bound
  obtained if every flow is re-shaped at every output.}
  \mylabel{fig-L21-ds}
\end{figure}

\paragraph{What happens when the bound is infinite ?}
Call $\nu_0$ the upper bound on utilization factors for which the
method in \sref{sec-timestoppingm} finds a finite bound. In the
example, we found $\nu_0\approx 0.564$, which is much less than
$1$. Does it mean that no finite bound exists when $\nu_0 <1$ ?
The answer to this question is not clear.

First, the $\nu_0$ found with the method can be improved if we
express more arrival constraints. Consider our particular example:
we have not exploited the fact that the fraction of input traffic
to node $i$ that originates from another node has to be
$\lambda_C$-smooth. If we do so, we will obtain better bounds.
Second, if we know that nodes have additional properties, such as
FIFO, then we may be able to find better bounds.

In general, the issue of stability, or existence of finite bounds,
for a given network with aggregate multiplexing is still
unresolved. The question of delay bounds for a network with
aggregate scheduling was first raised in \cite{cha91}. We will see
in \cref{L24} that if the network is a ring of constant bit rate
servers, then $\nu_0=1$, in other words, we can find a bound for
any $\nu<1$ \cite{tg96}. However, the proof does not extend to
general service curve guarantees, even strict service curves. In
\cite{andrews00}, the author exhibits an unstable network similar
to ours, with $\nu=1$ (a ``critical'' network), and claims that
this also holds for some sub-critical networks, but the proof is
not conclusive. In addition, a startling fact referred to as a
``non-monotone property'' of FIFO networks is described in
\cite{andrews00}, where it is shown that a network that is
\emph{stable} with a set of sessions with given rates may become
\emph{unstable} if the rates of some of these sessions are
\emph{reduced} for a period of time. On the other side of the
spectrum, we will also see in \cref{L24} strong and explicit
results \cite{leb99,CFZF98} with FIFO multiplexing.

\paragraph{The price for aggregate scheduling}
Consider again the example on \fref{fig-dsnet1}, but assume now
that every flow is reshaped at every output. This is not possible
with differentiated services, since there is no per-flow
information at nodes other than access nodes. However, we use this
scenario as a benchmark that illustrates the price we pay for
aggregate scheduling.

With this assumption, every flow has the same arrival curve at
every node. Thus we can compute a service curve $\beta_1$ for flow
$1$ (and thus for any flow) at every node, using \thref{theo-wcl};
we find that $\beta_1$ is the rate-latency function with rate $(C-
5 \rho)$ and latency $\frac{5 \sigma}{C-5 \rho}$. Thus a delay
bound for flow at any node, including the re-shaper, is
$h(\alpha_1, \alpha_1 \mpc \beta_1)= h(\alpha_1,
\beta_1)=\frac{6C}{C- 5 \rho}$ for $\rho \leq \frac{C}{6}$.
\fref{fig-L21-ds} shows this delay bound, compared to the delay
bound we found if no reshaper is used. As we already know, we see
that with per-flow information, we are able to guarantee a delay
bound for any utilization factor $\leq 1$. However, note also that
for relatively small utilization factors, the bounds are very
close.

\subsection{An Explicit Delay Bound for Differentiated Services Networks}
\mylabel{bounds-lossless}

%
%
%\mylabel{sec-networkgrand}
\begin{proposition}\cite{lebinfocom2001, qofis2000}
With the assumptions of \thref{theo-qofisbound}, if $\nu <
\frac{1}{h-1}$, then for any $D' >0$, there is a network in which
the worst case delay is at least $D'$. \mylabel{prop-grand}
\end{proposition}
In other words, the worst case queuing delay can be made
arbitrarily large; thus if we want to go beyond
\thref{theo-qofisbound}, any bound for differentiated services
must depend on the network topology or size, not only on the
utilization factor and the number of hops.

\pr
We build a family of networks, out of which, for any $D'$, we can
exhibit an example where the queuing delay is at least $D'$.

The thinking behind the construction is as follows. All flows are
low priority flows. We create a
  hierarchical network, where at the first level of the
  hierarchy we choose one ``flow" for which its first
  packet happens to encounter just \emph{one} packet of every
  other flow whose route it intersects, while its next packet
  does not encounter any queue at all.  This causes the first
  two packets of the chosen flow to come back-to-back after
  several hops.  We then construct the second level of the
  hierarchy by taking a new flow and making sure that its
  first packet encounters \emph{two} back-to-back packets of each
  flow whose routes it intersects, where the two back-to-back
  packet bursts of all these flows come from the output of a
  sufficient number of networks constructed as described at
  the first level of the hierarchy. Repeating this process
  recursively sufficient number of times,  for any chosen
  delay value $D$ we can create deep enough hierarchy so that
  the queuing delay of the first packet of some flow
  encounters a queuing delay more than $D$ (because it
  encounters a large enough back-to-back burst of packets
  of every other flow constructed in the previous iteration),
  while the second packet does not suffer any queuing delay
  at all.   We now describe in detail how to construct such a
  hierarchical network (which is really a family of networks)
  such that utilization factor of any link does not exceed a
  given factor $\nu$, and no flow traverses more than $h$ hops.

Now let us describe the networks in detail. We consider a family
of networks with a single traffic class and constant rate links,
all with same bit rate $C$. The network is assumed to be made of
infinitely fast switches, with one output buffer per link. Assume
that sources are all leaky bucket constrained, but are served in
an aggregate manner, first in first out. Leaky bucket constraints
are implemented at the network entry; after that point, all flows
are aggregated. Without loss of generality, we also assume that
propagation delays can be set to 0; this is because we focus only
on queuing delays. As a simplification, in this network, we also
assume that all packets have a unit size. We show that for any
fixed, but arbitrary delay budget $D$, we can build a network of
that family where the worst case queueing delay is larger than
$D$, while each flow traverses at most a specified number of hops.

A network in our family is called $\calN(h, \nu, J)$ and has three
parameters: $h$ (maximum hop count for any flow), $\nu$
(utilization factor) and $J$ (recursion depth). We focus on the
cases where $h \geq 3$ and $\frac{1}{h-1}< \nu <1$, which implies
that we can always find some integer $k$ such that
\begin{equation}\mylabel{eqk}
  \nu > \frac{1}{h-1}\frac{kh+1}{kh-1}
\end{equation}
Network $\calN(h, \nu, J)$ is illustrated in
Figures~\ref{fig-grand1} and \ref{fig-grand2}; it is a collection
of identical building blocks, arranged in a tree structure of
depth $J$. Every building block has one internal source of traffic
(called ``transit traffic"), $kh(h-1)$ inputs (called the
``building block inputs"), $kh(h-1)$ data sinks, $h-1$ internal
nodes, and one output. Each of the $h-1$ internal nodes receives
traffic from $kh$ building block inputs plus it receives transit
traffic from the previous internal node, with the exception of the
first one which is fed by the internal source. After traversing
one internal node, traffic from the building block inputs dies in
a data sink. In contrast, transit traffic is fed to the next
internal node, except for the last one which feeds the building
block output (\fref{fig-grand1}).
\begin{figure}[!htbp]
  \insfig{bb1}{0.7}
  \mycaption{The internal node (top) and the building block (bottom)
  used in our network example.}
  \mylabel{fig-grand1}
\end{figure}
\fref{fig-grand2} illustrates that our network has the structure
of a complete tree, with depth $J$. The building blocks are
organized in levels $j=1,...,J$. Each of the inputs of a level $j$
building block ($j\geq 2$) is fed by the output of one level $j-1$
building block. The inputs of level $1$ building blocks are data
sources. The output of one $j-1$ building block feeds exactly one
level $j$ building block input. At level $J$, there is exactly one
building block, thus at level $J-1$ there are $kh(h-1)$ building
blocks, and at level $1$ there are $(kh(h-1))^{J-1}$ building
blocks.
\begin{figure}[!htbp]
  \insfig{grandnet}{0.7}
  \mycaption{The network made of building blocks from \fref{fig-grand1}}
  \mylabel{fig-grand2}
\end{figure}
All data sources have the same rate $r=\frac{\nu C}{kh+1}$ and
burst tolerance $b=1$~packet. In the rest of this section we take
as a time unit the transmission time for one packet, so that
$C=1$. Thus any source may transmit one packet every $\theta =
\frac{kh+1}{\nu}$ time units. Note that a source may refrain from
sending packets, which is actually what causes the large delay
jitter. The utilization factor on every link is $\nu$, and every
flow uses $1$ or $h$ hops.

Now consider the following scenario. Consider some arbitrary level
1 building block. At time $t_0$, assume that a packet fully
arrives at each of the building block inputs of level $1$, and at
time $t_0+1$, let a packet fully arrive from each data source
inside every level $1$ building block (this is the first transit
packet). The first transit packet is delayed by $hk-1$ time units
in the first internal node. Just one time unit before this packet
leaves the first queue, let one packet fully arrive at each input
of the second internal node. Our first transit packet will be
delayed again by $hk-1$ time units. If we repeat the scenario
along all internal nodes inside the building block, we see that
the first transit packet is delayed by $(h-1)(hk-1)$ time units.
Now from \eref{eqk}, $\theta < (h-1)(hk-1)$, so it is possible for
the data source to send a second transit packet at time
$(h-1)(hk-1)$. Let all sources mentioned so far be idle, except
for the emissions already described. The second transit packet
will catch up to the first one, so the output of any level $1$
building block is a burst of two back-to-back packets. We can
choose $t_0$ arbitrarily, so we have a mechanism for generating
bursts of 2 packets.

Now we can iterate the scenario and use the same construction at
level $2$. The level-2 data source sends exactly three packets,
spaced by $\theta$. Since the internal node receives $hk$ bursts
of two packets originating from level 1, a judicious choice of the
level 1 starting time lets the first level 2 transit packet find a
queue of $2hk -1$ packets in the first internal node. With the
same construction as in level 1, we end up with a total queuing
delay of $(h-1)(2hk -1) > 2(h-1)(hk-1)> 2\theta$ for that packet.
Now this delay is more than $2 \theta$, and the first three
level-2 transit packets are delayed by the same set of non-transit
packets; as a result, the second and third level-2 transit packets
will eventually catch up to the first one and the output of a
level 2 block is a burst of three packets. This procedure easily
generalizes to all levels up to $J$. In particular, the first
transit packet at level $J$ has an end-to-end delay of at least
$J\theta$. Since all sources become idle after some time, we can
easily create a last level $J$ transit packet that finds an empty
network and thus a zero queuing delay.

Thus there are two packets in network $\calN(h, \nu, J)$, with one
packet having a delay larger than $J \theta$, and the other packet
has zero delay. This establishes that a bound on queuing delay,
and thus on delay variation in network $\calN(h, \nu, J)$ has to
be at least as large as~$J \theta$. \qed

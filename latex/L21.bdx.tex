\problemfile{L21-1}
\begin{problem}
\begin{enumerate}
    \item Assume $K$ connections, each with peak rate $p$, sustainable rate
    $m$ and burst tolerance $b$, are offered to a trunk with constant
    service rate $P$ and FIFO buffer of capacity $X$. Find the
    conditions on $K$ for the system to be loss-free.
    \item If $Km =P$, what is the condition on $X$ for $K$
    connections to be accepted ?
    \item What is the maximum number of connection if $p$ = 2 Mb/s,
    $m$ = 0.2 Mb/s, $X= 10 $MBytes, $b=1 $Mbyte and $P=$ 0.1, 1, 2 or
    10 Mb/s ?
    \item For a fixed buffer size $X$, draw the acceptance
    region when $K$ and $P$ are the variables.
\end{enumerate}
\end{problem}
\end{filecontents}
%%%%%%%%%%%%%%%%%%%%%%%%%%%%%%%%%%%%%%%%%%%%%%%%%%%%%%%%%%%%%%%%%%%%%%%%%%%%%%%%%
\problemfile{L21-2}
\begin{problem}
    Show the formulas giving the expressions for $f_B(R)$ and
    $f_B(\alpha)$.
\end{problem}
\end{filecontents}
%%%%%%%%%%%%%%%%%%%%%%%%%%%%%%%%%%%%%%%%%%%%%%%%%%%%%%%%%%%%%%%%%%%%%%%%%%%%%%%%%
\problemfile{L21-3}
\begin{problem}
\begin{enumerate}
    \item What is the effective bandwith for a connection with $p$ = 2 Mb/s,
    $m$ = 0.2 Mb/s,  $b=$ 100 Kbytes when $D=$ 1msec, 10 msec, 100 msec,
    1s ?
    \item Plot the effective bandwidth $e$ as a function of the delay
    constraint in the general case of a connection with parameters $p,
    m, b$.
\end{enumerate}
\end{problem}
\end{filecontents}
%%%%%%%%%%%%%%%%%%%%%%%%%%%%%%%%%%%%%%%%%%%%%%%%%%%%%%%%%%%%%%%%%%%%%%%%%%%%%%%%%
\problemfile{L21-4}
\begin{problem}
\begin{enumerate}
    \item Compute the effective bandwidth for a mix of VBR
    connections $1, \ldots, I$.

    \item Show how the homogeneous case can be derived from your formula

    \item Assume $K$ connections, each with peak rate $p$, sustainable rate
    $m$ and burst tolerance $b$, are offered to a trunk with constant
    service rate $P$ and FIFO buffer of capacity $X$. Find the
    conditions on $K$ for the system to be loss-free.

    \item  Assume that there are two classes of connections, with
    $K_{i}$ connections in class $i$, $i=1,2$, offered to a trunk with constant
    service rate $P$ and FIFO buffer of infinite capacity $X$. The
    connections are accepted as long as their queuing delay does
    not exceed some value $D$.
    Draw the acceptance region, that is, the set of $(K_{1}, K_{2})$
    that are accepted by CAC2. Is the
    acceptance region convex ? Is the complementary of the
    acceptance region in the positive orthant convex ? Does this
    generalize to more than two classes ?
\end{enumerate}
\end{problem}
\end{filecontents}%
%%%%%%%%%%%%%%%%%%%%%%%%%%%%%%%%%%%%%%%%%%%%%%%%%%%%%%%%%%%%%%%%%%%%%%%%
%%%%%%%%%%%%%%%%%%%%%%%%%%%%%%%%%%%%%%%%%%%%%%%%%%%%%%%%%%%%%%%%%%%%%%%%%%%%%%%%%
\problemfile{L21-11}
\begin{problem}
Consider a guaranteed rate scheduler, with rate $R$ and delay $v$,
that receives a packet flow with cumulative packet length $L$. The
(packetized) scheduler output is fed into a constant bit rate
trunk with rate $c > R$ and propagation delay $T$.
\begin{enumerate}
  \item Find a minimum service curve for the complete system.
 \sol{~\\
 The system is modelled by the concatenation
 $(N)(P^L)(S_{c,T})(P^L)$, where $N$ represents the scheduler.
 A minimum service curve for $(N)(P^L)$ is
 $S_{R,v+\frac{l_{\max}}{R}}$ and for $(\lambda_c)(P^L)$ is
 $S_{c,T+\frac{l_{\max}}{c}}$. Thus an end-to-end service curve is
 $S_{R,v+T+\frac{l_{\max}}{R}+\frac{l_{\max}}{c}}$.
 }
  \item Assume the flow of packets is $(r,b)$-constrained, with
  $b>l_{\max}$. Find a bound on the end-to-end delay and delay
  variation.
  \sol{~\\
  For the end-to-end delay is bound, we may omit the last packetizer. Thus
  a bound on the end-to-end delay is
  $$h(\gamma_{r,b}, S_{R,v+T+\frac{l_{\max}}{R}})
   = T+\frac{b+l_{\max}}{R}
  $$
  The minimum delay is $T + \frac{l_{\min}}{c}$ thus a bound on
  delay variation is
  $$
  v+\frac{b+l_{\max}}{R} - \frac{l_{\min}}{c}
  $$
}
\end{enumerate}
\end{problem}
\end{filecontents}%
%%%%%%%%%%%%%%%%%%%%%%%%%%%%%%%%%%%%%%%%%%%%%%%%%%%%%%%%%%%%%%%%%%%%%%%%%%%%%%%%%
\problemfile{L21-12}
\begin{problem}
Assume all nodes in a network are of the GR type with rate $R$ and latency $T$. A
  flow with T-SPEC $\alpha(t)=\min(rt+b, M+pt)$
  has performed a reservation with rate
  $R$ across a sequence of $H$ nodes, with $p\geq R$. Assume no reshaping is done.
  What is the buffer requirement at the $h$th node along the path,
  for $h=1,...H$~?
  \sol{~\\
An arrival curve for the input to the $h$th node is $\alpha_{h-1}$, obtained as the
deconvolution of $\alpha$ and a service curve $\beta_{h-1}$ for the concatenation of $h-1$
nodes. We have: $\beta_{h-1}$ is rate-latency, with rate $R$ and latency $(h-1)T'$, with $T'=T+
l_{\max}/R$. Thus, by basic network calculus:
$$
\alpha_{h-1}(t) = \min \left( b_{h-1} + rt, M_{h-1}+Rt \right)
$$
with
$$
b_{h-1}=b+ (h-1) r T'
$$
and
$$
M_{h-1}= (h-1)R T' + M +  \frac{b-M}{p-r}(p-R)
$$
The buffer requirement at node $h$, $v_h$, is the vertical distance $v(\alpha_{h-1},
\beta_{R,T'})$. As an intermediate step, consider the case $\alpha(t)=\min(b+rt,M+Rt)$ and
compute $v=v(\alpha, \beta_{R,T'})$. Define $\theta=\left[\frac{b-M}{R-r}\right]^+$; a simple
analysis gives
$$
\mif \theta \leq T' \mthen v=b+rT' \melse v=M+RT'
$$
In our case we have
$$
\theta=\frac{b-M}{p-r}-(h-1)T'
$$
whence we derive
$$
\mif \frac{b-M}{p-r} \leq h T' \mthen v_h = b + h r T' \melse v_h= M +  \frac{b-M}{p-r}(p-R) +
h R T'
$$
Thus for both small and large $h$, $v_h$ depends linearly on $h$.}
\end{problem}
\end{filecontents}%
%%%%%%%%%%%%%%%%%%%%%%%%%%%%%%%%%%%%%%%%%%%%%%%%%%%%%%%%%%%%%%%%%%%%%%%%%%%%%%%%%
\problemfile{L21-13}
\begin{problem}
Assume all nodes in a network are made of a GR type with rate $R$ and latency $T$, before which
a re-shaper with
  shaping curve $\sigma=\gamma_{r,b}$ is inserted. A
  flow with T-SPEC $\alpha(t)=\min(rt+b, M+pt)$
  has performed a reservation with rate
  $R$ across a sequence of $H$ such nodes, with $p\geq R$.  What is a buffer requirement at the $h$th node along the path,
  for $h=1,...H$~?
\sol{~\\ An arrival curve for the input to the $h$th GR component is $\gamma_{r,b}$; thus the
buffer required at the GR component (excluding the shaper) is $B_1=v(\gamma_{r,b}, \beta)$,
where $\beta$ is the rate-latency service curve with with rate $R$ and latency
$T'=T+l_{\max}/R$. Thus $B_1=b+rT'$.\\ However, we also need to allow some buffer for the
shaper. It is equal to $v(\alpha_{h-1}, \gamma_{r,b})$, where $\alpha'$ is an arrival curve for
the output of the $(h-1)$th node. We have
$$
\alpha' = \gamma_{r,b} \mpd \beta = \gamma_{r, b +rT'}
$$
Thus a buffer bound at the shaper is $B_2=rT'$. A total buffer bound is $B=b+ 2rT'$. Note that
it is independent of $h$.\\ We might also assume that the node employs buffer sharing between
the shaper and the GR component. If so, a buffer bound is obtained by viewing the complete node
as a concatenation of two service curve components; the bound is then
$$v_h= v(\alpha', \gamma_{r,b} \mpc
\beta )$$ In this particular case however, this does not improve the bound.\\ A more
sophisticated bound can be found if we account for the original peak rate limitation $(p,M)$.}
\end{problem}
\end{filecontents}%
%%%%%%%%%%%%%%%%%%%%%%%%%%%%%%%%%%%%%%%%%%%%%%%%%%%%%%%%%%%%%%%%%%%%%%%%%%%%%%%%%
\problemfile{L21-14}
\begin{problem}
Assume all nodes in a network are made of a shaper followed by a
FIFO multiplexer. Assume that flow $I$ has T-SPEC,
$\alpha_i(t)=\min(r_it+b_i, M+p_it)$, that the shaper at every
node uses the shaping curve $\sigma_i=\gamma_{r_i,b_i}$ for flow
$i$. Find the schedulability conditions for every node.
\end{problem}
\end{filecontents}%
%%%%%%%%%%%%%%%%%%%%%%%%%%%%%%%%%%%%%%%%%%%%%%%%%%%%%%%%%%%%%%%%%%%%%%%%%%%%%%%%%
\problemfile{L21-15}
\begin{problem}
A network consists of two nodes in tandem. There are $n_1$ flows
of type $1$ and $n_2$ flows of type $2$. Flows of type $i$ have
arrival curve $\alpha_i(t)=r_it+b_i$, $i=1,2$. All flows go
through nodes $1$ then $2$. Every node is made of a shaper
followed by an EDF scheduler. At both nodes, the shaping curve for
flows of type $i$ is some $\sigma_i$ and the delay budget for
flows of type $i$ is $d_i$. Every flow of type $i$ should have a
end-to-end delay bounded by $D_i$. Our problem is to find good
values of $d_1$ and $d_2$.
\begin{enumerate}
  \item We assume that $\sigma_i=\alpha_i$. What are the
  conditions on $d_1$ and $d_2$ for the end-to-end delay bounds to
  be satisfied ? What is the set of $(n_1, n_2)$ that are
  schedulable ?
  \item Same question if we set $\sigma_i=\lambda_{r_i}$
\end{enumerate}
\end{problem}
\end{filecontents}%
%%%%%%%%%%%%%%%%%%%%%%%%%%%%%%%%%%%%%%%%%%%%%%%%%%%%%%%%%%%%%%%%%%%%%%%%%%%%%%%%%
\problemfile{L21-16}
\begin{problem}
Consider the scheduler in \thref{theo-grissched}. Find an
efficient algorithm for computing the deadline of every packet.
\end{problem}
\end{filecontents}%
%%%%%%%%%%%%%%%%%%%%%%%%%%%%%%%%%%%%%%%%%%%%%%%%%%%%%%%%%%%%%%%%%%%%%%%%%%%%%%%%%
\problemfile{L21-17}
\begin{problem}
Consider a SCED scheduler with target service curve for flow $i$
given by
$$\beta_i= \gamma_{r_i, b_i}\mpc \delta_{d_i}
$$
Find an efficient algorithm for computing the deadline of every
packet.

Hint: use an interpretation as a leaky bucket. \sol{See page 69 of
\cite{saro96}}
\end{problem}
\end{filecontents}%
%%%%%%%%%%%%%%%%%%%%%%%%%%%%%%%%%%%%%%%%%%%%%%%%%%%%%%%%%%%%%%%%%%%%%%%%%%%%%%%%%
\problemfile{L21-18}
\begin{problem}
Consider the example of \fref{fig-dsnet1}. Apply the method of
\sref{sec-timestoppingm} but express now that the fraction of
input traffic to node $i$ that originates from another node must
have $\lambda_C$ as an arrival curve . What is the upper-bound on
utilization factors for which a bound is obtained ?
\end{problem}
\end{filecontents}%
%%%%%%%%%%%%%%%%%%%%%%%%%%%%%%%%%%%%%%%%%%%%%%%%%%%%%%%%%%%%%%%%%%%%%%%%%%%%%%%%%
\problemfile{L21-19}
\begin{problem}
Consider the delay bound in \thref{theo-qofisbound}. Take the same
assumptions but assume also that the network is feedforward. Which
better bound can you give ?
\end{problem}
\end{filecontents}%

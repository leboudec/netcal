\problemfile{L20-1}
\begin{problem}
Compute the maximum buffer size $X$ for a system that is initially
empty, and where the input function is $R(t)=\int_0^t r(s)ds$, for
the following cases.
 \begin{enumerate}
        \item  if $r(t) = a$ (constant)
        \item  one on-off connection with peak rate 1 Mb/s, on period 1 sec,
        off period $\tau$ seconds, and trunk bit rate $c=0.5$ Mb/s.
        \item  if $r(t) = c + c \sin \omega t$, with trunk bit rate $c >0$.
 \end{enumerate}
\end{problem}
\end{filecontents}
%%%%%%%%%%%%%%%%%%%%%%%%%%%%%%%%%%%%%%%%%%%%%%%%%%%%%%%%%%%%%%%%%%%%%%%%%%%%%%%%%%%%%%%%%%%%%%%%%%%
\problemfile{L20-2}
\begin{problem}
You have a fixed buffer of size $X$, that receives a data input
$r(t)$. Determine the output rate $c$ that is required to avoid
buffer overflow given that the buffer is initially empty.
\end{problem}
\end{filecontents}
%%%%%%%%%%%%%%%%%%%%%%%%%%%%%%%%%%%%%%%%%%%%%%%%%%%%%%%%%%%%%%%%%%%%%%%%%%%%%%%%%%%%%%%%%%%%%%%%%%%
\problemfile{L20-3}
\begin{problem}
\begin{enumerate}
        \item  For a flow with constant bit rate $c$, give some
        possible arrival curves.
        \item  Consider a flow with an arrival curve given by: $\alpha(t)=B$,
        where $B$ is constant. What does this mean for the flow ?
\end{enumerate}
\end{problem}
\end{filecontents}
%%%%%%%%%%%%%%%%%%%%%%%%%%%%%%%%%%%%%%%%%%%%%%%%%%%%%%%%%%%%%%%%%%%%%%%%%%%%%%%%%%%%%%%%%%%%%%%%%%%%
%\problemfile{ex-flui-pb1}
%\begin{problem}
%\begin{enumerate}
%        \item  Draw the arrival curve for a linearly constrained flow with
%parameters $(m,b)$ that also has a
%limit on the peak rate. What is the arrival curve ?
%
%        \item  Show that a flow is $(P,B)$ constrained iff $W_{P}(t) \leq B$
%        for all $t$.
%\end{enumerate}
%\end{problem}
%\end{filecontents}
%%%%%%%%%%%%%%%%%%%%%%%%%%%%%%%%%%%%%%%%%%%%%%%%%%%%%%%%%%%%%%%%%%%%%%%%%%%%%%%%%%%%%%%%%%%%%%%%%%%
\problemfile{L20-4}
\begin{problem}
\begin{enumerate}
We say that a flow is $(P, B)$ constrained if it has
$\gamma_{P,B}$ as an arrival curve.
        \item  A trunk system has a buffer size of $B$ and a trunk bitrate of $P$.
Fill in the dots: (1) there is no loss if the input is $(., .)$
constrained (2) the output is $(.,.)$ constrained.

        \item  A $(P, B)$ constrained flow is fed into an infinite buffer served at a
        rate of $c$. What is the maximum delay~?
\end{enumerate}

\end{problem}
\end{filecontents}
%%%%%%%%%%%%%%%%%%%%%%%%%%%%%%%%%%%%%%%%%%%%%%%%%%%%%%%%%%%%%%%%%%%%%%%%%%%%%%%%%%%%%%%%%%%%%%%%%%%
\problemfile{L20-5}
\begin{problem}
[On-Off flows]
\begin{enumerate}
\item Assume a data flow is periodical, with period $T$, and satisfies the
following: $r(t) = p$ for $0 \leq t < T_{0}$, and
$r(t) = 0$ for $T_{0}\leq t <  T$.
\begin{enumerate}
\item Draw $R(t) = \int_{0}^{t} r(s) ds$

\item Find an arrival curve for the flow. Find the minimum arrival
curve for the flow.
\item Find the minimum $(r, b)$ such that
the flow is $(r, b)$ constrained.

\sol{An arrival curve could be $\alpha(t)=pt$\\
The minimal arrival curve is: \[ \alpha (t) = \left\{ \begin{array}{ll}
                R(t) & \mbox{if R(t) is sub-additive}\\
                inf\{\delta (t),\alpha (t),\alpha (t)\mpc\alpha (t),..\} & \mbox{otherwise}
                \end{array}
                \right. \]
When the flow is (r,b) contrained, the minimum (r,b) values are $r = \frac{p
T_{0}}{T}$, $b=T_{0} (p-r)$}

\end{enumerate}

\item A traffic flow uses a link with bitrate $P$ (bits/s). Data is
sent as packets of variable length. The flow is controlled by a leaky
bucket $(r, b)$. What is the maximum packet size ? What is the minimum
time interval between packets of maximum size ?

Application: P = 2 Mb/s, r = 0.2 Mb/s; what is the required burst
tolerance $b$
if the packet length is 2 Kbytes ? What is then the minimum
spacing between packets ?

\sol{
maximum packet size $L = p \frac{b}{p-r}$ \\
minimum time between packets $T_{m}=\frac{b}{r}$ \\
application: $b=L \frac{p-r}{p} = 1.8$Kbytes, $T_{m}=72$ ms.
}
\end{enumerate}
\end{problem}
\end{filecontents}
%%%%%%%%%%%%%%%%%%%%%%%%%%%%%%%%%%%%%%%%%%%%%%%%%%%%%%%%%%%%%%%%%%%%%%%%%%%%%%%%%%%%%%%%%%%
\problemfile{L20-6}
\begin{problem}
\nfs{Alternative definition of GCRA} Consider the following
alternative definition of the GCRA:
\begin{definition}
The GCRA ($T,\tau$) is a controller that takes as input a cell
arrival time {\tt t} and returns {\tt result}. It has internal
(static) variables {\tt X} (bucket level) and {\tt LCT} (last
conformance time).

\begin{itemize}
        \item  initially, {\tt X = 0} and {\tt LCT = 0}
        \item  when a cell arrives at time {\tt t}, then
        \begin{verbatim}
                if (X - t + LCT > tau)
                   result = NON-CONFORMANT;
                else {
                     X = max (X - t + LCT, 0) + T;
                     LCT = t;
                     result = CONFORMANT;
                     }
        \end{verbatim}
\end{itemize}
\end{definition}
Show that the two definitions of GCRA are equivalent.\\
\sol{
Comparing the two algorithms we see that, if equivalence holds, we
should have \texttt{tat = X + LCT}.\\ The proof is now as follows.
We show by recurrence that for every input sequence, the output of
both algorithms is the same and that \texttt{tat = X + LCT} at the
end of the processing for the last input.  This is true for the
first input, at $t=0$. Now let us assume that this is true for all
inputs up to time $t$. The test done by both algorithms is the
same since \texttt{tat = X + LCT} so the output is the same. Then
\texttt{max(t, tat) + T = t + max(X- t + LCT, 0) + T} thus the
relation \texttt{tat = X + LCT} still holds after the processing
of this input. }
\end{problem}
\end{filecontents}
%%%%%%%%%%%%%%%%%%%%%%%%%%%%%%%%%%%%%%%%%%%%%%%%%%%%%%%%%%%%%%%%%%%%%%%%%%%%%%55
\problemfile{L20-7}
\begin{problem}
\begin{enumerate}
\item For the following flows and a GCRA(10, 2), give the conformant and
non-conformant cells. Times are in cell slots at the link rate. Draw
the leaky bucket behaviour assuming instantaneous cell arrivals.
 \begin{enumerate}
        \item  0, 10, 18, 28, 38

        \item  0, 10, 15, 25, 35

        \item  0, 10, 18, 26, 36

        \item  0, 10, 11, 18, 28
\end{enumerate}
\ifsol
\begin{enumerate}
        \item  \texttt{+ + + + +}
        \item  \texttt{+ + - + +}
        \item  \texttt{+ + + - +}
        \item  \texttt{+ + - + +}
\end{enumerate}
\fi
\item What is the maximum number of cells that can flow back to back
with GCRA(T, CDVT) (maximum ``clump'' size) ?
\sol{maximum clump size = floor$(\frac{CDVT}{T-1} + 1)$}
\end{enumerate}
\end{problem}
\end{filecontents}
%%%%%%%%%%%%%%%%%%%%%%%%%%%%%%%%%%%%%%%%%%%%%%%%%%%%%%%%%%%%%%%%%%%%%%%%%%%%%%
\problemfile{L20-8}
\begin{problem}
\begin{enumerate}
        \item For the following flows and a GCRA(100, 500), give the conformant and
  non-conformant cells. Times are in cell slots at the link rate.
  \begin{enumerate}
        \item  0, 100, 110, 12, 130, 140, 150, 160, 170, 180, 1000, 1010
        \item  0, 100, 130, 160, 190, 220, 250, 280, 310, 1000, 1030
        \item  0, 10, 20, 300, 310, 320, 600, 610, 620, 800, 810, 820, 1000,
        1010, 1020, 1200, 1210, 1220, 1400, 1410, 1420, 1600, 1610, 1620
    \end{enumerate}

   \ifsol
        \begin{enumerate}
        \item  \texttt{+ + + + + + + - - - + +}
        \item  \texttt{+ + + + + + + + + + + +}
        \item  \texttt{20(+) - + + -}
        \end{enumerate}
\fi

        \item Assume that a cell flow has a minimum spacing of $\gamma$
        time units between cell emission times ($\gamma$ is the minimum
        time between the beginnings of two cell transmissions).  What is
        the maximum burst size for GCRA($T, \tau)$ ?  What is the minimum
        time between bursts of maximum size ?
\sol{
Considering $\gamma$ as the cell separation between cell beginnings,\\
Maximum burst size(MBS) = floor$\lfloor \frac{\tau}{T - \gamma}+1 \rfloor$\\
Minimum time between bursts = MBS$(T-\gamma)+\gamma$
}

        \item Assume that a cell flow has a minimum spacing between cells of
        $\gamma$ time units, and a minimum spacing between bursts of
        $T_{I}$. What is the maximum burst size ?
        \sol{Maximum burst size = floor$ \lfloor \frac{\min(T_{i}-T,\tau)}{T
-\gamma}+1 \rfloor $}

\end{enumerate}
\end{problem}
\end{filecontents}
%%%%%%%%%%%%%%%%%%%%%%%%%%%%%%%%%%%%%%%%%%%%%%%%%%%%%%%%%%%%%%%%%%%%%%%%%%%%%%
\problemfile{L20-9}
\begin{problem}
For a CBR connection, here are some values from an ATM operator:
        \begin{verbatim}
 peak cell rate (cells/s)     100   1000  10000 100000
 CDVT     (microseconds)     2900   1200    400    135
        \end{verbatim}
        \begin{enumerate}
                \item What are the $(P, B)$ parameters in b/s and bits for
                each case ?  How does $T$ compare to $\tau$ ?
\ifsol

\begin{verbatim}
P          (b/s)  42400  424000   4240000   42400000
B            (b)    547     933      2120       6148
T    (microsecs)  10000    1000       100         10
CDVT (microsecs)   2900    1200       400        135
\end{verbatim}
\fi
                \item If a connection requires a peak cell rate of 1000 cells
                per second and a cell delay variation of 1400 microseconds,
                what can be done ?
\sol{
We need to find an arrival curve in the family given by Swiss Telecom
that can bound from above your arrival curve. In the  absence of information
about the link rate, this  means choosing a peak rate $P'$ larger  than 1000,
such that the  tolerance in cells, $B'_{p}= \delta + CDVT' \times P'$ is as
large as the requested tolerance  $B_{p}= \delta + CDVT \times P$. In other
words, call $f$ the function mapping the peak rate to the tolerance, we need
to find the minimum $P'$ such that $P' f(P') \geq Pf(P)$. A plot shows that
$\log f$ is linear in $\log P$. This gives $P' =$ 1738 cells/sec.
\begin{figure}[ht]
\begin{center}
%\includegraphics[angle=-0,scale=.4]{cdvtswisscom.eps}
\insfig{cdvtswisscom}{0.4} \caption{relationship between tolerance
and peak rate} \label{first}
\end{center}
\end{figure}
}
                \item Assume the operator allocates the peak rate to every
                connection at one buffer.  What is the amount of buffer
                required to assure absence of loss ?  Numerical Application
                for each of the following cases, where a number $N$ of
                identical connections with peak cell rate $P$ is multiplexed.
                \begin{verbatim}
 case                       1      2      3      4
 nb of connnections      3000    300     30      3
 peak cell rate (c/s)     100   1000  10000 100000
            \end{verbatim}
\ifsol
From proposition 1.3.2, it is sufficient to allocate a buffer equal to
$B_{p}$. This gives:
\begin{verbatim}
case                           1      2      3      4
number of connnections      3000    300     30      3
peak cell rate (cells/s)     100   1000  10000 100000
required buffer (cells)     3870    660    150     44
\end{verbatim}
It is possible to reduce the buffer size if we can exploit some constraints on
the peak rate.
\fi
        \end{enumerate}

\end{problem}
\end{filecontents}
%%%%%%%%%%%%%%%%%%%%%%%%%%%%%%%%%%%%%%%%%%%%%%%%%%%%%%%%%%%%%%%%%%%%%%%%%%%%%%%
%\problemfile{L20ex-fluid-gal0}
%\begin{problem}
%Define in at most four lines each of the concepts given in the
%Introduction
%\end{problem}
%\end{filecontents}
%
%\problemfile{L20ex-fluid-delay}
%\begin{problem}
%\begin{enumerate}
%        \item Is is true that the virtual delay defined in Section
%        \ref{d4-fluid-def} satisfies the following equation~?
%        (Justify your answer)
%        $$
%        d(t) = \max \left\{
%               T: T\geq 0 \mand R(t) \geq R^*(t + T)
%              \right\}
%        $$
%
%        \item  Show that if $R^{*}$ is continuous, then
%        $R^{*}\left(t +d(t)\right) = R(t)$. How can you interprete the
%        assumption on continuity ?
%\end{enumerate}
%\end{problem}
%\end{filecontents}
%
%\problemfile{L20ex-buflb2}
%\begin{problem}
%\begin{enumerate}
%        \item  Consider again Proposition \ref{prop-lb1} but assume that the trunk
%system is not first-in first-out, but only work conserving. What is
%now the bound on the delay ?
%        \item  Same question if we assume in addition that the trunk system
%        is first-in first-out {\em per flow}.
%\end{enumerate}
%\end{problem}
%\end{filecontents}
%%%%%%%%%%%%%%%%%%%%%%%%%%%%%%%%%%%%%%%%%%%%%%%%%%%%%%%%%%%%%%%%%%%%%%%%%%%%
%% Sept 98
%% PROBLEM FILES: Module D21 = LB and GCRA
%% supp problems to be merged with the new file
%%%%%%%%%%%%%%%%%%%%%%%%%%%%%%%%%%%%%%%%%%%%%%%%%%%%%%%%%%%%%%%%%%%%%%%%%%%%
%%%%%%%%%%%%%% EXAM ED 97 %%%%%%%%%%%%%%%%%%%%%%%%%%%%%%%%%%%%%%%%%%%%%%%%%%%
\problemfile{L20-10}
\begin{problem}
\nfs{September 1998}

The two questions in this problem are independent.
\begin{enumerate}
        \item  An ATM source is constrained by GCRA($T=30$ slots, $\tau=60$ slots),
where time is counted in slots. One slot is the time it takes to
transmit one cell on the link. The source sends cells according to
the following algorithm.
\begin{itemize}
        \item In a first phase, cells are sent at times $t(1)=0$,
        $t(2)=15$, $t(3)=30, \ldots,t(n)= 15(n-1)$ as long as all cells
        are conformant.  In other words, the number $n$ is the largest
        integer such that all cells sent at times $t(i)=15(i-1)$, $i\leq
        n$ are conformant. The sending of  cell $n$ at time $t(n)$ ends
        the first phase.

        \item  Then the source enters the second phase. The subsequent cell
        $n+1$ is sent at the earliest time after
        $t(n)$ at which a conformant cell can be sent, and the same is
        repeated for ever. In other words, call $t(k)$ the sending time for
        cell $k$, with $k>n$; we have then: $t(k)$ is the earliest time after
        $t(k-1)$ at which a conformant cell can be sent.
\end{itemize}

How many cells were sent by the source in time interval $[0, 151]$~?


        \item A network node can be modeled as a single buffer with a
        constant output rate $c$ (in cells per second).  It receives $I$
        ATM connections labeled $1, \ldots, I$.  Each ATM connection has a
        peak cell rate $p_{i}$ (in cells per second) and a cell delay
        variation tolerance $\tau_{i}$ (in seconds) for $1 \leq i \leq I$.
        The total input rate into the buffer is at least as large as
        $\sum_{i=1}^I p_{i}$ (which is equivalent to saying that it is
        unlimited). What is the buffer size (in cells) required for a
        loss-free operation~?

\end{enumerate}
\end{problem}
\end{filecontents}
%%%%%% ED Exam 98%%%%%%%%%%%%%%%%%%%%%%%%%%%%%%%%%%%%%%%%%%%%%%%%%%%%%%
\problemfile{L20-11}
\begin{problem}
In this problem, time is counted in slots. One slot is the duration to transmit
one ATM cell on the link.

\begin{enumerate}
        \item An ATM source $S_{1}$ is constrained by GCRA($T=50$ slots,
        $\tau=500$ slots), The source sends cells according to the
        following algorithm.
\begin{itemize}
        \item In a first phase, cells are sent at times $t(1)=0$,
        $t(2)=10$, $t(3)=20, \ldots,t(n)= 10(n-1)$ as long as all cells
        are conformant.  In other words, the number $n$ is the largest
        integer such that all cells sent at times $t(i)=10(i-1)$, $i\leq
        n$ are conformant. The sending of  cell $n$ at time $t(n)$ ends
        the first phase.

        \item  Then the source enters the second phase. The subsequent cell
        $n+1$ is sent at the earliest time after
        $t(n)$ at which a conformant cell can be sent, and the same is
        repeated for ever. In other words, call $t(k)$ the sending time for
        cell $k$, with $k>n$; we have then: $t(k)$ is the earliest time after
        $t(k-1)$ at which a conformant cell can be sent.
\end{itemize}

How many cells were sent by the source in time interval $[0, 401]$~?


        \item An ATM source $S_{2}$ is constrained by \emph{both}
        GCRA($T=10$ slots, $\tau=2$ slots) and GCRA($T=50$ slots,
        $\tau=500$ slots).  The source starts at time $0$, and has an
        infinite supply of cells to send.  The source sends its cells as
        soon as it is permitted by the combination of the GCRAs.  We call
        $t(n)$ the time at which the source sends the $n$th cell, with
        $t(1)=0$. What is the value of $t(15)$~?

\end{enumerate}
\end{problem}
\end{filecontents}
%%%%%%%%%%%%%%%%%%%%%%%%%%%%%%%%%%%%%%%%%%%%%%%%%%%%%%%%%%%%%%%%%%%%%%%%%%%%
\problemfile{L20-11a}
 \begin{problem}
 Consider a flow $R(t)$ receiving a minimum service curve
 guarantee $\beta$. Assume that
\begin{itemize}
  \item $\beta$ is concave and wide-sense increasing
  \item the $\inf$ in $R \mpc \beta$ is a $\min$
 \end{itemize}
For all $t$, call $\tau(t)$ a number such that
$$
(R \mpc \beta)(t)= R(\tau(t)) + \beta(t-\tau(t))
$$
Show that it is possible to choose $\tau$ such that if $t_1 \leq
t_2$ then  $\tau(t_1) \leq \tau(t_2)$.
 \sol{~\\Call $\tau_1=\tau(t_1)$ and
consider any $t' \leq \tau_1$. From the definition of $\tau_1$, we
have
$$
S(t') + \beta(t_1 - t' ) \geq S(\tau_1) + \beta(t_1 - \tau_1 )
$$
and thus
$$
S(t') + \beta(t_2 - t' ) \geq S(\tau_1) + \beta(t_1 - \tau_1 ) -
\beta(t_1 - t' ) + \beta(t_2 - t' )
$$
Now $\beta$ is convex, thus for any four numbers $a, b, c, d$ such
that $a \leq c  \leq b$,  $a \leq d  \leq b$ and $a+b=c+d$, we
have
$$\beta(a) + \beta(b) \geq \beta(c) + \beta(d)$$
(the interested reader will be convinced by drawing a small
figure). Applying this to $a=t_1- \tau_1, b=t_2 - t', c=t_1-t',
d=t_2-\tau_1$ gives
$$
S(t')+ \beta(t_2-t') \geq S(\tau_1)+\beta(t_2 - \tau_1)
$$
and the above equation holds for all $t'\leq \tau_1$. Consider now
the minimum, for a fixed $t_2$, of $S(t') +\beta(t_2-t')$ over all
$t' \leq t_2$. The above equation shows that the minimum is
reached for some $t' \geq \tau_1$.}
 \end{problem}
 \end{filecontents}
%%%%%%%%%%%%%%%%%%%%%%%%%%%%%%%%%%%%%%%%%%%%%%%%%%%%%%%%%%%%%%%%%%%%%%%%%%%%
\problemfile{L20-12}
\begin{problem}
        \begin{enumerate}
                        \item Find the maximum backlog and maximum delay for an ATM
                CBR connection with peak rate $P$ and cell delay variation
                $\tau$, assuming the service curve is $c(t) = r(t-T_{0})^{+}$
\sol{The CBR connection is defined by
        GCRA($\frac{\delta}{P}$, $\tau$)\\
        if p<r then,
        BackLog = \(B+PT_0\)\\
        Maximum Delay = \( \frac{B}{r}+T_0 \)\\
        otherwise,  backlog and max, delay are infinity.}

                \item Find the maximum backlog and maximum delay for an ATM
                VBR connection with peak rate $P$, cell delay variation
                $\tau$, sustainable cell rate $M$ and burst tolerance
$\tau_{B}$ (in
                seconds), assuming the service curve is $c(t) = r(t-T_{0})^{+}$
\sol{ }
\end{enumerate}

\end{problem}
\end{filecontents}
%%%%%%%%%%%%%%%%%%%%%%%%%%%%%%%%%%%%%%%%%%%%%%%%%%%%%%%%%%%%%%%%%%%%%%%%%%%%
\problemfile{L20-13}
\begin{problem}
Show the following statements:
\begin{enumerate}
 \item  Consider a $(P,B)$ constrained flow, served at a rate $c\geq P$.
       The output is also $(P,B)$ constrained.
\sol{ We wish to compute the output $\alpha \mpd \beta$ where
$\alpha(t)=Pt+B$ and $\beta(t)=ct$.
$$
\alpha \mpd \beta(t) = Pt+B + \sup_{u\geq0} (P-c)u
$$
Since $P\leq c$, this is largest when $u=0$. Therefore $\alpha
\mpd \beta(t)=\alpha(t)$. }
 \item  Assume $a()$ has a bounded right-handside derivative.
        Then the output for a flow constrained by $a()$, served in a buffer
at a
        constant rate $c \geq \sup_{t \geq 0} a'(t)$, is also constrained by
        $a()$.
\sol{
$$
a \mpd \beta(t) = \sup_{u\geq0} a(t+u) -cu
$$
By the mean-value theorem there is a $v\in[t,t+u]$ such that
$$a(t+u) = a(t) + u a'(v).$$
Thus,
$$
a \mpd \beta(t) = a(t) + \sup_{u\geq0} (a'(v)-c)u
$$
Since all derivatives are smaller than $c$, this is at its
largest when $u=0$, giving $a \mpd \beta(t) = a(t)$. }
\end{enumerate}
\end{problem}
\end{filecontents}
%%%%%%%%%%%%%%%%%%%%%%%%%%%%%%%%%%%%%%%%%%%%%%%%%%%%%%%%%%%%%%%%%%%%%%%%%%%%
\problemfile{L20-14}
\begin{problem}
\begin{enumerate}
                \item Find the the arrival curve constraining the output for
                an ATM CBR connection with peak rate $P$ and cell delay
                variation $\tau$, assuming the service curve is $c(t) =
                r(t-T_{0})^{+}$

                \item Find the arrival curve constraining the output
                for an ATM VBR connection with peak rate $P$, cell delay
                variation $\tau$, sustainable cell rate $M$ and burst
                tolerance $\tau_{B}$ (in seconds), assuming the service curve
                is $c(t) = r(t-T_{0})^{+}$
\end{enumerate}

\end{problem}
\end{filecontents}
%%%%%%%%%%%%%%%%%%%%%%%%%%%%%%%%%%%%%%%%%%%%%%%%%%%%%%%%%%%%%%%%%%%%%%%%%%%%
\problemfile{L20-15}
\begin{problem}
Consider the figure ``Derivation of arrival curve for the output
of a flow
  served in a node with rate-latency service curve $\beta_{R,T}$".
What can be said if $t_{0}$ in the Figure is infinite, namely, if
$a'(t) > r$ for all $t$ ?
\end{problem}
\end{filecontents}
%%%%%%%%%%%%%%%%%%%%%%%%%%%%%%%%%%%%%%%%%%%%%%%%%%%%%%%%%%%%%%%%%%%%%%%%%%%%
\problemfile{L20-16}
\begin{problem}

Consider a series of guaranteed service nodes with service curves
$c_{i}(t) = r_{i}(t-T_{i})^{+}$. What is the maximum delay through
this system for a flow constrained by $(m,b)$ ? \sol{Defining
\(T_0 = \sum_{\forall i} T_i \) and \(R_0 = min_{\forall i}
\{R_i\}\)
                \[ Max. Delay = \left\{ \begin{array}{ll}
                \frac{b}{R_0}+T_0 & \mbox{if $R_0>m$}\\
                \infty & \mbox{otherwise}
                \end{array}
                \right. \]
}
\end{problem}
\end{filecontents}
%%%%%%%%%%%%%%%%%%%%%%%%%%%%%%%%%%%%%%%%%%%%%%%%%%%%%%%%%%%%%%%%%%%%%%%%%%%%
\problemfile{L20-17}
\begin{problem}

A flow with T-SPEC $(p,M,r,b)$ traverses nodes 1 and 2.  Node $i$
offers a service curve $c_{i}(t) = R_{i}(t-T_{i})^{+}$. What
buffer size is required for the flow at node 2~? \sol{ Do this
proof again using max instead.\\
 Define $t_i=\frac{B-M}{p-r}$
\\
The buffer required for the flow at node 2 is:\\ if $t_i<T_1$
BackLog = \(b+r(T_1+T_2)\)\\ if $t_i>T_1$ and $p>R_1>r$\\
\hspace{5mm}if $T_2>t_i-T_1$ BackLog = \(b+rT_2+rT_1\)\\
\hspace{5mm}if $T_2<t_i-T_1$ and $R_2>R_1$ BackLog =
\(b+rt_i+(T_1-t_i)R_1+R_1T_2\)\\ \hspace{5mm}if $T_2<t_i-T_1$ and
$R_2<R_1$ BackLog = \(b+rt_i-R_2(t_i-T_1-T_2)\)\\ if $t_i>T_1$ and
$p<R_1$\\ \hspace{5mm}if $T_2>t_i-T_1$ BackLog =
\(M+pt_i+r(T_2-t_i+T_1)\)\\ \hspace{5mm}if $T_2<t_i-T_1$ and
$R_2>p$ BackLog = \(M+pT_1+pT_2\)\\ \hspace{5mm}if $T_2<t_i-T_1$
and $R_2<p$ BackLog = \(M+pt_i-R_2(t_i-T_1-T_2)\)\\ else BackLog =
$\infty$ }
\end{problem}
\end{filecontents}
%%%%%%%%%%%%%%%%%%%%%%%%%%%%%%%%%%%%%%%%%%%%%%%%%%%%%%%%%%%%%%%%%%%%%%%%%%%%
\problemfile{L20-18}
\begin{problem}
A flow with T-SPEC $(p,M,r,b)$ traverses nodes 1 and 2.  Node $i$
offers a service curve $c_{i}(t) = R_{i}(t-T_{i})^{+}$. A shaper
is placed between nodes 1 and 2. The shaper forces the flow to the
arrival curve $z(t)=\min(R_{2}t, bt+m)$.
\begin{enumerate}

\item
What buffer size is required for the flow at the shaper~? \sol{
Let $\tau=\frac{b-M}{p-r}$. If $T_1 > \tau$ then the buffer size
is $B=rT_1+b$. Otherwise,
$$
B = \bracket{ M+(P-R_1)*\tau+R_1T_1 \gap,R_1\leq P, R_1\leq R_2 \\
M+(P-R_2)*\tau+R_2T_1 \gap,R_1\leq P, R_1>R_2 \\ M+PT_1 \gap\gap
,R_1 > P, R_2\geq P \\ M+\tau*(P-R_2)+R_2T_1 \gap ,R_1 > P, R_2<
P }
$$
}
\item
What buffer size is required at node 2~? What value do you find if
$T_{1}=T_{2}$~?
\sol{\\
Let $\tau_{2}=\frac{m}{R_{2}-a}$. The buffer size is:
$$
B_{2}=\bracket{ aT_{2} + m \gap, \tau_{2}<T_{2} \\ R_{2}T_{2} \gap, otherwise}.
$$
$B_{2}$ does not depend on $T_{1}$.
}
\item Compare the sum of the preceding buffer sizes to the size
that would be required if no re-shaping is performed.
\item Give an arrival curve for the output of node 2.
\sol{ Choose any arrival curve that satifies the
conditions}
\end{enumerate}

\end{problem}
\end{filecontents}
%%%%%%%%%%%%%%%%%%%%%%%%%%%%%%%%%%%%%%%%%%%%%%%%%%%%%%%%%%%%%%%%%%%%%%%%%%%%
\problemfile{L20-19}
\begin{problem}
        Prove the formula giving of paragraph ``Buffer Sizing at a
        Re-shaper"
\sol{Follows}
\end{problem}
\end{filecontents}
%%%%%%%%%%%%%%%%%%%%%%%%%%%%%%%%%%%%%%%%%%%%%%%%%%%%%%%%%%%%%%%%%%%%%%%%%%%%
\problemfile{L20-20}
\begin{problem}
        Is Theorem ``Input-Output Characterization of Greedy Shapers" a stronger result
        than Corollary ``Service Curve offered by a Greedy Shaper"~?
\sol{ No, the theorem is equivalent to the corollary. Indeed, if
corollary is true, then
$$
R^*\geq R \mpc \sigma
$$
Now by the arrival curve property
$$
R* \leq R^* \mpc \sigma \leq R \otimes \sigma
$$
since $R^*\leq R$. Thus $R*=R\otimes \sigma.$ }
\end{problem}
\end{filecontents}
%%%%%%%%%%%%%%%%%%%%%%%%%%%%%%%%%%%%%%%%%%%%%%%%%%%%%%%%%%%%%%%%%%%%%%%%%%%%%
%\problemfile{L20ex-fluid-gal}
%\begin{problem}
%
%Define in at most ten lines each of the concepts given in the
%Introduction
%\end{problem}
%\end{filecontents}
%%%%%%%%%%%%%%%%%%%%%%%%%%%%%%%%%%%%%%%%%%%%%%%%%%%%%%%%%%%%%%%%%%%%%%%%%%%%
\problemfile{L20-21}
\begin{problem}

\begin{enumerate}

    \item Explain what is meant by ``we pay bursts only once''.

        \item Give a summary in at most 15 lines of the main properties of
        shapers

        \item  Define the following concepts by using the $\otimes $ operator:
        Service Curve, Arrival Curve, Shaper

        \item  What is a greedy source~?
\end{enumerate}
\end{problem}
\end{filecontents}
%%%%%%%%%%%%%%%%%%%%%%%%%%%%%%%%%%%%%%%%%%%%%%%%%%%%%%%%%%%%%%%%%%%%%%%%%%%%
\problemfile{L20-22}
\begin{problem}

\begin{enumerate}
        \item  Show that for a constant bit rate trunk with rate $c$, the backlog at
time $t$ is given by
$$
W(t) = \sup_{s \leq t} \left \{ R(t) - R^{*}(s) - c(t-s) \right \}
$$

        \item  What does the formula become if we assume only that,
        instead a constant bit rate trunk, the node is a scheduler
offering $\beta$ as a service curve~?
\end{enumerate}

\end{problem}
\end{filecontents}
%%%%%%%%%%%%%%%%%%%%%%%%%%%%%%%%%%%%%%%%%%%%%%%%%%%%%%%%%%%%%%%%%%%%%%%%%%%%
\problemfile{L20-23}
\begin{problem}
Is it true that offering a service curve $\beta$ implies that,
during any busy period of length $t$, the amount of service
received rate is at least $\beta(t)$ ? \sol{No, the property means
that $\beta$ is a strict service curve. In general, a service
curve property is not strict. Consider for example a shaper. }
\end{problem}
\end{filecontents}
%%%%%%%%%%%%%%%%%%%%%%%%%%%%%%%%%%%%%%%%%%%%%%%%%%%%%%%%%%%%%%%%%%%%%%%%%%%%
%% Sept 98
%% PROBLEM FILES: Module D22 = network calculus
%% supp problems to be merged with the new filw
%%%%%%%%%%%%%%%%%%%%%%%%%%%%%%%%%%%%%%%%%%%%%%%%%%%%%%%%%%%%%%%%%%%%%%%%%%%%
%%%%%%%%%%%%%%%%%%%%%%%%%%%%%%%%%%%%%%%%%%%%%%%%%%%%%%%%%%%%%%%%%%%%%%%%%%%%
\problemfile{L20-24}
\begin{problem}
A flow $S(t)$ is constrained by an arrival curve $\alpha$. The
flow is fed into a shaper, with shaping curve $\sigma$. We assume
that
$$
\alpha(s) = \min (m + ps, b +rs)
$$
and
$$
\sigma (s) = \min ( Ps, B+Rs )
$$
We assume that $p>r$, $m \leq b$ and $P \geq R$.

The shaper has a fixed buffer size equal to $X \geq m$. We require
that the buffer never overflows.

\begin{enumerate}
        \item Assume that $B=+\infty$.  Find the smallest
        of $P$ which guarantees that there is no buffer overflow.
        Let $P_{0}$ be this value.

        \item We do not assume that $B=+ \infty$ any more, but we
        assume that $P$ is set to the value $P_{0}$ computed in the
        previous question.  Find the value $(B_{0}, R_{0})$ of $(B,R)$  which
        guarantees that there is no buffer overflow and minimizes the cost
        function $c(B,R)= aB + R$, where $a$ is a positive constant.

        What is the
        maximum virtual delay if $(P, B, R)=(P_{0},B_{0}, R_{0})$~?
\end{enumerate}
\end{problem}
\end{filecontents}
%%%%%%%%%%%%%%  %%%%%%%%%%%%%%%%%%%
\problemfile{L20-25}
\begin{problem}
\nfs{GCRA and Eff Bw, easy; GCRA and Eff Bw exam ED 99, modified
feb 99 by leb\\}
We consider a buffer of size $X$ cells, served at
a constant
        rate of $c$ cells per second.  We put $N$ identical connections into
        the buffer; each of the $N$ connections is constrained both by
        GCRA($T_{1}$, $\tau_{1}$) and GCRA($T_{2}$, $\tau_{2}$).  What is
        the maximum value of $N$ which is possible if we want to guarantee
        that there is no cell loss at all~?

        Give the numerical application for $T_{1}=0.5$~ms, $\tau_{1}=4.5$~ms,
        $T_{2}=5$~ms, $\tau_{2}=495$~ms, $c=10^6$~cells/second, $X=10^4$~cells
        \ifsol

        Call $e$ the effective bandwidth for a delay of $\frac{X}{c}$. The
        answer is the largest integer $\leq$ $\frac{c}{e}$.

        NA: $D=  \frac{X}{c} =10$ms. $e=11000/6$ cells/second. $N=6000/11 = 545$
        \fi
\end{problem}%
\end{filecontents}
%%%%%%%%%%%%%%%%%%%%%exam ED 99%%%%%%%%%%%%%%%%%%%%%%%%%%%%%%
\problemfile{L20-26}
\begin{problem}
\nfs{non initial bucket level, hard\\}
 We consider a flow defined by
its function $R(t)$, with
                $R(t)=$ the number of bits observed since time $t=0$.
 \begin{enumerate}
  \item The
   flow is fed into a buffer, served at a rate $r$.  Call $q(t)$
   the buffer content at time $t$.  We do the same assumptions as
   in the lecture, namely, the buffer is large enough, and is
   initially empty. What is the expression of $q(t)$ assuming we
   know $R(t)$~?

   We assume now that, unlike what we saw in the lecture, the initial
   buffer content (at time $t=0$) is \emph{not} $0$, but some value $q_{0}
   \geq 0$. What is now the expression for $q(t)$~?
\sol{\\
The expression of $q(t)$ when the  buffer is initially empty is:
$$
q(t) = \sup_{s: s \leq t} \left \{ R(t) - R(s) - r(t-s) \right \}
$$
If the initial buffer content is some value $q_0$ then we have:
$$
q(t) = \max { ( \sup_{s: s \leq t} \left \{ R(t) - R(s) - r(t-s) \right \}, R(t)- rt+q_0  ) }
$$
}
 \item The flow is put into a leaky bucket policer, with rate
 $r$ and bucket size $b$.  This is a policer, not a shaper, so
 nonconformant bits are discarded.  We assume that the bucket
 is large enough, and is initially empty.  What is the
 condition on $R$ which ensures that no bit is discarded by the
 policer (in other words, that the flow is conformant)~?

 We assume now that, unlike what we saw in the lecture, the
 initial \emph{bucket} content (at time $t=0$) is \emph{not}
 $0$, but some value $b_{0} \geq 0$.  What is now the condition
 on $R$ which ensures that no bit is discarded by the policer
 (in other words, that the flow is conformant)~?
\sol{\\
The condition on $R$ which ensures that the flow is conformant when the bucket is initially empty is :
$$
\sup_{s: s \leq t} \left \{ R(t) - R(s) - r(t-s) \right \} \leq b
$$
If the initial bucket content is some value $b_0$ then the condition becomes:
$$
\max { ( \sup_{s: s \leq t} \left \{ R(t) - R(s) - r(t-s) \right \}, R(t)- rt+b_0  ) } \leq b
$$
}
\end{enumerate}
\end{problem}%
\end{filecontents}
%%%%%% de Thiele %%%%%%%%%%%%%%%%%%%%%%%%%%%%%%%%%%%%%%%%%%%%%%%%%%%%%%
\problemfile{L20-27}
\begin{problem}
Consider a variable capacity network node, with capacity curve
$M(t)$. Show that there is one maximum function $S^*(t)$ such that
for all $0 \leq s \leq t$, we have
$$M(t)-M(s) \geq S^*(t-s)
$$
Show that $S^*$ is super-additive.

Conversely, if a function $\beta$ is super-additive, show that
there is a variable capacity network node, with capacity curve
$M(t)$, such that for all $0 \leq s \leq t$, we have $M(t)-M(s)
\geq S^*(t-s) $.

Show that, with a notable exception, a shaper cannot be modeled as
a variable capacity node.
\end{problem}
\end{filecontents}
%%%%%% packetized shaper %%%%%%%%%%%%%%%%%%%%%%%%%%%%%%%%%%%%%%%%%%%%%%%%%%%%
\problemfile{L20-28}
\begin{problem}
\begin{enumerate}
  \item Consider a packetized greedy shaper with shaping curve
  $\sigma(t)=rt$ for $t \geq 0$. Assume that $L(k)= k M$ where $M$ is fixed.
  Assume that the input is given by $R(t)=10 M$ for $t>0$ and $R(0)=0$. Compute
  the sequence $R^{(i)}(t)$ used in the representation of the output of the
  packetized greedy shaper, for $i=1,2,3,...$.
  \sol{
  The output $R^{(i)}(t)$ is a train of packets $m=1,...,10$, all of size $M$ at
  times $T^{(i)}_m = (m+i-1) \frac{M}{r}$. As $i$ goes to infinity,
  $R^{(i)}(t)$ goes to $0$, for any fixed $t$.}
  \item Same question if $\sigma(t)=(rt+ 2M) 1_\{t>0\}$.
  \sol{
  $T^{(1)}_1 = T^{(1)}_2 = 0$ and otherwise $T^{(1)}_m = (m-2) \frac{M}{r}$.
  For $i \geq 1$ we have
  $R^{(i)}=R^{(1)}$ since \thref{theo-psps} applies.}
\end{enumerate}

\end{problem}
\end{filecontents}
%%%%%%%%%%%%%% equivalence GCRA1 et GCRA 2 %%%%%%%%%%%%%%%%%%%%%%%%%%%%%%%%%
\problemfile{L20-gcra2}
\begin{problem}
\nfs{April 1999} Consider the discrete time leaky bucket
algorithm, defined as follows. It operates like the GCRA, with the
following code.
\begin{itemize}
        \item  initially, {\tt X = 0} and {\tt LCT = 0}
        \item  when a cell arrives at time {\tt t}, then
        \begin{verbatim}
                if (X - t + LCT > tau)
                   result = NON-CONFORMANT;
                else {
                     X = max (X - t + LCT, 0) + T;
                     LCT = t;
                     result = CONFORMANT;
                     }
        \end{verbatim}
\end{itemize}
Show now that for any input, both the discrete time leaky bucket
and the GCRA produce the same output.

\ifsol {\em We show by induction on time that, at every point in
time, the following relation holds
\begin{equation}\label{eq-gcra-33}
  {\tt tat} = {\tt X} + {\tt LCT}
\end{equation}
and that they produce the same output.

This  is true for $t=0$. Now assume that this is also true up to
some time $s$ and consider the first cell arrival time $t > s$. It
follows from the induction hypothesis that both algorithms perform
the same test and thus produce the same output. In addition, the
new value of ${\tt tat}$ with the GCRA is
$${\tt tat'} = {\tt max (t, tat) + T}$$
while the discrete time leaky bucket produces
$${\tt X'= max (X - t + LCT, 0) + T }
\gap \mand \gap {\tt LCT' = t }$$ from where it follows
immediately that Equation(~\ref{eq-gcra-33}) continues to hold. }
\fi

Interpret the discrete time leaky bucket algorithm as a leaky
bucket with appropriate units for time and data, and use this to
show that the GCRA is equivalent to a leaky bucket. What is the
mapping between GCRA parameters and leaky bucket parameters~?
\ifsol {\em We map the discrete time flow to a continuous time
flow, using Equation~(~\ref{eq-map-inter}). This is equivalent to
assuming a continuous time model, with instantaneous cell
arrivals. We can interprete the discrete time leaky bucket as
follows. Upon a cell arrival, the level of fluid is updated,
accounting for a leak rate of 1; then, if the updated level is
less than {\tt tau}, the arriving cell pours into the bucket an
amount of {\tt T} units of fluid. This is equivalent to requiring
that the bucket never exceeds {\tt T + tau}. Remember that we
think of an arriving cell stream as a fluid with an impulse at a
cell arrival. The algorithm is thus equivalent to a leaky bucket
controller, with bucket depth {\tt T + tau}, leak rate equal to 1
unit of fluid per time unit, and with every cell pouring into the
bucket an instantaneous amount of {\tt T} units of fluid.

Now we change the unit of data in order to let the cell size be
equal to $\delta$ units of data. The leak rate is now equal to $r
= \frac{\delta}{\tt T}$, and the bucket size to $b =
\frac{(T+\tau)\delta}{T}$. } \fi
\end{problem}
\end{filecontents}
%%%%%%%%%%%%%%%%%%%%%%%%%%%%%%%%%%%%%%%%%%%%%%%%%%%%%%%%%%%%%%%%%%%%%%%%%%%%
\problemfile{L20-29}
\begin{problem}
\nfs{exam ed fev 2000; `` une classe au mus\'{e}e"} Consider a
source given by the function
$$
\bracket{R(t) = B \mf t > 0\\R(t)= 0 \mf t \leq 0 }
$$
Thus the flow consists of an instantaneous burst of $B$ bits.
\begin{enumerate}
\item What is the minimum arrival curve for the flow~?
  \item Assume that the flow is served in one node that offers a
  minimum service curve of the rate latency type, with rate $r$
  and latency $\Delta$. What is the maximum delay for the last bit
  of the flow~?
  \item We assume now that the flow goes through a series of two
  nodes, $\calN_1$ and $\calN_2$, where $\calN_i$ offers to the
  flow a
  minimum service curve of the rate latency type, with rate $r_i$
  and latency $\Delta_i$, for $i=1,2$. What is the the maximum delay for the last bit
  of the flow through the series of two nodes~?
  \item With the same assumption as in the previous item, call $R_1(t)$ the function describing the
  flow at the output of node $\calN_1$ (thus at the input of node $\calN_2$). What is
  the worst case minimum arrival curve for $R_1$~?
  \item We assume that we insert between $\calN_1$ and $\calN_2$ a
  ``reformatter" $\calS$.
  The input to $\calS$ is $R_1(t)$. We call $R'_1(t)$ the output of $\calS$.
  Thus $R'_1(t)$ is now the input to $\calN_2$.
   The function of the ``reformatter"$\calS$ is to delay the
  flow $R_1$ in order to output a flow $R'_1$ that is a delayed version of
  $R$. In other words, we must have $R'_1(t)=R(t-d)$ for some
  $d$. We assume that the reformatter $\calS$ is optimal in the
  sense that it chooses the smallest possible $d$. In the worst
  case, what is this optimal value of $d$~?
  \item With the same assumptions as in the previous item, what is
  the worst case end-to-end delay through the series of nodes
  $\calN_1, \calS, \calN_2$~? Is the reformatter transparent~?
\end{enumerate}
\end{problem}
\end{filecontents}
%%%%%%%%%%%%%%%%%%%%%%%%%%%%%%%%%%%%%%%%%%%%%%%%%%%%%%%%%%%%%%%%%%%%%%%%%%%%
 \problemfile{L20-31}
 \begin{problem}
 \nfs{another vl shaper}
 Let $\sigma$ be a good function. Consider the concatenation of a
 bit-by-bit greedy shaper, with curve $\sigma$, and an
 $L$-packetizer. Assume that $\sigma(0^+)=0$.
 Consider only inputs that are $L$-packetized
\begin{enumerate}
  \item Is this system a packetized shaper for $\sigma$ ?\sol{~No,
  the condition on $\sigma$ implies that a packetized shaper for
  $\sigma$ is a degenerate system, and this one is not.}
  \item Is it a packetized shaper for $\sigma+l_{\max}$ ?
  \sol{~This question is more reasonable. Indeed, since $P^L(x)
  \geq x -l_{\max}$, the answer is yes.}
  \item Is it a packetized greedy shaper for $\sigma+l_{\max}$ ?
  \sol{~This question is more difficult. In general, the answer is no.
  To see why, consider $\sigma(t)=rt$ and draw any sample path. With this
  system, the first packet is delayed by $\frac{l_1}{r}$ whereas the
  packetized greedy shaper for $rt + l_{\max}$ would not delay the
  first packet.}
\end{enumerate}
 \end{problem}
 \end{filecontents}
%%%%%%%%%%%%%%%%%%%%%%%%%%%%%%%%%%%%%%%%%%%%%%%%%%%%%%%%%%%%%%%%%%%%%%%%%%%%
\problemfile{L20-32}
\begin{problem}
Assume that $\sigma$ is a good function and $\sigma= \sigma_0 + l
u_0$ where $u_0$ is the step function with a step at $t=0$. Can we
conclude that $\sigma_0$ is sub-additive ? \sol{ No, see textbook
for a counter-example. }
\end{problem}
\end{filecontents}
%%%%%%%%%%%%%%%%%%%%%%%%%%%%%%%%%%%%%%%%%%%%%%%%%%%%%%%%%%%%%%%%%%%%%%%%%%%%
\problemfile{L20-33}
\begin{problem}
Is the operator $(P^L)$ upper-semi-continuous ? \sol{No, consider
a sequence $x_n=l_1 + \frac{l_2}{n}$. We have $P^L(x_n)=l_1$, thus
$\inf_n [P^L(x_n)]=l_1$. Now $\inf_n x_n = l_1$ thus $P^L[\inf_n
(x_n)]=0$.}
\end{problem}
\end{filecontents}
%%%%%%%%%%%%%%%%%%%%%%%%%%%%%%%%%%%%%%%%%%%%%%%%%%%%%%%%%%%%%%%%%%%%%%%%%%%%
 \problemfile{L20-34}
 \begin{problem}
\begin{enumerate}
  \item
 Consider the concatenation of an $L$-packetizer and a network
 element with minimum service curve $\beta$ and maximum service
 curve $\gamma$. Can we say that the combined system offer a
 minimum service curve $(\beta(t)-l_{\max})^+$ and a maximum service
 curve $\gamma$, as in the case where the concatenation would be
 in the reverse order ? \sol{ Yes, and the proof is similar}.
  \item Consider the concatenation of a GPS node offering a guarantee
  $\lambda_{r_1}$, an $L$-packetizer, and a second GPS node offering a guarantee
  $\lambda_{r_2}$. Show that the combined system offers a rate-latency
  service curve with rate $R=\min(r_1, r_2)$ and
  latency$E=\frac{l_{\max}}{\max(r_1, r_2)}$.
  \sol{If $r_1< r_2$, consider the combined system made of the packetizer and
  the second node. From the previous question, it offers a rate latency
  service curve with rate $r_2$ and latency
  $E=\frac{l_{\max}}{r_2}=\frac{l_{\max}}{\max(r_1, r_2)}$. Thus
  the global system offers a rate latency service curve with rate
  $R=\min(r_1, r_2)$ and latency $E$.\\
  Otherwise, consider first as combined system the concatenation
  of the first GPS node and the packetizer, and apply the textbook
  result with the same reasoning. Thus the result is true in all cases.}
\end{enumerate}
 \end{problem}
 \end{filecontents}
%%%%%%%%%%%%%%%%%%%%%%%%%%%%%%%%%%%%%%%%%%%%%%%%%%%%%%%%%%%%%%%%%%%%%%%%%%%%
 \problemfile{L20-35}
 \begin{problem}
 Consider a node that offers to a flow $R(t)$ a rate-latency
 service curve $\beta= S_{R,L}$. Assume that $R(t)$ is $L$-packetized, with
 packet arrival times called $T_1, T_2,...$  (and
 is left-continuous,
 as usual)

 Show that
 $(R\mpc \beta)(t) = \min_{T_i \in [0, t]} [R(T_i) + \beta(t-T_i)]$
  (and thus, the $\inf$ is attained).
\sol{ For a fixed $t$, study the function of $s$ given by $R(s) +
\beta(t-s)$. For $T_{i-1} < s \leq T_i$ we have $R(s)=R(T_i)$ thus
the $\inf$ for $s \in (T_{i-1}, T_i]$ is attained at $T_i$.}
 \end{problem}
 \end{filecontents}
%%%%%%%%%%%%%%%%%%%%%%%%%%%%%%%%%%%%%%%%%%%%%%%%%%%%%%%%%%%%%%%%%%%%%%%%%%%%
% \problemfile{}
% \begin{problem}
% \end{problem}
% \end{filecontents}

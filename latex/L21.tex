\chapter[Application to the Internet]{Application of Network Calculus
 to the Internet}
\mylabel{L21}

In this chapter we apply the concepts of \cref{L20} and explain
the theoretical underpinnings of integrated and differentiated
services. Integrated services define how reservations can be made
for flows. We explain in detail how this framework was deeply
influenced by GPS. In particular, we will see that it assumes that
every router can be modeled as a node offering a minimum service
curve that is a rate-latency function. We explain how this is used
in a protocol such as RSVP. We also analyze  the more efficient
framework based on service curve scheduling. This allows us to
address in a simple way the complex issue of schedulability.

We explain the concept of Guaranteed Rate node, which corresponds
to a service curve element, but with some differences, because it
uses a max-plus approach instead of min-plus. We analyze the
relation between the two approaches.

Differentiated services differ radically, in that reservations are
made per class of service, rather than per flow. We show how the
bounding results in \cref{L20} can be applied to find delay and
backlog bounds. We also introduce the ``damper", which is a way of
enforcing a maximum service curve, and show how it can radically
reduce the delay bounds.

\nfs{leftover 1:\\
 Fluid SCED: refaire SCED avec le lemme: si sum of beta < ct alors
 on peut scheduler; puis se demander quel est le delai et quelle
 est la sc apres packetisation globale. Devrait generaliser PGPS, voir
 Section 2.4.4 de Chang et reconcilier avec les resultats EDF de
 Liebeherr\\
 }
\nfs{leftover 2:\\ Reprendre le papier de PG a la lueur de
netcal\\
 }
\nfs{leftover mineur: explique best effort scheduling de SCED+\\}
\nfs{ papier Guerin ITIT 1997 sur buffer allocation policies\\}
\section{GPS and Guaranteed Rate Nodes}
\mylabel{sec-gps}
In this section we describe GPS and its
derivatives; they form the basis on which the Internet guaranteed
model was defined.%

\subsection{Packet Scheduling}
A guaranteed service network offers delay and throughput
guarantees to flows, provided that the flows satisfy some arrival
curve constraints (\sref{sec-L20-fcs}). This requires that network
nodes implement some form of packet scheduling, also called
service discipline. Packet scheduling is defined as the function
that decides, at every buffer inside a network node, the service
order for different packets.

A simple form of packet scheduling is FIFO: packets are served in
the order of arrival. The delay bound, and the required buffer,
depend on the minimum arrival curve of the aggregate flow
(Section~\ref{sec-effbw} on page~\pageref{sec-effbw}). If one flow
sends a large amount of traffic, then the delay increases for all
flows, and packet loss may occur. Thus FIFO scheduling requires
that arrival curve constraints on all flows be strictly enforced
at all points in the network. Also, with FIFO scheduling, the
delay bound is the same for all flows. We study FIFO scheduling in
more detail in \sref{L24}.

An alternative \cite{dks90,keshav-96} is to use per flow queuing,
in order to (1) provide isolation to flows and (2) offer different
guarantees. We consider first the ideal form of per flow queuing
called ``Generalized Processor Sharing" (GPS) \cite{pg93}, which
was already mentioned in Chapter~\ref{L20}.
\subsection{GPS and a Practical Implementation (PGPS)}
\mylabel{sec-gpsdet}

A GPS node serves several flows in parallel, and has a total
output rate equal to $c$ b/s. A flow $i$ is allocated a given
weight, say $\phi_i$. Call $R_i(t), R^*_i(t)$ the input and output
functions for flow $i$. The guarantee is that at any time $t$, the
service rate offered to flow $i$ is 0 is flow $i$ has no backlog
(namely, if $R_i(t)=R_i^*(t)$), and otherwise is equal to
$\frac{\phi_i}{\sum_{j \in B(t)} \phi_{j}}c$, where $B(t)$ is the
set of backlogged flows at time $t$. Thus
 $$R^*_i(t) = \int_0^t \frac{\phi_i}{\sum_{j \in B(s)} \phi_{j}} 1_{\{i \in B(s)\}}ds
 $$
 In the formula, we used the indicator function
 $1_{\{\mbox{\emph{expr}}\}}$, which is equal to $1$ if $\mbox{\emph{expr}}$ is
 true, and $0$ otherwise.

It follows immediately that the GPS node offers to flow $i$ a
service curve equal to $\lambda_{r_i c}$, with $r_i=\frac{\phi_i
C}{\sum_{j} \phi_{j}}$. It is shown in \cite{pg94} that a better
service curve can be obtained for every flow if we know some
arrival curve properties for all flows; however the simple
property is sufficient to understand the integrated service model.

GPS satisfies the requirement of isolating flows and providing
differentiated guarantees. We can compute the delay bound and
buffer requirements for every flow if we know its arrival curve,
using the results of Chapter~\ref{L20}. However, a GPS node is a
theoretical concept, which is not really implementable, because it
relies on a fluid model, and assumes that packets are infinitely
divisible. How can we make a practical implementation of GPS ? One
simple solution would be to use the virtual finish times as we did
for the buffered leaky bucket controller in \sref{sec-psps}: for
every packet we would compute its finish time $\theta$ under GPS,
then at time $\theta$ present the packet to a multiplexer that
serves packets at a rate $c$. \fref{fig-pgps} (left) shows the
finish times on an example. It also illustrates the main drawback
that this method would have: at times 3 and 5, the multiplexer
would be idle, whereas at time 6 it would have a burst of 5
packets to serve. In particular, such a scheduler would not be
work conserving.

This is what motivated researchers to find other practical
implementations of GPS. We study here one such implementation of
GPS, called
packet by packet generalized processor sharing (PGPS)~\cite{pg93}. %
\index{PGPS: packet generalized processor sharing}%
Other implementations of GPS are discussed in
Section~\ref{sec-gr}.

PGPS emulates GPS as follows. There is one FIFO queue per flow.
The scheduler handles packets one at a time, until it is fully
transmitted, at the system rate $c$. For every packet, we compute
the finish time that it would have under GPS (we call this the
``GPS-finish-time"). Then, whenever a packet is finished
transmitting, the next packet selected for transmission is the one
with the earliest GPS-finish-time, among all packets present.
Figure~\ref{fig-pgps} shows one example. We see that, unlike the
simple solution discussed earlier, PGPS is work conserving, but
does so at the expense of maybe scheduling a packet \emph{before}
its finish time under GPS.
\begin{figure}[!htbp]
  \insfig{pgps}{0.7}
  \mycaption{Scheduling with GPS (left) and PGPS (right).
  Flow 0 has weight 0.5, flows 1 to 5 have weight 0.1. All packets
  have the same transmission time equal to 1 time unit.}
  \mylabel{fig-pgps}
\end{figure}

We can quantify the difference between PGPS and GPS in the
following proposition. In \sref{sec-gr}, we will see how to derive
a service curve property.
\begin{proposition}[\cite{pg93}]
The finish time for PGPS is at most the finish time of GPS plus
$\frac{L}{c}$, where $c$ is the total rate and $L$ is the maximum
packet size.
 \mylabel{prop-pgps}
\end{proposition}
\pr Call $D(n)$ the finish time of the $n$th packet for the
aggregate
 input flow under PGPS, in the order of departure, and $\theta(n)$ under GPS.
 Call $n_0$ the
 number of the packet that started the busy period in which packet
 $n$ departs. Note that PGPS and GPS have the same busy periods,
 since if we observe only the aggregate flows, there is no
 difference between PGPS and GPS.

 There may be some packets that
 depart before packet $n$ in PGPS, but that nonetheless have a
 later departure time under GPS.
 Call $m_0 \geq n_0$ the largest packet
 number for which this occurs, if any; otherwise let $m_0=n_0-1$.
 In this proposition, we call $l(m)$ the length in bits of packet $m$.
 Under PGPS, packet $m_0$ started service at
 $D(m_0)- \frac{l(m_0)}{c}$, which must be earlier than the
 arrival times of packets $m= m_0+1, ..., n$.
 Indeed, otherwise, by definition of PGPS, the
 PGPS scheduler would have scheduled packets $m= m_0+1, ..., n$
 before packet $m_0$. Now let us observe the GPS system. Packets $m= m_0+1, ...,
 n$ depart no later than packet $n$, by definition of $m_0$; they
 have arrived after $D(m_0)- \frac{l(m_0)}{c}$. By expressing the amount of
 service in the interval $[D(m_0)- \frac{l(m_0)}{c}, \theta(n)]$
 we find thus
 $$ \sum_{m=m_0 +1}^n l(m) \leq c \left(\theta(n) - D(m_0)+ \frac{l(m_0)}{c}\right)
 $$
Now since packets $m_0, ..., n$ are in the same busy period, we
have
 $$D(n) = D(m_0) + \frac{\sum_{m=m_0 +1}^n l(m)}{c}$$
By combining the two equations above we find $D(n) \leq \theta(n)
+ \frac{l(m_0)}{c}$, which shows the proposition in the case where
$m_0 \geq n_0$.

 If $m_0 = n_0-1$, then all packets $n_0,...,n$ depart before packet $n$
 under GPS and thus the same reasoning shows that
 $$ \sum_{m=n_0}^n l(m) \leq c \left(\theta(n) - t_0\right) $$
 where $t_0$ is the beginning of the busy period, and that
 $$D(n) = t_0 + \frac{\sum_{m=n_0 }^n l(m)}{c}
 $$
 Thus $D(n) \leq \theta(n)$ in that case. \qed

%\subsection{GPS and a Practical Implementation (PGPS)}
%\mylabel{sec-gpsdet}
%
%A GPS node serves several flows in parallel, and has a total
%output rate equal to $c$ b/s. A flow $i$ is allocated a given
%weight, say $\phi_i$. Call $R_i(t), R^*_i(t)$ the input and output
%functions for flow $i$. The guarantee is that at any time $t$, the
%service rate offered to flow $i$ is 0 is flow $i$ has no backlog
%(namely, if $R_i(t)=R_i^*(t)$), and otherwise is equal to
%$\frac{\phi_i}{\sum_{j \in B(t)} \phi_{j}}c$, where $B(t)$ is the
%set of backlogged flows at time $t$. Thus
% $$R^*_i(t) = \int_0^t \frac{\phi_i}{\sum_{j \in B(s)} \phi_{j}} 1_{\{i \in B(s)\}}ds
% $$
% In the formula, we used the indicator function
% $1_{\{\mbox{\emph{expr}}\}}$, which is equal to $1$ if $\mbox{\emph{expr}}$ is
% true, and $0$ otherwise.
%
%It follows immediately that the GPS node offers to flow $i$ a
%service curve equal to $\lambda_{r_i c}$, with $r_i=\frac{\phi_i
%C}{\sum_{j} \phi_{j}}$. It is shown in \cite{pg94} that a better
%service curve can be obtained for every flow if we know some
%arrival curve properties for all flows; however the simple
%property is sufficient to understand the integrated service model.
%
%GPS satisfies the requirement of isolating flows and providing
%differentiated guarantees. We can compute the delay bound and
%buffer requirements for every flow if we know its arrival curve,
%using the results of Chapter~\ref{L20}. However, a GPS node is a
%theoretical concept, which is not really implementable, because it
%relies on a fluid model, and assumes that packets are infinitely
%divisible. How can we make a practical implementation of GPS ? One
%simple solution would be to use the virtual finish times as we did
%for the buffered leaky bucket controller in \sref{sec-psps}: for
%every packet we would compute its finish time $\theta$ under GPS,
%then at time $\theta$ present the packet to a multiplexer that
%serves packets at a rate $c$. \fref{fig-pgps} (left) shows the
%finish times on an example. It also illustrates the main drawback
%that this method would have: at times 3 and 5, the multiplexer
%would be idle, whereas at time 6 it would have a burst of 5
%packets to serve. In particular, such a scheduler would not be
%work conserving.
%
%This is what motivated researchers to find other practical
%implementations of GPS. We study here one such implementation of
%GPS, called
%packet by packet generalized processor sharing (PGPS)~\cite{pg93}. %
%\index{PGPS: packet generalized processor sharing}%
%Other implementations of GPS are discussed in
%Section~\ref{sec-gr}.
%
%PGPS emulates GPS as follows. There is one FIFO queue per flow.
%The scheduler handles packets one at a time, until it is fully
%transmitted, at the system rate $c$. For every packet, we compute
%the finish time that it would have under GPS (we call this the
%``GPS-finish-time"). Then, whenever a packet is finished
%transmitting, the next packet selected for transmission is the one
%with the earliest GPS-finish-time, among all packets present.
%Figure~\ref{fig-pgps} shows one example. We see that, unlike the
%simple solution discussed earlier, PGPS is work conserving, but
%does so at the expense of maybe scheduling a packet \emph{before}
%its finish time under GPS.
%\begin{figure}[!htbp]
%  \insfig{pgps}{0.7}
%  \mycaption{Scheduling with GPS (left) and PGPS (right).
%  Flow 0 has weight 0.5, flows 1 to 5 have weight 0.1. All packets
%  have the same transmission time equal to 1 time unit.}
%  \mylabel{fig-pgps}
%\end{figure}
%
%We can quantify the difference between PGPS and GPS in the
%following proposition. In \sref{sec-gr}, we will see how to derive
%a service curve property.
%\begin{proposition}[\cite{pg93}]
%The finish time for PGPS is at most the finish time of GPS plus
%$\frac{L}{c}$, where $c$ is the total rate and $L$ is the maximum
%packet size.
% \mylabel{prop-pgps}
%\end{proposition}
%\pr
%Call $D(n)$ the finish time of the $n$th packet for the aggregate
% input flow under PGPS, in the order of departure, and $\theta(n)$ under GPS.
% Call $n_0$ the
% number of the packet that started the busy period in which packet
% $n$ departs. Note that PGPS and GPS have the same busy periods,
% since if we observe only the aggregate flows, there is no
% difference between PGPS and GPS.
%
% There may be some packets that
% depart before packet $n$ in PGPS, but that nonetheless have a
% later departure time under GPS.
% Call $m_0 \geq n_0$ the largest packet
% number for which this occurs, if any; otherwise let $m_0=n_0-1$.
% In this proposition, we call $l(m)$ the length in bits of packet $m$.
% Under PGPS, packet $m_0$ started service at
% $D(m_0)- \frac{l(m_0)}{c}$, which must be earlier than the
% arrival times of packets $m= m_0+1, ..., n$.
% Indeed, otherwise, by definition of PGPS, the
% PGPS scheduler would have scheduled packets $m= m_0+1, ..., n$
% before packet $m_0$. Now let us observe the GPS system. Packets $m= m_0+1, ...,
% n$ depart no later than packet $n$, by definition of $m_0$; they
% have arrived after $D(m_0)- \frac{l(m_0)}{c}$. By expressing the amount of
% service in the interval $[D(m_0)- \frac{l(m_0)}{c}, \theta(n)]$
% we find thus
% $$ \sum_{m=m_0 +1}^n l(m) \leq c \left(\theta(n) - D(m_0)+ \frac{l(m_0)}{c}\right)
% $$
%Now since packets $m_0, ..., n$ are in the same busy period, we
%have
% $$D(n) = D(m_0) + \frac{\sum_{m=m_0 +1}^n l(m)}{c}$$
%By combining the two equations above we find $D(n) \leq \theta(n)
%+ \frac{l(m_0)}{c}$, which shows the proposition in the case where
%$m_0 \geq n_0$.
%
% If $m_0 = n_0-1$, then all packets $n_0,...,n$ depart before packet $n$
% under GPS and thus the same reasoning shows that
% $$ \sum_{m=n_0}^n l(m) \leq c \left(\theta(n) - t_0\right) $$
% where $t_0$ is the beginning of the busy period, and that
% $$D(n) = t_0 + \frac{\sum_{m=n_0 }^n l(m)}{c}
% $$
% Thus $D(n) \leq \theta(n)$ in that case. \qed

\subsection{Guaranteed Rate (GR) Nodes and the Max-Plus Approach}
\mylabel{sec-gr}

The service curve concept defined earlier can be approached from
the dual point of view, which consists in studying the packet
arrival and departure times instead of the functions $R(t)$ (which
count the bits arrived up to time $t$). This latter approach leads
to max-plus algebra (which has the same properties as min-plus),
is often more appropriate to account for details due to variable
packet sizes, but works well only when the service curves are of
the rate-latency type. It also useful when nodes cannot be assumed
to be FIFO per flow, as may be the case with DiffServ
(Section~\ref{sec-diffserv}).

GR also allows to show that many schedulers have the rate-latency
service curve property. Indeed, a large number of practical
implementations of GPS, other than PGSP, have been proposed in the
literature; let us mention: virtual clock scheduling \cite{zha90},
packet by packet generalized processor sharing \cite{pg93} and
self-clocked fair queuing \cite{gol94}(see also \cite{gue96}). For
a thorough discussion of practical implementations of GPS, see
\cite{zha96, gue96}). These implementations differ in their
implementation complexity and in the bounds that can be obtained.
It is shown in \cite{GLV95} that all of these implementations fit
in the following framework, called ``Guaranteed Rate", which we
define now. We will also analyze how it relates to the min-plus
approach.

\begin{definition}[GR Node\cite{GLV95}]
Consider a node that serves a flow. Packets are numbered in order
of arrival. Let $a_n \geq 0, d_n \geq 0$ be the arrival and
departure times. We say that a node is the a \emph{guaranteed
rate} (GR) node for this flow, with rate $r$ and delay $e$, if it
guarantees that $d_n \leq f_n + e$, where $f_n$ is defined by
\eref{eq-gr-def}.
\begin{equation}\mylabel{eq-gr-def}
\bracket{ f_0=0\\
    f_n= \max \left\{
                   a_n, f_{n-1}
                 \right\}
             + \frac{l_n}{r} \gap \mfa n \geq 1
            }
\end{equation}
\mylabel{def-gr}
\end{definition}
\index{Guaranteed Rate node}\index{GR}%
The variables $f_n$ (``Guaranteed Rate Clocks") can be interpreted
as the departures times from a FIFO constant rate server, with
rate $r$. The parameter $e$ expresses how much the node deviates
from it. Note however that \emph{a GR node need not be FIFO}. A GR
node is also called ``Rate-Latency server".

\textbf{Example: GPS.} Consider an ideal GPS scheduler, which
allocates a rate $R_i=\frac{c \phi_i}{\sum_j \phi_j}$ to some flow
$i$. It is a GR node for flow $i$, with rate $R_i$ and latency
$=0$ (the proof is left to the reader)

\begin{definition}[One Way Deviation of a scheduler from GPS]
We say that $S$ deviates from GPS by $e$ if for all packet $n$
the departure time satisfies
\begin{equation}\mylabel{eq-gr-eq6}
d_n\leq g_n + e
\end{equation}
where $g_n$ is the departure time from a hypothetical GPS node
that allocates a rate $r=\frac{c \phi_1}{\sum_j \phi_j}$ to
this flow (assumed to be flow 1).
\end{definition}

We interpret this definition as a comparison to a hypothetical
GPS reference scheduler that would serve the same flows.
\begin{theorem}
\label{theo-grgps} If a scheduler satisfies \eref{eq-gr-eq6},
then it is GR with rate $r$ and latency $e$.
\end{theorem}
\pr
$g_n \leq f_n$ and the rest is immediate.

\qed

\textbf{Example: PGPS.} Consider a PGPS scheduler, which allocates
a rate $R_i=\frac{c \phi_i}{\sum_j \phi_j}$ to some flow $i$. It
is a GR node for flow $i$, with rate $R_i$ and latency
$\frac{L}{c}$, where $L$ is the maximum packet size (among all
flows present at the scheduler) (this follows from
\pref{prop-pgps}).


\begin{theorem}[Max-Plus Representation of GR]
\mylabel{theo-grmaxplus} \mylabel{theo-GR} Consider a system where
packets are numbered $1, 2, ...$ in order of arrival. Call $a_n$,
$d_n$ the arrival and departure times for packet $n$, and $l_n$
the size of packet $n$. Define by convention $d_0=0$. The system
is a GR node with rate $r$ and latency $e$ if and only if for all
$n$ there is some $k \in \{1, ..., n\} $ such that
\begin{equation}\mylabel{eq-gr2091}\mylabel{eq-grmaxplus}
 d_n \leq e + a_k + \frac{l_k + ... + l_n}{r}
\end{equation}
\end{theorem}
\paragraph{Proof: } The recursion \eref{eq-gr-def} can be solved
iteratively, using the same max-plus method as in the proof of
\pref{theo-gcra-step}. Define
$$
 A^n_j=a_j +  \frac{l_j + ... + l_n}{r} \mfor 1 \leq j\leq n\\
$$
Then we obtain
 $$
 f_n  = \max(A_n^n , A_{n-1}^n , ... ,
    A_1^n)$$
The rest follows immediately. \qed

\eref{eq-gr2091} is the dual of the service curve definition
(\eref{eq-sercur-d2a} on \pgref{eq-gr2091}), with
$\beta(t)=r(t-e)^+$. We now elucidate this relationship.

\begin{theorem}[Equivalence with service curve]
Consider a node with $L$-packetized input.
\begin{enumerate}
  \item If the node guarantees a minimum service curve equal to
  the rate-latency function
$\beta=\beta_{r,v}$, and if it is FIFO, then it is a GR node with
rate $r$ and latency $v$.
  \item Conversely, a GR node with rate $r$ and latency $e$
is the concatenation of a service curve element, with service
curve equal to the rate-latency function $\beta_{r,v}$, and an
$L$-packetizer. If the GR node is FIFO, then so is the service
curve element.
\end{enumerate}
 \mylabel{theo-grcrep}
\end{theorem}
The proof is long and is given at the end of this section.

By applying \thref{theo-delvlp}, we obtain
\begin{corollary}
A GR node offers a minimum service curve $\beta_{r, v+
\frac{l_{\max}}{r}}$ \mylabel{cor-gr-sercur}
\end{corollary}
The service curve can be used to obtain backlog bounds.
\begin{theorem}[Delay Bound]
\mylabel{theo-grcdelaybound}
For an $\alpha$-smooth flow served in
a (possibly non FIFO) GR node with rate $r$ and latency $e$, the
delay for any packet is bounded by
\begin{equation}\mylabel{eq-psrgnf1}\mylabel{eq-mpdel}
  \sup_{t > 0} [\frac{\alpha(t)}{r} -  t] + e
\end{equation}
\end{theorem}
\paragraph{Proof: } By \thref{theo-grmaxplus}, for any fixed $n$, we can find a
$1\leq j\leq n$ such that
$$
f_n=a_j + \frac{l_j+...+l_n}{r} $$ The delay for packet $n$ is
 $$
 d_n - a_n \leq f_n + e -a_n
 $$
Define $t=a_n - a_j$. By hypothesis
$$l_j+...+l_n \leq \alpha_r(t)$$
where $\alpha_r(t)$ is the limit from the right of $\alpha$ at $t$.
Thus
 $$d_n -a_n \leq -t + \frac{\alpha_r(t)}{r} + e \leq \sup_{t \geq 0} [\frac{\alpha_r(t)}{r} -  t] + e
 $$
Now $\sup_{t > 0}[\frac{\alpha(t)}{r} -  t]=\sup_{t \geq
0}[\frac{\alpha_r(t)}{r} -  t]$.
 \qed
\paragraph{Comment: }
Note that \eref{eq-mpdel} is the horizontal deviation between the
arrival curve $\alpha$ and the rate-latency service curve with
rate $r$ and latency $e$. Thus, for FIFO GR nodes,
\thref{theo-grcdelaybound} follows from \thref{theo-GR} and the
fact that the packetizer can be ignored for delay computations.
The information in \thref{theo-grcdelaybound} is that it also
holds for non-FIFO nodes.

\subsection{Concatenation of GR nodes}

\paragraph{FIFO Nodes} For GR nodes that are FIFO per flow, the concatenation
result obtained with the service curve approach applies.

\begin{theorem} Specifically, the concatenation of $M$ GR nodes
(that are FIFO per flow) with rates $r_m$ and latencies $e_m$
is GR with rate $r= \min_m r_m$ and latency $e=
\sum_{i=1,...,n} e_i +
\sum_{i=1,...,n-1}\frac{L_{\max}}{r_i}$, where $L_{\max}$ is
the maximum packet size for the flow. \mylabel{theo-grconc}
\end{theorem}If $r_m=r$ for all $m$ then the extra term is $(M-1)\frac{L_{\max}}{r}$; it is due to packetizers.
\pr
By \thref{theo-grcrep}--(2), we can decompose system $i$ into
a concatenation $\calS_i, \calP_i$, where $\calS_i$ offers the
service curve $\beta_{r_i,e_i}$ and $\calP_i$ is a packetizer.

Call $\calS$ the concatenation
 $$\calS_1, \calP_1, \calS_2,
\calP_2,...,\calS_{n-1},\calP_{n-1},\calS_n$$ By
\thref{theo-grcrep}--(2), $\calS$ is FIFO and provides the
service curve $\beta_{r,e}$. By \thref{theo-grcrep}--(1), it
is GR with rate $r$ and latency $e$. Now $\calP_n$ does not
affect the finish time of the last bit of every packet.

\qed

Note that a slight change if the proof of the theorem shows
that the theorem is also valid if we replace $e=
\sum_{i=1,...,n} e_i + \sum_{i=1,...,n-1}\frac{L_{\max}}{r_i}$
by $e= \sum_{i=1,...,n} e_i +
\sum_{i=2,...,n}\frac{L_{\max}}{r_i}$.

\textbf{End-to-end Delay Bound.}

A bound on the end-to-end delay through a concatenation of GR
nodes is thus
\begin{equation}\mylabel{eq-e2e-dlgr}
 D=\sum_{m=1}^M v_m + l_{\max}\sum_{m=1}^{M-1} \frac{1}{r_m}+ \frac{\sigma}{\min_m r_m}
\end{equation}
which is the formula in \cite{GLV95}. It is a generalization of
\eref{eq-e2edelvlp} on \pgref{eq-e2edelvlp}.

\textbf{A Composite Node.}We analyze in detail one specific
example, which often arises in practice when modelling a
router. We consider a composite node, made of two components.
The former (``variable delay component") imposes to packets a
delay in the range $[\delta_{\max}-\delta, \delta_{\max}]$.
The latter is FIFO and offers to its input the packet scale
rate guarantee, with rate $r$ and latency $e$. We show that,
if the variable delay component is known to be FIFO, then we
have a simple result. We first give the following lemma, which
has some interest of its own.

\begin{lemma}[Variable Delay as GR]
\mylabel{lem-vdgrc} Consider a node which is known to
guarantee a delay $\leq \delta_{\max}$. The node need not be
FIFO. Call $l_{\min}$ the minimum packet size. For any $r>0$,
the node is GR with latency
$e=[\delta_{\max}-\frac{l_{\min}}{r}]^+$ and rate $r$.
\end{lemma}

\pr With the standard notation in this section, the hypothesis
implies that $d_n \leq a_n + \delta_{\max}$ for all $n \geq
1$. Define $f_n$  by \eref{eq-gr-def}. We have $f_n\geq a_n +
\frac{l_n}{r} \geq a_n + \frac{l_{\min}}{r}$, thus $d_n-f_n
\leq \delta_{\max} - \frac{l_{\min}}{r}\leq
[\delta_{\max}-\frac{l_{\min}}{r}]^+$.

\qed

\begin{theorem}(\emph{Composite GR Node with FIFO Variable Delay
Component}) \mylabel{theo-fgvd} Consider the concatenation of
two nodes. The former imposes to packets a delay  $\leq
\delta_{\max}$. The latter is a GR node with rate $r$ and
latency $e$. Both nodes are FIFO. The concatenation of the two
nodes, in any order, is GR with rate $r$ and latency $e''=e
+\delta_{\max}$.
\end{theorem}
\pr
The former node is
GR($r',e'=[\delta_{\max}-\frac{l_{\min}}{r'}]^+$) for any $r'
>r$. By \thref{theo-grconc} (and the note after it), the concatenation is
GR($r,e+e'+\frac{l_{\max}}{r'}$). Let $r'$ go to $\infty$.

\qed

\paragraph{GR nodes that are not FIFO per flow} The concatenation
result is no longer true. We study in detail the composite
node.

\begin{theorem}
\mylabel{theo-nfgrvd} Consider the concatenation of two nodes.
The first imposes to packets a delay in the range
$[\delta_{\max}-\delta, \delta_{\max}]$. The second is FIFO
and offers the guaranteed rate clock service to its input,
with rate $r$ and latency $e$. The first node is not assumed
to be FIFO, so the order of packet arrivals at the second node
is not the order of packet arrivals at the first one. Assume
that the fresh input is constrained by a continuous arrival
curve $\alpha(\cdot)$. The concatenation of the two nodes, in
this order,  offers to the fresh input the guaranteed rate
clock service with rate $r$ and latency $$e''=e+ \delta_{\max}
+
 \frac{\alpha( \delta) - l_{\min}}{r}
$$
\end{theorem}
The proof is given in the next section.

\textbf{Application: } For $\alpha(t)=\rho t + \sigma$, we
find
$$
 e''=e+ \delta_{\max}+\frac{\rho \delta +
\sigma -
   l_{\min}}{r}
$$


\subsection{Proofs}

\textbf{Proof of \thref{theo-grcrep}}

\textbf{Part 1: } Consider a service curve element $\calS$. Assume
to simplify the demonstration that the input and output functions
$R$ and $R^*$ are right-continuous. Consider the virtual system
$\calS^0$ made of a bit-by-bit greedy shaper with shaping curve
$\lambda_r$, followed by a constant bit-by-bit delay element. The
bit-by-bit greedy shaper is a constant bit rate server, with rate
$r$. Thus the last bit of packet $n$ departs from it exactly at
time $f_n$, defined by \eref{eq-gr-def}, thus the last bit of
packet $n$ leaves $\calS^0$ at $d^0_n=f_n + e$. The output
function of $\calS^0$ is $R^0=R \mpc \beta_{r,e}$. By hypothesis,
$R^* \geq R^0$, and by the FIFO assumption, this shows that the
delay in $\calS$ is upper bounded by the delay in $\calS'$. Thus
$d_n \leq f_n +e$.


\textbf{Part 2: } Consider the virtual system $\calS$ whose output
$S(t)$ is defined by
\begin{equation}\mylabel{eq-pr-defS}
  \begin{array}{c}
   \mif d_{i-1} < t \leq d_i\\
    \; \; \mthen S(t)=\min\{R(t), \max[L(i-1), L(i) -r(d_i-t)]\}
  \end{array}
\end{equation}
See \fref{fig-gr-sched} for an illustration. It follows
immediately that $R'(t)=P^L(S(t))$.

Also consider the virtual system $\calS^0$ whose output is
 $$S^0(t) = (\beta_{r,v}\mpc R)(t)$$
$\calS^0$ is the constant
 rate server, delayed by $v$. Our goal is now to show that
 $S \geq S^0$.

Call $d^0_i$ the departure time of the last bit of packet $i$ in
$\calS_0$ (see \fref{fig-gr-sched} for an example with $i=2$). Let
$u=d^0_i - d_i$. The definition of GR node means that $u \geq 0$.
Now since $\calS_0$ is a shifted constant rate server, we have:
 $$\mif d^0_i-\frac{l_i}{r} <s < d^0_i
 \mthen
 S^0(s) = L(i) - r(d^0_i-s)
 $$
Also $d^0_{i-1} \leq d^0_i-\frac{l_i}{r}$ thus
$S^0(d^0_i-\frac{l_i}{r}) = L(i-1)$ and
 $$\mif s \leq  d^0_i -\frac{l_i}{r}
 \mthen S^0(s) \leq L(i-1)
 $$
It follows that
\begin{equation}\mylabel{eq-gr-prv2}
\mif d_{i-1}+u <s < d^0_i
 \mthen
 S^0(s) \leq \max[L(i-1), L(i) -r(d^0_i-s)]
\end{equation}
Consider now some $t \in (d_{i-1},d_i]$ and let $s=t+u$. If
$S(t)=R(t)$, since $R \geq  S^0$, we then obviously have $S(t)
\geq S^0(t)$. Else, from \eref{eq-gr-def}, $S(t)=\max[L(i-1), L(i)
-r(d_i-t)]$. We have $d^0_i-s=d_i-t$ and thus, combining with
\eref{eq-gr-prv2}, we derive that $S^0(s) \leq S(t)$. Now $s \geq
t$, thus finally $S^0(t) \leq S(t)$. One can also readily see that
$\calS$ is FIFO if $d_{i-1}\leq d_i$ for all $i$. \qed
\begin{figure}[!htbp]
\insfig{grsercur2}{0.9}
    \mycaption{Arrival and departure functions for GR node. The virtual system
    output is $S(t)$.}
    \mylabel{fig-gr-sched}
\end{figure}


\textbf{Proof of \thref{theo-nfgrvd}. }

We use the same notation and convention as in the proof of
\thref{theo-nfvd}. We can also assume that all packet arrivals
are distinct, using the same type of reduction.

Fix some $n \geq 1$; due to \thref{theo-grmaxplus}, it is
sufficient to show that  there is some $k \in \{1, ..., n\}$
 such that
\begin{equation}\mylabel{eq-gr2091a}
 d_n \leq e_2 + a_k + \frac{l_k + ... + l_n}{r}
\end{equation}

By hypothesis, there exists some $j$ such that $b_j \leq b_n$
and
\begin{equation}\mylabel{eq-kjsd82}
  d_n \leq b_j+ e + \frac{B[b_j, b_n]}{r}
\end{equation}
We cannot assume that $j \leq n$; thus, define $k$ as the
oldest packet arrived in the interval $[b_j, b_n]$, in other
words: $k=\inf\{i \geq 1: b_j \leq b_i \leq b_n \}$.
Necessarily, we have now $k \leq n$.

Any packet that arrives at the second node in $[b_j, b_n]$
must have arrived at node $1$ after or with packet $k$, and
before $b_n$. Thus $B[b_j, b_n] \leq A[a_k, b_n]$. Now $b_n
\leq a_n + \delta$. Thus by \lref{lem-arrcur} in this
appendix:
$$
  \begin{array}{l}
   B[b_j, b_n] \leq A[a_k, a_n]+ A(a_n, b_n] \\
    \leq A[a_k, a_n] + \alpha(\delta) - l_{\min}
  \end{array}
$$
Also, $b_j \leq b_k \leq a_k + \delta$ and by
\eref{eq-kjsd82}:
$$
d_n \leq a_k + e + \delta + \alpha(\delta) + A[a_k, a_n] -
l_{\min}
$$
which shows \eref{eq-gr2091a}.




\section{The Integrated Services Model of the IETF}
\mylabel{sec-L20-fcs}

\subsection{The Guaranteed Service}

The Internet supports different reservation principles. Two
services are defined: the ``guaranteed"\index{guaranteed service}
service, and the `` controlled load"\index{controlled load
service} service. They differ in that the former provides real
guarantees, while the latter provides only approximate guarantees.
We outline the differences in the rest of this section. In both
cases, the principle is based on ``admission control", which
operates as follows.
\begin{itemize}
  \item In order to receive the guaranteed or
controlled load service, a flow must first perform a reservation
during a flow setup phase.
  \item A flow must confirm to an arrival curve of the form
  $\alpha(t) = \min(M+pt, rt+b)$, which is called the T-SPEC (see
  Section~\ref{sec-lbarcur} on page\pageref{eq-tspec}). The T-SPEC
  is declared during the reservation phase.
  \item All routers along the path accept or reject the
  reservation. With the guaranteed service, routers accept
  the reservation only if they are able to provide a service curve
  guarantee and enough buffer for loss-free operation. The service
  curve is expressed during the reservation phase, as explained
  below.

  For the
  controlled load service, there is no strict definition of what
  accepting a reservation means. Most likely, it means that the
  router has an estimation module that says that, with good
  probability, the reservation can be accepted and little loss
  will occur; there is no service curve or delay guarantee.
\end{itemize}

In the rest of this chapter we focus on the guaranteed service.
Provision of the controlled load service relies on models with
loss, which are discussed in Chapter~\ref{L30}.

\subsection{The Integrated Services Model for Internet Routers}

The reservation phase assumes that all routers can export their
characteristics using a very simple model. The model is based on
the view that an integrated services router implements a practical
approximation of GPS, such as PGPS, or more generally, a GR node.
We have shown in \sref{sec-gr} that the service curve offered to a
flow by a router implementing GR is a rate-latency function, with
rate $R$ and latency $T$ connected by the relationship
\begin{equation}\mylabel{eq-rtltc}
  T = \frac{C}{R} + D
\end{equation}
with $C=$ the maximum packet size for the flow and
$D=\frac{L}{c}$, where $L$ is the maximum packet size in the
router across all flows, and $c$ the total rate of the scheduler.
This is the model defined for an Internet node \cite{SV96}.

\begin{fact}
The Integrated Services model for a router is that the service
curve offered to a flow is always a rate-latency function, with
parameters related by a relation of the form~(\ref{eq-rtltc}).
\end{fact}

The values of $C$ and $D$ depend on the specific implementation of
a router, see \coref{cor-gr-sercur} in the case of GR nodes. Note
that a router does not necessarily implement a scheduling method
that approximates GPS. In fact, we discuss in
Section~\ref{sec-sced} a family of schedulers that has many
advantages above GPS. If a router implements a method that largely
differs from GPS, then we must find a service curve that
lower-bounds the best service curve guarantee offered by the
router. In some cases, this may mean loosing important information
about the router. For example, it is \emph{not} possible to
implement a network offering constant delay to flows by means of a
system like SCED+, discussed in Section~\ref{sec-sced+}, with the
Integrated Services router model.
% a VBR scheduler is another example
\subsection{Reservation Setup with RSVP}

Consider a flow defined by TSPEC $(M, p, r, b)$, that traverses
nodes $1, \ldots, N$. Usually, nodes 1 and $N$ are end-systems
while nodes $n$ for $1<n<N$ are routers. The Integrated Services
model assumes that node $n$ on the path of the flow offers a rate
latency service curve $\beta_{R_n,T_n}$, and further assumes that
$T_{n}$ has the form
 $$T_{n}=\frac{C_{n}}{R} +D_{n}$$
where $C_{n}$ and $D_{n}$ are constants that depend on the
characteristics of node $n$.

The reservation is actually put in place by means of a flow setup
procedure such as the resource reservation protocol (RSVP)%
\index{RSVP}. At the end of the procedure, node $n$ on the path
has allocated to the flow a value $ R_{n}\geq r$. This is
equivalent to allocating a service curve $\beta_{R_n,T_n}$. From
Theorem~\ref{theo-netSerCur} on page~\pageref{theo-netSerCur}, the
end-to-end service curve offered to the flow is the rate-latency
function with rate $R$ and latency $T$ given by
$$
\bracket{
 R = \min_{n=1...N}R_{n} \\
 T = \sum_{n=1}^N \left(\frac{C_n}{R_n} + D_n\right)
 }
$$
 Let
$C_{\mbox{tot}}=\sum_{n=1}^{N} C_n$ and
$D_{\mbox{tot}}=\sum_{n=1}^{N} D_n$. We can re-write the last
equation as
\begin{equation}\mylabel{eq-rtlcnete1}
 T = \frac{C_{\mbox{tot}}}{R} + D_{\mbox{tot}} -
 \sum_{n=1}^N S_n
\end{equation}
with
\begin{equation}\mylabel{eq-rtlcnete2}
 S_n =C_n \left(\frac{1}{R}- \frac{1}{R_n}\right)
\end{equation}
The term $S_n$ is called the ``local slack" term at node $n$.


From Proposition~\ref{theo-vbr-ratlat} we deduce immediately:
\begin{proposition}
    If $R \geq r$, the bound on the end-to-end delay,
    under the conditions described above is
    \begin{equation}
        \frac{b-M}{R} \left(
                  \frac{p-R}{p-r}
                  \right)^{+}
    + \frac{M + C_{tot}}{R}  +  D_{tot}  - \sum_{n=1}^N S_n
        \mylabel{eq-tot-del-1}
    \end{equation}
\end{proposition}

We can now describe the reservation setup with RSVP. Some details
of flow setup with RSVP are illustrated on
Figure~\ref{fig-rsvp-1}. It shows that two RSVP flows are
involved: an advertisement (\texttt{PATH}) flow and a reservation
(\texttt{RESV}) flow. We describe first the point-to-point case.
\begin{itemize}
   \item A \texttt{PATH} message is sent by the source; it
   contains
    the T-SPEC of the flow (source T-SPEC), which is not modified in
    transit, and another field, the ADSPEC, which is accumulated along
    the path.  At a destination, the ADSPEC field contains, among
    others,  the values of
   $C_{\mbox{tot}}, D_{\mbox{tot}}$ used in Equation \ref{eq-tot-del-1}.
    \texttt{PATH} messages do not cause any reservation to be made.
    \item  \texttt{RESV} messages are sent by the destination and cause
    the actual reservations to be made. They follow the reverse path
    marked by
    PATH messages. The \texttt{RESV} message contains a value, $R'$, (as part of the
    so-called R-SPEC), which is a lower bound on the rate parameters $R_n$ that routers
    along the path will have to reserve. The value of $R'$ is
    determined by the destination based on the end-to-end delay objective
    $d_{\mbox{obj}}$, following the procedure described below. It
    is normally not changed by the intermediate nodes.
\end{itemize}

\begin{figure}[!htbp]
\insfig{D2RSVP1}{0.8}
    \mycaption{Setup of Reservations, showing the PATH and RESV flows}
    \mylabel{fig-rsvp-1}
\end{figure}

Define function $f$ by
    $$
    f(R') :=
    \frac{b-M}{R'} \left(
                  \frac{p-R'}{p-r}
                  \right)^{+}
    + \frac{M + C_{tot}}{R'}  +  D_{tot}
    $$
In other words, $f$ is the function that defines the end-to-end
delay bound, assuming all nodes along the path would reserve $R_n=
R'$. The destination computes $R'$ as the smallest value $\geq r$
for which $f(R') \leq d_{\mbox{obj}}$. Such a value exists only if
$D_{tot} < d_{\mbox{obj}}$.

In the figure, the destination requires a delay variation
objective of 600 ms, which imposes a minimum value of $R'=$622
kb/s.  The value of $R'$ is sent to the next upstream node in the
R-SPEC field of the \texttt{PATH} message.  The intermediate nodes
do not know the complete values $C_{\mbox{tot}}$ and
$D_{\mbox{tot}}$, nor do they know the total delay variation
objective. Consider the simple case where all intermediate nodes
are true PGPS schedulers. Node $n$ simply checks whether it is
able to reserve $R_n=R'$ to the flow; this involves verifying that
the sum of reserved rates is less than the scheduler total rate,
and that there is enough buffer available (see below). If so, it
passes the RESV message upstream, up to the destination if all
intermediate nodes accept the reservation. If the reservation is
rejected, then the node discards it and normally informs the
source. In this simple case, all nodes should set their rate to
$R_n=R'$ thus $R=R'$, and Equation~(\ref{eq-tot-del-1}) guarantees
that the end-to-end delay bound is guaranteed.

In practice, there is a small additional element (use of the slack
term), due to the fact that the designers of RSVP also wanted to
support other schedulers. It works as follows.

\begin{figure}[!htbp]
\insfig{D2RSVP2}{0.8}
    \mycaption{Use of the slack term}
    \mylabel{fig-rsvp-2}
\end{figure}

There is another term in the R-SPEC, called the \emph{slack} term.
Its use is illustrated on Figure \ref{fig-rsvp-2}. In the figure,
we see that the end-to-end delay variation requirement, set by the
destination, is 1000 ms. In that case, the destination reserves
the minimum rate, namely, 512~kb/s. Even so, the delay variation
objective $D_{obj}$ is larger than the bound $D_{max}$ given by
Formula (\ref{eq-tot-del-1}). The difference $  D_{obj} - D_{max}$
is written in the slack term $S$ and passed to the upstream node
in the RESV message. The upstream node is not able to compute
Formula (\ref{eq-tot-del-1}) because it does not have the value of
the end-to-end parameters. However, it can use the slack term to
increase its internal delay objective, on top of what it had
advertised. For example, a guaranteed rate node may increase its
value of $v$ (Theorem \ref{theo-GR}) and thus reduce the internal
resources required to perform the reservation. The figure shows
that R1 reduces the slack term by 100 ms. This is equivalent to
increasing the $D_{tot}$ parameter by $100ms$, but without
modifying the advertised $D_{tot}$.

The delays considered here are the total (fixed plus variable)
delays. RSVP also contains a field used for advertising the fixed
delay part, which can be used to compute the end-to-end fixed
delay. The variable part of the delay (called delay jitter) is
then obtained by subtraction.

\subsection{A Flow Setup Algorithm}

There are many different ways for nodes to decide which parameter
they should allocate. We present here one possible algorithm. A
destination computes the worst case delay variation, obtained if
all nodes reserve the sustainable rate $r$. If the resulting delay
variation is acceptable, then the destination sets $R=r$ and the
resulting slack may be used by intermediate nodes to add a local
delay on top of their advertised delay variation defined by $C$
and $D$. Otherwise, the destination sets $R$ to the minimum value
$R_{min}$ that supports the end-to-end delay variation objective
and sets the slack to $0$. As a result, all nodes along the path
have to reserve $R_{min}$. As in the previous cases, nodes may
allocate a rate larger than the value of $R$ they pass upstream,
as a means to reduce their buffer requirement.

\begin{definition}[A Flow Setup Algorithm]

\begin{itemize}
    \item At a destination system $I$, compute
    $$
    D_{max}= f_{T}(r) +\frac{C_{tot}}{r}+ D_{tot}
    $$

    If $D_{obj}> D_{max}$ then assign to the flow a rate $R_{I}= r$
    and an additional delay variation $d_{I} \leq D_{obj}-D_{max}$; set
    $S_{I}= D_{obj}-D_{max} -d_{I}$ and
    send reservation request $R_{I}, S_{I}$ to
    station $I-1$.

    Else ($D_{obj}\leq D_{max}$) find the minimum $R_{min}$ such that
    $f_{T}(R_{min}) +\frac{C_{tot}}{R_{min}} \leq D_{obj} - D_{tot}$,
    if it exists.  Send reservation request $R_{I}=R_{min}, S_{I}=0$
    to station $I-1$.  If $R_{min}$ does not exist, reject the
    reservation or increase the delay variation objective $D_{obj}$.

    \item At an intermediate system $i$: receive from $i+1$ a
    reservation request $R_{i+1}, S_{i+1}$.

    If $S_{i}=0$, then perform reservation for rate $R_{i+1}$ and if
    successful, send reservation request $R_{i}=R_{i+1}, S_{i}=0$
    to station $i-1$.

    Else ($S_{i}>0$), perform a reservation for rate $R_{i+1}$ with
    some additional delay variation $d_{i}\leq S_{i+1}$.  if
    successful, send reservation request $R_{i}=R_{i+1},
    S_{i}=S_{i+1}-d_{i}$ to station $i-1$.
\end{itemize}
\mylabel{def-flowsetup}
\end{definition}

The algorithm ensures a constant reservation rate. It is easy to
check that the end to end delay variation is bounded by $D_{obj}$.

\subsection{Multicast Flows}

Consider now a multicast situation. A source $S$ sends to a number
of destinations, along a multicast tree. \texttt{PATH} messages
are forwarded along the tree, they are duplicated at splitting
points; at the same points, \texttt{RESV} messages are merged.
Consider such a point, call it node $i$, and assume it receives
reservation requests for the same T-SPEC but with respective
parameters $R'_{in}, S'_{in}$ and $R''_{in}, S''_{in}$. The node
performs reservations internally, using the algorithm in Definition~\ref{def-flowsetup}. Then it has to merge the reservation requests it will send to
node $i-1$. Merging uses the following rules:

\paragraph{R-SPEC Merging Rules} The merged reservation $R,S$ is
given by
$$R = \max(R', R'')$$
$$S= \min(S', S'')$$

Let us consider now a tree where the algorithm in Definition~\ref{def-flowsetup} is applied. We want
to show that the end-to-end delay bounds at all destinations are
respected.

The rate along the path from a destination to a source cannot
decrease with this algorithm. Thus the minimum rate along the tree
towards the destination is the rate set at the destination, which
proves the result.

A few more features of RSVP are:
\begin{itemize}
    \item  states in nodes need to be refreshed; if they are not
    refreshed, the reservation is released (``soft states'').

    \item  routing is not coordinated with the reservation of the flow

\end{itemize}
We have so far looked only at the delay constraints. Buffer
requirements can be computed using the values in
Proposition~\ref{theo-vbr-ratlat}.
%\pointex{ex-bufreqd-1} \pointex{ex-bufreqd-2}
\subsection{Flow Setup with ATM}
With ATM, there are the following differences:
\begin{itemize}
    \item  The path is determined at the flow setup time only.
    Different connections may follow different routes depending on their
    requirements, and once setup, a connection always uses the same path.

    \item  With standard ATM signaling, connection setup is initiated at
    the source and is confirmed by the destination and all intermediate
    systems.
\end{itemize}

\section{Schedulability}
\mylabel{sec-sced}

So far, we have considered one flow in isolation and assumed that
a node is able to offer some scheduling, or service curve
guarantee. In this section we address the global problem of
resource allocation.

When a node performs a reservation, it is necessary to check
whether local resources are sufficient. In general, the method for
this consists in breaking the node down into a network of building
blocks such as schedulers, shapers, and delay elements. There are
mainly two resources to account for: bit rate (called
``bandwidth") and buffer. The main difficulty is the allocation of
bit rate. Following \cite{scp95}, we will see in this section that
allocating a rate amounts to allocating a service curve. It is
also equivalent to the concept of schedulability.

Consider the simple case of a PGPS scheduler, with outgoing rate
$C$. If we want to allocate rate $r_i$ to flow $i$, for every $i$,
then we can allocate to flow $i$ the GPS weight
$\phi_i=\frac{r_i}{C}$. Assume that
\begin{equation}
 \sum_i r_i \leq C
 \mylabel{eq-sched-pgps}
\end{equation}
Then we know from \pref{prop-pgps} and \coref{cor-gr-sercur} that
every flow $i$ is guaranteed the rate-latency service curve with
rate $r_i$ and latency $\frac{L}{C}$. In other words, the
schedulability condition for PGPS is simply \eref{eq-sched-pgps}.
However, we will see now that a schedulability conditions are not
always as simple. Note also that the end-to-end delay depends not
only on the service curve allocated to the flow, but also on its
arrival curve constraints.

Many schedulers have been proposed, and some of them do not fit in
the GR framework. The most general framework in the context of
guaranteed service is given by SCED (Service Curve Earliest
Deadline first)~\cite{scp95},%
\index{SCED}%
which we describe now. We give the theory for constant size
packets and slotted time; some aspects of the general theory for
variable length packets are known \cite{Changbook}, some others
remain to be done. We assume without loss of generality that every
packet is of size 1 data unit.

\subsection{EDF Schedulers}%
\index{Earliest Deadline First (EDF) scheduler}%
\index{EDF see Earliest Deadline First}%
%
As the name indicates, SCED is based on the concept of Earliest
Deadline First (EDF) scheduler. An EDF scheduler assigns a
deadline $D^n_i$ to the $n$th packet of flow $i$, according to
some method. We assume that deadlines are wide-sense increasing
within a flow. At every time slot, the scheduler picks at  one of
the packets with the smallest deadline among all packets present.
There is a wide variety of methods for computing deadlines. The
``delay based" schedulers~\cite{LWF96} %
\index{Delay Based Scheduler}%
set $D^n_i= A^n + d_i$ where $A^n$ is the arrival time for the
$n$th packet for flow $i$, and $d_i$ is the delay budget allocated
to flow $i$. If $d_i$ is independent of $i$, then we have a FIFO
scheduler. We will see that those are special cases of SCED, which
we view as a very general method for computing deadlines.

An EDF scheduler is work conserving, that is, it cannot be idle if
there is at least one packet present in the system. A consequence
of this is that packets from different flows are not necessarily
served in the order of their deadlines. Consider for example a
delay based scheduler, and assume that flow $1$ has a lrage delay
budget $d_1$, while flow $2$ has a small delay budget $d_2$. It
may be that a packet of flow $1$ arriving at $t_1$ is served
before a packet of flow $2$ arriving at $t_2$, even though the
deadline of packet $1$, $t_1 + d_1$ is larger than the deadline of
packet $2$.

We will now derive a general schedulability criterion for EDF
schedulers. Call $R_i(t)$, $t \in \Nats$, the arrival function for
flow $i$. Call $Z_i(t)$ the number of packets of flow $i$ that
have deadlines $\leq t$. For example, for a delay based scheduler,
$Z_i(t)=R_i(t-d_i)$. The following is a modified version of
\cite{Changbook}.

\begin{proposition}
 \mylabel{prop-edf21}
 Consider an EDF scheduler with $I$ flows and outgoing rate $C$.
 A necessary condition for all packets to be served within their
 deadlines is
\begin{equation}\mylabel{eq-edf-cn}
 \mfa s \leq t : \;   \sum_{i=1}^{I} Z_i(t) - R_i(s) \leq C(t-s)
\end{equation}
 A sufficient condition is
 \begin{equation}\mylabel{eq-edf-cs}
 \mfa s \leq t  : \;  \sum_{i=1}^{I} [Z_i(t) - R_i(s)]^+ \leq C(t-s)
\end{equation}
\end{proposition}
\pr
We first prove the necessary condition. Call $R'_i$ the output for
flow $i$. Since the scheduler is work conserving, we have
$\sum_{i=1}^{I} R'_i = \lambda_C \mpc(\sum_{i=1}^{I} R_i)$. Now
$R'_i \geq Z_i$ by hypothesis. Thus
$$
\sum_{i=1}^{I} Z_i(t) \leq \inf_{s \in [0,t]} C(t-s) +
\sum_{i=1}^{I} R_i(s)
$$
which is equivalent to \eref{eq-edf-cn}

Now we prove the sufficient condition, by contradiction. Assume
that at some $t$ a packet with deadline $t$ is not yet served. In
time slot $t$, the packet served has a deadline $\leq t$,
otherwise our packet would have been chosen instead. Define $s_0$
such that the time interval $[s_0+1, t]$ is the maximum time
interval ending at $t$ that is within a busy period and for which
all packets served have deadlines $\leq t$.

Now call $\calS$ the set of flows that have a packet with deadline
$\leq t$ present in the system at some point in the interval
$[s_0+1, t]$. We show that if
\begin{equation}\mylabel{eq-proof-edf-2}
  \mif i \in \calS
  \mthen
  R'_i(s_0)=R_i(s_0)
\end{equation}
that is, flow $i$ is not backlogged at the end of time slot $s_0$.
Indeed, if $s_0+1$ is the beginning of the busy period, then the
property is true for any flow. Otherwise, we proceed by
contradiction. Assume that $i\in \calS$ and that $i$ would have
some backlog at the end of time slot $s_0$. At time $s_0$ some
packet with deadline $>t$ was served; thus the deadline of all
packets remaining in the queue at the end of time slot $s_0$ must
have a deadline $>t$. Since deadlines are assumed wide-sense
increasing within a flow, all deadlines of flow $i$ packets that
are in the queue at time $s_0$, or will arrive later, have
deadline $>t$, which contradicts that $i \in \calS$.

Further, it follows from the last argument that if $i \in \calS$,
then all packets served before or at $t$ must have a deadline
$\leq t$. Thus
 $$
 \mif i \in \calS \mthen R'_i(t) \leq Z_i(t)
 $$

Now since there is at least one packet with deadline $\leq t$ not
served at $t$, the previous inequality is strict for at least one
$i$ in $\calS$. Thus
\begin{equation}\mylabel{eq-proof-edf-1}
  \sum_{i \in \calS} R'_i(t) < \sum_{i \in \calS} Z_i(t)
\end{equation}
Observe that all packets served in $[s_0+1, t]$ must be from flows
in $\calS$. Thus
 $$
 \sum_{i=1}^I (R'_i(t)-R'_i(s_0)) = \sum_{i \in \calS}(R'_i(t)-R'_i(s_0))
 $$
Combining with \eref{eq-proof-edf-2} and \eref{eq-proof-edf-1}
gives
 $$
\sum_{i=1}^I (R'_i(t)-R'_i(s_0)) < \sum_{i \in
\calS}(Z_i(t)-R_i(s_0))
$$
Now $[s_0+1, t]$ is entirely in a busy period thus $\sum_{i=1}^I
(R'_i(t)-R'_i(s_0))=C(t-s_0)$; thus
 $$
C(t-s_0) < \sum_{i \in \calS}(Z_i(t)-R_i(s_0)) = \sum_{i \in
\calS}(Z_i(t)-R_i(s_0))^+ \leq \sum_{i=1}^{I}(Z_i(t)-R_i(s_0))^+
 $$
 which contradicts \eref{eq-edf-cs}.
 \qed

A consequence of the proposition that if a set of flows is
schedulable for some deadline allocation algorithm, then it is
also schedulable for any other deadline allocation method that
produces later or equal deadlines. Other consequences, of
immediate practical importance, are drawn in the next section.

\subsection{SCED Schedulers \cite{saro96}}
Given, for all $i$, a function $\beta_i$, SCED defines a deadline
allocation algorithm that guarantees, under some conditions, that
flow $i$ does have $\beta_i$ as a minimum service
curve\footnote{We use the original work in \cite{saro96}, which is
called there ``SCED-B". For simplicity, we call it SCED.}. Roughly
speaking, SCED sets $Z_i(t)$, the number of packets with deadline
up to $t$, to $(R_i \mpc \beta_i)(t)$.
\begin{definition}[SCED]
Call $A^n_i$ the arrival time for packet $n$ of flow $i$. Define
functions $R^n_i$ by:
$$
 R^n_i(t) = \inf_{s \in [0, A^n_i]} [R_i(s) + \beta_i (t-s)]
$$
With SCED, the deadline for packet $n$ of flow $i$ is defined by
$$
D^n_i = ( R^n_i)^{-1}(n) = \min \{t \in \Nats :  R^n_i(t) \geq n
\}
$$
Function $\beta_i$ is called the ``target service curve" for flow
$i$. \mylabel{def-sced}
\end{definition}
Function $R^n_i$ is similar to the min-plus convolution $R_i \mpc
\beta_i$, but the minimum is computed over all times up to
$A^n_i$. This allows to compute a packet deadline as soon as the
packet arrives; thus SCED can be implemented in real time. The
deadline is obtained by applying the pseudo-inverse of $R^n_i$, as
illustrated on \fref{fig-defsced}. If $\beta_i=\delta_{d_i}$, then
it is easy to see that $D^n_i=A^n_i+d_i$, namely, SCED is the
delay based scheduler in that case.
\begin{figure}[!htbp]
  \insfig{sced1}{0.8}
  \mycaption{Definition of SCED. Packet $n$ of flow $i$ arrives at time $A^n_i$.
  Its deadline is $D^n_i$.}
  \mylabel{fig-defsced}
\end{figure}
The following proposition is the main property of SCED. It shows
that SCED implements a deadline allocation method based on service
curves.
\begin{proposition}
 For the SCED scheduler, the number of packets with deadline
 $\leq t$ is given by $Z_i(t)=\lfloor (R_i \mpc \beta_i) (t)\rfloor$
 \mylabel{prop-sced-10}
\end{proposition}
\pr
We drop index $i$ in this demonstration. First, we show that
$Z(t)\geq \lfloor( R \mpc \beta)(t) \rfloor$. Let $n=\lfloor( R
\mpc \beta)(t)\rfloor$. Since $R \mpc \beta \leq R$ and $R$ takes
integer values, we must have $R(t) \geq n$ and thus $A^n \leq t$.
Now $R^n(t)\geq (R \mpc \beta)(t)$ thus
$$
R^n(t) \geq  (R \mpc \beta)(t) \geq n
$$
By definition of SCED, $D^n$ this implies that $D^n \leq t$ which
is equivalent to $Z(t) \geq n$.

Conversely, for some fixed but arbitrary $t$, let now $n=Z(t)$.
Packet $n$ has a deadline $\leq t$, which implies that $A^n \leq
t$ and for all  $s \in [0,A^n]$ :
\begin{equation}\mylabel{eq-sced-sercur}
R(s) + \beta(t-s) \geq n
\end{equation}
Now for $s \in [A^n, t]$ we have  $R(s) \geq n$ thus $R(s) +
\beta(t-s) \geq n $. Thus \eref{eq-sced-sercur} is true for all $s
\in [0,t]$, which means that $(R \mpc \beta)(t) \geq n$. \qed

\begin{theorem}[Schedulability of SCED, ATM]
Consider a SCED scheduler with $I$ flows, total outgoing rate $C$,
and target service curve $\beta_i$ for flow $i$.
\begin{enumerate}
  \item If
\begin{equation}\mylabel{eq-schedofsced-1}
 \sum_{i=1}^I \beta_i(t) \leq Ct \mfa t \geq 0
\end{equation}
then every packet is served before or at its deadline and every
flow $i$
  receives $\lfloor \beta_i \rfloor$ as a service curve.
  \item Assume that in addition we know that every flow $i$ is constrained
  by an arrival curve $\alpha_i$. If
  \begin{equation}\mylabel{eq-schedofsced-2}
 \sum_{i=1}^I (\alpha_i \mpc \beta_i)(t) \leq Ct \mfa t \geq 0
\end{equation}
  then the same conclusion holds
\end{enumerate}\mylabel{theo-schedofsced}
\end{theorem}
\pr
\begin{enumerate}
  \item \pref{prop-sced-10} implies that $Z_i(t) \leq R_i(s) +
  \beta_i(t-s)$ for $0 \leq s \leq t$. Thus $Z_i(t) - R_i(s) \leq
  \beta_i(t-s)$. Now $0 \leq \beta_i(t-s)$ thus
  $$
  [Z_i(t) - R_i(s)]^+ = \max [ Z_i(t) - R_i(s),0] \leq \beta_i(t-s)
  $$
  By hypothesis, $\sum_{i=1}^I \beta_i(t-s) \leq C(t-s)$ thus by
  application of \pref{prop-edf21}, we know that every packet is
  served before or at its deadline. Thus $R'_i  \geq Z_i$ and from
  \pref{prop-sced-10}:
  $$
  R'_i  \geq Z_i  = \lfloor \beta_i \mpc R_i \rfloor
  $$
 Now $R_i$ takes only integer values thus $\lfloor \beta_i \mpc R_i
 \rfloor= \lfloor \beta_i \rfloor  \mpc R_i$.
  \item By hypothesis, $R_i = \alpha_i \mpc R_i  $ thus
  $Z_i=\lfloor \alpha_i \mpc \beta_i  \mpc R_i \rfloor$ and we can
  apply the same argument, with $ \alpha_i \mpc  \beta_i$ instead of
$\beta_i$. \qed
\end{enumerate}

\paragraph{Schedulability of delay based schedulers}

A delay based scheduler assigns a delay objective $d_i$ to all
packets of flow $i$. A direct application of
\thref{theo-schedofsced} gives the following schedulability
condition.
\begin{theorem}[\cite{LWF96}]
Consider a delay based scheduler that serves $I$ flows, with delay
$d_i$ assigned to flow $i$. All packets have the same size and
time is slotted. Assume flow $i$ is $\alpha_i$-smooth, where
$\alpha_i$ is sub-additive. Call $C$ the total outgoing bit rate.
Any mix of flows satisfying these assumptions is schedulable if
$$
\sum_i \alpha_i(t - d_i) \leq Ct
$$
If $\alpha_i(t) \in \Nats$ then the condition is necessary.
\mylabel{theo-edf-sched}
\end{theorem}
\pr
A delay based scheduler is a special case of SCED, with target
service curve $\beta_i=\delta_{d_i}$. This shows that the
condition in the theorem is sufficient. Conversely, consider the
greedy flows given by $R_i(t)=\alpha_i(t)$. This is possible
because $\alpha_i$ is assumed to be sub-additive. Flow $R_i$ must
be schedulable, thus the output $R'_i$ satisfies $R'_i(t) \geq
\alpha_i(i- d_i)$. Now $\sum_i R'_i(t) \leq ct$, which proves that
the condition must hold. \qed

It is shown in \cite{LWF96} that a delay based scheduler has the
largest schedulability region among all schedulers, given arrival
curves and delay budgets for every flow. Note however that in a
network setting, we are interested in the end-to-end delay bound,
and we know (\sref{sec-concat}) that it is generally less than the
sum of per hop bounds.

The schedulability of delay based schedulers requires that an
arrival curve is known and enforced at every node in the network.
Because arrival curves are modified by network nodes, this
motivates the principle of Rate Controlled Service Disciplines
(RCSDs) \cite{chuck90,zf94,gue96}, which implement in every node a
packet shaper followed by a delay based scheduler. The packet
shaper guarantees that an arrival curve is known for every flow.
Note that such a combination is not work conserving.

Because of the "pay bursts only once" phenomenon, RCSD might
provide end-to-end delay bounds that are worse than guaranteed
rate nodes. However, it is possible to avoid this by aggressively
reshaping flows in every node, which, from \thref{theo-edf-sched},
allows us to set smaller deadlines. If the arrival curves
constraints on all flows are defined by a single leaky bucket,
then it is shown in \cite{boyer,peris97} that one should reshape a
flow to its sustained rate at every node in order to achieve the
same end-to-end delay bounds as GR nodes would.


\paragraph{Schedulability of GR nodes}
Consider the family of GR nodes, applied to the ATM case. We
cannot give a general schedulability condition, since the fact
that a scheduler is of the GR type does not tell us exactly how
the scheduler operates. However, we show that for any rate $r$ and
delay $v$ we can implement a GR node with SCED.
\begin{theorem}[GR node as SCED, ATM case]
 Consider the SCED scheduler with $I$ flows and outgoing rate $C$.
 Let the target service curve for flow $i$ be
 equal to the rate-latency service curve with rate $r_i$ and
 latency $v_i$. If
 $$
 \sum_{i=1}^I r_i \leq C
 $$
 then the scheduler is a GR node for each flow $i$, with rate
$r_i$ and delay $v_i$.
 \mylabel{theo-grissched}
\end{theorem}
\pr
From \pref{prop-sced-10}:
$$Z_i(t)=\lfloor (R_i \mpc\lambda_{r_i})(t-v_i) \rfloor$$
thus $Z_i$ is the output of the constant rate server, with rate
$r_i$, delayed by $v_i$. Now from \thref{theo-schedofsced} the
condition in the theorem guarantees that $R'_i \geq Z_i$, thus the
delay for any packet of flow $i$ is bounded by the delay of the
constant rate server with rate $r_i$, plus $v_i$. \qed

Note the fundamental difference between rate based and delay based
schedulers. For the former, schedulability is a condition on the
sum of the rates; it is independent of the input traffic. In
contrast, for delay based schedulers, schedulability imposes a
condition on the arrival curves. Note however that in order to
obtain a delay bound, we need some arrival curves, even with delay
based schedulers.

%\paragraph{Notes on Network Wide Allocation of service curves}
%\thref{}
%\paragraph{Optimal target service curve}
\paragraph{Better than Delay Based scheduler}
A scheduler need not be either rate based or delay based. Rate
based schedulers suffer from coupling between delay objective and
rate allocation: if we want a low delay, we may be forced to
allocate a large rate, which because of \thref{theo-grissched}
will reduce the number of flows than can be scheduled. Delay based
schedulers avoid this drawback, but they require that flows be
reshaped at every hop. Now, with clever use of SCED, it is
possible to obtain the benefits of delay based schedulers without
paying the price of implementing shapers.

Assume that for every flow $i$ we know an arrival curve $\alpha_i$
and we wish to obtain an end-to-end delay bound $d_i$. Then the
smallest network service curve that should be allocated to the
flow is $\alpha_i \mpc \delta_{d_i}$ (the proof is easy and left
to the reader). Thus a good thing to do is to build a scheduler by
allocating to flow $i$ the target service curve $\alpha_i \mpc
\delta_{d_i}$. The schedulability condition is the same as with a
delay based scheduler, however, there is a significant difference:
the service curve is guaranteed even if some flows are not
conforming to their arrival curves. More precisely, if some flows
do not conform to the arrival curve constraint, then the service
curve is still guaranteed, but the delay bound is not.

This observation can be exploited to allocate service curves in a
more flexible way than what is done in \sref{sec-L20-fcs}
\cite{CruzSCED}. Assume flow $i$ uses the sequence of nodes
$m=1,...,M$. Every node receives a part $d^m_i$ of the delay
budget $d_i$, with $\sum_{m=1}^M d^m_i \leq d_i$. Then it is
sufficient that every node implements SCED with a target service
curve $\beta^m_i= \delta_{d^m_i} \mpc \alpha_i$ for flow $i$. The
schedulability condition at node $m$ is
 $$
 \sum_{j \in E_m} \alpha_j(t-d^m_j) \leq C_m t
 $$
where $E_m$ is the set of flows scheduled at node $m$ and $C_m$ is
the outgoing rate of node $m$. If it is satisfied, then flow $i$
receives $\alpha_i \mpc \delta_{d_i} $ as end-to-end service curve
and therefore has a delay bounded by $d_i$. The schedulability
condition is the same as if we had implemented at node $m$ the
combination of a delay based scheduler with delay budget $d^m_i$,
and a reshaper with shaping curve $\alpha_i$; but we do not have
to implement a reshaper. In particular, the delay bound for flow
$i$ at node $m$ is larger than $d^m_i$; we find again the fact
that the end-to-end delay bound is less than the sum of individual
bounds.

In \cite{saro96}, it is explained how to allocate a service curves
$\beta_i^m$ to every network element $m$ on the path of the flow,
such that $\beta_i^1 \mpc \beta_i^2 \mpc ...= \alpha_i \mpc
\delta_i$, in order to obtain a large schedulability set. This
generalizes and improves the schedulability region of RCSD.

\paragraph{Extension to variable length packets}
We can extend the previous results to variable length packets; we
follow the ideas in \cite{Changbook}. The first step is to
consider a fictitious preemptive EDF scheduler (system I), that
allocates a deadline to every bit. We define $Z_i^{I}(t)$ as
before, as the number of bits whose deadline is $\leq t$. A
preemptive EDF scheduler serves the bits present in the system in
order of their deadlines. It is preemptive (and fictitious) in
that packets are not delivered entirely, but, in contrast, are
likely to be interleaved. The results in the previous sections
apply with no change to this system.

The second step is to modify system I by allocating to every bit a
deadline equal to the deadline of the last bit in the packet. Call
it system II. We have $Z_i^{II}(t)=P^{L_i}(Z_i^{I}(t))$ where
$P^{L_i}$ is the cumulative packet length (\sref{sec-vlp}) for
flow $i$. From the remarks following \pref{prop-edf21}, it follows
that if system I is schedulable, then so is system II. System II
is made of a preemptive EDF scheduler followed by a packetizer.

The third step consists in defining ``packet-EDF" scheduler
(system III); this is derived from system II in the same way as
PGSP is from GPS. More precisely, the packet EDF scheduler picks
the next packet to serve among packets present in the system with
minimum deadline. Then, when a packet is being served, it is not
interrupted. We also say that system III is the non-preemptive EDF
scheduler. Then the departure time of any packet in system III is
bounded by its departure time in system II plus
$\frac{l_{\max}}{C}$ where $l_{\max}$ is the maximum packet size
across all flows and $C$ is the total outgoing rate. The proof is
similar to \pref{prop-pgps} and is left to the reader (it can also
be found in \cite{Changbook}).

We can apply the three steps above to a SCED scheduler with
variable size packets, called ``Packet-SCED".
\begin{definition}[Packet SCED]
A PSCED schedulers is a non-premptive EDF schedulers, where
deadlines are allocated as follows. Call $A^n_i$ the arrival time
for packet $n$ of flow $i$. Define functions $R^n_i$ by:
$$
 R^n_i(t) = \inf_{s \in [0, A^n_i]} [R_i(s) + \beta_i (t-s)]
$$
With PSCED, the deadline for packet $n$ of flow $i$ is defined by
$$
D^n_i = ( R^n_i)^{-1}(L_i(n)) = \min \{t \in \Nats :  R^n_i(t)
\geq (L_i(n)) \}
$$
where $L_i$ is the cumulative packet length for flow $i$. Function
$\beta_i$ is called the ``target service curve" for flow $i$.
\mylabel{def-psced}
\end{definition}

The following proposition follows from the discussion above.
\begin{proposition}\cite{Changbook}
Consider a PSCED scheduler with $I$ flows, total outgoing rate
$C$, and target service curve $\beta_i$ for flow $i$. Call
$l^{i}_{\max}$ the maximum packet size for flow $i$ and let
$l_{\max}=\max_i l^{i}_{\max}$.
\begin{enumerate}
  \item If
\begin{equation}\mylabel{eq-schedofsced-21}
 \sum_{i=1}^I \beta_i(t) \leq Ct \mfa t \geq 0
\end{equation}
then every packet is served before or at its deadline plus
$\frac{l_{\max}}{C}$. A bound on packet delay is $h(\alpha_i,
\beta_i)+\frac{l_{\max}}{C}$. Moreover, every flow $i$
  receives
  $[\beta_i(t-l^{i}_{\max}) - \frac{l_{\max}}{C}]^+$
  as a service curve.
  \item Assume that, in addition, we know that every flow $i$ is constrained
  by an arrival curve $\alpha_i$. If
  \begin{equation}\mylabel{eq-schedofsced-22}
 \sum_{i=1}^I (\alpha_i \mpc \beta_i)(t) \leq Ct \mfa t \geq 0
\end{equation}
  then the same conclusion holds.
\end{enumerate}\mylabel{theo-schedofpsced}
\end{proposition}
Note that the first part of the conclusion means that the maximum
packet delay can be computed by assuming that flow $i$ would
receive $\beta_i$ (not $\beta_i(t-l^{i}_{\max})$) as a service
curve, and adding $\frac{\max}{C}$.
\pr
It follows from the three steps above that the PSCED scheduler can
be broken down into a preemptive EDF scheduler, followed by a
packetizer, followed by a delay element. The rest follows from the
properties of packetizers and \thref{theo-schedofsced}.
%
%\begin{theorem}[Schedulability of EDF\cite{LWF96}]
%Take the same assumptions as in \thref{theo-edf-sched}, but with
%variable length packets, and with continuous time. Assume further
%that flows are numbered in order of increasing delays, so $d_1
%\leq d_2 \leq ... \leq d_F$. Call $l_f$ the maximum packet size
%for flow $f$. Any mix of flows satisfying these assumptions is
%schedulable if and only if for all $t \geq 0$
%\begin{equation}\mylabel{eq-edf-pr}
%  \sum_f \alpha_f(t - d_f) \leq Ct
%\end{equation}
%and for all $t$ such that $d_1 \leq t <d_F$
% \begin{equation}\mylabel{eq-edf-npr}
%\sum_f \alpha_f(t - d_f) \leq Ct - \max_{f: d_f > t  } l_f
% \end{equation}
%\mylabel{theo-edf-sched2}
%\end{theorem}
%The idea of the proof is the same as for the ATM case and is not
%given here (see \cite{LWF96}). Note that if we would allow the EDF
%scheduler to be preemptive, then the condition would be
%\eref{eq-edf-pr} only. In other words, \eref{eq-edf-pr} expresses
%the additional delay due to packets that cannot be preempted.
\subsection{Buffer Requirements}
\mylabel{sec-bufalloc}

As we mentioned at the beginning of this section, buffer
requirements have to be computed in order to accept a reservation.
The condition is simply $\sum_i X_i \leq X$ where $X_i$ is the
buffer required by flow $i$ at this network element, and $X$ is
the total buffer allocated to the class of service. The
computation of $X_i$ is based on \thref{theo-vbr-ratlat}; it
requires computing an arrival curve of every flow as it reaches
the node. This is done using \thref{theo-vbr-ratlat2} and the flow
setup algorithm, such as in \dref{def-flowsetup}.

It is often advantageous to reshape flows at every node. Indeed,
in the absence of reshaping, burstiness is increased linearly in
the number of hops. But we know that reshaping to an initial
constraint does not modify the end-to-end delay bound and does not
increase the buffer requirement at the node where it is
implemented. If reshaping is implemented per flow, then the
burstiness remains the same at every node.
%\section{A network calculus view of scheduling}


%\subsection{A second look at GPS}
%  loop effects
%
%  garanties deduites des limites sur les autres flux
%
%

% section Non-Work Conserving Schedulers
%\mylabel{sec-nws}
%
%
%Combination of policing and FIFO queues
%
%Spacer Controller + FIFO comme alternative.
%Mettre la discussion du papier de Cruz et al dans ToN
%
%Comprendre le lien entre eff bw et la regionde schedulabilite de
%Liebeherr
%  Scheduling
%
%  Output of a packet based shaper
%
%  Scheduling conditions for EDF
%
%  Schedulers faits d'un edf et d'un shaper
%
%  Schedulability conditions in a network with feedback (Peris,
%  Chang)
%
%  RCS thesis a reecrire
%
%
%     strict service curve guarantees
%
%     (1) si source est localement stable alors service curve
%rt (facile)
%
% (2) sinon on a quand meme des bornes par le principe des greedy
%sources -> ce point reste a elucider (definir la service curve
%dans ce cas)
%
%
%
%
%idees de Maher: reprendre WFQ avec W(s,t) pb: definir WFQ comme un
%probleme d'optimisation max-min PG montrent
%
%
%- Guerin et al: shaping by buffer constraints (paper in Sigcomm
%98)
%
%
%
%Definition de service curve universelle et transformee de Maher
\section{Application to Differentiated Services}
\mylabel{sec-ds} \mylabel{sec-diffserv}
\subsection{Differentiated Services}
\mylabel{sec-dsgen}

In addition to the reservation based services we have studied in
\sref{sec-L20-fcs}, the Internet also proposes differentiated
services \cite{RFC2475}. The major goal of differentiated services
is to provide some form of better service while avoiding per flow
state information as is required by integrated services. The idea
to achieve this is based on the following principles.
\begin{itemize}
  \item Traffic classes are defined; inside a network,
  all traffic belonging to
  the same class is treated as one single aggregate flow.
  \item At the network edge, individual flows (called ``micro-flows")
  are assumed to conform to some arrival curve, as with integrated
  services.
\end{itemize}
\begin{figure}[!htbp]
  \insfig{efmod}{0.7}
  \mycaption{Network Model for EF. Microflows
  are individually shaped and each conform to some
   arrival curve. At all nodes, microflows $R_1$ to $R_3$
   are handled as one aggregate flow, with a guaranteed rate (GR) guarantee. Upon leaving a node, the different microflows
   take different paths and become part of other aggregates at other nodes.}
  \mylabel{fig-diffservmod}
\end{figure}
If the aggregate flows receive appropriate service curves in the
network, and if the total traffic on every aggregate flow is not
too large, then we should expect some bounds on delay and loss.
The condition on microflows is key to ensuring that the total
aggregate traffic remains within some arrival curve constraints. A
major difficulty however, as we will see, is to derive bounds for
individual flows from characteristics of an aggregate.

Differentiated services is a framework that includes a number of
different services. The main two services defined today are
expedited forwarding\index{EF}\index{Expedited Forwarding}
(EF)\cite{efnew-april,psrgton} and assured
forwarding\index{AF}\index{Assured Forwarding} (AF)\cite{RFC2597}.
The goal of EF is to provide to an aggregate some hard delay
guarantees, and no loss. The goal of AF is to separate traffic
between a small number of classes (4); inside each class, three
levels of drop priorities are defined. One of the AF classes could
be used to provide a low delay service with no loss, similar to
EF.

In this chapter, we focus on the fundamental issue of how
aggregate scheduling impacts delay and throughput guarantees. In
the rest of this section, we use the network model shown on
\fref{fig-diffservmod}. Our problem is to find bounds for
end-to-end delay jitter on one hand, for backlog at all nodes on
the other hand, under the assumptions mentioned above. Delay
jitter is is the difference between maximum and minimum delay; its
value determines the size of playout buffers
(\sref{sec-playoutBuf}).



\subsection{An Explicit Delay Bound for EF}
\mylabel{bounds-lossless}

We consider EF, the low delay traffic class, as mentioned in
\sref{sec-dsgen}, and find a closed form expression for the worst
case delay, which is valid in any topology, in a lossless network.
This bound is based on a general time stopping method explained in
detail in \cref{L24}. It was obtained in \cite{qofis2000} and
\cite{jiang01}.

\paragraph{Assumption and Notation}
(See Figure~\ref{fig-diffservmod})
\begin{itemize}
\item Microflow $i$ is constrained
by the arrival curve $\rho_i t + \sigma_i$ at the network access.
Inside the network, EF microflows are \emph{not} shaped.

\item Node $m$ acts as a Guaranteed Rate node for the entire EF aggregate, with rate $r_m$
and latency $e_m$. This is true in particular if the aggregate is
served as one flow in a FIFO service curve element, with a
rate-latency service curve; but it also holds quite generally,
even if nodes are non-FIFO (\sref{sec-gr}). In \cref{L24}, we
explain that the generic node model used in the context of EF is
packet scale rate guarantee, which satisfies this assumption.

Let $e$ be an upper bound on $e_m$ for all $m$.

\item $h$ is a bound on the number of hops used by any flow. This is
typically 10 or less, and is much less than the total number of
nodes in the network.

\item
Utilization factors: Define $\nu_m =\frac{1}{r_m}\sum_{i \ni m}
\rho_i$, where the notation $i\ni m$ means that node $m$ is on the
path of microflow $i$. Let $\nu$ be an upper bound on all $v_m$.


\item Scaled burstiness factors: Define $\tau_m =\frac{1}{r_m}\sum_{i \ni m}
\sigma_i$. Let $\tau$ be an upper bound on all $\tau_m$.

\item $L_{\max}$ is an upper bound on the
size (in bits) of any EF packet.

\end{itemize}
\begin{theorem}[Closed form bound for delay and backlog \cite{qofis2000}]
If $\nu < \frac{1}{h-1}$ then a bound on end-to-end delay
variation for EF is $h D_1$ with
$$
D_1 = \frac{e + \tau}{1-(h-1) \nu}
$$
At node $m$, the buffer required for serving low delay traffic
without loss is bounded by $B_{\mbox{req}}=r_m D_1 + L_{\max}$.
\mylabel{theo-qofisbound}
\end{theorem}
\paragraph{Proof: }
(Part 1:) Assume that a finite bound exists and call $D$ the least
upper bound. The data that feeds node $m$ has undergone a variable
delay in the range $[0,(h-1)D]$, thus an arrival curve for the EF
aggregate at node $m$ is $\nu r_m (t + (h-1) D)  + r_m \tau $. By
application of \eref{eq-mpdel}, the delay seen by any packet is
bounded by $e + \tau + (h-1) D \nu$; thus $D \leq e + \tau + (h-1)
D \nu$. If the utilization factor $\nu$ is less than
$\frac{1}{h-1}$, it follows that $D \leq D_1$.

(Part 2:) We prove that a finite bound exists, using the
time-stopping method. For any time $t>0$, consider the virtual
system made of the original network, where all sources are stopped
at time $t$. This network satisfies the assumptions of part 1,
since there is only a finite number of bits for the entire
lifetime of the network. Call $D^{\prime }(t)$ the worst case
delay across all nodes for the virtual network indexed by $t$.
From the above derivation we see that $D^{\prime }(t)\leq D_{1}$
for all $t$. Letting $t$ tend to $+\infty $ shows that the worst
case delay at any node remains bounded by $D_{1}$.

(Part 3:) By \coref{cor-gr-sercur}, the backlog is bounded by the
vertical deviation between the arrival curve $\nu r_m (t + (h-1)
D) + r_m \tau $ and the service curve $[r_m (t - e_m) -
L_{\max}]^+$, which after some algebra gives $B_{\mbox{req}}$ \qed

The theorem can be slightly improved by avoiding to take maxima
for $\nu_m$; this gives the following result (the proof is left to
the reader):
\begin{corollary}
If $\nu < \frac{1}{h-1}$ then a bound on end-to-end delay
variation for EF is $h D'_1$ with
$$
D'_1 = \min_m \left\{\frac{e_m + \tau_m}{1-(h-1) \nu_m}\right\}
$$
\end{corollary}

\paragraph{Improved Bound
When Peak Rate is Known: }

A slightly improved bound can be obtained if, in addition, we have
some information about the total incoming bit rate at every node.
We add the following assumptions to the previous list.
\begin{itemize}
\item Let $C_{m}$ denote a bound on the peak rate of all incoming
low delay traffic traffic at node $m$. If we have no information
about this peak rate, then $C_{m}=+\infty $. For a router with
large internal speed and buffering only at the output, $C_{m}$ is
the sum of the bit rates of all incoming links (the delay bound is
better for a smaller $C_{m}$).

\item Fan-in: Let $I_m$ be the number of incident links at node $m$.
Let $F$ be an upper bound on $\frac{I_m L_{\max}}{r_m}$. $F$ is
the maximum time to transmit a number of EF packets that
simultaneously appear on multiple inputs.

\item
Redefine $\tau_m:= \max\{ \frac{I_m L{\max}}{r_m},
\frac{1}{r_m}\sum_{i \ni m} \sigma_i \}$. Let $\tau$ be an upper
bound on all $\tau_m$.

\item  Let $u_{m}=\frac{[C_{m}-r_{m}]^+}{C_{m}-\nu_m r_{m}}$.
Note that $0 \leq u_{m} \leq 1$, $u_{m}$ increases with $C_{m}$,
and if $C_{m}=+\infty $, then $u_{m}=1$. Call $u=\max_{m}u_{m}$.
The parameter $u \in[0,1]$ encapsulates how much we gain by
knowing the maximum incoming rates $C_m$ ($u$ is small for small
values of $C_m$).
\end{itemize}

\begin{theorem}[Improved Delay Bound
When Peak Rate is Known \cite{qofis2000,jiang01}] Let
$\nu^*=\min_m\{\frac{C_m}{(h-1)(C_m-r_m)^+ + r_m}\}$. If $\nu <
\nu^*$, a bound on end-to-end delay variation for EF is $h D_2$
with
$$
D_2 = \frac{e +  u \tau + (1-u) F }{1-(h-1) u \nu}
$$
\mylabel{theo-qofisbound2}
\end{theorem}
\paragraph{Proof: }

The proof is similar to the proof of \thref{theo-qofisbound}. Call
$D$ the least bound, assuming it exists.

An arrival curve for the flow of EF packets arriving at node $m$
on some incident link $l$ is $C^l_m t + L_{\max}$, where $C^l_m$
is the peak rate of the link (this follows from item 4 in
\thref{theo-delvlp}). Thus an arrival curve for the incoming flow
of EF packets at node $m$ is $C_m t + I_m L_{\max}$. The incoming
flow is thus constrained by the T-SPEC $(M,p, r,b)$ (see Page
\pageref{eq-tspec}) with $M=I_m L_{\max}$, $p=C_m$, $r=r_m \nu_m$,
$b=r_m \tau_m + (h-1)D r_m \nu_m$. By \pref{theo-vbr-ratlat}, it
follows that
$$
D \leq \frac{I_m L_{\max}  (1-u_m)}{r_m} + (\tau_m + (h-1)D \nu_m)
u_m
$$
The condition $\nu < \nu^*$ implies that $1- (h-1) \nu_m u_m >0$,
thus
$$
D \leq  \frac{e_m + \tau_m u_m + \frac{I_m L_{\max}
(1-u_m)}{r_m}}{1- (h-1) \nu_m u_m  }
$$
The above right-hand-side is an increasing function of $u_m$, due
to $\tau_m \geq \frac{I_m L{\max}}{r_m}$. Thus we have a bound by
replacing $u_m$ by $u$:
$$
D \leq \frac{e_m + \tau_m u  + \frac{I_m L_{\max} (1-u )}{r_m}}{1-
(h-1) \nu_m u}
 \leq D_2
$$
The rest of the proof follows along lines similar to the proof of
\thref{theo-qofisbound}. \qed


It is also possible to derive an improved backlog bound, using
\pref{theo-vbr-ratlat}. As with \thref{theo-qofisbound2}, we also
have the following variant.
\begin{corollary}
If $\nu < \nu^*$, a bound on end-to-end delay variation for EF is
$h D'_2$ with
$$
D'_2 = \min_m\left\{ \frac{e_m + \tau_m u_m + \frac{I_m L_{\max}
(1-u_m)}{r_m}}{1- (h-1) \nu_m u_m  } \right\}
$$
\end{corollary}


\paragraph{Discussion:}

If we have no information about the peak incoming rate $C_l$, then
we set $C_l = + \infty$ and \thref{theo-qofisbound2} gives the
same bound as \thref{theo-qofisbound2}. For finite values of
$C_m$, the delay bound is smaller, as illustrated by
\fref{fig-qofis1}.
%\begin{table}[tbp]
%\mylabel{tab-qofis1} \centering
%\begin{tabular}{|r|c|c|c|c|c|}
%\hline $\nu$ & 0.04 & 0.08 & 0.12 & 0.16 & 0.20 \\ \hline\hline
%$D$($C_m=+\infty$) & 16.88 & 74.29 & $+ \infty$ & $+\infty$ &
%$+\infty $ \\ \hline $D$ ($C_m = 2 r_m$) & 12.75 & 32.67 & 71.50 &
%186.00 & $+ \infty$ \\ \hline
%\end{tabular}
%\caption{The bound $D$ (in ms)\ in \thref{theo-qofisbound}  for
%$h=10$. $\sigma_{i}=100$B for all flows, $\rho_{i}=32$kb/s for all
%flows, $r_m=149.760$Mb/s and $C_m=+\infty $ or $C_m=2r_m$.}
%\end{table}
\begin{figure}[!htbp]
  \insfig{dsqofis}{0.7}
  \mycaption{The bound $D$ (in seconds)\ in \thref{theo-qofisbound} versus the
  utilization factor $\nu$ for
$h=10$, $e=2 \frac{1500B}{r_m}$, $L_{\max}=1000$ b,
$\sigma_{i}=100$B and $\rho_{i}=32$kb/s for all flows,
$r_m=149.760$Mb/s, and $C_m=+\infty$ (thin line) or $C_m=2r_m$
(thick line).}
  \mylabel{fig-qofis1}
\end{figure}

The bound is valid only for small utilization factors; it explodes
at $\nu > \frac{1}{h-1}$, which does not mean that the worst case
delay does grow to infinity \cite{LeBoudec2000mMay}. In some cases
the network may be unbounded; in some other cases (such as the
unidirectional ring, there is always a finite bound for all $\nu
<1$. This issue is discussed in \cref{L24}, where we we find
better bounds, at the expense of more restrictions on the routes
and the rates. Such restrictions do not fit with the
differentiated services framework. Note also that, for
feed-forward networks, we know that there are finite bounds for
$\nu <1$. However we show now that the condition $\nu <
\frac{1}{h-1}$ is the best that can be obtained, in some sense.
%
%\mylabel{sec-networkgrand}
\begin{proposition}\cite{lebinfocom2001, qofis2000}
With the assumptions of \thref{theo-qofisbound}, if $\nu >
\frac{1}{h-1}$, then for any $D' >0$, there is a network in which
the worst case delay is at least $D'$. \mylabel{prop-grand}
\end{proposition}
In other words, the worst case queuing delay can be made
arbitrarily large; thus if we want to go beyond
\thref{theo-qofisbound}, any bound for differentiated services
must depend on the network topology or size, not only on the
utilization factor and the number of hops.

\pr
We build a family of networks, out of which, for any $D'$, we can
exhibit an example where the queuing delay is at least $D'$.

The thinking behind the construction is as follows. All flows are
low priority flows. We create a
  hierarchical network, where at the first level of the
  hierarchy we choose one ``flow" for which its first
  packet happens to encounter just \emph{one} packet of every
  other flow whose route it intersects, while its next packet
  does not encounter any queue at all.  This causes the first
  two packets of the chosen flow to come back-to-back after
  several hops.  We then construct the second level of the
  hierarchy by taking a new flow and making sure that its
  first packet encounters \emph{two} back-to-back packets of each
  flow whose routes it intersects, where the two back-to-back
  packet bursts of all these flows come from the output of a
  sufficient number of networks constructed as described at
  the first level of the hierarchy. Repeating this process
  recursively sufficient number of times,  for any chosen
  delay value $D$ we can create deep enough hierarchy so that
  the queuing delay of the first packet of some flow
  encounters a queuing delay more than $D$ (because it
  encounters a large enough back-to-back burst of packets
  of every other flow constructed in the previous iteration),
  while the second packet does not suffer any queuing delay
  at all.   We now describe in detail how to construct such a
  hierarchical network (which is really a family of networks)
  such that utilization factor of any link does not exceed a
  given factor $\nu$, and no flow traverses more than $h$ hops.

Now let us describe the networks in detail. We consider a family
of networks with a single traffic class and constant rate links,
all with same bit rate $C$. The network is assumed to be made of
infinitely fast switches, with one output buffer per link. Assume
that sources are all leaky bucket constrained, but are served in
an aggregate manner, first in first out. Leaky bucket constraints
are implemented at the network entry; after that point, all flows
are aggregated. Without loss of generality, we also assume that
propagation delays can be set to 0; this is because we focus only
on queuing delays. As a simplification, in this network, we also
assume that all packets have a unit size. We show that for any
fixed, but arbitrary delay budget $D$, we can build a network of
that family where the worst case queueing delay is larger than
$D$, while each flow traverses at most a specified number of hops.

A network in our family is called $\calN(h, \nu, J)$ and has three
parameters: $h$ (maximum hop count for any flow), $\nu$
(utilization factor) and $J$ (recursion depth). We focus on the
cases where $h \geq 3$ and $\frac{1}{h-1}< \nu <1$, which implies
that we can always find some integer $k$ such that
\begin{equation}\mylabel{eqk}
  \nu > \frac{1}{h-1}\frac{kh+1}{kh-1}
\end{equation}
Network $\calN(h, \nu, J)$ is illustrated in
Figures~\ref{fig-grand1} and \ref{fig-grand2}; it is a collection
of identical building blocks, arranged in a tree structure of
depth $J$. Every building block has one internal source of traffic
(called ``transit traffic"), $kh(h-1)$ inputs (called the
``building block inputs"), $kh(h-1)$ data sinks, $h-1$ internal
nodes, and one output. Each of the $h-1$ internal nodes receives
traffic from $kh$ building block inputs plus it receives transit
traffic from the previous internal node, with the exception of the
first one which is fed by the internal source. After traversing
one internal node, traffic from the building block inputs dies in
a data sink. In contrast, transit traffic is fed to the next
internal node, except for the last one which feeds the building
block output (\fref{fig-grand1}).
\begin{figure}[!htbp]
  \insfig{bb1}{0.7}
  \mycaption{The internal node (top) and the building block (bottom)
  used in our network example.}
  \mylabel{fig-grand1}
\end{figure}
\fref{fig-grand2} illustrates that our network has the structure
of a complete tree, with depth $J$. The building blocks are
organized in levels $j=1,...,J$. Each of the inputs of a level $j$
building block ($j\geq 2$) is fed by the output of one level $j-1$
building block. The inputs of level $1$ building blocks are data
sources. The output of one $j-1$ building block feeds exactly one
level $j$ building block input. At level $J$, there is exactly one
building block, thus at level $J-1$ there are $kh(h-1)$ building
blocks, and at level $1$ there are $(kh(h-1))^{J-1}$ building
blocks.
\begin{figure}[!htbp]
  \insfig{grandnet}{0.7}
  \mycaption{The network made of building blocks from \fref{fig-grand1}}
  \mylabel{fig-grand2}
\end{figure}
All data sources have the same rate $r=\frac{\nu C}{kh+1}$ and
burst tolerance $b=1$~packet. In the rest of this section we take
as a time unit the transmission time for one packet, so that
$C=1$. Thus any source may transmit one packet every $\theta =
\frac{kh+1}{\nu}$ time units. Note that a source may refrain from
sending packets, which is actually what causes the large delay
jitter. The utilization factor on every link is $\nu$, and every
flow uses $1$ or $h$ hops.

Now consider the following scenario. Consider some arbitrary level
1 building block. At time $t_0$, assume that a packet fully
arrives at each of the building block inputs of level $1$, and at
time $t_0+1$, let a packet fully arrive from each data source
inside every level $1$ building block (this is the first transit
packet). The first transit packet is delayed by $hk-1$ time units
in the first internal node. Just one time unit before this packet
leaves the first queue, let one packet fully arrive at each input
of the second internal node. Our first transit packet will be
delayed again by $hk-1$ time units. If we repeat the scenario
along all internal nodes inside the building block, we see that
the first transit packet is delayed by $(h-1)(hk-1)$ time units.
Now from \eref{eqk}, $\theta < (h-1)(hk-1)$, so it is possible for
the data source to send a second transit packet at time
$(h-1)(hk-1)$. Let all sources mentioned so far be idle, except
for the emissions already described. The second transit packet
will catch up to the first one, so the output of any level $1$
building block is a burst of two back-to-back packets. We can
choose $t_0$ arbitrarily, so we have a mechanism for generating
bursts of 2 packets.

Now we can iterate the scenario and use the same construction at
level $2$. The level-2 data source sends exactly three packets,
spaced by $\theta$. Since the internal node receives $hk$ bursts
of two packets originating from level 1, a judicious choice of the
level 1 starting time lets the first level 2 transit packet find a
queue of $2hk -1$ packets in the first internal node. With the
same construction as in level 1, we end up with a total queuing
delay of $(h-1)(2hk -1) > 2(h-1)(hk-1)> 2\theta$ for that packet.
Now this delay is more than $2 \theta$, and the first three
level-2 transit packets are delayed by the same set of non-transit
packets; as a result, the second and third level-2 transit packets
will eventually catch up to the first one and the output of a
level 2 block is a burst of three packets. This procedure easily
generalizes to all levels up to $J$. In particular, the first
transit packet at level $J$ has an end-to-end delay of at least
$J\theta$. Since all sources become idle after some time, we can
easily create a last level $J$ transit packet that finds an empty
network and thus a zero queuing delay.

Thus there are two packets in network $\calN(h, \nu, J)$, with one
packet having a delay larger than $J \theta$, and the other packet
has zero delay. This establishes that a bound on queuing delay,
and thus on delay variation in network $\calN(h, \nu, J)$ has to
be at least as large as~$J \theta$. \qed

\subsection{Bounds for Aggregate Scheduling with Dampers}
\mylabel{sec-sced+}

At the expense of some protocol complexity, the previous bounds
can be improved without losing the feature of aggregate
scheduling. It is even possible to avoid bound explosions at all,
using the concepts of \emph{damper}\index{damper}. Consider an EDF
scheduler (for example a SCED scheduler) and assume that every
packet sent on the outgoing link carries a field with the
difference $d$ between its deadline and its actual emission time,
if it is positive, and $0$ otherwise. A damper is a regulator in
the next downstream node that picks for the packet an eligibility
time that lies in the interval $[a+d-\Delta, a+d]$, where $\Delta$
is a constant of the damper, and $a$ is the arrival time of the
packet in the node where the damper resides.
We call $\Delta$ the ``damping tolerance".%
\index{damping tolerance} %
The packet is then withheld until its eligibility time
\cite{jitterEDD91,CruzSCED}, see \fref{fig-dsdamper}. In addition,
we assume that the damper operates in a FIFO manner; this means
that the sequence of eligibility times for consecutive packets is
wide-sense increasing.

Unlike the scheduler, the damper does not exist in isolation. It
is associated with the next scheduler on the path of a packet. Its
effect is to forbid scheduling the packet before the eligibility
time chosen for the packet. Consider \fref{fig-dsdamper}.
Scheduler $m$ works as follows. When it has an opportunity to send
a packet, say at time $t$, it picks a packet with the earliest
deadline, among all packets that are present in node $N$, and
whose eligibility date is $\leq t$. The timing information $d$
shown in the figure is carried in a packet header, either as a
link layer header information, or as an IP hop by hop header
extension. At the end of a path, we assume that there is no damper
at the destination node.

The following proposition is obvious, but important, and is given
without proof.
\begin{proposition}
Consider the combination $\calS$ of a scheduler and its associated
damper. If all packets are served by the scheduler before or at
their deadlines, then $\calS$ provides a bound on delay variation
equal to $\Delta$.
\end{proposition}
\begin{figure}[!htbp]
  \insfig{dsdamper2}{0.8}
  \insfig{dsdamper1}{0.8}
  \mycaption{Dampers in a differentiated services context. The model
  shown here assumes that routers are made of infinitely fast
  switching fabrics and output schedulers. There is one logical
  damper for each upstream scheduler. The damper decides when an
  arriving packet becomes visible in the node.}
  \mylabel{fig-dsdamper}
\end{figure}

It is possible to let $\Delta=0$, in which case the delay is
constant for all packets. A bound on the end-to-end delay
variation is then the delay bound at the last scheduler using the
combination of a scheduler and a damper (this is called ``jitter
EDD" in \cite{jitterEDD91}). In practice, we consider $\Delta >0$
for two reasons. Firstly, it is impractical to assume that we can
write the field $d$ with absolute accuracy. Secondly, having some
slack in the delay variation objective provides better performance
to low priority traffic \cite{CruzSCED}.
%Thirdly, we have
%considered so far that all propagation times are constant, which,
%from the analysis in \sref{sec-fixedDelay}, allows us to ignore
%such delays. In some cases (multiple access links, links with
%forward error correction), this is not exactly true. We know in
%general that this can be handled by adding a variable delay
%component, with appropriate minimum and maximum service curves.
%More simply here, if the link fed by scheduler $m$ introduces a
%delay whose variable part is bounded by $\Delta'_m$, then it can
%easily be seen that this is equivalent to adding equal $\Delta'_m$
%to the damping tolerance of damper $m$.

There is no complicated feasibility condition for a damper, as
there is for schedulers. The operation of a damper is always
possible, as long as there is enough buffer.

\begin{proposition}[Buffer requirement for a damper]
If all packets are served by the scheduler before or at their
deadlines, then the buffer requirement at the associated damper is
bounded by the buffer requirement at the scheduler.
\end{proposition}
\pr
Call $R(t)$ the total input to the scheduler, and $R'(t)$ the
amount of data with deadline $\leq t$. Call $R^*(t)$ the input to
the damper, we have $R^*(t) \leq R(t)$. Packets do not stay in the
damper longer than until their deadline in the scheduler, thus the
output $R_1(t)$ of the damper satisfies $R_1(t) \geq R'(t)$. The
buffer requirement at the scheduler at time $t$ is $R(t)- R'(t)$;
at the damper it is $R^*(t)-R_1(t)\geq R(t)- R'(t)$. \qed

\begin{theorem}[Delay and backlog bounds with dampers]
 Take the same assumptions as in \thref{theo-qofisbound}, we
assume that every scheduler $m$ that is not an exit point is
associated with a damper in the next downstream node, with damping
tolerance $\Delta_m$. Let $\Delta$ be a bound on all $\Delta_m$.

If $\nu \leq 1$, then a bound on the end-to-end delay jitter for
low delay traffic is
$$
D  = e + (h-1) \Delta (1 +  \nu) + \tau \nu
$$
A bound on the queuing delay at any scheduler is
 $$
 D_0 = e +  \nu  [\tau + (h-1) \Delta]
 $$
The buffer required at scheduler $m$, for serving low delay
traffic without loss is bounded by
$$
B_{req}=  r_m D_0
$$
A bound on the buffer required at damper $m$ is the same as the
buffer required at scheduler $m$.
 \mylabel{theo-dampersfords}
\end{theorem}
\pr
The variable part of the delay between the input of a scheduler
and the input of the next one is bounded by $\Delta$. Now let us
examine the last scheduler, say $m$, on the path of a packet. The
delay between a source for a flow $i \ni m$ and scheduler $m$ is a
constant plus a variable part bounded by $(h-1)\Delta$. Thus an
arrival curve for the aggregate low-delay traffic arriving at
scheduler $m$ is
$$
\alpha_2(t)= \nu r_m (t + \tau + (h-1) \Delta)
$$
By applying \thref{theo-delay}, a delay bound at scheduler $m$ is
given by
$$D_2 = E + u \nu  [\tau + (h-1) \Delta]
$$
A bound on end-to-end delay variation is $(h-1) \Delta + D_2$,
which is the required formula.

The derivation of the backlog bound is similar to that in
\thref{theo-qofisbound}. \qed

The benefit of dampers is obvious: there is no explosion to the
bound, it is finite (and small if $\Delta$ is small) for any
utilization factor up to $1$ (see \fref{fig-dsdamp1}).
Furthermore, the bound is dominated by $h \Delta$, across the
whole range of utilization factors up to $1$. A key factor in
obtaining little delay variation is to have a small damping
tolerance $\Delta$.
\begin{figure}[!htbp]
  \insfig{dsdamp}{0.7}
  \mycaption{The bound $D$ (in seconds) in \thref{theo-dampersfords}
  the same parameters as \fref{fig-qofis1}, for a damping
  tolerance $\Delta=5$ ms per damper, and $C_m=+\infty$ (thick line).
   The figure also shows the two curves of \fref{fig-qofis1}, for
   comparison. The bound is very close to $h \Delta=0.05$s, for all
   utilization factors up to $1$.}
   \mylabel{fig-dsdamp1}
\end{figure}

There is a relation between a damper and a maximum service curve.
Consider the combination of a scheduler with minimum service curve
$\beta$ and its associate damper with damping tolerance $\Delta$.
Call $p$ the fixed delay on the link between the two. It follows
immediately that the combination offers the maximum service curve
$\beta \mpc \delta_{p-\Delta }$ and the minimum service curve
$\beta \mpc \delta_{p}$. Thus a damper may be viewed as a way to
implement maximum service curve guarantees. This is explored in
detail in \cite{CruzSCED}.


%\section{Application to MPLS and IPv6}
%Jitter-EDD
%
%Requires IP hop by hop option and flow based routing.
%\section{Annotated Bibliography}
%
%GPS et les cas instables
%\cite{GLV95} gives a bound for end-to-end delay guarantees for
%what they call ``guaranteed rate" schedulers

%\cite{zha96} is an overview of scheduling policies, their
%implementation complexity and other properties. See also the
%scheduling chapter in~\cite{keshav-96}.
%
%\cite{gue96} provides a general overview of all components
%involved in Intserv.

\subsection{Static Earliest Time First (SETF)}


A simpler alternative to the of dampers is proposed by Z.-L. Zhang
et al under the name of Static Earliest Time First
(SETF)~\cite{zhili01}.
\paragraph{Assumptions}
We take the same assumptions as with \thref{theo-qofisbound}, with
the following differences.
\begin{itemize}
  \item At network access, packets are stamped with
their time of arrival. At any node, they are served within the EF
aggregate at one node in order of time stamps. Thus we assume that
nodes offer a GR guarantee to the EF aggregate, as defined by
\eref{eq-gr-def} or \eref{eq-gr2091}, but where packets are
numbered in order of time stamps (i.e. their order at the network
access, not at this node).
\end{itemize}
\begin{theorem}
If the time stamps have infinite precision, for all $\nu <1$, the
end-to-end delay variation for the EF aggregate is bounded by
$$
D = (e + \tau)\frac{1-(1-\nu)^{h}}{\nu (1-\nu)^{h-1}}
$$
\mylabel{theo-setf}
\end{theorem}
\paragraph{Proof: }
The proof is similar to the proof of \thref{theo-qofisbound}. Call
$D_k$ the least bound, assuming it exists, on the end-to-end delay
after $k$ hops, $k \leq h$. Consider a tagged packet, with label
$n$, and call $d_k$ its delay in $k$ hops. Consider the node $m$
that is the $h$th hop for this packet. Apply \eref{eq-grmaxplus}:
there is some label $k\leq n$ such that
\begin{equation}\label{eq-setf-1}
d_n \leq e + a_k + \frac{l_k + ... + l_n}{r}
\end{equation}
where $a_j$ and $d_j$ are the arrival and departure times at node
$m$ of the packet labeled $j$, and $l_j$ its length in bits. Now
packets $k$ to $n$ must have arrived at the network access before
$a_n - d_k$ and after $a_m -D_{Hh-1}$. Thus
$$
l_k + ... + l_n \leq \alpha(a_n - a_m - d_k + D_{h-1})
$$
where $\alpha$ is an arrival curve at network access for the
traffic that will flow through node $m$. We have $\alpha(t)\leq
r_m (\nu t +\tau)$. By \eref{eq-mpdel}, the delay $d_n-a_n$ for
our tagged packet is bounded by
$$
e + \sup_{t \geq 0}\left[ \frac{\alpha(t- d_k + D_{h-1})}{r_m}
-t\right] = e + \tau+ \nu (D_{h-1}- d_k)
$$
thus
$$
d_{k+1}\leq d_k +e + \tau+ \nu (D_{h-1}- d_k)
$$
The above inequation can be solved iteratively for $d_k$ as a
function of $D_{h-1}$; then take $k=h-1$ and assume the tagged
packet is one that achieves the worst case $k$-hop delay, thus
$D_{h-1}=d_{h-1}$ which gives an inequality for $D_{h-1}$; last,
take $k=h$ and obtain
%\begin{equation}\label{eq-dh}
%  d_k \leq ( e + \tau + \nu D_{h-1})\frac{1-(1-\nu)^k}{\nu}
%\end{equation}
%Take $k=h-1$ and assume the tagged packet is one that achieves the
%worst case $k$ hop delay, thus $D_{h-1}=d_{h-1}$; (\ref{eq-dh})
%then gives an inequality for $D_{h-1}$:
%$$
%D_{h-1} \leq (e + \tau)\frac{1-(1-\nu)^{h-1}}{\nu (1-\nu)^{h-1}}
%$$
the end-to-end delay bound as desired. \qed

\paragraph{Comments: }
 The bound is finite for all
values of the utilization factor $\nu<1$, unlike the end-to-end
bound in \thref{theo-qofisbound}. Note that for small values of
$\nu$, the two bounds are equivalent.

We have assumed here infinite precision about the arrival time
stamped in every packet. In practice, the timestamp is written
with some finite precision; in that case, Zhang \cite{zhili01}
finds a bound which lies between \thref{theo-qofisbound} and
\thref{theo-setf} (at the limit, with null precision, the bound is
exactly \thref{theo-setf}).



\section{Bibliographic Notes}
The delay bound for EF in \thref{theo-qofisbound2} was originally
found in \cite{qofis2000}, but neglecting the $L_{\max}$ term; a
formula that accounts for $L_{\max}$ was found in \cite{jiang01}.

Bounds that account for statistical multiplexing can be found in
\cite{VL2002EF}.
\section{Exercises}
%% LaTeX2e file `./temp/L21-11'
%% generated by the `filecontents' environment
%% from source `netCalBookv4' on 2019/02/19.
%%
\begin{problem}
Consider a guaranteed rate scheduler, with rate $R$ and delay $v$,
that receives a packet flow with cumulative packet length $L$. The
(packetized) scheduler output is fed into a constant bit rate
trunk with rate $c > R$ and propagation delay $T$.
\begin{enumerate}
  \item Find a minimum service curve for the complete system.
 \sol{~\\
 The system is modelled by the concatenation
 $(N)(P^L)(S_{c,T})(P^L)$, where $N$ represents the scheduler.
 A minimum service curve for $(N)(P^L)$ is
 $S_{R,v+\frac{l_{\max}}{R}}$ and for $(\lambda_c)(P^L)$ is
 $S_{c,T+\frac{l_{\max}}{c}}$. Thus an end-to-end service curve is
 $S_{R,v+T+\frac{l_{\max}}{R}+\frac{l_{\max}}{c}}$.
 }
  \item Assume the flow of packets is $(r,b)$-constrained, with
  $b>l_{\max}$. Find a bound on the end-to-end delay and delay
  variation.
  \sol{~\\
  For the end-to-end delay is bound, we may omit the last packetizer. Thus
  a bound on the end-to-end delay is
  $$h(\gamma_{r,b}, S_{R,v+T+\frac{l_{\max}}{R}})
   = T+\frac{b+l_{\max}}{R}
  $$
  The minimum delay is $T + \frac{l_{\min}}{c}$ thus a bound on
  delay variation is
  $$
  v+\frac{b+l_{\max}}{R} - \frac{l_{\min}}{c}
  $$
}
\end{enumerate}
\end{problem}

%% LaTeX2e file `./temp/L21-12'
%% generated by the `filecontents' environment
%% from source `netCalBookv4' on 2019/02/19.
%%
\begin{problem}
Assume all nodes in a network are of the GR type with rate $R$ and latency $T$. A
  flow with T-SPEC $\alpha(t)=\min(rt+b, M+pt)$
  has performed a reservation with rate
  $R$ across a sequence of $H$ nodes, with $p\geq R$. Assume no reshaping is done.
  What is the buffer requirement at the $h$th node along the path,
  for $h=1,...H$~?
  \sol{~\\
An arrival curve for the input to the $h$th node is $\alpha_{h-1}$, obtained as the
deconvolution of $\alpha$ and a service curve $\beta_{h-1}$ for the concatenation of $h-1$
nodes. We have: $\beta_{h-1}$ is rate-latency, with rate $R$ and latency $(h-1)T'$, with $T'=T+
l_{\max}/R$. Thus, by basic network calculus:
$$
\alpha_{h-1}(t) = \min \left( b_{h-1} + rt, M_{h-1}+Rt \right)
$$
with
$$
b_{h-1}=b+ (h-1) r T'
$$
and
$$
M_{h-1}= (h-1)R T' + M +  \frac{b-M}{p-r}(p-R)
$$
The buffer requirement at node $h$, $v_h$, is the vertical distance $v(\alpha_{h-1},
\beta_{R,T'})$. As an intermediate step, consider the case $\alpha(t)=\min(b+rt,M+Rt)$ and
compute $v=v(\alpha, \beta_{R,T'})$. Define $\theta=\left[\frac{b-M}{R-r}\right]^+$; a simple
analysis gives
$$
\mif \theta \leq T' \mthen v=b+rT' \melse v=M+RT'
$$
In our case we have
$$
\theta=\frac{b-M}{p-r}-(h-1)T'
$$
whence we derive
$$
\mif \frac{b-M}{p-r} \leq h T' \mthen v_h = b + h r T' \melse v_h= M +  \frac{b-M}{p-r}(p-R) +
h R T'
$$
Thus for both small and large $h$, $v_h$ depends linearly on $h$.}
\end{problem}

%% LaTeX2e file `./temp/L21-13'
%% generated by the `filecontents' environment
%% from source `netCalBookv4' on 2019/02/19.
%%
\begin{problem}
Assume all nodes in a network are made of a GR type with rate $R$ and latency $T$, before which
a re-shaper with
  shaping curve $\sigma=\gamma_{r,b}$ is inserted. A
  flow with T-SPEC $\alpha(t)=\min(rt+b, M+pt)$
  has performed a reservation with rate
  $R$ across a sequence of $H$ such nodes, with $p\geq R$.  What is a buffer requirement at the $h$th node along the path,
  for $h=1,...H$~?
\sol{~\\ An arrival curve for the input to the $h$th GR component is $\gamma_{r,b}$; thus the
buffer required at the GR component (excluding the shaper) is $B_1=v(\gamma_{r,b}, \beta)$,
where $\beta$ is the rate-latency service curve with with rate $R$ and latency
$T'=T+l_{\max}/R$. Thus $B_1=b+rT'$.\\ However, we also need to allow some buffer for the
shaper. It is equal to $v(\alpha_{h-1}, \gamma_{r,b})$, where $\alpha'$ is an arrival curve for
the output of the $(h-1)$th node. We have
$$
\alpha' = \gamma_{r,b} \mpd \beta = \gamma_{r, b +rT'}
$$
Thus a buffer bound at the shaper is $B_2=rT'$. A total buffer bound is $B=b+ 2rT'$. Note that
it is independent of $h$.\\ We might also assume that the node employs buffer sharing between
the shaper and the GR component. If so, a buffer bound is obtained by viewing the complete node
as a concatenation of two service curve components; the bound is then
$$v_h= v(\alpha', \gamma_{r,b} \mpc
\beta )$$ In this particular case however, this does not improve the bound.\\ A more
sophisticated bound can be found if we account for the original peak rate limitation $(p,M)$.}
\end{problem}

%% LaTeX2e file `./temp/L21-14'
%% generated by the `filecontents' environment
%% from source `netCalBookv4' on 2019/02/19.
%%
\begin{problem}
Assume all nodes in a network are made of a shaper followed by a
FIFO multiplexer. Assume that flow $I$ has T-SPEC,
$\alpha_i(t)=\min(r_it+b_i, M+p_it)$, that the shaper at every
node uses the shaping curve $\sigma_i=\gamma_{r_i,b_i}$ for flow
$i$. Find the schedulability conditions for every node.
\end{problem}

%% LaTeX2e file `./temp/L21-15'
%% generated by the `filecontents' environment
%% from source `netCalBookv4' on 2019/02/19.
%%
\begin{problem}
A network consists of two nodes in tandem. There are $n_1$ flows
of type $1$ and $n_2$ flows of type $2$. Flows of type $i$ have
arrival curve $\alpha_i(t)=r_it+b_i$, $i=1,2$. All flows go
through nodes $1$ then $2$. Every node is made of a shaper
followed by an EDF scheduler. At both nodes, the shaping curve for
flows of type $i$ is some $\sigma_i$ and the delay budget for
flows of type $i$ is $d_i$. Every flow of type $i$ should have a
end-to-end delay bounded by $D_i$. Our problem is to find good
values of $d_1$ and $d_2$.
\begin{enumerate}
  \item We assume that $\sigma_i=\alpha_i$. What are the
  conditions on $d_1$ and $d_2$ for the end-to-end delay bounds to
  be satisfied ? What is the set of $(n_1, n_2)$ that are
  schedulable ?
  \item Same question if we set $\sigma_i=\lambda_{r_i}$
\end{enumerate}
\end{problem}

%% LaTeX2e file `./temp/L21-16'
%% generated by the `filecontents' environment
%% from source `netCalBookv4' on 2019/02/19.
%%
\begin{problem}
Consider the scheduler in \thref{theo-grissched}. Find an
efficient algorithm for computing the deadline of every packet.
\end{problem}

%% LaTeX2e file `./temp/L21-17'
%% generated by the `filecontents' environment
%% from source `netCalBookv4' on 2019/02/19.
%%
\begin{problem}
Consider a SCED scheduler with target service curve for flow $i$
given by
$$\beta_i= \gamma_{r_i, b_i}\mpc \delta_{d_i}
$$
Find an efficient algorithm for computing the deadline of every
packet.

Hint: use an interpretation as a leaky bucket. \sol{See page 69 of
\cite{saro96}}
\end{problem}

%%% LaTeX2e file `./temp/L21-18'
%% generated by the `filecontents' environment
%% from source `netCalBookv4' on 2019/02/19.
%%
\begin{problem}
Consider the example of \fref{fig-dsnet1}. Apply the method of
\sref{sec-timestoppingm} but express now that the fraction of
input traffic to node $i$ that originates from another node must
have $\lambda_C$ as an arrival curve . What is the upper-bound on
utilization factors for which a bound is obtained ?
\end{problem}
moved to L24
%% LaTeX2e file `./temp/L21-19'
%% generated by the `filecontents' environment
%% from source `netCalBookv4' on 2019/02/19.
%%
\begin{problem}
Consider the delay bound in \thref{theo-qofisbound}. Take the same
assumptions but assume also that the network is feedforward. Which
better bound can you give ?
\end{problem}

%
% put below the list of exercises you want to see in the document
% you must have imported the database in the preamble, with commands
% such as \input{MMMM.bdx.tex}, where MMMM is the name of an exercise
% database file
%
% there is also an automatic mechanism to print all exercises, ask the author
%

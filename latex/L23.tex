\chapter{Adaptive and Packet Scale Rate Guarantees}
\mylabel{L23}

\section{Introduction}
In \cref{L20} we defined a number of service curve concepts:
minimum service curve, maximum service curve and strict service
curves. In this chapter we go beyond and define some concepts that
more closely capture the properties of generalized processor
sharing (GPS).

We start by a motivating section, in which we analyze some
features of service curves or Guaranteed Rate node that do not
match GPS. Then we provide the theoretical framework of packet
scale rate guarantee (PSRG); it is a more complex node
abstraction than Guaranteed Rate, which better captures some
of the properties of GPS. A major difference is the
possibility to derive information on delay when the buffer
size is known -- a property that is not possible with service
curve or guaranteed rate. This is important for low delay
services in the internet. PSRG is used in the definition of
the Internet Expedited Forwarding service.

Just like GR is the max-plus correspondant of the min-plus
concept of service curve, PSRG is the max-plus correspondant
of \emph{adaptive service curves}. These were first proposed
in Okino's dissertation in \cite{oki98} and by Agrawal, Cruz,
Okino and Rajan in \cite{cruzADAP}. We explain the
relationship between the two and give practical applications
to the concatenation of PSRG nodes.


In the context of differentiated services, a flow is an
aggregate of a number of micro-flows that belong to the same
service class. Such an aggregate may join a router by
different ports, and may follow different paths inside the
router. It follows that it can generally not be assumed that a
router is FIFO per flow. This is why the definition of PSRG
(like GR) does not assume the FIFO property.

In all of this chapter, we assume that flow functions are
left-continuous, unless stated otherwise.


\section{Limitations of the Service Curve and GR  Node
Abstractions} \mylabel{sec-sercurnotenough} The definition of
service curve introduced in \sref{sec-sercur} is an
abstraction of nodes such as GPS and its practical
implementations, as well as guaranteed delay nodes. This
abstraction is used in many situations, described all along
this book. However, it is not always sufficient.

Firstly, it does not provide a guarantee over any interval.
Consider for example a node offering to a flow $R(t)$ the service
curve $\lambda_C$. Assume $R(t)=B$ for $t>0$, so the flow has a
very large burst at time $0$ and then stops. A possible output is
illustrated on \fref{fig-ascintro}. It is perfectly possible that
there is no output during the time interval $(0,
\frac{B-\epsilon}{C}]$, even though there is a large backlog. This
is because the server gave a higher service than the minimum
required during some interval of time, and the service property
allows it to be lazy after that.
\begin{figure}[!htbp]
  \insfig{ascintro}{0.7}
  \mycaption{The service curve property is not sufficient.}
  \mylabel{fig-ascintro}
\end{figure}


Secondly, there are case where we would like to deduce a bound on
the delay that a packet will suffer given the backlog that we can
measure in the node. This is used for obtaining bounds in FIFO
systems with aggregate scheduling. In \cref{L24} we use such a
property for a constant delay server with rate $C$: given that the
backlog at time $t$ is $Q$, the last bit present at time $t$ will
depart before within a time of $\frac{Q}{C}$. If we assume instead
that the server has a service curve $\lambda_C$, then we cannot
draw such a conclusion. Consider for example \fref{fig-ascintro}:
at time $t >0$, the backlog, $\epsilon$, can be made arbitrily
small, whereas the delay $\frac{B - \epsilon}{C}-t$ can be made
arbitrarily large.

The same limitation applies to the concept of Guaranteed Rate
node. Indeed, the example in \fref{fig-ascintro} could very
well be for GR node. The main issue here is that a GR node,
like a service curve element, may serve packets \emph{earlier}
than required.

A possible fix is the use of \emph{strict service curve}, as
defined in \dref{def-ssc} on \pgref{def-ssc}. Indeed, it follows
from the next section (and can easily be shown independently) that
if a FIFO node offers a strict service curve $\beta$, then the
delay at time $t$ is bounded by $\beta^{-1}(Q(t))$, where $Q(t)$
is the backlog at time $t$, and $\beta^{-1}$ is the pseudo-inverse
(\dref{def:pseudo-inverse} on \pgref{def:pseudo-inverse}).
%\end{definition}
%\pr
%Call  $R(t), R^*(t)$ the input and output. If the last bit present
%at time $t$ has not yet departed at time $t+ \tau$, then we must
%have
%$$R^*(t+ \tau) < R(t)
% $$
% and
% $$
% R^*(t + \tau) \geq R^*(t) + \beta (\tau)
% $$
% thus
% $$
% \beta(\tau) <R(t)-R^*(t)=Q(t)
% $$
% thus the delay is bounded by $\sup\{\tau: \beta(\tau) > Q(t) \}$
% which is equal to $\beta^{-1}(Q(t))$.
% \qed

We know that the GPS node offers to a flow a strict service
curve equal of the form $\lambda_{R}$. However, we cannot
model delay nodes with a strict service curve. Consider for
example a node with input $R(t)=\epsilon t$, which delays all
bits by a constant time $d$. Any interval $[s,t]$ with $s \geq
d$ is within a busy period, thus if the node offers a strict
service curve $\beta$ to the flow, we should have $\beta(t-s)
\epsilon (t-s)$, and $\epsilon$ can be arbitrarily small.
Thus, the strict service curve does not make much sense for a
constant delay node.
\section{Packet Scale Rate Guarantee}
\subsection{Definition of Packet Scale Rate
Guarantee}

In \sref{sec-gr} on \pgref{sec-gr} we have introduced the
definition of guaranteed rate scheduler, which is the
practical application of rate latency service curves. Consider
a node where packets arrive at times $a_1\geq 0, a_2, ...$ and
leave at times $d_1, d_2, ...$. A guaranteed rate scheduler,
with rate $r$ and latency $e$ requires that $d_i \leq f'_i +
e$, where $f'_i$ is defined iteratively by $f'_0=0$ and
 $$
 f'_i=\max\{a_i, f'_{i-1}\} + \frac{l_i}{r}
 $$
where $l_i$ is the length of the $i$th packet.

A \emph{packet scale rate guarantee} is similar, but avoids
the limitations of the service curve concept discussed in
\sref{sec-sercurnotenough}. To that end, we would like that
the deadline $f'_i$ is reduced whenever a packet happens to be
served early. This is done by replacing $f'_{i-1}$ in the
previous equation by $\min\{f'_i, d_i\}$. This gives the
following definition.
\begin{definition}[Packet Scale Rate Guarantee]
Consider a node that serves a flow of packets numbered $i=1,
2, ...$. Call $a_i, d_i, l_i$ the arrival time, departure
time, and length in bits for the $i$th packet, in order of
arrival. Assume $a_1 \geq 0$.We say that the node offers to
the flow a packet scale rate guarantee with rate $r$ and
latency $e$ if the departure times satisfy
$$d_i \leq f_i + e$$ where $f_i$ is defined by:
\begin{equation}\mylabel{eq-psrg-def}
\bracket{f_0= d_0= 0 \\
    f_i = \max \left\{
                   a_i, \min\left(d_{i-1}, f_{i-1}\right)
                 \right\}
             + \frac{l_i}{r} \gap \mfa i \geq 1
            }
\end{equation}
\index{Packet Scale Rate Guarantee}%
 \mylabel{def-psrg}
\end{definition}

See \fref{fig-psrgdef} and \fref{fig-psrggr} for an
illustration of the definition.

\begin{figure}[!htbp]
  \insfig{psrgdef}{0.85}
  \mycaption{Definition of PSRG.}
  \mylabel{fig-psrgdef}
\end{figure}

\begin{figure}[!htbp]
  \insfig{psrggr}{0.85}
  \mycaption{Difference between PSRG and GR when packet $n-1$ leaves before $f_n$.}
  \mylabel{fig-psrggr}
\end{figure}

\begin{theorem}
A PSRG node with rate $r$ and latency $e$ is GR($r,e$).
\mylabel{theo-psrgisgr}
\end{theorem}
\pr Follows immediately from the definition. \qed

\textbf{Comment.} It follows that a PSRG node enjoys all the
properties of a GR node. In particular:
\begin{itemize}
  \item Delay bounds for input traffic with arrival curves can
  be obtained from \thref{theo-grcdelaybound}.
  \item PSRG nodes have a  rate latency service curve property (\thref{theo-grcrep})
  that can be used for buffer dimensioning.
\end{itemize}

We now obtain a characterization of packet scale rate
guarantee that does not contain the virtual finish times
$f_n$. It is the basis for many results in this chapter. We
start with an expansion of the recursive definition of packet
scale rate guarantee,

\begin{lemma}[Min-max expansion of PSRG]
\mylabel{lem-psrgmaxplus} Consider three arbitrary sequences
of non-negative numbers $(a_n)_{n \geq 1}$, $(d_n)_{n \geq
0}$, and $(m_n)_{n \geq 1}$, with $d_0 =0$. Define the
sequence $(f_n)_{n\geq 0}$, by
$$\bracket{
 f_0 =0\\
 f_n = \max \left[a_n, \min \left(d_{n-1}, f_{n-1} \right)\right] + m_n \mfor
 n \geq 1
 }
$$
Also define
$$
 \bracket{
 A^n_j=a_j +  m_j + ... + m_n \mfor 1 \leq j\leq n\\
 D^n_j=d_j + m_{j+1} + ... + m_n \mfor 0 \leq j\leq n -1
 }
$$
For all $n \geq 1$, we have
  \begin{eqnarray*}
    f_n  = \min & [ & \max(A_n^n , A_{n-1}^n , ... ,
    A_1^n),\\
 &&  \max(A_n^n , A_{n-1}^n ... , A_2^n , D_1^n),\\
 &&    ...\\
  &&  \max(A_n^n , A_{n-1}^n ... , A_{j+1}^n , D_j^n),\\
  &&   ...\\
 &&  \max(A_n^n , A_{n-1}^n , D_{n-2}^n),\\
 && \max(A_n^n , D_{n-1}^n)\\
 & ]\end{eqnarray*}
\end{lemma}

%\begin{theorem}[Max-Min-Plus expansion of PSRG]
%\mylabel{theo-psrgmaxplus} Consider three arbitrary sequences of
%non-negative numbers $(a_n)_{n \geq 1}$, $(d_n)_{n \geq 0}$, and
%$(m_n)_{n \geq 1}$, with $d_0 =0$. Define the sequence
%$(f_n)_{n\geq 0}$, by
%$$\bracket{
% f_0 =0\\
% f_n = a_n \vee \left[d_{n-1}\wedge f_{n-1} \right] + m_n \mfor
% n \geq 1
% }
%$$
%Also define
%$$
% \bracket{
% A^n_j=a_j +  m_j + ... + m_n \mfor 1 \leq j\leq n\\
% D^n_j=d_j + m_{j+1} + ... + m_n \mfor 0 \leq j\leq n
% }
%$$
%with the convention that $D_n^n=d_n$.  For all $n \geq 1$, we have
%  \begin{eqnarray*}
%    \lefteqn{f_n = \left[A_n^n \vee A_{n-1}^n ... \vee A_1^n\right]}\\
% &&  \wedge  \left[A_n^n \vee A_{n-1}^n ... \vee A_2^n \vee D_1^n\right]\\
% &&  \wedge ...\\
%  &&  \wedge  \left[A_n^n \vee A_{n-1}^n ... \vee A_{j+1}^n \vee D_j^n\right]\\
%  &&  \wedge ...\\
% &&  \wedge \left[A_n^n \vee A_{n-1}^n \vee D_{n-2}^n\right]\\
% &&  \wedge \left[A_n^n \vee D_{n-1}^n\right]\\
%  \end{eqnarray*}
%\end{theorem}
The proof is long and is given in a separate section
(\sref{sec-ascproof}); it is based on min-max algebra.

\textbf{Comment: }The expansion in \lref{lem-psrgmaxplus} can
be interpreted as follows. The first term $\max(A_n^n ,
A_{n-1}^n , ... ,
    A_1^n)$ corresponds to the guaranteed rate clock recursion
    (see \thref{theo-GR}). The following terms have the
    effect of reducing $f_n$, depending on the values of $d_j$.


We now apply the previous lemma to packet scale rate guarantee
and obtain the required characterization without the virtual
finish times $f_n$:
\begin{theorem}
\mylabel{theo-psrgmaxplus} Consider a system where packets are
numbered $1, 2, ...$ in order of arrival. Call $a_n$, $d_n$
the arrival and departure times for packet $n$, and $l_n$ the
size of packet $n$. Define by convention $d_0=0$. The packet
scale rate guarantee with rate $r$ and latency $e$ is
equivalent to: For all $n$ and all $0\leq j\leq n-1$, one of
the following holds
\begin{equation}\mylabel{eq-psrgmaxplus1}
 d_n \leq e + d_j + \frac{l_{j+1} + ... + l_n}{r}
 \end{equation}
or there is some $k \in \{j+1, ..., n\}$ such that
\begin{equation}\mylabel{eq-psrgmaxplus2}
 d_n \leq e + a_k + \frac{l_k + ... + l_n}{r}
\end{equation}
\end{theorem}
The proof is also given in \sref{sec-ascproof}. It is a
straightforward application of \lref{lem-psrgmaxplus}.

\textbf{Comment 1: }The original definition of EF in
\cite{RFC2598} was based on the informal intuition that a node
guarantees to the EF aggregate a rate equal to $r$, at all
time scales (this informal definition was replaced by PSRG).
\thref{theo-psrgmaxplus} makes the link to the original
intuition: a rate guarantee at all time scales means that
either \eref{eq-psrgmaxplus1} or \eref{eq-psrgmaxplus2} must
hold. For a simple scheduler, the former means that $d_j, d_n$
are in the same backlogged period; the latter is for the
opposite case, and here $a_k$ is the beginning of the
backlogged period. But note that we do not assume that the
PSRG node is a simple scheduler; as mentioned earlier, it may
be any complex, non work conserving node. It is a merit of the
abstract PSRG definition to avoid using the concept of
backlogged period, which is not meaningful for a composite
node \cite{efsup-rfc,psrgton}.

\textbf{Comment 2: }In \thref{theo-grmaxplus} we give a
similar result for GR nodes. It is instructive to compare both
in the case of a simple scheduler, where the interpretation in
terms of backlogged period can be made.  Let us assume the
latency term is $0$, to make the comparison simple. For such a
simple scheduler,  PSRG means that during \emph{any}
backlogged period, the scheduler guarantees a rate at least
equal to $r$. In contrast, and again for such simple
schedulers, GR means that during the backlogged period
starting at the first packet arrival that finds the system
empty (this is called ``busy period" in queuing theory), the
average rate of service is at least $r$. GR allows the
scheduler to serve some packets more quickly than at rate $r$,
and take advantage of this to serve other packets at a rate
smaller than $r$, as long as the overall average rate is at
least $r$. PSRG does not allow such a behaviour.

A special case of interest is when $e=0$.

\begin{definition}
We call \emph{minimum rate server}, with rate $r$, a PSRG node
for with latency $e=0$.
\index{minimum rate server}
\end{definition}
For a minimum rate server we have
\begin{equation}\mylabel{eq-mr-def}
\bracket{ d_0=0\\
    d_i \leq \max \left\{
                   a_i, d_{i-1}
                 \right\}
             + \frac{l_i}{r} \gap \mfa i \geq 1
            }
\end{equation}
Thus, roughly speaking, a
minimum rate server guarantees that during any busy period,
the instantaneous output rate is at least $r$. A GPS node with
total rate $C$ and weight $w_i$ for flow $i$ is a minimum rate
server for flow $i$, with rate $r_i=\frac{w_i C }{\sum_j
w_j}$.



\subsection{Practical Realization of Packet Scale Rate Guarantee}

We show in this section that a wide variety of schedulers
provide the packet scale rate guarantee. More schedulers can
be obtained by using the concatenation theorem in the previous
section.

A simple but important realization is the priority scheduler.
\begin{proposition}
Consider a non-preemptive priority scheduler in which all
packets share a single FIFO queue with total output rate $C$.
The high priority flow receives a packet scale rate guarantee
with rate $C$ and latency $e= \frac{l_{max}}{C}$, where
$l_{max}$ is the maximum packet size of all low priority
packets.
 \mylabel{prop-asc-pq}
\end{proposition}
\pr
By \pref{prop-priostric}, the high priority traffic receives a
strict service curve $\beta_{r,c}$. \qed

We have already introduced in \sref{sec-gr} schedulers that
can be thought of as derived from GPS and we have modeled
their behaviour with a rate-latency service curve. In order to
give a PSRG for such schedulers, we need to define more.

%% local definition
\newcommand{\wf} {\textrm{WF}^2\textrm{Q}}
\begin{definition}[PSRG Accuracy of a scheduler with respect to rate $r$]
Consider a scheduler $S$ and call $d_i$ the time of the $i$-th
departure. We say that the PSRG accuracy of $S$ with respect
to rate $r$ is $(e_1, e_2)$ if there is a minimum rate server
with rate $r$ and departure times $g_i$ such that for all $i$
\begin{equation}\mylabel{eq-asc-eq6}
  g_i-e_1 \leq d_i\leq g_i + e_2
\end{equation}
\end{definition}
We interpret this definition as a comparison to a hypothetical
GPS reference scheduler that would serve the same flows. The
term $e_2$ determines the maximum  per-hop delay bound,
whereas $e_1$ has an effect on the jitter at the output of the
%XXX change all the W2FQ to  $\textrm{WF}^{\texterm{2}}\texterm{Q}$
%XXX or at least WF2Q if the math verson does not work :-)
scheduler. For example, it is shown in\ \cite{BZ96a} that
$\wf$ satisfies $e_1(\wf) = l_{max}/r$, $e_2(\wf) =
l_{max}/C$, where $l_{max}$ is maximum packet size and $C$ is
the total output rate. In contrast, for PGPS \cite{pg94}
$e_2(\textrm{PGPS}) = e_2(\wf)$, while $e_1(\textrm{PGPS})$ is
linear in the number of queues in the scheduler. This
illustrates that, while $\wf$ and PGPS have the same delay
bounds, PGPS may result in substantially burstier departure
patterns.
\begin{theorem}
\label{theo-ascgps} If a scheduler satisfies
\eref{eq-asc-eq6}, then it offers the packet scale rate
guarantee with rate $r$ and latency $e= e_1 + e_2$.
\end{theorem}
The proof is in \sref{sec-ascproof}.

\subsection{Delay From Backlog}
 A main feature of the packet scale rate
guarantee definition is that it allows to bound delay from
backlog. For a FIFO node, it could be derived from
\thref{theo-ascdelay} and \thref{theo-ascrep}. But the
important fact is that the bound is the same, with or without
FIFO assumption.

\begin{theorem} \mylabel{theo-delfrbknf}
Consider a node offering the Packet Scale Rate Guarantee with
rate $r$ and latency $e$, not necessarily FIFO. Call $Q$ the
backlog at time $t$. All packets that are in the system at
time $t$ will leave the system no later than at time
$t+Q/r+e$,
\end{theorem}
The proof is in \sref{sec-ascproof}.


\textbf{Application to Differentiated Services} Consider a
network of nodes offering the EF service, as in
\sref{sec-dsgen}. Assume node $m$ is a PSRG node with rate
$r_m$ and latency $e_m$. Assume the buffer size at node $m$ is
limited to $B_m$. A bound $D$ on delay at node $m$ follows
directly
$$
D=\frac{B_m}{r_m} + e_m
$$
Compare to the bound in \thref{theo-qofisbound}: this bound is
valid for all utilization levels and is independent of traffic
load. \fref{fig-boundlal7} shows a numerical example.
\begin{figure}[!htbp]
  \insfig{efboundpsrg}{0.85}
  \mycaption{End to end delay bound versus the
utilization factor $\alpha$ for an infinite buffer (left
curve) and buffers sizes of $1$MB (top), $0.38$MB (middle) and
$0.1$MB (bottom).  There are $h=10$ hops, $e_m=2
\frac{1500B}{r_m}$, $\sigma_{i}=100$B and $\rho_{i}=32$kb/s
for all flows, $r_m=149.760$Mb/s.}
  \mylabel{fig-boundlal7}
\end{figure}

However, forcing a small buffer size may cause some packet
loss. The loss probability can be computed if we assume in
addition that the traffic at network edge is made of
stationary, independent flows \cite{VL2002EF}.

\section{Adaptive Guarantee}
\subsection{Definition of Adaptive Guarantee} Much in the spirit of PSRG, we know introduce a
stronger service curve concept, called \emph{adaptive
guarantee}, that better captures the properties of GPS
\cite{oki98,cruzADAP}, and helps finding concatenation
properties for PSRG. Before giving the formula, we motivate it
on three examples.

\textbf{Example 1.} Consider a node offering a strict service
curve $\beta$. Consider some fixed, but arbitrary times $s<t$.
Assume that $\beta$ is continuous. If $[s,t]$ is within a busy
period, we must have
 $$
 R^*(t) \geq R^*(s) + \beta (t-s)
 $$
Else, call $u$ the beginning of the busy period at $t$. We
have
$$
R^*(t) \geq R(u) + \beta(t-u)
$$
thus in all cases
\begin{equation}\mylabel{eq-introasc1}
 R^*(t) \geq \left( R^*(s) + \beta(t-s)\right) \wedge \inf_{u \in
[s,t] }\left(R(u) + \beta(t-u)\right)
\end{equation}

\textbf{Example 2.} Consider a node that guarantees a virtual
delay $\leq d$. If $t-s\leq d$ then trivially
$$
 R^*(t) \geq R^*(s) + \delta_d (t-s)
 $$
 and if $t-s >d$ then the virtual delay property means that
 $$
 R^*(t) \geq R(t-d)= \inf_{u \in
[s,t] }\left(R(u) + \delta_d(t-u)\right)
 $$
 thus we have the same relation as in \eref{eq-introasc1}
with $\beta=\delta_d$.

\textbf{Example 3.} Consider a greedy shaper with shaping
function $\sigma$ (assumed to be a good function). Then
$$
R^*(t) = \inf_{u\leq t} [R(u) + \sigma(t-u)]
$$
Breaking the inf into $u <s$ and $u\geq s$ gives
\begin{equation}\mylabel{eq-ascsh1}
  R^*(t) = \inf_{u < s}[R(u)+\sigma(t-u)] \wedge \inf_{u \in
[s,t]}[R(u)+\sigma(t-u)]
\end{equation}
Define $\tilde{\sigma}:=\sigma \Mpd \sigma$, namely,
\begin{equation}\mylabel{eq-ascsigtil}
\tilde{\sigma}(u) = \inf_t [\sigma(t+u) - \sigma(u)]
\end{equation}
For example, for a piecewise linear concave arrival curve
(conjunction of leaky buckets), $\sigma(t)=\min_i (r_i u +
b_i)$, we have $\tilde{\sigma}(u)=\min_i r_i u$. Back to
\eref{eq-ascsh1}, we have
$$
\sigma(t-u) \geq \sigma(s-u) + \tilde{\sigma}(t-s)
$$
and finally
\begin{equation}\mylabel{eq-introasc2}
 R^*(t) \geq \left( R^*(s) + \tilde{\sigma}(t-s)\right) \wedge \inf_{u \in
[s,t] }\left(R(u) + \sigma(t-u)\right)
\end{equation}
We see that these three cases fall under a common model:
\begin{definition}[Adaptive Service Curve]
Let $\tilde{\beta},\beta$ be in $\calF$.
%with
%$\tilde{\beta} \leq \beta$.
Consider a system $\mathcal{S}$ and a flow through
$\mathcal{S}$ with input and output functions $R$ and $R^{*}$.
We say that $\calS$ offers the \emph{adaptive guarantee}
$(\tilde{\beta},\beta)$ if for any $s \leq t$ it holds:
$$
R^*(t) \geq \left( R^*(s) + \tilde{\beta}(t-s)\right)  \wedge
\inf_{u \in [s,t]}[R(u)+\beta(t-u)]
$$%
\index{adaptive guarantee}%
If $\tilde{\beta}=\beta$ we say that the node offers the
adaptive guarantee $\beta$.
 \mylabel{def-asc}
\end{definition}
The following proposition summarizes the examples discussed
above:
\begin{proposition}
\begin{itemize}
  \item If $\calS$ offers to a flow a strict service curve $\beta$, then
it also offers the \emph{adaptive guarantee} $\beta$.
  \item If $\calS$ guarantees a virtual delay bounded by $d$, then
  it also offers the \emph{adaptive guarantee} $\delta_d$
  \item A greedy shaper with shaping curve $\sigma$, where
  $\sigma$ is a good function, offers the \emph{adaptive guarantee}
  $(\tilde{\sigma},\sigma)$,
with $\tilde{\sigma}$ defined in \eref{eq-ascsigtil}.
\end{itemize}
\end{proposition} Similar to \cite{oki98}, we use the notation $R
\rightarrow (\tilde{\beta}, \beta) \rightarrow R^*$ to express
that \dref{def-asc} holds. If $\tilde{\beta}=\beta$ we write
$R \rightarrow (\beta) \rightarrow R^*$.

Assume that $R$ is left-continuous and $\beta$ is continuous.
It follows from \thref{theo-minplust0} on
\pgref{theo-minplust0} that the adaptive guarantee is
equivalent to saying that for all $s \leq t$, we have either
$$
R^*(t) - R^*(s) \geq \tilde{\beta}(t-s)
$$
or
$$
R^*(t) \geq R(u) + \beta(t-u)
$$
for some $u \in [s,t]$.
\subsection{Properties of Adaptive
Guarantees} \mylabel{sec-ascprop}
\begin{theorem}
Let $R \rightarrow (\tilde{\beta}, \beta) \rightarrow R^*$. If
$\tilde{\beta} \leq \beta$ then $\beta$ is a minimum service
curve for the flow. \mylabel{theo-ascsc}
\end{theorem}
\pr Apply \dref{def-asc} with $s=0$ and use the fact that
$\tilde{\beta} \leq \beta$.
 \qed



\begin{theorem}[Concatenation]
If $R \rightarrow (\tilde{\beta_1}, \beta_1) \rightarrow R_1$
and $R_1 \rightarrow (\tilde{\beta_2}, \beta_2) \rightarrow
R^*$ then $R \rightarrow (\tilde{\beta}, \beta) \rightarrow
R^*$ with
$$
\tilde{\beta}=\left( \tilde{\beta_1} \mpc \beta_2 \right)
\wedge \tilde{\beta_2}
$$
and
$$
\beta= \beta_1 \mpc \beta_2
$$
 \mylabel{theo-ascconc}
\end{theorem}
The proof is in \sref{sec-ascproof}
 \begin{corollary}
If $R_{i-1} \rightarrow (\tilde{\beta_i}, \beta_i) \rightarrow
R_i$ for $i=1$ to $n$ then $R_0 \rightarrow (\tilde{\beta},
\beta) \rightarrow R_n$ with
$$
\tilde{\beta}= \left( \tilde{\beta_1} \mpc \beta_2 \mpc ...
\mpc \beta_n \right)
 \wedge
 \left( \tilde{\beta_2} \mpc \beta_3 \mpc
... \mpc \beta_n \right)
 \wedge
 ...
 \wedge
 \left( \tilde{\beta}_{n-1} \mpc \beta_n \right)
 \wedge
 \tilde{\beta_n}
$$
and
$$
\beta= \beta_1 \mpc ... \mpc \beta_n
$$
 \mylabel{coro-ascconc}
 \end{corollary}
 \pr
 Apply \thref{theo-ascconc} iteratively and use
Rule 6 in \thref{thm:rule1-7} on \pgref{thm:rule1-7}. \qed

\begin{theorem}[Delay from Backlog]
If $R \rightarrow (\tilde{\beta}, \beta) \rightarrow R^*$,
then the virtual delay at time $t$ is bounded by
$\tilde{\beta}^{-1}(Q(t))$, where $Q(t)$ is the backlog at
time $t$, and $\tilde{\beta}^{-1}$ is the pseudo-inverse of
$\tilde{\beta}$ (see \dref{def:pseudo-inverse} on
\pgref{def:pseudo-inverse}).
 \mylabel{theo-ascdelay}
\end{theorem}
The proof is in \sref{sec-ascproof}. Note that if the node is
FIFO, then the virtual delay at time $t$ is the real delay for
a bit arriving at time $t$.

Consider a system (\emph{bit-by-bit system}) with
$L$-packetized input $R$ and bit-by-bit output $R^*$, which is
then $L$-packetized to produce a final packetized output $R'$.
We call \emph{combined system} the system that maps $R$ into
$R'$. Assume both systems are first-in-first-out and lossless.
Remember from \thref{theo-delvlp} that the per-packet delay
for the combined system is equal the maximum virtual delay for
the bit-by-bit system.

\begin{theorem}[Packetizer and Adaptive Guarantee]
If the bit-by-bit system offers to the flow the adaptive
guarantee $(\tilde{\beta}, \beta)$, then the combined system
offers to the flow the adaptive guarantee $(\tilde{\beta}',
\beta')$ with
$$\tilde{\beta}'(t) = [\tilde{\beta}(t)-l_{\max}]^+
$$ and
$$\beta'(t) = [\beta(t)-l_{\max}]^+
$$
where $l_{\max}$ is the maximum packet size for the flow.
\mylabel{theo-ascpack}
\end{theorem}
The proof is in \sref{sec-ascproof}.



\subsection{PSRG and Adaptive Service Curve}

We now relate packet scale rate guarantee to an adaptive
guarantee. We cannot expect an exact equivalence, since a
packet scale rate guarantee does not specify what happens to
bits at a time other than a packet departure or arrival.
However, the concept of packetizer allows us to establish an
equivalence.

\begin{theorem}[Equivalence with adaptive guarantee]
Consider a node $\calS$ with $L$-packetized input $R$ and with
output $R^*$.
\begin{enumerate}
  \item If $R \rightarrow (\beta)\rightarrow R^*$,
where $\beta=\beta_{r,e}$ is the rate-latency function with
rate $r$ and latency $e$, and if $\calS$ is FIFO, then $\calS$
offers to the flow the packet scale rate guarantee with rate
$r$ and latency $e$.
  \item Conversely, if $\calS$ offers to the flow the packet
scale rate guarantee with rate $r$ and latency $e$ and if
$R^*$ is $L$-packetized, then $\calS$ is the concatenation of
a node $\calS'$ offering the adaptive guarantee $\beta_{r,e}$
and the $L$-packetizer. If $\calS$ is FIFO, then so is
$\calS'$.
\end{enumerate}
 \mylabel{theo-ascrep}
\end{theorem}
The proof is long and is given in a separate section
(\sref{sec-ascproof}). Note that the packet scale rate
guarantee does not mandate that the node be FIFO; it is
possible that $d_i < d_{i-1}$ in some cases. However, part 1
of the theorem requires the FIFO assumption in order for a
condition on $R, R^*$ to be translated into a condition on
delays.

\section{Concatenation of PSRG Nodes}
%%%ici

\subsection{Concatenation of FIFO PSRG Nodes}
We have a simple concatenation result for FIFO systems:
\begin{theorem}
Consider a concatenation of FIFO systems numbered $1$ to $n$.
The output of system $i-1$ is the input of system $i$, for
$i>1$. Assume system $i$ offers the packet scale rate
guarantee with rate $r_i$ and latency $e_i$. The global system
offers the packet scale rate guarantee with rate
$r=\min_{i=1,...,n} r_i$ and latency $e= \sum_{i=1,...,n} e_i
+ \sum_{i=1,...,n-1}\frac{L_{\max}}{r_i}$. \mylabel{prop-conc}
\end{theorem}
 \pr
 By \thref{theo-ascrep}--(2), we can decompose system $i$ into a
concatenation $\calS_i, \calP_i$, where $\calS_i$ offers the
adaptive guarantee $\beta_{r_i,e_i}$ and $\calP_i$ is a
packetizer.

Call $\calS$ the concatenation
 $$\calS_1, \calP_1, \calS_2,
\calP_2,...,\calS_{n-1},\calP_{n-1},\calS_n$$ By
\thref{theo-ascrep}--(2), $\calS$ is FIFO. By
\thref{theo-ascpack}, it provides the adaptive guarantee
$\beta_{r,e}$. By \thref{theo-ascrep}--(1), it also provides
the packet scale rate guarantee with rate $r$ and latency $e$.
Now $\calP_n$ does not affect the finish time of the last bit
of every packet.

\qed

\textbf{A Composite Node} We analyze in detail one specific
example, which often arises in practice when modelling a
router. We consider a composite node, made of two components.
The former (``variable delay component") imposes to packets a
delay in the range $[\delta_{\max}-\delta, \delta_{\max}]$.
The latter is FIFO and offers to its input the packet scale
rate guarantee, with rate $r$ and latency $e$. We show that,
if the variable delay component is known to be FIFO, then we
have a simple result. We first give the following lemma, which
has some interest of its own.

\begin{lemma}[Variable Delay as PSRG]
\mylabel{lem-vdpsrg} Consider a node which is known to
guarantee a delay $\leq \delta_{\max}$. The node need not be
FIFO. Call $l_{\min}$ the minimum packet size. For any $r>0$,
the node offers the packet scale rate guarantee with latency
$e=[\delta_{\max}-\frac{l_{\min}}{r}]^+$ and rate $r$.
\end{lemma}
\begin{preuve} With the standard notation in this section, the
hypothesis implies that $d_n \leq a_n + \delta_{\max}$ for all
$n \geq 1$. Define $f_n$  by \eref{eq-psrg-def}. We have
$f_n\geq a_n + \frac{l_n}{r} \geq a_n + \frac{l_{\min}}{r}$,
thus $d_n-f_n \leq \delta_{\max} - \frac{l_{\min}}{r}\leq
[\delta_{\max}-\frac{l_{\min}}{r}]^+$. \end{preuve}

We will now apply known results on the concatenation of FIFO
elements and solve the case where the variable delay component
is FIFO.
\begin{theorem}(\emph{Composite Node with FIFO Variable Delay Component})
\mylabel{theo-fvd} Consider the concatenation of two nodes.
The former imposes to packets a delay  $\leq \delta_{\max}$.
The latter offers the packet scale rate guarantee to its
input, with rate $r$ and latency $e$. Both nodes are FIFO. The
concatenation of the two nodes, in any order, offers the
packet scale rate guarantee with rate $r$ and latency $e'=e
+\delta_{\max}$.
\end{theorem}
\begin{preuve} Combine \thref{theo-ascconc} with
\lref{lem-vdpsrg}: for any $r' \geq r$, the combined node
offers the packet scale guarantee with rate $r$ and latency
$e(r')=e + \delta_{\max} + \frac{l_{\max}-l_{\min}}{r'}$.
Define $f_n$ for all $n$ by \eref{eq-psrg-def}. Consider some
fixed but arbitrary $n$. We have $d_n - f_n \leq e(r')$, and
this is true for any $r'\geq r$. Let $r' \rightarrow +\infty$
and obtain $d_n - f_n \leq \inf_{r' \geq r} e(r') = e +
\delta_{\max}$ as required.
\end{preuve}

\subsection{Concatenation of non FIFO PSRG Nodes}
In general, we cannot say much about the concatenation of non
FIFO PSRG nodes. We analyze in detail composite node described
above,  but now the delay element is non FIFO. This is a
frequent case in practice. The results are of interest for
modelling a router. The also serve the purpose of showing that
the results in \thref{prop-conc} do not hold here.

To obtain a result, we need to an arrival curve for the
incoming traffic. This is because some packets may take over
some other packets in the non-FIFO delay element
(\fref{fig-expsrgnf}); an arrival curve puts a bound on this.

\begin{figure}[!htbp]
  \insfig{expsrgnf}{0.7}
  \mycaption{Composite Node with non-FIFO Variable Delay Component. Packet $n$ arrives at times
  $a_n$ at the first component, at time $b_n$ at the second component, and leaves the system
  at time $d_n$. Since the first component is not FIFO, overtaking may occur; $(k)$ is the packet number of
  the $k$th packet arriving at the second component. }
  \mylabel{fig-expsrgnf}
\end{figure}

\begin{theorem}(\emph{Composite Node with non-FIFO Variable Delay
Component}) \mylabel{theo-nfvd} Consider the concatenation of
two nodes. The first imposes to packets a delay in the range
$[\delta_{\max}-\delta, \delta_{\max}]$. The second is FIFO
and offers the packet scale rate guarantee to its input, with
rate $r$ and latency $e$. The first node is not assumed to be
FIFO, so the order of packet arrivals at the second node is
not the order of packet arrivals at the first one. Assume that
the fresh input is constrained by a continuous arrival curve
$\alpha(\cdot)$. The concatenation of the two nodes, in this
order,  satisfies the packet scale rate guarantee with rate
$r$ and latency
\begin{equation}\mylabel{eq-theonfvd}
  \begin{array}{rl}
    e'=&  e+ \delta_{\max} +  \\
    & \min \{
  \sup_{t \geq 0}[ \frac{\alpha(t+ \delta) - l_{\min}}{r}- t], \\
    & \; \; \sup_{0 \leq t \leq \delta}
  [ \frac{\alpha(t)+ \alpha(\delta) -
2 l_{\min}}{r}- t] \}
  \end{array}
\end{equation}
%$$e'= e+ \delta_{\max} + \frac{\alpha(\delta)}{r} + \sup_{0 \leq u
%\leq \delta}[\frac{\alpha(u)}{r}- u]
%$$
%where $\alpha(u+)$ is the limit to the right of $\alpha$.
\end{theorem}
The proof is long, and is given in \sref{sec-ascproof}.

Figures \ref{fig-numapp} to \ref{fig-numapp3} show numerical
applications when the arrival curve includes both peak rate
and mean rate constraints.

\textbf{Special Case : }For $\alpha(t)=\rho t + \sigma$, a
direct computation of the suprema in \thref{theo-nfvd} gives:
$$\begin{array}{rl}
\mif \rho \leq r
 \mthen & e'=e+ \delta_{\max}+\frac{\rho \delta +\sigma - l_{\min}}{r}
\\
 \melse & e'=e+ \delta_{\max} -
 \delta + 2 \frac{\rho \delta + \sigma - l_{\min}}{r}
\end{array}
 $$
The latency of the composite node has a discontinuity equal to
$\sigma/r$ at $\rho=r$. It may seem irrelevant to consider the
case $\rho
> r$. However, PSRG gives a delay from backlog
bound; there may be cases where the only information available
on the aggregate input is a bound on sustainable rate $\rho$,
with $\rho
>r$. In such cases, there are probably other mechanisms (such as
window flow control \cite{CO96}) to prevent buffer overflow;
here, it is useful to be able to bound $e'$ as in
\thref{theo-nfvd}.
\begin{figure}[!htbp]
  \insfig{psrgnf1}{0.4}
  \mycaption{Numerical Application of \thref{theo-nfvd} and \thref{theo-nfgrvd},
  showing the additional latency $e'-e$ for a composite
  node, made of a variable delay element ($\delta=\delta_{\max}=0.01 s$)
  followed by a PSRG or GR component with rate $r=100$Mb/s and latency $e$. The
  fresh traffic has arrival curve $\rho t + \sigma$, with
  $\sigma=50$KBytes. The figure shows $e'-e$ as a function of $\rho$, for $\l_{\min}=0$.
  Top graph: delay element is non-FIFO, second component is PSRG
  (\thref{theo-nfvd}); middle graph: delay element is non-FIFO, second component is
  GR (\thref{theo-nfgrvd}); bottom line: delay element is FIFO, both cases (\thref{theo-fvd} and \thref{theo-nfvd}).
  Top and middle graph coincide for $\rho \leq r$.
} \mylabel{fig-numapp}
\end{figure}
\begin{figure}[!htbp]
\begin{center}
  \ifig{psrgnf31}{0.4}\ifig{psrgnf33}{0.4}
\end{center}
  \mycaption{Same as \fref{fig-numapp}, but the fresh traffis has a peak rate
  limit. The arrival curve for the
  fresh traffic is $\min(pt+MTU,\rho t + \sigma)$, with $MTU=500$B, $p=200Mb/s$ (top picture) or $p=2 \rho$ (bottom picture).
} \mylabel{fig-numapp2}
\end{figure}
\begin{figure}[!htbp]
  \insfig{psrgnf3d}{0.5}
  \mycaption{Latency increase as a function of peak rate and mean rate. The parameters are the same as for \fref{fig-numapp2}.
} \mylabel{fig-numapp3}
\end{figure}


\textbf{Comment 1 : } We now justify why \thref{theo-nfvd} is
needed, in other words: if we relax the FIFO assumption for
the variable delay component, then \thref{theo-fvd} does not
hold any more. Intuitively, this is because a tagged packet
(say $P3$ on \fref{fig-expsrgnf}) may be delayed at the second
stage by packets ($P4$ on the figure) that arrived later, but
took over our tagged packet. Also, the service rate may appear
to be reduced by packets ($P1$ on the figure) that had a long
delay in the variable delay component. Formally, we have:
\begin{proposition}[Tightness] The bound in \thref{theo-nfvd} is
tight in the case of an arrival curve of the form
$\alpha(t)=\rho t + \sigma$ and if $l_{\max} \geq 2 l_{min}$.
\end{proposition}\mylabel{prop-psrgtight}
The proof is in \sref{sec-ascproof}.

The proposition shows that the concatenation of non-FIFO PSRG
nodes does not follow the rule as for FIFO nodes, which is
recalled in the proof of \thref{theo-fvd}. Note that if the
condition $l_{\max} \geq 2 l_{min}$ is not satisfied then the
bound in \thref{theo-nfvd} is tight up to a tolerance of $2
\l_{\min}/r$.

\textbf{Comment 2 : } \eref{eq-theonfvd} for the latency is
the minimum of two terms. In the case $\alpha(t)=\rho t +
\sigma$, for $\rho \leq r$, the bound is equal to its former
term, otherwise to its second term. For a general $\alpha$
however, such a simplification does not occur. \nfs{For
example, for $\alpha(t)=\min(pt+M, \rho t + \sigma)$, with
$\rho \leq r \leq p$, the bound is equal to its former term
for large values of $\delta$, and otherwise to the latter
term.}

%
%\textbf{Comment 3 : } We have shown in Comment 1 that
%\thref{theo-nfvd} is tight for $\rho \leq r$. While our initial
%simulations indicate that the bound is almost tight for $\rho >
%r$, a general statement about tightness in this case is for
%further study.
%

\textbf{Comment 3 : }If $\alpha$ is not continuous (thus has
jumps at some values), then it can be shown that
\thref{theo-nfvd} still holds, with \eref{eq-theonfvd}
replaced by
$$
  \begin{array}{rl}
    e'=  & e+ \delta_{\max} + \\
    & \min \{
  \sup_{t \geq 0}[ \frac{\alpha(t+ \delta) }{r}- t] ,  \\
    &\;\;\sup_{0 \leq t \leq \delta}
  [ \frac{\alpha_0(t)+ \alpha_0(\delta)}{r}- t] \}
  \end{array}
$$
with $\alpha_0(u)=\min[\alpha(u+)- l_{\min}, \alpha(u)]$.

\section{Comparison of GR and PSRG} \mylabel{sec-grc}
First, we know that a PSRG node is GR with the same
parameters. This can be used to obtain delay and backlog
bounds for arrival curve constrained input traffic. Compare
however \thref{theo-grgps} to \thref{theo-ascgps}: the PSRG
characterization has a larger latency $e$ than the GR
characterization, so it is better not to use the two
characterizations separately: GR to obtain delay and backlog
bounds, PSRG to obtain delay-from-backlog bounds.

Second, we have shown  that for GR there cannot exist a
delay-from-backlog bound as in \thref{theo-delfrbknf}.

Third, there are similar concatenation results as for PSRG in
\thref{theo-nfgrvd}. The value of latency increase $e'$ for
the composite node is the same for PSRG and GR when the total
incoming rate $\rho$ is less than the scheduler rate $r$.
However, the guarantee expressed by PSRG is stronger than that
of GR. Thus the stronger guarantee of PSRG comes at no cost,
in that case.


%
%The delay bound is the same, whether we have the (old) guaranteed
%rate clock definition, or the (new) packet identity aware
%definition. The results about concatenation with a variable delay
%node are also the same for a guaranteed rate clock model.
\section{Proofs} \mylabel{sec-ascproof}

\subsection{Proof of \lref{lem-psrgmaxplus}} In order to
simplify the notation, we use, locally to this proof, the
 following convention: first, $\vee$ has precedence over $\wedge$; second, we denote $A \vee B$ with
 $A B$. Thus, in this proof only, the expression
 $$
 A  B \wedge C D
 $$ means
  $$
 (A \vee B) \wedge (C \vee D)
 $$
 The reason for this convention is to simplify the use of the distributivity
 of $\vee$ with respect to $\wedge$ \cite{maxPlus}, which
 is here written as
 $$
 A ( B \wedge C) = A B \wedge A C
 $$

Our convention is typical of ``min-max" algebra, where $\min$
takes the role of addition and $\max$ the role of
multiplication. Armed with this facilitating notation, the
proof becomes simple, but lengthy, calculus. In the rest of
the proof we consider some fixed $n$ and drop superscript
$^n$.

For $0 \leq j\leq n-1$, define
$$F_j=f_j + m_{j+1} + ... + m_n
$$
and let $F_n=f_n$. Also let $D_0=d_0 + m_1 + ...+ m_n = m_1 +
...+ m_n$

First note that for all $j \geq 1$:
$$
f_j = \left(a_j + m_j \right) \vee \left[\left(f_{j-1} + m_j
\right) \wedge \left(d_{j-1} + m_j \right)  \right]
$$
then, by adding $m_{j+1}+...+m_n$ to all terms of the right
hand side of this equation, we find
$$
F_j = A_j \vee (F_{j-1} \wedge D_{j-1})
$$
or, with our notation:
$$
F_j = A_j   \left(F_{j-1} \wedge D_{j-1}\right)
$$
and by distributivity:
\begin{equation}\mylabel{eq-maxpluspsrg22}
  F_j = A_j F_{j-1} \wedge A_j D_{j-1}
\end{equation}
Now we show by downwards induction on $j= n-1, ...,0$ that
\begin{eqnarray}
  f_n  =  & \; & A_n A_{n-1} ... A_{j+1}F_j \nonumber \\
  &   \wedge & A_n A_{n-1} ... A_{j+1}D_j \nonumber \\
  &  \wedge & ...\nonumber \\
  &  \wedge &A_n A_{n-1} ... A_{k+1} D_k  \nonumber \\
  &  \wedge & ...\nonumber \\
  &  \wedge & A_n A_{n-1} D_{n-2} \nonumber \\
  &  \wedge & A_n D_{n-1} \mylabel{eq-maxplusleme1}
\end{eqnarray}
where $k$ ranges from $j$ to $n-1$. For $j=n-1$, the property
follows from \eref{eq-maxpluspsrg22} applied for $j=n$. Assume
now that \eref{eq-maxplusleme1} holds for some $j \in \{1,
..., n-1\}$. By \eref{eq-maxpluspsrg22}, we have
\begin{eqnarray*}
\lefteqn{ A_n A_{n-1} ... A_{j+1}F_j = }\\ & & A_n A_{n-1} ...
A_{j+1} (A_j F_{j-1} \wedge A_j D_{j-1} )
\end{eqnarray*}
thus
\begin{eqnarray*}
\lefteqn{ A_n A_{n-1} ... A_{j+1}F_j =} \\
 & & A_n A_{n-1} ... A_{j+1}
A_j F_{j-1} \wedge A_n A_{n-1} ... A_{j+1}  A_j D_{j-1}
\end{eqnarray*}
which, combined with \eref{eq-maxplusleme1} for $j$ shows the
property for $j-1$.

Now we apply \eref{eq-maxplusleme1} for $j=0$ and find
$$
  \begin{array}{l}
   f_n= A_n A_{n-1} ... A_{1}F_0  \wedge A_n A_{n-1} ... A_{1}
  D_0 \wedge ... \\
   \;\; \wedge A_n A_{n-1} D_{n-2} \wedge A_n D_{n-1}
  \end{array}
$$
First note that $F_0=D_0$ so we can remove the first term in
the right hand side of the previous equation. Second, it
follows from $a_1 \geq 0$ that $D_0 \leq A_1$ thus
$$A_n A_{n-1} ... A_{1}  D_0 = A_n A_{n-1} ... A_{1}$$
thus finally
\begin{eqnarray*}
\lefteqn{ f_n = A_n A_{n-1} ... A_{1}\wedge A_n A_{n-1} ...
A_{2}
  D_1 \wedge  ... }\\
  & & \wedge A_n A_{n-1} D_{n-2}
\wedge A_n D_{n-1}
\end{eqnarray*}
which is precisely the required formula.
\subsection{Proof of \thref{theo-psrgmaxplus}} First, assume that the packet scale
rate guarantee holds. Apply \lref{lem-psrgmaxplus} with
$m_n=\frac{l_n}{r}$. It follows that, for $1\leq j\leq n-1$.
$$
f_n \leq \max\left[ A_n^n , A_{n-1}^n , ... , A_{j+1}^n ,
D_j^n \right]
$$
thus $f_n$ is bounded by one of the terms in the right hand
side of the previous equation. If it is the last term, we have
$$
f_n \leq D_j^n= d_j + \frac{l_{j+1} + ... + l_n}{r}
$$
now $d_n \leq f_n + e$, which shows \eref{eq-psrgmaxplus1}.
Otherwise, there is some $k \in \{j+1, ..., n\}$ such that
$$
f_n \leq A_{k}^n =  a_k + \frac{l_k + ... + l_n}{r}
$$ which shows \eref{eq-psrgmaxplus2}.
For $j=0$, \lref{lem-psrgmaxplus} implies that
$$
f_n \leq \max\left[A_n^n , A_{n-1}^n, ... , A_1^n \right]
$$ and the rest follows similarly.

Second, assume conversely that \eref{eq-psrgmaxplus1} or
\eref{eq-psrgmaxplus2} holds. Consider some fixed $n$, and
define $A_j^n, D_j^n, F_j^n$ as in \lref{lem-psrgmaxplus},
with $m_n=\frac{l_n}{r}$. For $1 \leq j\leq n-1$, we have
$$
d_n -e \leq \max\left[ A_n^n , A_{n-1}^n , ... , A_{j+1}^n ,
D_j^n \right]
$$
and for $j=0$:
 $$
d_n -e \leq \max\left[A_n^n , A_{n-1}^n , ... , A_1^n\right]
 $$
thus $d_n -e$ is bounded by the minimum of all right-handsides
in the two equations above, which, by  \lref{lem-psrgmaxplus},
is precisely $f_n$.

\subsection{Proof of
\thref{theo-ascgps}} We first prove that for all $i \geq 0$
\begin{equation}\mylabel{eq-asc-c2}
  f_i \geq g_i - e_1
\end{equation}
where $f_i$ is  defined by \eref{eq-psrg-def}. Indeed, if
\eref{eq-asc-c2} holds, then by \eref{eq-asc-eq6}):
$$
d_{i}  \leq g_i + e_2 \leq f_i + e_1 + e_2 $$ which means that
the scheduler offers the packet scale rate guarantee with rate
$r$ and latency $e= e_1 + e_2$.

Now we prove \eref{eq-asc-c2} by induction. \eref{eq-asc-c2}
trivially holds for $i=0$.

Suppose now that it holds for $i-1$, namely,
$$f_{i-1} \geq g_{i-1} -e_1$$
By hypothesis, \eref{eq-asc-eq6} holds:
$$d_{i-1} \geq g_{i-1} - e_1$$
thus
\begin{equation}\mylabel{eq-asc-c4a}
 \min [f_{i-1}, d_{i-1}]  \geq g_{i-1} - e_1
\end{equation}
Combining this with \eref{eq-psrg-def}, we obtain
\begin{equation}\mylabel{eq-asc-c5a}
  f_i \geq g_{i-1} - e_1 + \frac{L(i)}{R}
\end{equation}
Again from \eref{eq-psrg-def} we have
\begin{equation}\mylabel{eq-asc-c6a}
  \begin{array}{rl}
   f_i \geq & a_i+ \frac{l_i}{r}\\
   \geq &a_i - e_1 + \frac{l_i)}{r}
  \end{array}
\end{equation}
Now by \eref{eq-mr-def}
\begin{equation}\mylabel{eq-asc-c3a}
  g_i \leq \max [ a_i, g_{i-1}] + \frac{l_i}{r}
\end{equation}

 Combining  \eref{eq-asc-c5a}),
\eref{eq-asc-c6a}) and (\ref{eq-asc-c3a}) gives
$$f_i \geq g_i-e_1$$
\qed

\subsection{Proof of \thref{theo-delfrbknf}}
Consider a fixed packet $n$ which is present at time $t$. Call
$a_j$ [resp. $d_j$] the arrival [resp. departure] time of
packet $j$. Thus $a_n\leq t \leq d_n$. Let $\calB$ be the set
of packet numbers that are present in the system at time $t$,
in other words:
$$
\calB = \left\{k \geq 1 | a_k \leq t \leq d_k \right\}
$$
The backlog at time $t$ is $ Q= \sum_{i \in \calB} l_i$. The
absence of FIFO assumption means that $\calB$ is not
necessarily a set of consecutive integers. However, define $j$
as the minimum packet number such that the interval $[j,n]$ is
included in $\calB$. There is such a $j$ because $n \in
\calB$. If $j \geq 2$ then $j-1$ is not in $\calB$ and
$a_{j-1} \leq a_n \leq t$ thus necessarily
\begin{equation}\mylabel{eq-kjas881}
 d_{j-1} < t
\end{equation}
If $j=1$, \eref{eq-kjas881} also holds with our convention
$d_0=0$. Now we apply the alternate characterization of packet
scale rate guarantee (\thref{theo-psrgmaxplus}) to $n$ and
$j-1$. One of the two following equations must hold:
\begin{equation}\mylabel{eq-myxpoik21}
  d_n \leq e + d_{j-1} + \frac{l_j + ... + l_n}{r}
\end{equation}
or there exists a $k \geq j$, $k\leq n$ with
\begin{equation}\mylabel{eq-myxpoik22}
  d_n \leq e + a_{k} + \frac{l_k + ... + l_n}{r}
\end{equation}
Assume that \eref{eq-myxpoik21} holds. Since $[j,n] \subset
\calB$, we have $Q_n \geq l_j + ... + l_n $. By
\eref{eq-kjas881} and \eref{eq-myxpoik21} it follows that
$$
 d_n \leq e + t + \frac{Q}{r}
$$ which shows the result in this case. Otherwise, use
\eref{eq-myxpoik22}; we have $Q \geq l_k + ... + l_n$ and $a_k
\leq t$ thus
$$
 d_n \leq e + t + \frac{Q}{r}
$$

\subsection{Proof of \thref{theo-ascconc}}
 Consider some fixed but arbitrary times $s\leq t$ and let $u
 \in [s,t]$. We have
 $$
 R_1(u) \geq \left[R_1(s) + \tilde{\beta}(u-s) \right]
 \wedge
 \inf_{v \in [s,u]} \left[ R(v) + \beta_1(u-v)\right]
 $$
 thus
 $$
  \begin{array}{rl}
   R_1(u)  + \beta_2(t-u) &\geq
    \left[R_1(s) + \tilde{\beta}(u-s)+\beta_2(t-u)
 \right]
 \wedge \\
 &\inf_{v \in [s,u]} \left[ R(v) + \beta_1(u-v)
 +\beta_2(t-u)\right]
  \end{array}
 $$
 and
 \begin{eqnarray*}
  \lefteqn{\inf_{u \in [s,t]} \left[R_1(u) + \beta_2(t-u)\right]
  \geq  } \\
  & & \inf_{u \in [s,t]}
 \left[R_1(s) + \tilde{\beta}(u-s)+\beta_2(t-u) \right]\\
  & &
 \wedge
 \inf_{u \in [s,t], v \in [s,u]}
 \left[ R(v) + \beta_1(u-v) +\beta_2(t-u)\right]
 \end{eqnarray*}
 After re-arranging the infima, we find
 \begin{eqnarray*}
  \lefteqn{
 \inf_{u \in [s,t]} \left[R_1(u) + \beta_2(t-u)\right]
 \geq } \\
 & &   \left(R_1(s) +
  \inf_{u \in [s,t]} \left[\tilde{\beta}(u-s)+\beta_2(t-u)
  \right]\right)
 \wedge \\
& & \inf_{v \in [s,t]}\left( R(v) + \inf_{u \in [v,t]} \left[
\beta_1(u-v) +\beta_2(t-u)\right]\right)
\end{eqnarray*}
which can be rewritten as
 \begin{eqnarray*}
  \lefteqn{\inf_{u \in [s,t]} \left[R_1(u) + \beta_2(t-u)\right]
  \geq}\\
 & &  \left( R_1(s) + (\tilde{\beta_1} \mpc \beta_2) (t-s)
 \right) \wedge\\
 & & \inf_{v \in [s,t]}\left[R(v) + \beta(t-v)\right]
\end{eqnarray*}
Now by hypothesis we have
$$
R^*(t) \geq \left( R^*(s) + \tilde{\beta_2}(t-s)\right) \wedge
\inf_{u \in [s,t]}[R(u)+\beta_2(t-u)]
$$
Combining the two gives
\begin{eqnarray*}
  \lefteqn{
R^*(t) \geq }\\
 & & \left( R^*(s) + \tilde{\beta_2}(t-s)\right) \wedge
\left( R_1(s) + (\tilde{\beta_1} \mpc \beta_2) (t-s) \right)\\
 & & \wedge
 \inf_{v \in [s,t]}\left[R(v) + \beta(t-v)\right]
\end{eqnarray*}
Now $R_1(s) \geq R^*(s)$ thus
\begin{eqnarray*}
  \lefteqn{
R^*(t) \geq }\\ & & \left( R^*(s) +
\tilde{\beta_2}(t-s)\right) \wedge\left( R^*(s) +
(\tilde{\beta_1} \mpc \beta_2) (t-s)\right)\\
 & & \wedge
 \inf_{v \in [s,t]}\left[R(v) + \beta(t-v)\right]
\end{eqnarray*}
 \qed

\subsection{Proof of \thref{theo-ascdelay}}
If the virtual delay at time $t$ is larger than $t+ \tau$ for
some $\tau \geq 0$, then we must have
\begin{equation}\mylabel{eq-ascdel01}
  R^*(t+ \tau) < R(t)
\end{equation}

By hypothesis
\begin{equation}\mylabel{eq-ascdel1}
 R^*(t + \tau) \geq \left(R^*(t) + \tilde{\beta} (\tau) \right)
\wedge \inf_{[u \in[t, t + \tau]} \left[ R(u) + \beta(t+
\tau-u)\right]
\end{equation}
now for $u \in[t, t + \tau]$
$$
 R(u) + \beta(t+\tau-u) \geq R(t) + \beta(0) \geq R^*(t + \tau)
$$
thus \eref{eq-ascdel1} implies that
$$
R^*(t + \tau) \geq R^*(t) + \tilde{\beta} (\tau)
$$
combining with \eref{eq-ascdel01} gives
$$
Q(t)=R(t)-R^*(t) \geq \tilde{\beta} (\tau)
$$
thus the virtual delay is bounded by $\sup\{\tau:
\tilde{\beta}(\tau) > Q(t) \}$ which is equal to
$\tilde{\beta}^{-1}(Q(t))$. \qed


\subsection{Proof of \thref{theo-ascpack}}
 \pr
 Let $s \leq t$. By hypothesis we have
 $$
R^*(t) \geq \left( R^*(s) + \tilde{\beta}(t-s)\right)  \wedge
\inf_{u \in [s,t]}[R(u)+\beta(t-u)]
 $$
 We do the proof when the $\inf$ in the above formula is a
 minimum, and leave it to the alert reader to extend it to the
 general case. Thus assume that for some $u_0 \in [s,t]$:
 $$
\inf_{u \in [s,t]}[R(u)+\beta(t-u)] = R(u_0) +  \beta(t-u_0)
 $$
it follows that either
$$ R^*(t) - R^*(s) \geq \tilde{\beta}(t-s)
$$
or
$$
R^*(t) \geq R(u_0) + \beta(t-u_0)
$$
Consider the former case. We have $R'(t) \geq R^*(t) -
l_{\max}$ and $R'(s) \leq R^*(s)$ thus
$$ R'(t) \geq R^*(t) - l_{\max} \geq R'(s) + \tilde{\beta}(t-s) - l_{\max}
$$
Now also obviously $R'(t) \geq R'(s)$, thus finally
$$ R'(t) \geq R'(s) +\max[0,  \tilde{\beta}(t-s) -
l_{\max}]=R'(s)+\tilde{\beta}'(t-s)
$$
Consider now the latter case. A similar reasoning shows that
$$ R'(t) \geq R(u_0) + \beta(t-u_0) - l_{\max}
$$
but also
$$
R^*(t) \geq R(u_0)
$$
now the input is $L$-packetized. Thus
$$
R'(t)=P^L(R^*(t)) \geq P^L(R(u_0))=R(u_0)
$$
from which we conclude that $R'(t) \geq R(u_0) +
\beta'(t-u_0)$.

 Combining the two cases provides the required adaptive
guarantee.
 \qed

%%%ici

\subsection{Proof of \thref{theo-ascrep}} The first part uses
the min-max expansion of packet scale rate guarantee in
\lref{lem-psrgmaxplus}. The second part relies on the
reduction to the minimum rate server.

We use the same notation as in \dref{def-psrg}.
 $L(i)=\sum_{j=1}^i l_j$ is the cumulative packet length.

 \paragraph{Item 1: } Define the sequence of times $f_k$ by
 \eref{eq-psrg-def}. Consider now some fixed but arbitrary packet index $n\geq
 1$.
 By the FIFO assumption, it is sufficient to show that
\begin{equation}\mylabel{eq-ascpr9832}
  R^*(t) \geq L(n)
\end{equation}
 with $t=f_n+e$. By \lref{lem-psrgmaxplus}, there is some
 index $1 \leq j \leq n$ such that
\begin{equation}\mylabel{eq-maxplusasc1}
 f_n=
 \left( s +\frac{ L(n) -L(j-1)}{r} \right)
  \bigvee
      \max_{k=j+1}^i \left(a_k + \frac{L(n)-L(k-1)}{r}
       \right)
\end{equation}
with $$s=a_j \vee d_{j-1}$$ and with the convention that
$d_0=0$.

Let us now apply the definition of an adaptive guarantee to
the time interval $[s,t]$:
$$
R^*(t) \geq A \wedge B
$$
with
 $$
 A:=R^*(s) + r(t-s-e)^+ \mand
B:=\inf_{u \in [s,t]} B(u)
 $$
 where
 $$B(u):=\left(
 R(u) + r(t-u-e)^+
 \right)
 $$
 Firstly, since $s \geq d_{j-1}$, we have $R^*(s) \geq L(j-1)$. By
 \eref{eq-maxplusasc1}, $f_n \geq s +\frac{ L(n) -L(j-1)}{r}$
 thus $t \geq s +\frac{ L(n) -L(j-1)}{r} + e$. It follows that
 $$
 t-s-e \geq \frac{ L(n) -L(j-1)}{r}
 $$
 and thus $A\geq L(n)$.

 Secondly, we show that $B\geq L(n)$ as well. Consider some $u \in
 [s,t]$. If $u\geq a_n$ then $R(u) \geq L(n)$ thus $B(u) \geq
 L(n)$. Otherwise, $u<a_n$; since $s\geq a_{j}$, it follows that
 $a_{k-1} \leq u <a_{k}$ for some $k \in\{j+1, ..., n\}$ and
 $R(u) = L(k-1)$. By \eref{eq-maxplusasc1},
 $$
 f_n \geq a_k + \frac{L(n)-L(k-1)}{r}
 $$
 thus
 $$
 t-u-e \geq \frac{L(n)-L(k-1)}{r}
 $$
 It follows that $B(u) \geq L(n)$ also in that case. Thus we have
 shown that $B\geq L(n)$.

 Combining the two shows that $R^*(t) \geq L(n)$ as required.

 \paragraph{Item 2: }
 We use a reduction to a minimum rate server as follows. Let
 $d'_i:=\min(d_i, f_i)$ for $i\geq 0$. By \eref{eq-psrg-def} we
 have
\begin{equation}\mylabel{eq-ascpr98132}
  a_i \leq d'_i \leq \max(a_i, d'_{i-1}) + \frac{l_i}{r}
\end{equation}
and
\begin{equation}\mylabel{eq-ascpr98133}
  d'_i \leq d_i \leq d'_i + e
\end{equation}
The idea of the proof is now to interpret $d'_i$ as the output
time for packet $i$ out of a virtual minimum rate server.

Construct a virtual node $\calR$ as follows. The input is the
original input $R(t)$. The output is defined as follows. The
number of bits of packet $i$ that are output up to time $t$ is
$\psi_i(t)$, defined by
 $$\bracket{
 \mif t > d'_i \mthen \psi_i(t)=L(i) \\
 \melse \mif a_i< t \leq d'_i \mthen \psi_i(t)=\left[L(i) -r (d'_i -t)\right]^+
 \\
 \melse \psi_i(t)=0
}
$$
so that the total output of $\calR$ is $R_1(t)=\sum_{i\geq
1}\psi_i(t)$.

The start time for packet $i$ is thus
$\max[a_i,d'_i-\frac{l_i}{r}]$ and the finish time is $d'_i$.
Thus $\calR$ is causal (but not necessarily FIFO, even if the
original system would be FIFO).  We now show that during any
busy period, $\calR$ has an output rate at least equal to $r$.

Let $t$ be during a busy period. Consider now some time $t$
during a busy period. There must exist some $i$ such that $a_i
\leq t \leq d'_i$. Let $i$ be the smallest index such that
this is true. If $a_i \geq d'_{i-1}$ then by
\eref{eq-ascpr98132} $d'_i-t \leq \frac{l_i}{r}$ and thus
$\psi'_r(t) =r$ where $\psi'_r$ is the derivative of $\psi_i$
to the right. Thus the service rate at time $t$ is at least
$r$.

Otherwise, $ a_i < d'_{i-1}$. Necessarily (because we number
packets in order of increasing $a_i$'s -- this is not a FIFO
assumption) $a_{i-1} \leq a_i$; since $i$ is the smallest
index such that $a_i \leq t < d'_i$, we must have $t\geq
d'_{i-1}$. But then  $d'_i-t \leq \frac{l_i}{r}$ and the
service rate at time $t$ is at least $r$. Thus, node $\calR$
offers the strict service curve $\lambda_r$ and
\begin{equation}\mylabel{eq-ascpr287}
  R \rightarrow (\lambda_r) \rightarrow R_1
\end{equation}



Now define node $\calD$. Let $\delta(i):=d_{i}-d'_i$, so that
$0 \leq \delta(i) \leq E$.  The input of $\calD$ is the output
of $\calR$. The output is as follows; let a bit of packet $i$
arrive at time $t$; we have $t\leq d'_{i}\leq d_{i}$. The bit
is output at time $t'=\max[\min[d_{i-1}, d_{i}],t+\delta_i]$.
Thus all bits of packet $i$ are delayed in $\calD$ by at most
$\delta(i)$, and if $d_{i-1}<d_{i}$ they depart after $d_{i}$.
It follows that the last bit of packet $i$ leaves $\calD$ at
time $d_{i}$. Also, since $t'\geq t$, $\calD$ is causal.
Lastly, if the original system is FIFO, then $d_{i-1}<d_{i}$,
all bits of packet $i$ depart after $d_{i-1}$ and thus the
concatenation of $\calR$ and $\calD$ is FIFO. Note that
$\calR$ is not necessarily FIFO, even if the original system
is FIFO.


The aggregate output of $\calD$ is
$$
R_2(t)\geq \sum_{i\geq 1}\psi_i(t-\delta(i))\geq R_1(t-e)
$$

thus the virtual delay for $\calD$ is bounded by $e$ and
\begin{equation}\mylabel{eq-ascpr288}
  R_1 \rightarrow (\delta_e) \rightarrow R_2
\end{equation}
Now we plug the output of $\calD$ into an $L$-packetizer.
Since the last bit of packet $i$ leaves $\calD$ at time $d_i$,
the final output is $R^*$. Now it follows from
\eref{eq-ascpr287}, \eref{eq-ascpr288} and
\thref{theo-ascconc} that
 $$R \rightarrow (\lambda_r \mpc \delta_e) \rightarrow R_2
 $$
\qed

\subsection{Proof of \thref{theo-nfvd}}

We first introduce some notation (see \fref{fig-expsrgnf}).
Call $a_n \geq 0$ the arrival times for the fresh input.
Packets are numbered in order of arrival, so $0 \leq a_1 \leq
a_2 \leq ...$. Let $l_n$ be the size of packet $n$. Call $b_n$
the arrival time for packet $n$ at the second component; $b_n$
is not assumed to be monotonic with $n$, but for all $n$:
\begin{equation}\mylabel{eq-Elkxj872}
 a_n \leq b_n \leq a_n + \delta
\end{equation}
Also call $d_n$ the departure time of packet $n$ from the
second component. By convention, $a_0=d_0=0$.

Then, define
$$e_1=e+ \delta_{\max} +
  \sup_{t \geq 0}[ \frac{\alpha(t+ \delta) - l_{\min}}{r}- t]
$$
and
$$
e_2
=
e+ \delta_{\max} +
 \sup_{0 \leq t \leq \delta}
  [ \frac{\alpha(t)+ \alpha(\delta) -
l_{\min}}{r}- t]
$$
so that $e'= \min[e_1, e_2]$. It is sufficient to show that
the combined node separately satisfies the packet scale rate
guarantee with rate $r$ and with latencies $e_1$ and $e_2$. To
see why, define $f_n$ by \eref{eq-psrg-def}. If $d_n - f_n
\leq e_1$ and $d_n - f_n \leq e_2$ for all $n$, then $d_n -
f_n \leq e'$.

\textbf{Part 1: }We show that the combined node satisfies the
packet scale rate guarantee with rate $r$ and latency $e_1$.

An arrival curve for the input traffic to the second component
is $\alpha_2(t)=\alpha(t+ \delta)$. Thus, by
\thref{theo-grcdelaybound}, $d_n \leq b_n + D_2$, with
$$
d_n \leq b_n + e +  \sup_{t \geq 0}[ \frac{\alpha(t+
\delta)}{r}- t]
$$
By \eref{eq-Elkxj872}:
$$d_n -a_n \leq e+ \delta_{\max}+ \sup_{t \geq 0}[ \frac{\alpha(t+ \delta)}{r}- t]
$$
Now we apply \lref{lem-vdpsrg} which ends the proof for this
part.

\textbf{Part 2: }We show that the combined node satisfies the
packet scale rate guarantee with rate $r$ and latency $e_2$.

Let $\delta_{\min}= \delta_{\max}-\delta$ the constant part of
the delay. We do the proof for $\delta_{\min}=0$ since we can
eliminate the constant delay by observing packets
$\delta_{\min}$ time units after their arrival, and adding
$\delta_{\min}$ to the overall delay.

\textbf{Part 2A:}

We assume in this part that there cannot be two arrivals at
the same instant; in part 2B, we will show how to relax this
assumption.
%also assume in a
%first pthat there cannot be two arrivals at the same instant --
%since we use a continuous time model, this is not really a
%limitation.

For a time interval $(s,t]$ (resp. $[s,t]$), define $A(s,t]$
as the total number of bits at the fresh input during the
interval $(s,t]$ (resp. $[s,t]$); similarly, define $B(s,t]$
and $B[s,t]$ at the input of the second node. We have the
following relations:
$$
A(s,t] = \sum_{n\geq 1} 1_{\{ s<a_n \leq t]\}} l_n \;,\;
 A[s,t] =
\sum_{n\geq 1} 1_{\{ s \leq a_n \leq t]\}} l_n
$$
$$
B(s,t] = \sum_{n\geq 1} 1_{\{ s<b_n \leq t]\}} l_n \;,\;
 B[s,t] =
\sum_{n\geq 1} 1_{\{ s \leq b_n \leq t]\}} l_n
$$
Note that
$$A(a_j,a_n]=\sum_{i=j+1}^n l_i$$
but, by lack of FIFO assumption, there is no such relation for
$B$.


By definition of an arrival curve, we have $A(s,t] \leq
\alpha(t-s)$.
 \nfs{
\begin{lemma}
\mylabel{lem-arrcur0}For $0\leq t, u$, $A[t,t+u) \leq
\alpha(u)$
\end{lemma}
\begin{preuve}
 $A[t,t+u)=\lim_{n \rightarrow
+\infty}A(t-\frac{1}{n}, t+u-\frac{1}{n}]$ and
$A(t-\frac{1}{n}, t+u-\frac{1}{n}]\leq \alpha(u)$.\end{preuve}
}

\begin{lemma}
\mylabel{lem-arrcur}For $0\leq t, u$ and $0\leq v \leq t$, if
there is an arrival at $t$, then $A(t,t+u] \leq \alpha(u) -
l_{\min}$ and $A[t-v,t) \leq \alpha(v) - l_{\min}$
\end{lemma}
\begin{preuve} First note that $A[t,t+u]\leq \inf_{\epsilon
> 0} A(t-\epsilon, t+u]\leq \inf_{\epsilon > 0}
\alpha(u+\epsilon)=\alpha(u)$  (the last equality is because
$\alpha$ is continuous).

Second, let $l$ be the packet length for one packet arriving
at time $t$. Then
 $A(t,t+u]+l\leq A[t, t+u]\leq \alpha(u)$ thus $A(t,t+u]\leq \alpha(u)-l \leq \alpha(u) -
l_{\min}$. The same reasoning shows the second inequality in
the lemma. \end{preuve}

Now we apply \thref{theo-psrgmaxplus}. Consider some fixed
packets numbers $0\leq j<n$. We have to show that one of the
following holds:
\begin{equation}\mylabel{eq-psrgmaxplus1p}
 d_n \leq e_2 + d_j + \frac{A(a_j, a_n]}{r}
 \end{equation}
or there is some $k \in \{j+1, ..., n\}$ such that
\begin{equation}\mylabel{eq-psrgmaxplus2p}
 d_n \leq e_2 + a_k + \frac{A[a_k, a_n]}{r}
\end{equation}

\textbf{(Case 1:)} Assume that $b_j \geq b_n$. Since the
second node is FIFO, we have
$$
d_n \leq d_j
$$
and thus \eref{eq-psrgmaxplus1p} trivially holds.

\textbf{(Case 2:)} Assume that $b_j < b_n$. By
\thref{theo-psrgmaxplus} applied to the second node, we have
\begin{equation}\mylabel{eq-jpoiuwejfhn81}
  d_n \leq e+ d_j + \frac{1}{r} B(b_j,b_n]
\end{equation}
or there exists some $k$ such that $b_j \leq b_k \leq b_n$ and
\begin{equation}\mylabel{eq-jpoiuwejfhn82}
  d_n \leq e+ b_k + \frac{1}{r} B[b_k,b_n]
\end{equation}

\textbf{(Case 2a: )} Assume that \eref{eq-jpoiuwejfhn81}
holds. By \eref{eq-Elkxj872}, any packet that arrives at node
2 in the interval $(b_j,b_n]$ must have arrived at node $1$ in
the interval $(a_j - \delta, b_n] \subset (a_j - \delta,
a_n+\delta]$. Thus
$$
  \begin{array}{l}
    B(b_j,b_n] \leq A(a_j - \delta, a_n+\delta] \\
     \leq A(a_j,a_n] + A[a_j -\delta, a_j)+
    A(a_n, a_n + \delta]\\
    \leq
A(a_j,a_n]+ 2 \alpha(\delta) -2 l_{\min}
  \end{array}
$$
the last part being due to \lref{lem-arrcur}. Thus
$$
\begin{array}{l}
 d_n \leq e + \delta + \frac{\alpha(\delta)}{r}- \delta +
  \frac{\alpha(\delta)}{r} +
 d_j \\
  \;\; \;\; \;\; \;\;+ \frac{1}{r} A(a_j,a_n] -2 l_{\min} \\
 \leq e_2 +
 d_j + \frac{1}{r} A(a_j,a_n]
 \end{array}
$$
which shows \eref{eq-psrgmaxplus1p}.

\textbf{(Case 2b: )} Assume that \eref{eq-jpoiuwejfhn82}
holds. Note that we do not know the order of $k$ with respect
to $j$ and $n$. However, in all cases, by \eref{eq-Elkxj872}:
\begin{equation}\mylabel{eq-dsakl8236}
  B[b_k, b_n] \leq A[b_k-\delta, a_n + \delta]
\end{equation}

We further distinguish three cases.

\textbf{(Case 2b1: )}   $ k \leq j $:

Define
\begin{equation}\mylabel{eq-defuaiw8927}
  u=  a_j -b_k +\delta
\end{equation}
By hypothesis, $a_k\leq a_j$ and $b_k -\delta \leq  a_k$ so
that $u \geq 0$. Note also that $a_j\leq b_j
 \leq b_k$ and thus $u \leq \delta$.

 By \eref{eq-dsakl8236}:
 $$
B[b_k, b_n] \leq A[b_k - \delta, a_j) + A[a_j, a_n] + A(a_n,
a_n + \delta ]
 $$
 Now by
\lref{lem-arrcur} $A(a_n, a_n + \delta ] \leq \alpha(\delta)$
and $A[b_k - \delta, a_j) \leq \alpha(u)- l_{\min}$. Thus
$$
B[b_k, b_n] \leq A[a_j, a_n]+ \alpha(u) + \alpha(\delta) - 2
l_{\min}
$$
Combine with \eref{eq-jpoiuwejfhn82}, \eref{eq-defuaiw8927}
and obtain
$$
d_n  \leq a_j + \frac{A[a_j, a_n]}{r} + e_2
$$
which shows that \eref{eq-psrgmaxplus2p} holds.

\textbf{(Case 2b2: )} $j < k  \leq n$:

Define $u= \delta -b_k + a_k$. By \eref{eq-dsakl8236}
$$B[b_k,b_n] \leq A[a_k,a_n]+ \alpha(u) +\alpha(\delta) -2 l_{\min}
$$
which shows that
$$
d_n \leq e_2 + a_k +\frac{1}{r}A[a_k, a_n]
$$

\textbf{(Case 2b3: )} $k >n$:

Define $u=\delta - b_k + a_n$. By $b_k \leq b_n$ and $b_n \leq
a_n + \delta$ we have $u \geq 0$. By $b_k \geq a_k$ and $a_k
\geq a_n$ we have $u \leq \delta$.

Now by \eref{eq-jpoiuwejfhn82}:
$$
 d_n \leq e+ b_k + \frac{1}{r} B[b_k, b_n]
 = e + \delta - u + a_n +  \frac{1}{r} B[b_k, b_n]
$$
By \eref{eq-dsakl8236}
$$
  \begin{array}{l}
    B[b_k, b_n] \leq A[a_n -u , a_n +
 \delta]\\
   =A[a_n -u , a_n) + l_n + A(a_n,a_n +
 \delta]\\
 \leq \alpha(u) + l_n + \alpha(\delta) -2 l_{\min}
  \end{array}
$$
which shows that
$$
d_n \leq e_2 + a_n + \frac{l_n}{r}
$$

\textbf{Part 2B:}
 %\iffullText {
Now it remains to handle the case where packet arrivals at
either component may be simultaneous. We assume that packets
are ordered at component 2 in order of arrival, with some
unspecified mechanism for breaking ties. Packets also have a
label which is their order of arrival at the first component;
we call $(k)$ the label of the $k$th packet in this order (see
\fref{fig-expsrgnf} for an illustration).

Call $\calS$ the original system. Fix some arbitrary integer
$N$. Consider the truncated system $\calS^N$ that is derived
from the original system by ignoring all packets that arrive
at the first component after time $a_N + \delta$. Call $a^N_n,
b^N_n, d^N_n, f^N_n$ the values of arrival, departure, and
virtual finish times in the truncated system (virtual finish
times are defined by \eref{eq-psrg-def}). Packets with numbers
$\leq N$ are not affected by our truncation, thus $a^N_n=a_n,
b^N_n=b_n, d^N_n=d_n, f^N_n=f_n$ for $n \leq N$. Now the
number of arrival events at either component 1 or 2 in the
truncated system is finite; thus we can find a positive number
$\eta$ which separates arrival events. Formally: for any $m,n
\leq N$:
$$a_m=a_n \mor |a_m -a_n|> \eta$$
and
$$b_m=b_n \mor |b_m -b_n|> \eta$$
Let $\epsilon < \frac{\eta}{2}$. We define a new system,
called $\calS^{N,\epsilon}$, which is derived from $\calS^N$
as follows.
\begin{itemize}
  \item We can find some sequence of numbers $x_n \in (0, \epsilon)$,
  $n\leq N$ such that:  (1) they are all distinct; (2) if the
  packet
  labeled $m$ is ordered before the packet labeled  $n$ in the order of arrival
  at the second
  component, then $x_m < x_n$. Building such a sequence is easy,
  and any sequence satisfying (1) and (2) will do. For example, take
  $x_n =\frac{k}{N+1}
  \epsilon$ where $k$ is the order of arrival of packet $n$ (in other words, $(k)=n$).
  \item Define the new arrival and departure times by
  $$
  a_n^{\epsilon}= a_n + x_n \; , \;
  b_n^{\epsilon}= b_n + x_n \; , \;
  d_n^{\epsilon}= d_n + x_n
  $$
It follows from our construction that all $a_n^{\epsilon}$ are
distinct for $n\leq N$,
  and the same holds for $b_n^{\epsilon}$. Also, the arrival order of
  packets at the second component is the same as in the original system.
  \end{itemize}
Thus we have built a new system $\calS^{N,\epsilon}$ where all
arrivals times are distinct, the order of packets at the
second component is the same as in $\calS^{N}$, arrival and
departure times are no earlier than in $\calS^{N}$, and differ
by at most $\epsilon$.

For $k \leq N$, call $F_{(k)}^{\epsilon}$ the virtual finish
times at the second component. By definition:
$$
\bracket{
 F_{(0)}^{\epsilon} =0\\
 F_{(k)}^{\epsilon}= \max \left[
 b_{(k)}^{\epsilon}, \min \left(d_{(k-1)}^{\epsilon}, F_{(k-1)}^{\epsilon} \right)\right]
 \\ \;\;+ \frac{l_{(k)}}{r} \mfor
 k \geq 1
 }
$$
and a similar definition holds for $F_{(k)}$ by dropping
$\epsilon$. It follows by induction that
$$
F_{(k)}^{\epsilon} \geq  F_{(k)}
$$
thus
$$ d_{(k)}^{\epsilon} \leq d_k +  \epsilon  \leq e+ F_{(k)}
\leq e+ F_{(k)}^{\epsilon} +
  \epsilon $$
Similarly, $b_{k}^{\epsilon} \leq a_k^{\epsilon} + \delta$.
This shows that $\calS^{N,\epsilon}$ satisfies the assumptions
of the theorem, with $e$ replaced by $e+
  \epsilon$

Thus the conclusion of Part 2A holds for $\calS^{N,\epsilon}$.
Define now $f_n^{\epsilon}$ by \eref{eq-psrg-def} applied to
$a_n^{\epsilon}$ and $d_n^{\epsilon}$. We have:
\begin{equation}\mylabel{eq-kjas823}
 d_n^{\epsilon} \leq f_n^{\epsilon} + e_2 + \epsilon
\end{equation}
It also follows by induction that
$$
f_n^{\epsilon} \leq f_n + \epsilon
$$
Now $d_n\leq d_n^{\epsilon}$ thus
$$d_n - f_n \leq d_n^{\epsilon} - f_n^{\epsilon} + \epsilon
$$
Combining with \eref{eq-kjas823} gives:
$$
d_n - f_n \leq e_2 + 2\epsilon
$$
Now $\epsilon$ can be arbitrarily small, thus we have shown
that for all $n\leq N$:
$$
d_n - f_n \leq e_2
$$
Since $N$ is arbitrary, the above is true for all $n$.


\subsection{Proof of \pref{prop-psrgtight}}
\begin{preuve}
Case $\rho \leq r$: Assume that the source is greedy from time
0, with packet $n=1$, of size $l_1=l_{\min}$, $a_1=0$,
$b_1=\delta_{\max}$. Assume all subsequent packets have a
delay in the first component equal to $\delta_{\max}-\delta$.
We can build an example where packet $1$ is overtaken by
packets $n=2, ...,n_1$ that arrive in the interval $(0,
\delta]$, with $l_2 + ... + l_{n_1}= \rho \delta + \sigma -
l_1$. Assume that packet $1$ undergoes the maximum delay
allowed by PSRG at the second component. It follows after some
algebra that $d_1=e+ \delta_{\max}+\frac{\rho \delta +
\sigma}{r}$. Now $f_1=\frac{ l_{\min}}{r}$ thus $d_1-f_1=e'$
and the characterization is tight.

Case $\rho > r$: We build a worst case scenario as follows. We
let $e=0$, without loss of generality (add a delay element to
this example and obtain the general case). The principle is to
first build a maximum size burst which is overtaken by a
tagged packet $j$. Later, a tagged packet $n$ is overtaken by
a second maximum size burst. Between the two, packets arrive
at a data rate $r$; the second burst is possible because $r<
\rho$ and $a_n-a_j$ is long enough. Details are in
\fref{fig-wc} and \tref{tab-wc}. We find finally $d_n-f_n=
2(\rho \delta +\sigma - l_{\min})/r$ which shows that the
bound is achieved.
\begin{figure}[!htbp]
 \insfig{expsrgnf2}{0.5}
  \mycaption{Worst-case example for \thref{theo-nfvd}.
  All packets have $0$ delay through the first component except packets $1 ... j-1$ and $n$.}
  \mylabel{fig-wc}
\end{figure}
{\small

\begin{table*}[!htbp]
 \hspace{-1cm}
   $$\begin{array}{|c|c|c|c|c|c|}
    % after \\: \hline or \cline{col1-col2} \cline{col3-col4} ...
 \hline
  k & a_k & l_k & b_k & f_k & d_k \\
 \hline
 \hline
 1 & 0 & \sigma -  l_{\min} & \delta^+ & \mbox{not relevant} & d_j+l_1/r\\
 2 & l_2/\rho & l_{2} & \delta^+ & \mbox{not relevant} & d_j+(l_1+l_2)/r\\
 ... & ... & ... & ... &... & ... \\
 j-1 &  \delta & l_{j-1} & \delta^+ & \mbox{not relevant} & d_j +A \\
 \hline
 j & \delta & l_{\min} & \delta & \geq \delta +l_{\min}/r  & \delta +l_{\min}/r \\
 \hline
 j+1& \delta+l_{\min}/r & l_{\min} & a_{j+1} & \delta+2 l_{\min}/r
 & f_{j+1} + A \\
 ... & ... & ... & ... &... & ... \\
 n-1 & \delta+ (n-j-1)l_{\min}/r & l_{\min} & a_{n-1} & \delta+(n-j) l_{\min}/r & f_{n-1} + A \\
 \hline
 n & \delta+ (n-j)l_{\min}/r & l_{\min} & a_n + \delta & \delta+(n-j+1) l_{\min}/r & f_n + 2 A \\
 \hline
 n+1 & a_n^+ & \sigma - l_{\min} & a_{n+1} & \mbox{not relevant} & f_{n-1}+A+(\sigma - l_{\min} )/r\\
 n+2 & a_n+a_2 & l_{2} & a_{n+2} & \mbox{not relevant} & f_{n-1}+A+(\sigma - l_{\min}+l_2 )/r\\
 ... & ... & ... & ... &... &...  \\
 n+j-1 & (a_n + \delta)^- & l_{j-1} & (a_n +\delta)^- & \mbox{not relevant} & f_{n-1}+2A\\
 \hline
 \multicolumn{6}{|l|}{
\mbox{Notes: } A=(\rho \delta +\sigma - l_{\min})/r }\\
 \multicolumn{6}{|l|}{
%\nu \mbox{ is defined by } \nu = (\lfloor\sigma + \rho \delta -
%l_{\min}) \bmod l_{\min}. l_1 \mbox{is feasible because
%}l_{\max}\geq 2 l_{\min}
  (j, l_2,...,l_{j-1}) \mbox{ is a solution to } l_2 + ... + l_{j-1}= \rho \delta,
  \mbox{ sc }  l_2, ..., l_{j-1}\in [l_{\min}, l_{\max}].
  \mbox{ For example, let }j=2+\lfloor \frac{\rho
  \delta}{l_{\min}}\rfloor,}\\
  \multicolumn{6}{|l|}{l_2=\rho \delta-(j-3)l_{\min},
  l_3=...=l_{j-1}=l_{\min}. \mbox{ We have }l_2\leq l_{\max} \mbox{ because }l_{\max}\geq 2 l_{\min}}\\
 \hline
  \end{array}
  $$
  \mycaption{Details for \fref{fig-wc}. Assume for this table that
  $\sigma - l_{\min}\leq  l_{\max}$, otherwise replace packets $1$ and $n+1$ by a number
  of smaller packets arriving in batch.}
  \mylabel{tab-wc}
\end{table*}
}
\end{preuve}


\section{Bibliographic Notes}

The concept of adaptive service curve was introduced in Okino's
dissertation in \cite{oki98} and was published by Agrawal, Cruz,
Okino and Rajan in \cite{cruzADAP}, which contains most results in
\sref{sec-ascprop}, as well as an application to a window flow
control problem that extends \sref{sec:windowflowcontrol1} on
\pgref{sec:windowflowcontrol1}. They call $\tilde{\beta}$ an
``adaptive service curve" and $\beta$ a ``partial service curve".

The packet scale rate guarantee was first defined
independently of adaptive service guarantees in
\cite{lebinfocom2001}. It serves as a basis for the definition
of the Expedited Forwarding capability of the Internet.

\section{Exercises}
%% LaTeX2e file `./temp/L23-1'
%% generated by the `filecontents' environment
%% from source `netCalBookv4' on 2019/02/19.
%%
\begin{problem}
Assume that $R \rightarrow (\tilde{\beta}, \beta) \rightarrow
R^*$.
\begin{enumerate}
  \item Show that the node offers to the flow a strict service
curve equal to $\tilde{\beta}\mpc \overline{\beta}$, where
$\overline{\beta}$ is the sub-additive closure of $\beta$.
  \item If $\tilde{\beta}=\beta$ is a rate-latency function, what is the
  value obtained for the strict service curve~?
\end{enumerate}
\end{problem}

%% LaTeX2e file `./temp/L23-2'
%% generated by the `filecontents' environment
%% from source `netCalBookv4' on 2019/02/19.
%%
\begin{problem}
 Consider a system with
input $R$ and output $R^*$. We call ``input flow restarted at time
$t$" the flow $R_t$ defined for $u \geq 0$ by
$$
R_t(u)=R(t+u)-R^*(t) = R(t,u]+Q(t)
$$
where $Q(t):=R(t)-R^*(t)$ is the backlog at time $t$. Similarly,
let the``output flow restarted at time $t$" be the flow $R^*_t$
defined for $u \geq 0$ by
$$R^*_t(u)=R^*(t+u)-R^*(t)$$
Assume that the node guarantees a service curve $\beta$ to all
couples of input, output flows $(R_t, R^*_t)$. Show that $R
\rightarrow (\beta) \rightarrow R^*$.
\end{problem}


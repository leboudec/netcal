\paragraph{Example: VBR flow with rate-latency service curve}
Consider a VBR flow, defined by T-SPEC $(M, p, r, b)$. This means
that the flow has $\alpha(t) = \min(M+pt, rt+b)$ as an arrival
curve (Section~\ref{con-flows}). Assume that the flow is served in
one node that guarantees a service curve equal to the rate-latency
function $\beta=\beta_{R,T}$. This example is the standard model
used in Intserv. Let us apply Theorems~\ref{theo-backlog}
and~\ref{theo-delay}.  Assume that $R \geq r$, that is, the
reserved rate is as large as the sustainable rate of the flow.

From the convexity of the region between $\alpha$ and $\beta$
(Figure~\ref{fig-calculei}), we see that the vertical deviation
$v=\sup_{s \geq 0}[\alpha(s)-\beta(s)]$ is reached for at an
angular point of either $\alpha$ or $\beta$. Thus
$$v = \max [\alpha(T), \alpha(\theta)-\beta(\theta)]
$$with $\theta =\frac{b-M}{p-r}$.
Similarly, the horizontal distance is reached an angular point. In
the figure, it is either the distance marked as $AA'$ or $BB'$.
Thus, the bound on delay $d$ is given by
$$d = \max \left( \frac{\alpha(\theta)}{R}+T-\theta, \frac{M}{R}+T \right)$$
After some max-plus algebra, we can re-arrange these results as
follows.
\begin{proposition}[Intserv model, buffer and delay bounds]
\mylabel{theo-vbr-ratlat}Consider a VBR flow, with TSPEC $(M, p,
r, b)$, served in a node that guarantees to the flow a service
curve equal to the rate-latency function $\beta=\beta_{R,T}$. The
buffer required for the flow is bounded by
$$v=b + r T + \left(\frac{b-M}{p-r} - T\right)^+ [(p-R)^+ -p + r]
$$ The maximum delay for the flow is bounded by
$$d=\frac{M + \frac{b-M}{p-r}(p-R)^+ }{R} +T$$
\end{proposition}
\begin{figure}[htbp]
  \insfig{calculeI}{0.5}
  \mylabel{fig-calculei}
  \mycaption{Computation of buffer and delay bound for one VBR flow
  served in one Intserv node.}
\end{figure}

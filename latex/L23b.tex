\chapter{Time Varying Shapers}
\mylabel{L23b}

\section{Introduction}
Throughout the book we usually assume that systems are idle at
time $0$. This is not a limitation for systems that have a renewal
property, namely, which visit the idle state infinitely often --
for such systems we choose the time origin as one such instant.

There are cases however where we are interested in the effect at
time $t$ of non zero initial conditions. This occurs for example
for re-negotiable services, where the traffic contract is changed
at periodic renegotiation moments.  An example for this service is
the Integrated Service of the IETF with the Resource reSerVation
Protocol (RSVP), where the negotiated contract may be modified
periodically  \cite{GP98}. A similar service is the
ATM Available Bit Rate service (ABR).%
\index{Available Bit Rate} \index{ABR} \index{Re-negotiable
service}
With a renegotiable service, the shaper employed by the
source is time-varying. With ATM, this corresponds to the concept
of Dynamic
Generic Cell Rate Algorithm (DGCRA).%
\index{DGCRA}. At renegotiation moments, the system cannot
generally be assumed to be idle. This motivates the need for
explicit formulae that describe the transient effect of non-zero
initial condition.

In \sref{sec-tvs} we define time varying shapers. In general,
there is not much we can say apart from a direct application of
the fundamental min-plus theorems in
\sref{sec:fixedpointequation}. In contrast, for shapers made of a
conjunction of leaky buckets, we can find some explicit formulas.
In \sref{sec-nzic} we derive the equations describing a shaper
with non-zero initial buffer. In \sref{sec-nzib} we add the
constraint that the shaper has some history. Lastly, in
\sref{sec-tvlb}, we apply this to analyze the case where the
parameters of a shaper are periodically modified.

This chapter also provides an example of the use of time shifting.

\section{Time Varying Shapers}
\mylabel{sec-tvs}

We define a time varying shaper as follows.

\begin{definition}
Consider a flow $R(t)$. Given a function of two time variables
$H(t,s)$, a \emph{time varying shaper} forces the output
$R^{*}(t)$ to satisfy the condition
$$
R^{*}(t) \leq H(t,s) +  R^* (s)
$$
for all $s \leq t$, possibly at the expense of buffering some
data. An optimal time varying shaper, or greedy time varying
shaper, is one that maximizes its output among all possible
shapers. \mylabel{def-tvsh}
\index{time varying shaper}
\end{definition}
The existence of a greedy time varying shaper follows from the
following proposition.
\begin{proposition}
For an input flow $R(t)$ and a function of two time variables
$H(t,s)$, among all flows $R^* \leq R$ satisfying
$$R^{*}(t) \leq H(t,s)
+ R^* (s)$$ there is one flow that upper bounds all. It is given
by
\begin{equation}\mylabel{eq-vls-1}
  R^*(t)= \inf_{s \geq 0} \left[\overline{H}(t,s) + R(s)\right]
\end{equation}
where $\overline{H}$ is the min-plus closure of $H$, defined in
\eref{eq:closurelinearoperator} on
\pgref{eq:closurelinearoperator}.
\mylabel{prop-tvs}
\end{proposition}
\pr
The condition defining a shaper can be expressed as
 $$
 \bracket{R^* \leq \calL_H ( R^* )\\
 R^* \leq R
 }
  $$
where $\calL_H$ is the min-plus linear operator whose impulse
response is $H$ (\thref{thm:impulseresponse}). The existence of a
maximum solution follows from \thref{thm:spacemethod} and from the
fact that, being min-plus linear, $\calL_H$ is
upper-semi-continuous. The rest of the proposition follows from
\thref{thm:linearoperatorclosure} and \thref{thm:spacemethod}.
\qed

The output of the greedy shaper is given by \eref{eq-vls-1}. A
time invariant shaper is a special case; it corresponds to $H(s,t)
= \sigma(t-s)$, where $\sigma$ is the shaping curve. In that case
we find the well-known result in \thref{theo-gen-shaper}.

In general, \pref{prop-tvs} does not help much. In the rest of
this chapter, we specialize to the class of concave piecewise
linear time varying shapers.
\begin{proposition}
Consider a set of $J$ leaky buckets with time varying rates
$r_j(t)$ and  bucket sizes $b_j(t)$. At time $0$, all buckets are
empty. A flow $R(t)$ satisfies the conjunction of the $J$ leaky
bucket constraints if and only if for all $0\leq s \leq t$:
$$R(t) \leq H(t,s) + R(s)$$
with
\begin{equation}\mylabel{eq-1sdoasj}
  H(t,s)= \min_{1 \leq j \leq J} \{ b_{j}(t)+ \int_{s}^t r_{j}(u) du \}
\end{equation}
\mylabel{prop-tv-plc}
\end{proposition}
\pr
Consider the level of the $j$th bucket. It is the backlog of the
variable capacity node (\sref{sec-clasc}) with cumulative function
$$M_j(t) = \int_{0}^t r_{j}(u) du $$
We know from \cref{L11} that the output of the variable capacity
node is given by
 $$
 R'_j(t)=\inf_{0 \leq s \leq t} \{ M_j(t) -M_j(s) + R(s) \}
 $$
The $j$th leaky bucket constraint is
 $$R(t)-R'_j(t) \leq b_j(t)$$
Combining the two expresses the $j$th constraint as
$$
R(t)-R(s) \leq M_j(t) -M_j(s) + b_j(t)
$$
for all $0\leq s \leq t$. The conjunction of all these constraints
gives \eref{eq-1sdoasj}.

In the rest of this chapter, we give a practical and explicit
computation of $\overline{H} $ for $H$ given in \eref{eq-1sdoasj},
when the functions $r_j(t)$ and $b_j(t)$ are piecewise constant.

\section[Time Invariant Shaper with Initial Conditions]{Time Invariant Shaper with Non-zero Initial Conditions}
We consider in this section some time invariant shapers. We start
with a general shaper with shaping curve $\sigma$, whose buffer is
not assumed to be initially empty. Then we will apply this to
analyze leaky bucket shapers with non-empty initial buckets.

\subsection{Shaper with Non-empty Initial Buffer}
\mylabel{sec-nzic}

\begin{proposition}[Shaper with non-zero initial buffer]
\mylabel{prop-proNZB} Consider a shaper system with shaping curve
$\sigma$. Assume that $\sigma$ is a good function.  Assume that
the initial buffer content is $w_0$. Then the output $R^*$ for a
given input $R$ is
\begin{equation}
\mylabel{l23b-enzb}
\begin{array}{lr}
R^*(t)= \sigma(t) \wedge \displaystyle\inf_{0 \leq s \leq t} \{
R(s) + w_0 + \sigma(t-s) \}& \mfa t \geq 0
\end{array}
\end{equation}
\end{proposition}
\pr First we derive the constraints on the output of the shaper.
$\sigma$ is the shaping function thus, for all $t \geq s \geq 0$
\[
R^*(t) \leq R^*(s) + \sigma(t-s)
\]
and given that the bucket at time zero is not empty, for any $t
\geq 0$, we have that
\[
R^*(t) \leq R(t) +  w_0
\]
At time $s=0$, no data has left the system; this is expressed with
\[
R^*(t) \leq \delta_0(t)
\]
The output is thus constrained by
$$
R^{*}\leq (\sigma \mpc R^{*}) \wedge (R + w_{0}) \wedge \delta_{0}
$$
where $\mpc$ is the min-plus convolution operation, defined by $(f
\mpc g) (t) = \inf_{s} f(s) + g(t-s)$. Since the shaper is an
optimal shaper, the output is the maximum function satisfying this
inequality. We know from \lref{lem-gs} that
\[
\begin{array}{ll}
R^* & = \sigma \mpc [(R+w_0)\wedge \delta_0]\\ &= [\sigma \mpc
(R+w_0)] \wedge [\sigma \mpc \delta_0]\\ &= [\sigma \mpc (R+w_0)]
\wedge \sigma
\end{array}
\]
which after some expansion gives the formula in the proposition.
\qed.

Another way to look at the proposition consists in saying that the
initial buffer content is represented by an instantaneous burst at
time $0$.

The following is an immediate consequence.
\begin{corollary}[Backlog for a shaper with non-zero initial buffer]
\mylabel{l23b-proLB} The backlog of the shaper buffer with the
initial buffer content $w_0$ is given by
\begin{equation}
w(t)= \left(R(t)- \sigma(t) + w_0 \right) \vee \sup_{0 < s\leq t}
\{R(t)-R(s)-\sigma(t-s)\} \mylabel{l23b-elb}
\end{equation}
\end{corollary}


\subsection{Leaky Bucket Shapers with Non-zero Initial Bucket Level}
\mylabel{sec-nzib} Now we characterize a leaky-bucket shaper
system with non-zero initial bucket levels.

\begin{proposition}[Compliance with $J$ leaky buckets with non-zero initial
bucket levels] \mylabel{l23b-corLB} A flow $S(t)$ is compliant
with $J$ leaky buckets with leaky bucket specifications
$(r_j,b_j)$, $j=1,2 \dots J$ and initial bucket level $q_j^0$ if
and only if
\[
\begin{array}{lr}
S(t) - S(s) \leq
        \displaystyle\min_{1 \leq j \leq J}[r_j \cdot (t-s) +b_j]
        & \mfa 0 < s\leq t \\
S(t) \leq \displaystyle \min_{1 \leq j \leq J} [ r_j \cdot t +
                b_j - q_j^0 ]
        & \mfa t \geq 0 \\
\end{array}
\]
\end{proposition}
\pr
Apply \sref{l23b-proLB} to each of the buckets. \qed

\begin{proposition}[Leaky-Bucket Shaper with non-zero
initial bucket levels] \mylabel{l23b-proNZLB} Consider a greedy
shaper system defined by the conjunction of $J$ leaky buckets
$(r_j,b_j)$, with $j=1,2 \dots J$. Assume that the initial bucket
level of the j-th bucket is $q_j^0$. The initial level of the
shaping buffer is zero. The output $R^*$ for a given input $R$ is
\begin{equation}
\mylabel{l23b-enzlb}
\begin{array}{lr}
R^*(t)= \min [\sigma^0(t), (\sigma \mpc R)(t)]&\mfa t \geq 0
\end{array}
\end{equation}
where $\sigma$ is the shaping function
\[
\sigma(u)=\displaystyle \min_{1 \leq j \leq J}\{\sigma_j(u)\}
        = \displaystyle\min_{1 \leq j \leq J}\{r_j \cdot u+b_j\}
\]
and $\sigma^0$ is defined as
\[
\sigma^0(u)=\displaystyle \min_{1 \leq j \leq J} \{ r_j \cdot u +
                b_j - q_j^0 \}
\]
\end{proposition}
\pr By \coref{l23b-corLB} applied to $S=R^*$, the condition that
the output is compliant with the $J$ leaky buckets is
\[
\begin{array}{lr}
R^*(t) - R^*(s) \leq \sigma(t-s) & \mfa 0 < s\leq t \\ R^*(t) \leq
\sigma^0(t) & \mfa t \geq 0 \\
\end{array}
\]
Since $\sigma^0(u) \leq \sigma(u)$ we can extend the validity of
the first equation to $s=0$. Thus we have the following
constraint:
\[
R^*(t) \leq [(\sigma \mpc R^*) \wedge (R  \wedge \sigma^0)](t)
\]
Given that the system is a greedy shaper, $R^*(\cdot)$ is the
maximal solution satisfying those constraints. Using the same
min-plus result as in \pref{prop-proNZB}, we obtain:
$$
R^* = \sigma \mpc (R  \wedge \sigma^0)= (\sigma \mpc R) \wedge
(\sigma \mpc \sigma^0)$$
 As $\sigma^0 \leq \sigma$, we obtain
 $$R^* = (\sigma \mpc R) \wedge
\sigma^0
 $$
\qed

We can now obtain the characterization of a leaky-bucket shaper
 with non-zero initial conditions.
\begin{theorem}[Leaky-Bucket Shaper with non-zero
initial conditions] \mylabel{l23b-proNZ} Consider a shaper defined
by $J$ leaky buckets $(r_j,b_j)$, with $j=1,2 \dots J$
(leaky-bucket  shaper). Assume that the initial buffer level of
 is $w_0$ and the initial level
of the $j$th bucket is $q_j^0$. The output $R^*$ for a given input
$R$ is
\begin{equation}
\mylabel{l23b-enz}
\begin{array}{lr}
R^*(t)= \min \{ \sigma^0(t), w_0 + \displaystyle \inf_{u>0} \{
R(u) + \sigma(t-u) \} \} &\mfa t \geq 0
\end{array}
\end{equation}
with
\[
\sigma^0(u)=\displaystyle \min_{1 \leq j \leq J} ( r_j \cdot u +
                b_j - q_j^0 )
\]
\end{theorem}
\pr
Apply \pref{l23b-proNZLB} to the input $R'=(R+ w_0) \wedge
\delta_0$ and observe that $\sigma^0 \leq \sigma$. \qed

An interpretation of \eref{l23b-enz} is that the output of the
shaper with non-zero initial conditions is either the output of
the ordinary leaky-bucket shaper, taking into account the initial
level of the buffer, or, if smaller, the output imposed by the
initial conditions, independent of the input.
%
\section{Time Varying Leaky-Bucket Shaper}
 \mylabel{sec-tvlb}

We consider now time varying leaky-bucket shapers that are
piecewise constant. The shaper is defined by a fixed number $J$ of
leaky buckets, whose parameters change at times $t_i$. For $t \in
[t_i, t_{i+1}):=I_i$, we have thus
$$
r_j(t)=r_j^i \; \mand \; b_j(t)=b_j^i
$$
At times $t_i$, where the leaky bucket parameters are changed, we
keep the leaky bucket level $q_j(t_i)$ unchanged.

We say that $\sigma_i(u):=\min_{1\leq j J}\{r_j^i u + b_j^i\}$ is
the value of the time varying shaping curve during interval $I_i$.
With the notation in \sref{sec-tvs}, we have
$$
H(t,t_i)=\sigma_i(t-t_i) \;\mif t \in I_i
$$

We can now use the results in the previous section.

\begin{proposition}[Bucket Level]
\mylabel{l23b-proQ} Consider a piecewise constant time varying
leaky-bucket shaper with output $R^*$. The bucket level $q_j(t)$
of the j-th bucket is, for $t \in I_i$:
\begin{equation}
\mylabel{l23b-q}
  \begin{array}{rl}
    q_j(t)= &\left[ R^*(t)-R^*(t_i)-r_j^i \cdot(t-t_i)+q_j(t_i) \right]
 \vee  \\
     & \sup_{t_i < s\leq t}
              \{R^*(t)-R^*(s)-r_j^i \cdot(t-s)\}
  \end{array}
\end{equation}
\end{proposition}
\pr
We use a time shift, defined as follows. Consider a fixed interval
$I_i$ and define
$$x^*(\tau):=R^*(t_i + \tau) - R^*(t_i)$$

Observe that $q_j(t_i+\tau)$ is the backlog at time $\tau$ (call
it $w(\tau)$ at the shaper with shaping curve $\sigma(\tau)=r_j^i
\cdot t$, fed with flow $x^*$, and with an initial buffer level
$q_j(t_i)$. By \cref{l23b-proLB} we have
 $$
w(\tau)=\left[x^*(\tau)-r_j^i \cdot \tau + q_j(t_i) \right] \vee
\sup_{0 <s'\leq \tau}
                \{x^*(\tau)-x^*(s')-r_j^i \cdot(\tau-s')\}
 $$
 which after re-introducing $R^*$ gives \eref{l23b-q}
 \qed

%We can now characterize a time varying leaky-bucket shaper in the
%interval $I_i$ by using the input-output characterization given
%for a leaky-bucket
% shaper  with
%non-zero initial conditions. The initial conditions are
%represented by $q_j(t_i)$ and $w(t_i)$, which are respectively the
%bucket level and the backlog that are found by the traffic
%arriving in the interval $I_i$. We also derive the backlog at any
%time.
%
\begin{theorem}[Time Varying Leaky-Bucket
Shapers] \mylabel{l23b-proDS} Consider a piecewise constant time
varying leaky-bucket shaper with time varying shaping curve
$\sigma_i$ in the interval $I_i$. The output $R^*$ for a given
input $R$ is
\begin{equation}
\mylabel{l23b-resultDS} R^*(t)=\min \left[ \sigma^0_i(t-t_i)+
R^*(t_i), \displaystyle\inf_{t_i <s \leq t} \{ \sigma_i(t-s)
+R(s)\}\right]
\end{equation}
with $\sigma^0_i$ is defined by
\[
\sigma^0_i(u)=\displaystyle \min_{1 \leq j \leq J} \left[ r_j^i
\cdot u +
                b_i^j - q_j(t_i) \right]
\]
and $q_j(t_i)$ is defined recursively by \eref{l23b-q}. The
backlog at time $t$ is defined recursively by
\begin{equation}
\mylabel{l23b-shbuf} w(t)=\begin{array}{lr}
      \max \left [ \begin{array}{l}
        \displaystyle\sup_{t_i < s \leq t}\{R(t)-R(s)-\sigma_i(t-s)\},\\
                     R(t)-R(t_i)-\sigma^0_i(t-t_i) +w(t_i)\\
                        \end{array}
                        \right] &
                t \in I_i\\
\end{array}
\end{equation}
\end{theorem}
\pr
Use the same notation as in the proof of \pref{l23b-proQ} and
define in addition
$$x(\tau):=R(t_i + \tau) - R(t_i)$$

We can now apply \thref{l23b-proNZ}, with initial bucket levels
equal to $q_j(t_i)$ as given in \eref{l23b-q} and with an initial
buffer level equal to $w(t_i)$. The input-output characterization
of this system is given by \eref{l23b-enz}, thus
\[
x^*(\tau) = \sigma^0_i(\tau) \wedge [\sigma_i \mpc x'](\tau)
\]
where
\[
x'(\tau)=\left \{ \begin{array}{lr} x(\tau)+w(t_i) & \tau > 0 \\
x(\tau) & \tau \leq 0\\
\end{array}
\right .
\]
Hence, re-introducing the original notation, we obtain
\[
R^*(t) - R^*(t_i)=\left[ \sigma^0_i(t-t_i)\wedge
\displaystyle\inf_{t_i <s \leq t} \{ \sigma_i(t-s) +R(s)
-R(t_i)+w(t_i)\}\right]
\]
which gives \eref{l23b-resultDS}.

The backlog at time $t$ follows immediately. \qed

Note that \thref{l23b-proDS} provides a representation of
$\overline{H}$. However, the representation is recursive: in order
to compute $R^*(t)$, we need to compute $R^*(t_i)$ for all $t_i <
t$.


\section{Bibliographic Notes}
\cite{Gioleb} illustrates how the formulas in \sref{sec-tvlb} form
the basis for defining a renegotiable VBR service. It also
illustrates that, if some inconsistency exists between network and
user sides whether leaky buckets should be reset or not at every
renegotiation step, then this may result in inacceptable losses
(or service degradation) due to policing.

 \cite{changCruz98} analyzes the general concept of time varying
 shapers.

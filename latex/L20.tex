%\chapter[Network Calculus]{Network Calculus, with Application to Intserv and
%ATM}%
\chapter[Network Calculus]{Network Calculus}%
\mylabel{L20}
%\section{Introduction}
\nfs{Leftover 1: does
packetized re-shaping increase delay bound ?}
%Modifier intro pour tenir compte du texte packetizer
%
%Implementation of shaper in real packet world: Peris
%
%Put Viniotis and Steve Wright's work on ATM schedulers.
%
%Explain CDVT in terms of a periodic function with delay due to
%multiplexing.
%
%Application: representation de la convolution quand $\sigma$ est
%PCL comme l'output d'un controlleur a buckets multiples.

In this chapter we introduce the basic network calculus concepts
of arrival, service curves and shapers.  The application given in
this chapter concerns primarily networks with reservation services
such as ATM or the Internet integrated services
(``Intserv''\index{Intserv}). Applications to other settings are
given in the following chapters.

We begin the chapter by defining cumulative functions, which can
handle both continuous and discrete time models.  We show how
their use can give a first insight into playout buffer issues,
which will be revisited with more detail in Chapter~\ref{L22}.
Then the concepts of Leaky Buckets and Generic Cell Rate
algorithms are described in the appropriate framework, of arrival
curves.  We address in detail the most important arrival curves:
piecewise linear functions and stair functions. Using the stair
functions, we clarify the relation between spacing and arrival
curve.

We introduce the concept of service curve as a common model for a
variety of network nodes. We show that all schedulers generally
proposed for ATM or the Internet integrated services can be
modeled by a family of simple service curves called the
rate-latency service curves.  Then we discover physical properties
of networks, such as ``pay bursts only once" or ``greedy shapers
keep arrival constraints".  We also discover that greedy shapers
are min-plus, time invariant systems. Then we introduce the
concept of maximum service curve, which can be used to account for
constant delays or for maximum rates. We illustrate all along the
chapter how  the results can be used for practical buffer
dimensioning. We give practical guidelines for handling fixed
delays such as propagation delays. We also address the distortions
due to variability in packet size.

\section{Models for Data Flows}

\subsection{Cumulative Functions, Discrete Time versus Continuous Time Models}
It is convenient to describe data flows by means of the cumulative
function $R(t)$, defined as the number of bits seen on the flow in
time interval $[0,t]$. By convention, we take $R(0)=0$, unless
otherwise specified. Function $R$ is always wide-sense increasing,
that is, it belongs to the space $\calF$ defined in
\sref{sec:wsifunctions} on \pgref{sec:wsifunctions}. We can use a
discrete or continuous time model. In real systems, there is
always a minimum granularity (bit, word, cell or packet),
therefore  discrete time with a finite set of values for $R(t)$
could always be assumed. However, it is often computationally
simpler to consider continuous time, with a function $R$ that may
be continuous or not. If $R(t)$ is a continuous function, we say
that we have a {\em fluid model}. Otherwise, we take the
convention that the function is either right or left-continuous
(this makes little difference in practice).%
\footnote{It would be nice to stick to either left- or
right-continuous functions. However, depending on the model, there
is no best choice: see \sref{sec-vd}, \sref{arrcur-def} and \sref{sec-vlp}.}
Figure~\ref{modF} illustrates these definitions.

\hrulefill
\paragraph{Convention: } A flow is described by a wide-sense
increasing function $R(t)$; unless otherwise specified, in this
book, we consider  the following types of models:
\begin{itemize}
  \item discrete time: $t \in \Nats=\{0,1,2,3,...\}$\index{1Nats@$\Nats$}
  \item fluid model: $t \in \Reals^+=[0,+\infty)$\index{1Reals+@$\Reals^+$} and $R$ is a continuous
  function
  \item general, continuous time model: $t \in \Reals^+$ and $R$ is
   a left- or right-continuous
  function
\end{itemize}
\hrulefill

\begin{figure}[htbp]
		\insfig{modf}{0.8}
        \mycaption{Examples of Input and Output functions, illustrating our terminology and
        convention.
        $R_1$ and $R^*_1$ show a
       continuous function of continuous time (fluid model); we assume
        that packets arrive bit by bit, for a duration of one time unit
        per packet arrival. $R_2$ and $R^*_2$ show
        continuous time with discontinuities at packet
        arrival times (times 1, 4, 8, 8.6 and 14); we assume here that packet arrivals
        are observed only when the packet has been fully received; the dots
        represent the value at the point of discontinuity; by convention, we
        assume that the function is
        left- or right-continuous.
        $R_3$ and $R^*_3$ show a discrete time model;
        the system is observed only at times
        $0, 1, 2...$}
        \mylabel{modF}
\end{figure}

If we assume that $R(t)$ has a derivative $\frac{dR}{dt}=r(t)$
such that $R(t)=\int_{0}^t r(s) ds$ (thus we have a fluid model),
then $r$ is called the rate function. Here, however, we will see
that it is much simpler to consider cumulative functions such as
$R$ rather than rate functions. Contrary to standard algebra, with
min-plus algebra we do not need functions to have ``nice"
properties such as having a derivative.

It is always possible to map a continuous time model $R(t)$ to a
discrete time model $S(n), n \in \Nats$ by choosing a time slot
$\delta$ and sampling by

\begin{equation} \mylabel{eq-map-samp}
  S(n) = R( n \delta)
\end{equation}
In general, this results in a loss of information. For the reverse
mapping, we use the following convention. A continuous time model
can be derived from
$S(n), n \in \Nats$ by letting%
\footnote{$\lceil x \rceil$ (``ceiling of $x$'')
is defined as the smallest integer $\geq x$;
for example $\lceil 2.3\rceil = 3$ and
$\lceil 2\rceil = 2$}%
\begin{equation} \mylabel{eq-map-inter}
 R'(t)= S(\lceil
\frac{t}{\delta}\rceil)
\end{equation}
The resulting function $R'$ is
always left-continuous, as we already required. Figure~\ref{modF} illustrates this
mapping with $\delta = 1$, $S=R_3$ and $R'=R_2$.

Thanks to the mapping in Equation~(\ref{eq-map-samp}), any result
for a continuous time model also applies to discrete time. Unless
otherwise stated, all results in this book apply to both
continuous and discrete time. Discrete time models are generally
used in the context of ATM; in contrast, handling variable size
packets is usually done with a continuous time model (not
necessarily fluid). Note that handling variable size packets
requires some specific mechanisms, described in
Section~\ref{sec-vlp}.

Consider now a system $\calS$, which we view as a blackbox;
$\calS$ receives input data, described by its cumulative function
$R(t)$, and delivers the data after a variable delay. Call
$R^{*}(t)$ the \emph{output function}, namely, the cumulative
function at the output of system $\calS$. System $\calS$ might be,
for example, a single buffer served at a constant rate, a complex
communication node, or even a complete network. Figure~\ref{modF}
shows input and output functions for a single server queue, where
every packet takes exactly 3 time units to be served. With output
function $R^*_1$ (fluid model) the assumption is that a packet can
be served as soon as a first bit has arrived (cut-through
assumption), and that a packet departure can be observed bit by
bit, at a constant rate. For example, the first packet arrives
between times 1 and 2, and leaves between times 1 and 4. With
output function $R^*_2$ the assumption is that a packet is served
as soon as it has been fully received and is considered out of the
system only when it is fully transmitted (store and forward
assumption). Here, the first packet arrives immediately after time
1, and leaves immediately after time 4. With output function
$R^*_3$ (discrete time model), the first packet arrives at time 2
and leaves at time 5.

\subsection{Backlog and Virtual Delay}
\label{sec-vd}
From the input and output functions, we derive the two following
quantities of interest.
\begin{definition}[Backlog and Delay]
\mylabel{L20-intro} For a lossless system:
\begin{itemize}
        \item  The \emph{backlog} at time $t$ is $R(t) - R^{*}(t)$.
        \item  The \emph{virtual delay} at time $t$ is
        $$d(t) = \inf \left\{
               \tau \geq 0 : R(t) \leq R^*(t + \tau)
              \right\}
    $$
\end{itemize}
\end{definition}

The backlog is the amount of bits that are held inside the system;
if the system is a single buffer, it is the queue length. In
contrast, if the system is more complex, then the backlog is the
number of bits ``in transit'', assuming that we can observe input
and output simultaneously.  %The virtual delay at time $t$ is the
%delay that would be experienced by a bit arriving at time $t$ if
%all bits received before it are served before it.
In
Figure~\ref{modF}, the backlog, called $x(t)$, is shown as the
vertical deviation between input and output functions.

Roughly speaking, the virtual delay at time $t$ is the
delay that is experienced by a bit arriving at time $t$ if
all bits received before it are served before it (we say that such a system is \nt{FIFO} for this flow, where FIFO stands for ``First In, First Out"). On the figure, it is the horizontal deviation
between input and output functions.  More precisely, if these functions %continuous (fluid model), then it is easy to see that
%
are left-continuous (as in Figure~\ref{modF}, subfigure 2), and for a FIFO system, $d(t)$ is the delay for a  hypothetical bit that would arrive \emph{just before} time $t$; the delay for a bit that arrives \emph{at} time $t$ is the limit from the right of $d(t)$, namely, $d_r(t)\eqdef\limit{s}{t, s>t}d(s)$
\index{limit from
the right}\footnote{Other authors often use the
notation $d(t^+)$ for $d_r(t)$.}.
In contrast, if the input and output functions are right-continuous, and for a FIFO system, $d(t)$ is the delay for a bit that arrives at time $t$.
%If, as we usually assume
%in this book, $R$ and $R^*$ are left-continuous, then $d(t)= \min
%\left\{ \tau \geq 0 : R(t) \leq R^*(t + \tau) \right\}$, in other
%words, the minimum is attained in the definition of $d(t)$.

In Figure~\ref{modF}, we see that the values of backlog and
virtual delay slightly differ for the three models. Thus the delay
experienced by the last bit of the first packet is $d(2)=2$ time
units for the first subfigure; in contrast, it is equal to
$d_r(1)=3$ time units on the second subfigure. This is of course in
accordance with the different assumptions made for each of the
models. Similarly, the delay for the fourth packet on subfigure 2
is $d_r(8.6)=5.4$ time units, which corresponds to 2.4 units of
waiting time and 3 units of service time. In contrast, on the
third subfigure, it is equal to $d(9)=6$ units; the difference is
the loss of accuracy resulting from discretization.



\subsection{Example: The Playout Buffer}
\mylabel{sec-playoutBuf} Cumulative functions are a powerful tool
for studying delays and buffers. In order to illustrate this,
consider the simple playout buffer problem that we describe now.
Consider a packet switched network that carries bits of
information from a source with a constant bit rate $r$ (Figure
\ref{fig-playoutBuf}) as is the case for example, with circuit
emulation. We take a fluid model, as illustrated in
Figure~\ref{fig-playoutBuf}. We have a first system $\calS$, the
network, with input function $R(t)=rt$. The network imposes some
variable delay, because of queuing points, therefore the output
$R^{*}$ does not have a constant rate $r$. What can be done to
recreate a constant bit stream ?
\begin{figure}[!htbp]
\insfig{playBuf}{1.0}
        \mycaption{A Simple Playout Buffer Example}
        \mylabel{fig-playoutBuf}
\end{figure}
A standard mechanism is to smooth the delay variation in a playout
buffer. It operates as follows. When the first bit of data
arrives, at time $d_r(0)$, %, where $d_r(0)=\lim_{t \rightarrow 0,
%t>0}d(t)$ is the limit to the right of function $d$\index{limit to
%the right}\footnote{It is the virtual delay for a hypothetical bit
%that would arrive just after time $0$. Other authors often use the
%notation $d(0+)$},
it is stored in the buffer until a fixed time
$\Delta$ has elapsed. Then the buffer is served at a constant rate
$r$ whenever it is not empty. This gives us a second system
$\calS'$, with input $R^*$ and output $S$.

%\index{limit to the right}

Let us assume that the network delay variation is bounded by
$\Delta$. This implies that for every time $t$, the virtual delay
(which is the real delay in that case) satisfies
 $$ - \Delta \leq
d(t) - d_r(0) \leq \Delta $$
 Thus, since we have a fluid model, we
have
 $$r (t - d_r(0) - \Delta)  \leq R^*(t) \leq r (t - d_r(0) + \Delta)$$
 which is illustrated in the figure by the two lines (D1) and (D2)
parallel to $R(t)$.
 The figure suggests that,  for the playout
buffer $\calS'$ the input function $R^*$ is always above the
straight line (D2), which means that the playout buffer never
underflows. This suggests in turn that the output function $S(t)$
is given by $S(t)=r( t - d_r(0) - \Delta)$.

Formally, the proof is as follows. We proceed by contradiction.
Assume the buffer starves at some time, and let $t_{1}$ be the
first time at which this happens. Clearly the playout buffer is
empty at time $t_{1}$, thus $R^*(t_1)=S(t_1)$. There is a time
interval $[t_{1}, t_{1}+ \epsilon]$ during which the number of
bits arriving at the playout buffer is less than $r \epsilon$ (see
Figure~\ref{fig-playoutBuf}). Thus, $d(t_{1}+ \epsilon) > d_r(0) +
\Delta$ which is not possible. Secondly, the backlog in the buffer
at time $t$ is equal to $R^*(t)-S(t)$, which is bounded by the
vertical deviation between (D1) and (D2), namely, $2r \Delta$.

We have thus shown that the playout buffer is able to remove the
delay variation imposed by the network. We summarize this as
follows.

\begin{proposition}
        Consider a constant bit rate stream of rate $r$, modified by a network
        that imposes a variable delay variation and no loss. The
        resulting flow is put into a playout buffer, which
        operates by delaying the first bit of the flow by
        $\Delta$, and reading the flow at rate $r$. Assume that the
        delay variation imposed by the network is bounded by $\Delta$, then
\begin{enumerate}
                \item  the playout buffer never starves and produces a constant
                output at rate $r$;
                \item  a buffer size of $2 \Delta r$ is sufficient to avoid overflow.
        \end{enumerate}
\end{proposition}
%
%
%Consider a packet switched network that carries bits of
%information from a source with a constant bit rate $r$ (Figure
%\ref{fig-playoutBuf}) as is the case for example, with circuit
%emulation. We take a fluid model, as illustrated in
%Figure~\ref{fig-playoutBuf}. We have a first system $\calS$, the
%network, with input function $R(t)=rt$. The network imposes some
%variable delay, because of queuing points, therefore the output
%$R^{*}$ does not have a constant rate $r$. What can be done to
%recreate a constant bit stream ?
%\begin{figure}[!htbp]
%\insfig{playBuf}{1.0}
%        \mycaption{A Simple Playout Buffer Example}
%        \mylabel{fig-playoutBuf}
%\end{figure}
%A standard mechanism is to smooth the delay variation in a playout
%buffer. It operates as follows. When the first bit of data
%arrives, at time $d_r(0)$, where $d_r(0)=\lim_{t \rightarrow 0,
%t>0}d(t)$ is the limit to the right of function $d$\index{limit to
%the right}\footnote{It is the virtual delay for a hypothetical bit
%that would arrive just after time $0$. Other authors often use the
%notation $d(0+)$}, it is stored in the buffer until a fixed time
%$\Delta$ has elapsed. Then the buffer is served at a constant rate
%$r$ whenever it is not empty. This gives us a second system
%$\calS'$, with input $R^*$ and output $S$.
%
%\index{limit to the right}
%
%Let us assume that the network delay variation is bounded by
%$\Delta$. This implies that for every time $t$, the virtual delay
%(which is the real delay in that case) satisfies
% $$ - \Delta \leq
%d(t) - d_r(0) \leq \Delta $$
% Thus, since we have a fluid model, we
%have
% $$r (t - d_r(0) - \Delta)  \leq R^*(t) \leq r (t - d_r(0) + \Delta)$$
% which is illustrated in the figure by the two lines (D1) and (D2)
%parallel to $R(t)$.
% The figure suggests that,  for the playout
%buffer $\calS'$ the input function $R^*$ is always above the
%straight line (D2), which means that the playout buffer never
%underflows. This suggests in turn that the output function $S(t)$
%is given by $S(t)=r( t - d_r(0) - \Delta)$.
%
%Formally, the proof is as follows. We proceed by contradiction. Assume the buffer starves at
%some time, and let $t_{1}$ be the first time at which this happens. Clearly the playout buffer
%is empty at time $t_{1}$, thus $R^*(t_1)=S(t_1)$. There is a time interval $[t_{1}, t_{1}+
%\epsilon]$ during which the number of bits arriving at the playout buffer is less than $r
%\epsilon$ (see Figure~\ref{fig-playoutBuf}). Thus, $d(t_{1}+ \epsilon) > d_r(0) + \Delta$ which
%is not possible. Secondly, the backlog in the buffer at time $t$ is equal to $R^*(t)-S(t)$,
%which is bounded by the vertical deviation between (D1) and (D2), namely, $2r \Delta$.
%
%We have thus shown that the playout buffer is able to remove the
%delay variation imposed by the network. We summarize this as
%follows.
%
%\begin{proposition}
%        Consider a constant bit rate stream of rate $r$, modified by a network
%        that imposes a variable delay variation and no loss. The
%        resulting flow is put into a playout buffer, which
%        operates by delaying the first bit of the flow by
%        $\Delta$, and reading the flow at rate $r$. Assume that the
%        delay variation imposed by the network is bounded by $\Delta$, then
%\begin{enumerate}
%                \item  the playout buffer never starves and produces a constant
%                output at rate $r$;
%                \item  a buffer size of $2 \Delta r$ is sufficient to avoid overflow.
%        \end{enumerate}
%\end{proposition}

We study playout buffers in more details in Chapter~\ref{L22},
using the network calculus concepts further introduced in this
chapter.

% ``space method'' = applies to both continuous time and discrete time
% ``time methods'' = applies to discrete time only


\section{Arrival Curves}
\mylabel{con-flows}

\subsection{Definition of an Arrival Curve}
\mylabel{arrcur-def}


Assume that we want to provide guarantees to data flows. This
requires some specific support in the network, as explained in
Section~\ref{sec-sercur}; as a counterpart, we need to limit the
traffic sent by sources. With integrated services networks (ATM or
the integrated services internet), this is done by using the
concept of arrival curve, defined below.
%We will see an
%alternative in Chapter~\ref{L21}.
\begin{definition} [Arrival Curve]
Given a wide-sense increasing function $\alpha$ defined for $t
\geq 0$ we say that a flow $R$ is constrained by $\alpha$ if and
only if for all $s \leq t$:
$$R(t)-R(s) \leq \alpha(t-s)$$
We say that $R$ has $\alpha$ as an arrival curve, or also that $R$
is $\alpha$-smooth.%
\index{arrival curve}%
\index{smooth ($\alpha$-smooth for some function $\alpha(t)$}
\end{definition}
Note that the condition is over a set of overlapping intervals, as
Figure~\ref{affArrCur} illustrates.
\begin{figure}[!htbp]
   \insfig{arrcurdef}{0.8}
  \mycaption{Example of Constraint by arrival curve,
  showing a cumulative function
  $R(t)$ constrained by the arrival curve $\alpha(t)$.}
   \mylabel{affArrCur}
\end{figure}

\paragraph{Affine Arrival Curves: }
For example, if $\alpha(t)=rt$, then the constraint means that, on
any time window of width $\tau$, the number of bits for the flow
is limited by $r \tau$. We say in that case that the flow is peak
rate limited. This occurs if we know that the flow is arriving on
a link whose physical bit rate is limited by $r$ b/s. A flow where
the only constraint is a limit on the peak rate is often
(improperly) called a ``constant bit rate" (CBR) flow, or
``deterministic bit rate" (DBR) flow.

Having $\alpha(t)=b$, with $b$ a constant, as an arrival curve
means that the maximum number of bits that may ever be sent on the
flow is at most $b$.

More generally, because of their relationship with leaky buckets,
we will often use \emph{affine} arrival curves $\gamma_{r,b}$,
defined by:
%\index{1gamma@$\gamma_{r,b}$ (affine arrival curve)} \index{affine
%arrival curve}
$\gamma_{r,b}(t)=rt +b$ for $t>0$ and $0$ otherwise. Having
$\gamma_{r,b}$ as an arrival curve allows a source to send $b$
bits at once, but not more than $r$ b/s over the long run.
Parameters $b$ and $r$ are called the burst tolerance (in units of
data) and the rate (in units of data per time unit).
Figure~\ref{affArrCur} illustrates such a constraint.

%\begin{definition} [Arrival Curve]
%Given a wide-sense increasing function $\alpha$ defined for $t
%\geq 0$ (namely, $\alpha \in \calF$), we say that a flow $R$ is
%constrained by $\alpha$ if and only if for all $s \leq t$:
%$$R(t)-R(s) \leq \alpha(t-s)$$
%We say that $R$ has $\alpha$ as an arrival curve, or also that $R$
%is $\alpha$-smooth.%
%\index{arrival curve}%
%\index{smooth ($\alpha$-smooth for some function $\alpha(t)$}
%\end{definition}
%Note that the condition is over a set of overlapping intervals, as
%Figure~\ref{affArrCur} illustrates.
%\begin{figure}[!htbp]
%   \insfig{arrcurdef}{0.8}
%  \mycaption{Example of Constraint by arrival curve,
%  showing a cumulative function
%  $R(t)$ constrained by the arrival curve $\alpha(t)$.}
%   \mylabel{affArrCur}
%\end{figure}
%
%\paragraph{Affine Arrival Curves: }
%For example, if $\alpha(t)=rt$, then the constraint means that, on
%any time window of width $\tau$, the number of bits for the flow
%is limited by $r \tau$. We say in that case that the flow is peak
%rate limited. This occurs if we know that the flow is arriving on
%a link whose physical bit rate is limited by $r$ b/s. A flow where
%the only constraint is a limit on the peak rate is often
%(improperly) called a ``constant bit rate" (CBR) flow, or
%``deterministic bit rate" (DBR) flow.
%
%Having $\alpha(t)=b$, with $b$ a constant, as an arrival curve
%means that the maximum number of bits that may ever be sent on the
%flow is at most $b$.
%
%More generally, because of their relationship with leaky buckets,
%we will often use \emph{affine} arrival curves $\gamma_{r,b}$,
%defined by:
%%\index{1gamma@$\gamma_{r,b}$ (affine arrival curve)} \index{affine
%%arrival curve}
%$\gamma_{r,b}(t)=rt +b$ for $t>0$ and $0$ otherwise (see
%Section~\ref{sec:wsifunctions} for an illustration). Having
%$\gamma_{r,b}$ as an arrival curve allows a source to send $b$
%bits at once, but not more than $r$ b/s over the long run.
%Parameters $b$ and $r$ are called the burst tolerance (in units of
%data) and the rate (in units of data per time unit).
%Figure~\ref{affArrCur} illustrates such a constraint.
%
\paragraph{Stair Functions as Arrival Curves: }
In the context of ATM, we also use arrival curves of the form $k
v_{T, \tau}$, where $v_{T, \tau}$ is the stair functions defined
by
  $v_{T, \tau}(t) = \lceil \frac{t+ \tau}{T} \rceil$ for $t>0$ and
  $0$ otherwise
(see Section~\ref{sec:wsifunctions} for an illustration). Note
that $v_{T, \tau}(t) = v_{T, 0}(t+\tau)$, thus $v_{T, \tau}$
results from $v_{T, 0}$ by a time shift to the left. Parameter $T$
(the ``interval") and $\tau$ (the ``tolerance") are expressed in
time units.
%Figure~\ref{arrcurcata} summarizes and illustrates the
%definitions of $\gamma_{r,b}$ and $v_{T, \tau}$.
%\begin{figure}[!htbp]
%  \mylabel{arrcurcata}
%$$
%\begin{array}{lccr}
%\gamma_{r,b}(t)=rt +b \mif t > 0 &\hspace{3cm}& k v_{T, \tau}(t) =
%k \lceil \frac{t+ \tau}{T} \rceil&\hspace{2cm}\\
%\end{array}
%$$
%  \insfig{arrcurcata}{0.8}
%  \mycaption{Definition of arrival curves used in this book: $\gamma_{r,b}$
%  with rate $r$ and burst tolerance $b$;
%  $k v_{T, \tau}$ with step $k$, interval $T$ and tolerance
%  $\tau$.
%  }
%\end{figure}
In order to understand the use of $v_{T, \tau}$, consider a flow
that sends packets of a fixed size, equal to $k$ unit of data (for
example, an ATM flow). Assume that the packets are spaced by at
least $T$ time units. An example is a constant bit rate voice
encoder, which generates packets periodically during talk spurts,
and is silent otherwise. Such a flow has $k v_{T, 0}$ as an
arrival curve.

Assume now that the flow is multiplexed with some others. A simple
way to think of this scenario is to assume that the packets are
put into a queue, together with other flows. This is typically
what occurs at a workstation, in the operating system or at the
ATM adapter. The queue imposes a variable delay; assume it can be
bounded by some value equal to $\tau$ time units. We will see in
the rest of this chapter and in Chapter~\ref{L21} how we can
provide such bounds. Call $R(t)$ the input function for the flow
at the multiplexer, and $R^*(t)$ the output function. We have
$R^*(s) \geq R(s - \tau)$, from which we derive:
$$
R^*(t) - R^*(s) \leq R(t) - R(s - \tau) \leq k v_{T,0}(t-s+\tau) =
k v_{T,\tau}(t-s)
$$
Thus $R^*$ has $k v_{T,\tau}$ as an arrival curve. We have shown
that \emph{a periodic flow, with period $T$, and packets of
constant size $k$, that suffers a variable delay $\leq \tau$, has
$k v_{T, \tau}$ as an arrival curve}. The parameter $\tau$ is
often called the ``one-point cell delay variation", as it
corresponds to a
deviation from a periodic flow that can be observed at one point.%
%
%
%
%
% is equivalent to requiring that packets are spaced by at
%least $T$ time units.
%
%
%For $\tau >0$, packets are spaced by $T$, but there is an initial
%tolerance of $\tau$. For example, an ATM flow constrained by $3
%v_{10, 4}$ may send, over any interval, 3 cells in the first
%$T-\tau=6$ time units, then no more than 3 cells per 10 time
%units.

In general, function $v_{T, \tau}$ can be used to express
\emph{minimum spacing} between packets, as the following
proposition shows.

\begin{proposition}[Spacing as an arrival constraint]
\mylabel{theo-l20-jud} Consider a flow, with cumulative function
$R(t)$, that generates packets of constant size equal to $k$ data
units, with instantaneous packet arrivals. Assume time is discrete
or time is continuous and $R$ is left-continuous. Call $t_{n}$ the
arrival time for the $n$th packet. The following two properties
are equivalent:
\begin{enumerate}
    \item  for all $m,n$, $t_{m+n}-t_{m} \geq nT - \tau$
    \item  the flow has $k v_{T, \tau}$ as an arrival curve
\end{enumerate}
\end{proposition}
The conditions on packet size and packet generation mean that $R(t)$
has the form $nk$, with $n \in \Nats$.  The spacing condition implies
that the time interval between two consecutive packets is $\geq T-
\tau$, between a packet and the next but one is $\geq 2T- \tau$, etc.
\pr Assume that property 1 holds.
% We prove by contradiction that property
% 2 also holds.
Consider an arbitrary interval $]s, t]$, and call $n$ the number
of packet arrivals in the interval. Say that these packets are
numbered $m+1, \ldots, m+n$, so that $s < t_{m+1}\leq \ldots \leq
t_{m+n} \leq t$, from which we have
$$
t-s > t_{m+n}- t_{m+1}
$$
Combining with  property 1,  we get
$$
t-s > (n-1) T - \tau
$$
From the definition of $v_{T, \tau}$ it follows that $v_{T,
\tau}(t-s) \geq n$. Thus $R(t)-R(s) \leq k v_{T, \tau}(t-s)$,
which shows the first part of the proof.

Conversely, assume now that property 2 holds. If time is discrete,
we convert the model to continuous time using the mapping in
Equation~\ref{eq-map-inter}, thus we can consider that we are in
the continuous time case. Consider some arbitrary integers $m, n$;
for all $\epsilon >0$, we have, under the assumption in the
proposition: $$ R(t_{m+n}+\epsilon)- R(t_{m}) \geq (n+1)k $$ thus,
from the definition of $v_{T, \tau}$, $$ t_{m+n} - t_{m} +\epsilon
> nT -\tau $$ This is true for all  $\epsilon >0$, thus $t_{m+n} -
t_{m} \geq  nT -\tau$. \qed

In the rest of this section we clarify the relationship between
arrival curve constraints defined by affine and by stair
functions. First we need a technical lemma, which amounts to
saying that we can always change an arrival curve to be
left-continuous.
\begin{lemma}[Reduction to left-continuous arrival curves]
\mylabel{lem-red-lc} Consider a flow $R(t)$ and a wide sense
increasing function $\alpha(t)$, defined for $t\geq 0$. Assume
that $R$ is either left-continuous, or right-continuous. Denote
with $\alpha_l(t)$ the limit from the left%
\index{limit from the left} of $\alpha$ at $t$ (this limit exists at
every point because $\alpha$ is wide sense increasing); we have
$\alpha_l(t)=\sup_{s<t}\alpha(s)$. If $\alpha$ is an arrival curve
for $R$, then so is $ \alpha_l$.
\end{lemma}
\pr Assume first that $R$ is left-continuous. For some $s < t$,
let $t_{n}$ be a sequence of increasing times converging towards
$t$, with $s < t_{n} \leq t$.  We have $R(t_{n})- R(s) \leq
\alpha(t_{n}-s) \leq \alpha_l(t-s)$.  Now $\lim_{n \rightarrow
+\infty} R(t_{n})=R(t)$ since we assumed that $R$ is
left-continuous.  Thus $R(t)-R(s) \leq \alpha_l(t-s)$.

If in contrast $R$ is right-continuous, consider a sequence
$s_{n}$ converging towards $s$ from above. We have similarly
$R(t)- R(s_{n}) \leq \alpha(t-s_{n}) \leq \alpha_l(t-s)$ and
$\lim_{n \rightarrow +\infty} R(s_{n})=R(s)$, thus $R(t)-R(s) \leq
\alpha_l(t-s)$ as well. \qed

Based on this lemma, we can always reduce an arrival curve to be
left-continuous%
\footnote{If we consider $\alpha_r(t)$, the limit from the right of
$\alpha$ at $t$, then $\alpha \leq \alpha_r$ thus $\alpha_r$ is
always an arrival curve, however it is not better than $\alpha$.}.
Note that $\gamma_{r,b}$ and $v_{T, \tau}$ are left-continuous.
Also remember that, in this book, we use the convention that
cumulative functions such as $R(t)$ are left continuous; this is a
pure convention, we might as well have chosen to consider only
right-continuous cumulative functions. In contrast, an arrival
curve can always be assumed to be left-continuous, but not
right-continuous.


In some cases, there is equivalence between a constraint defined
by $\gamma_{r,b}$ and $v_{T, \tau}$. For example, for an ATM flow
(namely, a flow where every packet has a fixed size equal to one
unit of data) a constraint $\gamma_{r,b}$ with $r=\frac{1}{T}$ and
$b=1$ is equivalent to sending one packet every $T$ time units,
thus is equivalent to a constraint by the arrival curve $v_{T,
0}$. In general, we have the following result.
\begin{proposition}
        \mylabel{theo-gamma-u}
Consider either a left- or right- continuous flow $R(t), t \in
\Reals^+$, or a discrete time flow $R(t), t \in \Nats$, that
generates packets of constant size equal to $k$ data units, with
instantaneous packet arrivals. For some $T$ and $\tau$, let
$r=\frac{k}{T}$ and $b=k(\frac{\tau}{T}+1)$. It is equivalent to
say that $R$ is constrained by $\gamma_{r,b}$ or by $k v_{T,
\tau}$.
\end{proposition}
\pr Since we can map any discrete time flow to a left-continuous,
continuous time flow, it is sufficient to consider a
left-continuous flow $R(t), t \in \Reals^+$. Also, by changing the
unit of data to the size of one packet, we can assume without loss
of generality that $k=1$. Note first, that with the parameter
mapping in the proposition, we have $v_{T, \tau} \leq
\gamma_{r,b}$, which shows that if $v_{T, \tau}$ is an arrival
curve for $R$, then so is $\gamma_{r,b}$.

Conversely, assume now that $R$ has $\gamma_{r,b}$ as an arrival
curve. Then for all $s\leq t$, we have
$
R(t)-R(s) \leq rt +b
$,
and since $R(t)- R(s) \in \Nats$, this implies
$
R(t) -R(s) \leq \lfloor rt + b \rfloor $, Call $\alpha(t)$ the
right handside in the above equation and apply
Lemma~\ref{lem-red-lc}. We have $\alpha_l(t)=\lceil rt + b -1
\rceil = v_{T, \tau}(t)$. \qed

Note that the equivalence holds if we can assume that the packet
size is constant and equal to the step size in the constraint $k
v_{T, \tau}$.  In general, the two families of arrival curve do
not provide identical constraints.  For example, consider an ATM
flow, with packets of size 1 data unit, that is constrained by an
arrival curve of the form $ k v_{T, \tau}$, for some $k > 1$. This
flow might result from the superposition of several ATM flows. You
can convince yourself that this constraint cannot be mapped to a
constraint of the form $\gamma_{r,b}$.  We will come back to this
example in Section~\ref{3bornes}.

\subsection{Leaky Bucket and Generic Cell Rate Algorithm}
\mylabel{sec-lbarcur}

Arrival curve constraints find their origins in the concept of
leaky bucket and generic cell rate algorithms, which we describe
now. We show that leaky buckets correspond to affine arrival
curves $\gamma_{r,b}$, while the generic cell rate algorithm
corresponds to stair functions $v_{T, \tau}$. For flows of fixed
size packets, such as ATM cells, the two are thus equivalent.

\begin{definition} [Leaky Bucket Controller]
A Leaky Bucket Controller is a device that analyzes the data on a
flow $R(t)$ as follows. There is a pool (bucket) of fluid of size
$b$. The bucket is initially empty. The bucket has a hole and
leaks at a rate of $r$ units of fluid per second when it is not
empty.

Data from the flow $R(t)$ has to pour into the bucket an amount of
fluid equal to the amount of data. Data that would cause the
bucket to overflow is declared non-conformant, otherwise the data
is declared conformant.
\end{definition}
Figure~\ref{leabu} illustrates the definition.  Fluid in the leaky
bucket does not represent data, however, it is counted in the same
unit as data.

Data that is not able to pour fluid into the bucket is said to be
``non-conformant'' data. In ATM systems, non-conformant data is
either discarded, tagged with a low priority for loss (``red''
cells), or can be put in a buffer (buffered leaky bucket
controller). With the Integrated Services Internet, non-conformant
data is in principle not marked, but simply passed as best effort
traffic (namely, normal IP traffic).
\begin{figure}[!htbp]
\mylabel{leabu}
  \insfig{leabu}{0.8}
  \mycaption{A Leaky Bucket Controller. The second part of the figure shows (in grey) the
  level of the bucket $x(t)$ for a sample input, with $r=0.4$ kbits per time
  unit and $b=1.5$ kbits. The packet arriving at time $t=8.6$ is not
  conformant, and no fluid is added to the bucket. If $b$ would be
  equal to $2$ kbits, then all packets would be conformant.}
\end{figure}

We want now to show that a leaky bucket controller enforces an
arrival curve constraint equal to $\gamma_{r,b}$. We need the
following lemma.
\begin{lemma}
        \mylabel{lem-buf-simple}
        Consider a buffer served at a constant rate $r$. Assume that the
        buffer is empty at time $0$. The input is described by the cumulative
        function $R(t)$. If there is no overflow during $[0,t]$, the buffer
        content at time $t$ is given by
        $$
        x(t)=\sup_{s : s \leq t} \{ R(t) - R(s) -r (t-s)\}
        $$
\end{lemma}
\pr The lemma can be obtained as a special case of
Corollary~\ref{coro-shaper-buffer-occupancy} on
page~\pageref{coro-shaper-buffer-occupancy}, however we give here
a direct proof. First note that for all $s$ such that $s\leq t$,
$(t-s)r$ is an upper bound on the number of bits output in
$]s,t]$, therefore:
$$
R(t)-R(s) - x(t) +x(s) \leq (t-s)r
$$

Thus
$$
x(t) \geq R(t)-R(s) + x(s) - (t-s)r \geq R(t)-R(s) - (t-s)r
$$

which proves that $x(t) \geq \sup_{s : s \leq t} \{ R(t) - R(s) -r
(t-s)\}$.

Conversely, call $t_{0}$ the latest time at which the buffer was
empty before time $t$: $$t_{0} = \sup \{s: s\leq t, x(s) = 0\}$$
(If $x(t)  > 0$ then $t_{0}$ is the beginning of the busy period
at time $t$). During $]t_{0}, t]$, the queue is never empty,
therefore it outputs bit at rate $r$, and thus
\begin{equation} \mylabel{eq-d4-37443}
        x(t)=x(t_{0 })+R(t)-R(t_{0 }) - (t-t_{0 })r
\end{equation}

We assume that $R$ is left-continuous (otherwise the proof is a
little more complex); thus $x(t_{0 })=0$ and thus $x(t) \leq
\sup_{s : s \leq t} \{ R(t) - R(s) -r (t-s)\}$ %\qed

Now the content of a leaky bucket behaves exactly like a buffer
served at rate $r$, and with capacity $b$. Thus, a flow $R(t)$ is
conformant if and only if the bucket content $x(t)$ never exceeds
$b$. From Lemma~\ref{lem-buf-simple}, this means that
$$
\sup_{s : s \leq t} \{ R(t) - R(s) -r (t-s)\} \leq b
$$
which is equivalent to
$$
R(t)-R(s) \leq r(t-s) +b
$$
for all $s \leq t$. We have thus shown the following.
\begin{proposition}
A leaky bucket controller with leak rate $r$ and bucket size $b$
forces a flow to be constrained by the arrival curve
$\gamma_{r,b}$, namely:
\begin{enumerate}
        \item  the flow of conformant data has $\gamma_{r,b}$ as an arrival
        curve;
\item  if the input already has $\gamma_{r,b}$ as an arrival
        curve, then all data is conformant.
\end{enumerate}
\end{proposition}

%
%\begin{definition} [Leaky Bucket Controller]
%A Leaky Bucket Controller is a device that analyzes the data on a
%flow $R(t)$ as follows. There is a pool (bucket) of fluid of size
%$b$. The bucket is initially empty. The bucket has a hole and
%leaks at a rate of $r$ units of fluid per second when it is not
%empty.
%
%Data from the flow $R(t)$ has
%to pour into the bucket an amount of fluid equal to the amount of
%data. Data that would cause the bucket to overflow is declared
%non-conformant, otherwise the data is declared conformant.
%\end{definition}
%Figure~\ref{leabu} illustrates the definition.  Fluid in the leaky
%bucket does not represent data, however, it is counted in the same
%unit as data.
%
%Data that is not able to pour fluid into the bucket is said to be
%``non-conformant'' data. In ATM systems, non-conformant data is either
%discarded, tagged with a low priority for loss (``red'' cells), or can
%be put in a buffer (buffered leaky bucket controller). With the
%Integrated Services Internet, non-conformant data is in principle not
%marked, but simply passed as best effort traffic (namely, normal IP
%traffic).
%\begin{figure}[!htbp]
%\mylabel{leabu}
%  \insfig{leabu}{0.8}
%  \mycaption{A Leaky Bucket Controller. The second part of the figure shows (in grey) the
%  level of the bucket $x(t)$ for a sample input, with $r=0.4$ kbits per time
%  unit and $b=1.5$ kbits. The packet arriving at time $t=8.6$ is not
%  conformant, and no fluid is added to the bucket. If $b$ would be
%  equal to $2$ kbits, then all packets would be conformant.}
%\end{figure}
%
%We want now to show that a leaky bucket controller enforces an
%arrival curve constraint equal to $\gamma_{r,b}$. We need the
%following lemma.
%\begin{lemma}
%        \mylabel{lem-buf-simple}
%        Consider a buffer served at a constant rate $r$. Assume that the
%        buffer is empty at time $0$. The input is described by the cumulative
%        function $R(t)$. If there is no overflow during $[0,t]$, the buffer
%        content at time $t$ is given by
%        $$
%        x(t)=\sup_{s : s \leq t} \{ R(t) - R(s) -r (t-s)\}
%        $$
%\end{lemma}
%\pr
%The lemma can be obtained as a special case of
%Corollary~\ref{coro-shaper-buffer-occupancy} on
%page~\pageref{coro-shaper-buffer-occupancy}, however we give here
%a direct proof. First note that for all $s$ such that $s\leq t$,
%$(t-s)r$ is an upper bound on the number of bits output in
%$]s,t]$, therefore:
%$$
%R(t)-R(s) - x(t) +x(s) \leq (t-s)r
%$$
%
%Thus
%$$
%x(t) \geq R(t)-R(s) + x(s) - (t-s)r \geq R(t)-R(s) - (t-s)r
%$$
%
%which proves that $x(t) \geq \sup_{s : s \leq t} \{ R(t) - R(s) -r (t-s)\}$.
%
%Conversely, call $t_{0}$ the latest time at which the buffer was
%empty before time $t$: $$t_{0} = \sup \{s: s\leq t, x(s) = 0\}$$
%(If $x(t)  > 0$ then $t_{0}$ is the beginning of the busy period
%at time $t$). During $]t_{0}, t]$, the queue is never empty,
%therefore it outputs bit at rate $r$, and thus
%\begin{equation} \mylabel{eq-d4-37443}
%        x(t)=x(t_{0 })+R(t)-R(t_{0 }) - (t-t_{0 })r
%\end{equation}
%
%We assume that $R$ is left-continuous (otherwise the
%proof is a little more complex); thus $x(t_{0 })=0$
%and thus $x(t) \leq \sup_{s : s \leq t} \{ R(t) - R(s) -r (t-s)\}$ \qed
%
%Now the content of a leaky bucket behaves exactly like a buffer
%served at rate $r$, and with capacity $b$. Thus, a flow $R(t)$ is
%conformant if and only if the bucket content $x(t)$ never exceeds
%$b$. From Lemma~\ref{lem-buf-simple}, this means that
%$$
%\sup_{s : s \leq t} \{ R(t) - R(s) -r (t-s)\} \leq b
%$$
%which is equivalent to
%$$
%R(t)-R(s) \leq r(t-s) +b
%$$
%for all $s \leq t$. We have thus shown the following.
%\begin{proposition}
%A leaky bucket controller with leak rate $r$ and bucket size $b$
%forces a flow to be constrained by the arrival curve $\gamma_{r,b}$,
%namely:
%\begin{enumerate}
%        \item  the flow of conformant data has $\gamma_{r,b}$ as an arrival
%        curve;
%\item  if the input already has $\gamma_{r,b}$ as an arrival
%        curve, then all data is conformant.
%\end{enumerate}
%\end{proposition}

We will see in Section~\ref{3bornes} a simple interpretation of
the leaky bucket parameters, namely: $r$ is the minimum rate
required to serve the flow, and $b$ is the buffer required to
serve the flow at a constant rate.

Parallel to the concept of leaky bucket is the Generic Cell Rate
Algorithm (GCRA)\index{GCRA (Generic Cell Rate Algorithm!definition},
used with ATM.
\begin{definition}[GCRA ($T,\tau$)]
The Generic Cell Rate
Algorithm (GCRA) with parameters ($T,\tau$) is used with fixed size
packets, called cells, and defines conformant cells as follows.
It takes as input a cell
arrival time {\tt t} and returns {\tt result}. It has an internal
(static) variable {\tt tat} (theoretical arrival time).
\begin{itemize}
        \item  initially, {\tt tat = 0}
        \item  when a cell arrives at time {\tt t}, then
        \begin{verbatim}
                if (t < tat - tau)
                   result = NON-CONFORMANT;
                else {
                     tat = max (t, tat) + T;
                     result = CONFORMANT;
                     }
        \end{verbatim}
\end{itemize}
\end{definition}
Table~\ref{tab-gcra} illustrate the definition of GCRA. It
illustrates that $\frac{1}{T}$ is the long term rate that can be
sustained by the flow (in cells per time unit); while $\tau$ is a
tolerance that quantifies how early cells may arrive with respect
to an ideal spacing of $T$ between cells. We see on the first
example that cells may be early by 2 time units (cells arriving at
times 18 to 48), however this may not be cumultated, otherwise the
rate of $\frac{1}{T}$ would be exceeded (cell arriving at time
57).
\begin{table}[htbp]
        \begin{center}
        \begin{tabular}{|r||c|c|c|c|c|c|c|}
                \hline
                arrival time & 0 & 10 & 18 & 28 & 38 & 48 & 57  \\
                \hline
                tat before arrival & 0 & 10 & 20 & 30 & 40 & 50 & 60  \\
                \hline
                result & c &  c  &  c  & c  &
                c  & c &
                non-c   \\
                \hline
        \end{tabular}
         \end{center}
        \begin{center}
        \begin{tabular}{|r||c|c|c|c|c|}
                \hline
                arrival time & 0 & 10 & 15 & 25 & 35  \\
                \hline
                tat before arrival & 0 & 10 & 20 & 20 & 30  \\
                \hline
                result & c &  c  &  non-c  & c  &  c   \\
                \hline
        \end{tabular}
        \end{center}
        \mycaption{Examples for GCRA(10,2). The table gives the cell arrival
        times, the value of the \texttt{tat} internal variable just before
        the cell arrival, and the result for the cell (c = conformant,
        non-c = non-conformant).}
        \protect\mylabel{tab-gcra}
\end{table}

In general, we have the following result, which establishes the
relationship between GCRA and the stair functions $v_{T, \tau}$.
\begin{proposition}
    \mylabel{theo-gcra-step}
Consider a flow, with cumulative function $R(t)$, that generates
packets of constant size equal to $k$ data units, with
instantaneous packet arrivals. Assume time is discrete or time is
continuous and $R$ is left-continuous. The following two
properties are equivalent:
\begin{enumerate}
    \item  the flow is conformant to GCRA($T, \tau$)
    \item  the flow has $(k \; v_{T, \tau})$ as an arrival curve
\end{enumerate}
\end{proposition}
\pr
The proof uses max-plus algebra.
Assume that property 1 holds.  Denote with $\theta_{n}$ the value
of $\texttt{tat}$ just after the arrival of the $n$th packet (or
cell), and by convention $\theta_{0}=0$.
Also call $t_{n}$ the arrival time of the $n$th packet.
From the definition of the GCRA we have $\theta_{n}= \max (t_{n},
\theta_{n-1}) + T$. We write this equation for all $m \leq n$, using
the notation $\vee$ for $\max$. The distributivity of addition with
respect to $\vee$ gives:
$$
\bracket
  {
  \theta_{n} = (\theta_{n-1}+T) \vee (t_{n}+T) \\
  \theta_{n-1}+T = (\theta_{n-2}+2T) \vee (t_{n-1}+2T) \\
  \ldots \\
  \theta_{1}+(n-1)T = (\theta_{0}+nT) \vee (t_{1}+nT)
  }
$$ Note that $(\theta_{0}+nT) \vee (t_{1}+nT) = t_{1}+nT$ because
$\theta_{0}=0$ and $t_{1} \geq 0$, thus the last equation can be
simplified to $\theta_{1}+(n-1)T =t_{1}+nT$. Now the iterative
substitution of one equation into the previous one, starting from
the last one, gives
\begin{equation} \mylabel{eq-thetan}
    \theta_{n} = (t_{n}+T) \vee (t_{n-1}+2T) \vee \ldots \vee (t_{1}+nT)
\end{equation}
Now consider the $(m+n)$th arrival, for some $m, n \in \Nats$, with
$m\geq 1$. By
property 1, the packet is conformant, thus
 \begin{equation} \mylabel{eq-l20-as8}
    t_{m+n} \geq \theta_{m+n-1} -\tau
 \end{equation}
Now from Equation~(\ref{eq-thetan}), $\theta_{m+n-1} \geq
t_{j}+(m+n-j)T$ for all $1 \leq j\leq m+n-1$. For $j=m$, we obtain
$\theta_{m+n-1}\geq t_{m}+ nT$. Combining this with
Equation~(\ref{eq-l20-as8}), we have $t_{m+n} \geq t_{m} + nT -
\tau$. With proposition~\ref{theo-l20-jud}, this shows property 2.

Conversely, assume now that property 2 holds. We show by induction
on $n$ that the $n$th packet is conformant. This is always true
for $n=1$. Assume it is true for all $m \leq n$. Then, with the
same reasoning as above, Equation~(\ref{eq-thetan}) holds for $n$.
We rewrite it as $\theta_{n}= \max_{1 \leq j \leq n
}\{t_{j}+(n-j+1)T \}$. Now from proposition~\ref{theo-l20-jud},
$t_{n+1} \geq t_{j}+(n-j+1)T - \tau$ for all $1 \leq j \leq n$,
thus $t_{n+1} \geq \max_{1 \leq j \leq n }\{t_{j}+(n-j+1)T \}-
\tau$. Combining the two, we find that $t_{n+1} \geq \theta_{n} -
\tau$, thus the $(n+1)$th packet is conformant. \qed

Note the analogy between Equation~(\ref{eq-thetan}) and
Lemma~\ref{lem-buf-simple}.  Indeed, from
proposition~\ref{theo-gamma-u}, for packets of constant size,
there is equivalence between arrival constraints by affine
functions $\gamma_{r,b}$ and by stair functions $v_{T, \tau}$.
This shows the following result.
\begin{corollary}
    \mylabel{coro-lb-eq-gcra}
    For a flow with packets of constant size, satisfying the
GCRA($T,\tau$) is equivalent to satisfying a leaky bucket
controller, with rate $r$ and burst tolerance $b$ given by:
$$ b = (\frac{\tau}{T} + 1) \delta $$
$$ r = \frac{\delta}{T} $$
In the formulas, $\delta$ is the packet size in units of data.
\end{corollary}
The corollary can also be shown by a direct equivalence of the GCRA
algorithm to a leaky bucket controller%
\ifexos
~(see Problem~\ref{L20-gcra2}).%
\else
.%
\fi

Take the ATM cell as unit of data. The results above show that for
an ATM cell flow, being conformant to GCRA($T, \tau$) is
equivalent to having $v_{T, \tau}$ as an arrival curve. It is also
equivalent to having $\gamma_{r,b}$ as an arrival curve, with
$r=\frac{1}{T}$ and $b=\frac{\tau}{T}+1$.

Consider a family of $I$ leaky bucket controllers (or GCRAs), with
parameters $r_{i}, b_{i}$, for $1 \leq i \leq I$. If we apply all
of them in parallel to the same flow, then the conformant data is
data that is conformant for each of the controllers in isolation.
The flow of conformant data has as an arrival curve
$$\alpha(t) = \min_{1 \leq i \leq I}(\gamma_{r_{i}, b_{i}}(t))=
\min_{1 \leq i \leq I}  (r_{i}t + b_{i})$$

It can easily be shown that the family of arrival curves that can be
obtained in this way is the set of concave, piecewise linear
functions, with a finite number of pieces.  We will see in
Section~\ref{shapers} some examples of functions that do not belong to
this family.

\paragraph{Application to ATM and the Internet}

Leaky buckets and GCRA are used by standard bodies to define
conformant flows in Integrated Services Networks. With ATM, a
constant bit rate connection (CBR) is defined by one GCRA (or
equivalently, one leaky bucket), with parameters $(T, \tau)$. $T$
is called the ideal cell interval, and $\tau$ is called the Cell
Delay Variation Tolerance (CDVT)\index{CDVT (cell delay variation
tolerance)}. Still with ATM, a variable bit rate (VBR) connection
is defined as one connection with an arrival curve that
corresponds to 2 leaky buckets or GCRA controllers. The Integrated
services framework of the Internet (Intserv) uses the same family
of arrival curves, such as
\begin{equation}\mylabel{eq-tspec}
 \alpha(t) = \min(M+pt, rt+b)
\end{equation}
where $M$ is interpreted as the maximum packet size, $p$ as the
peak rate, $b$ as the burst tolearance, and $r$ as the sustainable
rate (Figure~\ref{fig-vbrenv}). In Intserv jargon, the 4-uple
$(p,M,r,b)$ is also called a T-SPEC (traffic specification).
\index{T-SPEC (traffic specification)}
\begin{figure}[!htbp]
    \insfig{figvbr}{0.4}
    \mycaption{Arrival curve for ATM VBR and
    for Intserv flows}
    \protect\mylabel{fig-vbrenv}
 \end{figure}

\subsection{Sub-additivity and Arrival Curves}

In this Section we discover the fundamental relationship between
min-plus algebra and arrival curves. Let us start with a
motivating example.

Consider a flow $R(t) \in \Nats$ with $t\in \Nats$; for example
the flow is an ATM cell flow, counted in cells. Time is discrete
to simplify the discussion. Assume that we know that the flow is
constrained by the arrival curve $3v_{10,0}$; for example, the
flow is the superposition of 3 CBR connections of peak rate $0.1$
cell per time unit each. Assume in addition that we know that the
flow arrives at the point of observation over a link with a
physical characteristic of 1 cell per time unit. We can conclude
that the flow is also constrained by the arrival curve $v_{1,0}$.
Thus, obviously, it is constrained by $\alpha_{1}=\min(3v_{10,0},
v_{1,0})$. Figure~\ref{fig-alpha-sadd} shows the function
$\alpha_{1}$.


\begin{figure}[!htbp]
\insfig{alphasadd}{0.9}
    \mycaption{The arrival curve
    $\alpha_{1}=\min(3v_{10,0}, v_{1,0})$ on the left, and its sub-additive closure
    (``good'' function) $\bar{\alpha_{1}}$ on the right.
    Time is discrete, lines are put for ease of reading.}
    \protect\mylabel{fig-alpha-sadd}
\end{figure}
Now the arrival curve $\alpha_{1}$ tells us that $R(10)\leq 3$ and $R(11)
\leq 6$. However, since there can arrive at most 1 cell per time
unit , we can also conclude that $R(11) \leq R(10) + [R(11)-R(10)]
\leq \alpha_{1}(10) + \alpha_{1}(1) = 4$. In other words, the sheer knowledge
that $R$ is constrained by $\alpha_{1}$ allows us to derive a better
bound than $\alpha_{1}$ itself. This is because $\alpha_{1}$ is not a
``good'' function, in a sense that we define now.

\begin{definition}
\mylabel{def-subaddi}
    Consider a function $\alpha$ in $\calF$.
    We say that $\alpha$ is a ``good'' function if any one of
    the following equivalent properties is satisfied%
\index{Good function}%
    \begin{enumerate}
        \item  $\alpha$ is sub-additive and $\alpha(0)=0$
        \item  $\alpha= \alpha \mpc \alpha$
        \item $\alpha \mpd \alpha = \alpha$
        \item  $\alpha = \bar{\alpha}$ (sub-additive closure of $\alpha$).
    \end{enumerate}
\end{definition}

The definition uses the concepts of sub-additivity, min-plus
convolution, min-plus deconvolution and sub-additive closure,
which are defined in Chapter~\ref{L10}. The equivalence between
the four items comes from
Corollaries~\ref{cor:sub-additiveclosure} on
page~\pageref{cor:sub-additiveclosure}
and~\ref{thm:mpdsub-additive} on
page~\pageref{thm:mpdsub-additive}. Sub-additivity (item 1) means
that $\alpha(s+t) \leq \alpha(s) + \alpha(t)$.  If $\alpha$ is not
sub-additive, then $\alpha(s) + \alpha(t)$ may be a better bound
than $\alpha(s+t)$, as is the case with $\alpha_{1}$ in the
example above. Item 2, 3 and 4 use the concepts of min-plus
convolution, min-plus deconvolution and sub-additive closure,
defined in Chapter~\ref{L10}. We know in particular
(Theorem~\ref{thm:sub-additiveclosure}) that the sub-additive
closure of a function $\alpha$ is the largest ``good'' function
$\bar{\alpha}$ such that $\bar{\alpha} \leq \alpha$. We also know
that $\bar{\alpha}\in \calF$ if $\alpha \in \calF $.

The main result about arrival curves is that {\em any} arrival curve
can be replaced by its sub-additive closure, which is a ``good'' arrival
curve. Figure~\ref{fig-alpha-sadd} shows $\bar{\alpha_{1}}$ for our example
above.
\begin{theorem}[Reduction of Arrival Curve to a
    Sub-Additive One] Saying that a flow is constrained by a
    wide-sense increasing function $\alpha$ is equivalent to saying that
    it is constrained by the sub-additive closure $\bar{\alpha}$.
    \mylabel{theo-goodarrcur}
\end{theorem}
The proof of the theorem leads us to the heart of the concept of
arrival curve, namely, its correspondence with  a fundamental,
linear relationships in min-plus algebra, which we will now
derive.
\begin{lemma}
\mylabel{lem-l20-lwius}
    A flow $R$ is constrained by arrival curve $\alpha$ if and only if
    $R \leq R \mpc \alpha$
\end{lemma}
\pr Remember that an equation such as $R \leq R \mpc \alpha$ means
that for all times $t$, $R(t) \leq (R \mpc \alpha)(t)$.  The
min-plus convolution $R \mpc \alpha$ is defined in
Chapter~\ref{L10}, page~\pageref{def:minplusconvolution}; since
$R(s)$ and $\alpha(s)$ are defined only for $s \geq 0$, the
definition of $R \mpc \alpha$ is: $(R \mpc \alpha)(t) = \inf_{0
\leq s \leq t} (R(s) + \alpha(t-s))$. Thus $R \leq R \mpc \alpha$
is equivalent to $R(t) \leq R(s) + \alpha(t-s)$ for all $0 \leq s
\leq t$. \qed
\begin{lemma}
    If $\alpha_{1}$ and
    $\alpha_{2}$ are arrival curves for a flow $R$, then so is
    $\alpha_{1} \mpc \alpha_{2}$
\end{lemma}
\pr We know from Chapter~\ref{L10} that $\alpha_{1}
\mpc \alpha_{2}$ is wide-sense increasing if $\alpha_{1}$ and
$\alpha_{2}$ are.  The rest of the proof follows immediately from
Lemma~\ref{lem-l20-lwius} and the associativity of $\mpc$.
\qed
\paragraph{Proof of Theorem}
Since $\alpha$ is an arrival curve, so is $\alpha \mpc \alpha$,
and by iteration, so is $\alpha^{(n)}$ for all $n \geq 1$. By the
definition of $\delta_{0}$, it is also an arrival curve. Thus so
is $\bar{\alpha}=\inf_{n\geq 0}\alpha^{(n)}$.

Conversely, $\bar{\alpha} \leq \alpha$; thus, if $\bar{\alpha}$ is
an arrival curve, then so is $\alpha$. \qed

\paragraph{Examples}

We should thus restrict our choice of arrival curves to
sub-additive functions.  As we can expect, the functions
$\gamma_{r,b}$ and $v_{T, \tau}$ introduced in
Section~\ref{arrcur-def} are sub-additive and since their value is
$0$ for $t=0$, they are ``good'' functions, as we now show.
Indeed, we know from Chapter~\ref{L20} that any concave function
$\alpha$ such that $\alpha(0)=0$ is sub-additive.  This explains
why the functions $\gamma_{r,b}$ are sub-additive.

Functions $v_{T, \tau}$ are not concave, but they still are
sub-additive. This is because, from its very definition, the
ceiling function is sub-additive, thus $$v_{T, \tau}(s+t)=\lceil
\frac{s+t+\tau}{T} \rceil \leq  \lceil \frac{s+\tau}{T} \rceil +
\lceil \frac{t}{T} \rceil \leq \lceil \frac{s+\tau}{T} \rceil +
\lceil \frac{t + \tau}{T} \rceil = v_{T, \tau}(s) + v_{T, \tau}(t)
$$ Let us return to our introductory example with $\alpha_{1}=
\min(3v_{10,0}, v_{1,0})$.  As we discussed, $\alpha_{1}$ is not
sub-additive.  From Theorem~\ref{theo-goodarrcur}, we should thus
replace $\alpha_{1}$ by its sub-additive closure
$\bar{\alpha_{1}}$, which can be computed by
Equation~(\ref{eq:sub-additiveclosure}). The computation is
simplified by the following remark, which follows immediately from
Theorem ~\ref{thm:propertiessub-additiveclosure}:
\begin{lemma}
    \mylabel{lem-cmpsou}
    Let $\gamma_{1}$ and
$\gamma_{2}$ be two ``good'' functions. The sub-additive closure of
$\min (\gamma_{1},\gamma_{2}) $ is
$\gamma_{1} \mpc \gamma_{2}$.
\end{lemma}
We can apply the lemma to $\alpha_{1}= 3v_{10,0} \wedge v_{1,0}$,
since $v_{T, \tau}$ is a ``good'' function. Thus
$\bar{\alpha_{1}}= 3v_{10,0}\mpc v_{1,0}$, which the alert reader
will enjoy computing. The result is plotted in
Figure~\ref{fig-alpha-sadd}.

Finally, let us mention the following equivalence, the proof of
which is easy and left to the reader.

\begin{proposition}
\mylabel{theo-greedy} For a given wide-sense increasing function
$\alpha$, with $\alpha(0)=0$, consider a source defined by
$R(t)=\alpha(t)$ (greedy source). The source has $\alpha$ as an arrival curve if and only if $\alpha$ is a ``good" function.%
\index{greedy source}
\end{proposition}

\paragraph{VBR arrival curve}
Now let us examine the family of arrival curves obtained by
combinations of leaky buckets or GCRAs (concave piecewise linear
functions). We know from Chapter~\ref{L10} that if $\gamma_{1}$
and $\gamma_{2}$ are concave, with
$\gamma_{1}(0)=\gamma_{2}(0)=0$, then $\gamma_{1} \mpc \gamma_{2}
= \gamma_{1} \wedge \gamma_{2}$. Thus any concave piecewise linear
function $\alpha$ such that $\alpha(0)=0$ is a ``good'' function.
In particular, if we define the arrival curve for VBR connections
or Intserv flows by
$$
\bracket{
\alpha(t) = \min (pt + M, rt +b) \gap \mif t >0\\
 \alpha(0) = 0
} $$ (see Figure~\ref{fig-vbrenv}) then $\alpha$ is a ``good"
function.

%\mymarginpar{Skip these two paragraphs at a first reading}%
%\petitcar{%
We have seen in Lemma~\ref{lem-red-lc} that an arrival curve
$\alpha$ can always be replaced by its limit from the left
$\alpha_l$. We might wonder how this combines with the
sub-additive closure, and in particular, whether these two
operations commute (in other words, do we have $(\bar{\alpha})_l =
\overline{\alpha _l}$~?). In general, if $\alpha$ is
left-continuous, then we cannot guarantee that $\bar{\alpha}$ is
also left-continuous, thus we cannot guarantee that the operations
commute. However, it can be shown that $(\bar{\alpha})_l$ is
always a ``good" function, thus
$\overline{(\bar{\alpha})_l}=(\bar{\alpha})_l$. Starting from an
arrival curve $\alpha$ we can therefore improve by taking the
sub-additive closure first, then the limit from the left. The
resulting arrival curve $(\bar{\alpha})_l$ is a ``good" function
that is also left-continuous (a ``very good" function%
\index{Very good function}), and the constraint by $\alpha$ is
equivalent to the constraint by $(\bar{\alpha})_l$

Lastly, let us mention that it can easily be shown, using an
argument of uniform continuity, that if $\alpha$ takes only a
finite set of values over any bounded time interval, and if
$\alpha$ is left-continuous, then so is $\bar{\alpha}$ and then we
do have $(\bar{\alpha})_l = \overline{\alpha _l}$. This assumption
is always true in discrete time, and in most cases in practice.
%}%% end of peticars

\subsection{Minimum Arrival Curve}
Consider now a given flow $R(t)$, for which we would like to
determine a minimal arrival curve. This problem arises, for
example, when $R$ is known from measurements. The following
theorem says that there is indeed one minimal arrival curve.
\begin{theorem}[Minimum Arrival Curve]\mylabel{theo-mac}
    Consider a flow $R(t)_{t \geq 0}$. Then
    \begin{itemize}
        \item  function $R \mpd R$ is an arrival curve for the flow
        \item  for any arrival curve $\alpha$ that constrains the flow, we
        have: $(R \mpd R) \leq \alpha$
        \item  $R \mpd R$ is a ``good" function
    \end{itemize}
    Function $R \mpd R$ is called the \emph{minimum arrival curve}
    for flow $R$.
\end{theorem}
The minimum arrival curve uses min-plus deconvolution, defined in
Chapter~\ref{L10}. Figure~\ref{minarrcur} shows an example of $R
\mpd R$ for a measured function $R$.
\pr
By definition of $\mpd$, we have
 $(R\mpd R)(t)
= \sup_{v \geq 0} \{ R(t+v) -R(v) \}$, it follows that $(R \mpd
R)$ is an arrival curve.

Now assume that some $\alpha$ is also an arrival curve for $R$.
From Lemma~\ref{lem-l20-lwius}, we have $R \leq R \mpc \alpha)$.
From Rule~14 in Theorem~\ref{thm:rule11-14} in Chapter~\ref{L10},
it follows that $R \mpd R \leq \alpha$, which shows that $R\mpd R$
is the minimal arrival curve for $R$. Lastly, $R\mpd R$ is a
``good" function from Rule 15 in  Theorem~\ref{thm:rule11-14}.
\qed

\begin{figure}[!htbp]
 \insfig{trace0}{0.6}
 \insfig{trace1}{0.6}
 \insfig{mina01}{0.6}
  \mylabel{minarrcur}
  \mycaption{Example of minimum arrival curve.
  Time is discrete, one
  time unit is 40 ms.
  The top figures shows, for two similar traces, the
  number of packet arrivals at every time slot.
  Every packet is of constant size (416 bytes). The bottom figure
  shows the minimum arrival curve for the first trace (top curve) and
  the second trace (bottom curve). The large burst in the first trace
  comes earlier, therefore its minimum arrival curve is slightly larger.}
\end{figure}
Consider a greedy source, with $R(t)=\alpha(t)$, where $\alpha$ is
a ``good" function. What is the minimum arrival curve~?%
\footnote{Answer: from the equivalence in
Definition~\ref{def-subaddi}, the minimum arrival curve is
$\alpha$ itself.} Lastly, the curious reader might wonder whether
$R \mpd R$ is left-continuous. The answer is as follows. Assume
that $R$ is either right or left-continuous. By
lemma~\ref{lem-red-lc}, the limit from the left $(R \mpd R)_l$ is
also an arrival curve, and is bounded from above by $R \mpd R$.
Since $R \mpd R$ is the minimum arrival curve, it follows that $(R
\mpd R)_l=R \mpd R$, thus $R \mpd R$ is left-continuous (and is
thus a ``very good" function).

In many cases, one is interested not in the absolute minimum
arrival curve as presented here, but in a minimum arrival curve
within a family of arrival curves, for example, among all
$\gamma_{r,b}$ functions. For a development along this line, see
\cite{naudts99}.


\section{Service Curves}
\mylabel{sec-sercur}
\subsection{Definition of Service Curve}
\mylabel{sec-sercurdef} We have seen that one first principle in
integrated services networks is to put arrival curve constraints
on flows. In order to provide reservations, network nodes in
return need to offer some guarantees to flows. This is done by
packet schedulers \cite{keshav-96}. The details of packet
scheduling are abstracted using the concept of service curve,
which we introduce and study in this section. Since the concept of
service curve is more abstract than that of arrival curve, we
introduce it on some examples.

A first, simple example of a scheduler is a Generalized Processor
Sharing (GPS) node \cite{pg93}\index{GPS (generalized processor
sharing}. We define now a simple view of GPS; more details are
given in Chapter~\ref{L21}. A GPS node serves several flows in
parallel, and we can consider that every flow is allocated a given
rate. The guarantee is that during a period of duration $t$,  for
which a flow has some backlog in the node, it receives an amount
of service at least equal to $r t$, where $r$ is its allocated
rate. A GPS node is a theoretical concept, which is not really
implementable, because it relies on a fluid model, while real
networks use packets. We will see in Section~\ref{sec-gps} on
page~\pageref{sec-gps} how to account for the difference between a
real implementation and GPS. Consider a input flow $R$, with
output $R^*$, that is served in a GPS node, with allocated rate
$r$. Let us also assume that the node buffer is large enough so
that overflow is not possible. We will see in this section how to
compute the buffer size required to satisfy this assumption. Lossy
systems are the object of Chapter~\ref{L30}. Under these
assumptions, for all time $t$, call $t_0$ the beginning of the
last busy period for the flow up to time $t$. From the GPS
assumption, we have
$$R^{*}(t)-R^{*}(t_{0})\geq r(t-t_{0})$$
Assume as usual that $R$ is left-continuous; at time $t_0$ the
backlog for the flow is $0$, which is expressed by $R(t_0) -
R^*(t_0)=0$. Combining this with the previous equation, we obtain:
$$R^{*}(t)-R(t_{0})\geq r(t-t_{0}) $$
We have thus shown that, for all time $t$: $R^{*}(t) \geq
\inf_{0\leq s \leq t} [R(s) + r(t-s)]$, which can be written as
\begin{equation}\mylabel{eq-gps}
  R^{*} \geq R\mpc \gamma_{r,0}
\end{equation}
Note that a limiting case of GPS node is the constant bit rate
server with rate $r$, dedicated to serving a single flow. We will
study GPS in more details in Chapter~\ref{L21}.

Consider now a second example. Assume that the only information we
have about a network node is that the maximum delay for the bits
of a given flow $R$ is bounded by some fixed value $T$, and that
the bits of the flow are served in first in, first out order. We
will see in Section~\ref{shapers} that this is used with a family
of schedulers called ``earliest deadline first" (EDF). We can
translate the assumption on the delay bound to $d(t) \leq T$ for
all $t$.  Now since $R^*$ is always wide-sense increasing, it
follows from the definition of $d(t)$ that $R^*(t+T) \geq R(t)$.
Conversely, if $R^*(t+T) \geq R(t)$, then $d(t) \leq T$. In other
words, our condition that the maximum delay is bounded by $T$ is
equivalent to $R^*(t+T) \geq R(t)$ for all $t$. This in turn can
be re-written as $$R^*(s) \geq R(s-T) $$for all $s \geq T$. We
have introduced in Chapter~\ref{L10} the ``impulse" function
$\delta_{T}$ defined by $\delta_{T}(t)=0$ if $0\leq t \leq T$ and
$\delta_{T}(t)=+\infty $ if $ t > T$. It has the property that,
for any wide-sense increasing function $x(t)$, defined for $t \geq
0$, $(x \mpc \delta_T)(t)=x(t-T)$ if $ t \geq T$ and $(x \mpc
\delta_T)(t)=x(0)$ otherwise. Our condition on the maximum delay
can thus be written as
\begin{equation}\mylabel{eq-delaynode2}
 R^* \geq R\mpc \delta_T
\end{equation}

For the two examples above, there is an input-output relationship
of the same form (Equations~(\ref{eq-gps}) and
(\ref{eq-delaynode2})). This suggests the definition of service
curve, which, as we see in the rest of this section, is indeed
able to provide useful results.
\begin{figure}[!htbp]
  \insfig{sercurd}{0.8}
  \mycaption{Definition of service curve. The output $R^*$ must be
  above $R \mpc \beta$, which is
  the lower
  envelope of all curves $t \mapsto R(t_0) + \beta(t-t_0)$.}
  \mylabel{fig-sercurd}
\end{figure}

\begin{definition}[Service Curve]
\mylabel{def-sercur} Consider a  system $\mathcal{S}$ and a
flow through $\mathcal{S}$ with input and output function $R$
and $R^{*}$. We say that $\mathcal{S}$ offers to the flow a
\emph{service curve} $\beta$ if and only if $\beta$ is wide
sense increasing, $\beta(0)=0$ and $R^{*} \geq R \mpc \beta$
\end{definition}
Figure~\ref{fig-sercurd} illustrates the definition.

The definition means that $\beta$ is a wide sense increasing
function, with $\beta(0)=0$, and that for all $t\geq 0$,
%\begin{equation}\mylabel{eq-sercur-d21}
$$
  R^{*}(t) \geq \inf_{s \leq t} \left(R(s)  + \beta(t-s)\right)
$$
In practice, we can avoid the use of an infimum if $\beta$ is
continuous. The following proposition is an immediate consequence
of \thref{theo-minplust0} on \pgref{theo-minplust0}.
\begin{proposition}
\mylabel{prop-sercurt0} If $\beta$ is continuous, the service
curve property means that for all $t$ we can find $t_0 \leq t$
such that
\begin{equation}\mylabel{eq-sercur-d2a}
  R^{*}(t) \geq R_l(t_{0}) + \beta(t-t_{0})
\end{equation}
where $R_l(t_0)= \sup_{\{s < t_0\}} R(s)$ is the limit from the left
of $R$ at $t_0$. If $R$ is left-continuous, then
$R_l(t_0)=R(t_0)$.
\end{proposition}

For a constant rate server (and also for any \emph{strict} service
curve), the number $t_0$ in \eref{eq-sercur-d2a} can be taken as the
beginning of the busy period, for other cases, we do not know.
However, in some cases we can pick a $t_0$ that increases with $t$:

\begin{proposition}
\mylabel{prop-t0augmente} If the service curve $\beta$ is convex,
then we can find some wide sense increasing function $\tau(t)$
such that we can choose $t_0=\tau(t)$ in \eref{eq-sercur-d2a}.
\end{proposition}
Note that since a service curve is assumed to be wide-sense
increasing, $\beta$, being convex, is necessarily continuous; thus
we can apply \pref{prop-sercurt0}.
\pr
We give the proof when $R$ is left-continuous. The proof for the
general case is essentially the same but involves some $\epsilon$
cutting. Consider some $t_1 < t_2$ and call $\tau_1$ a value of
$t_0$ as in \eref{eq-sercur-d2a}) at $t=t_1$. Also consider any
$t' \leq \tau_1$.  From the definition of $\tau_1$, we have
$$
R^*(t') + \beta(t_1 - t' ) \geq R^*(\tau_1) + \beta(t_1 - \tau_1 )
$$
and thus
$$
R^*(t') + \beta(t_2 - t' ) \geq R^*(\tau_1) + \beta(t_1 - \tau_1 )
- \beta(t_1 - t' ) + \beta(t_2 - t' )
$$
Now $\beta$ is convex, thus for any four numbers $a, b, c, d$ such
that $a \leq c  \leq b$,  $a \leq d  \leq b$ and $a+b=c+d$, we
have
$$\beta(a) + \beta(b) \geq \beta(c) + \beta(d)$$
(the interested reader will be convinced by drawing a small
figure). Applying this to $a=t_1- \tau_1, b=t_2 - t', c=t_1-t',
d=t_2-\tau_1$ gives
$$
R^*(t')+ \beta(t_2-t') \geq R^*(\tau_1)+\beta(t_2 - \tau_1)
$$
and the above equation holds for all $t'\leq \tau_1$. Consider now
the minimum, for a fixed $t_2$, of $R^*(t') +\beta(t_2-t')$ over
all $t' \leq t_2$. The above equation shows that the minimum is
reached for some $t' \geq \tau_1$. \qed

We will see in Section~\ref{bounds} that the combination of a
service curve guarantee with an arrival curve constraint forms the
basis for deterministic bounds used in integrated services
networks. Before that, we give the fundamental service curve
examples that are used in practice.


\subsection{Classical Service Curve Examples}
\mylabel{sec-clasc}

\paragraph{Guaranteed Delay Node}
\index{guaranteed delay node} The analysis of the second example
in Section~\ref{sec-sercurdef} can be rephrased as follows.
\begin{proposition}
    For a lossless bit processing system, saying that the delay
    for any bit is bounded by some fixed $T$ is equivalent to saying
    that the system offers to the flow a service curve equal to
    $\delta_{T}$.
\end{proposition}
\paragraph{Non Premptive Priority Node}
\index{Priority Node}
 Consider a node that serves two flows,
$R_H(t)$ and $R_L(t)$. The first flow has non-preemptive priority
over the second one (\fref{fig-HOL}). This example explains the
general framework used when some traffic classes have priority
over some others, such as with the Internet differentiated
services \cite{RFC2475}. The rate of the server is constant, equal
to $C$. Call $R_H^*(t)$ and $R_L^*(t)$ the outputs for the two
flows.
\begin{figure}[!htbp]
    \insfig{sercurhol}{0.6}
    \mycaption{Two priority flows (H and L) served with a preemptive head of
    the line (HOL) service discipline. The high priority flow is constrained by
    arrival curve $\alpha$.}
\mylabel{fig-HOL}
\end{figure}
Consider first the high priority flow. Fix some time $t$ and call
$s$ the beginning of the backlog period for high priority traffic.
The service for high priority can be delayed by a low priority
packet that arrived shortly before $s'$, but as soon as this
packet is served, the server is dedicated to high priority as long
as there is some high priority traffic to serve. Over the interval
$(s,t]$, the output is $C(t-s)$Thus
$$R_H^*(t)- R_H^*(s) \geq C(t-s) -l^H_{\max}
$$
where $l^L_{\max}$ is the maximum size of a low priority packet.
Now by definition of $s$: $R_H^*(s)=R_H(s)$ thus
$$R_H^*(t)\geq R_H(s) + C(t-s) -l^L_{\max}
$$
Now we have also
$$R_H^*(t) - R_H(s) = R_H^*(t) - R_H^*(s) \geq 0$$
from which we derive
$$R_H^*(t)\geq R_H(s) + [C(t-s) -l^L_{\max}]^+$$
The function $u \rightarrow [Cu -l^L_{\max}]^+$ is called the
rate-latency function with rate $C$ and latency
$\frac{l^L_{\max}}{C}$ \cite{SV96} (in this book we note it
$\beta_{C,\frac{l^L_{\max}}{C}}$, see also
Figure~\ref{fig:examplesoffunctions} on
page~\pageref{fig:examplesoffunctions}). Thus the high priority
traffic receives this function as a service curve.

Now let us examine low priority traffic. In order to assure that
it does not starve, we assume in such situations that the high
priority flow is constrained by an arrival curve $\alpha_H$.
Consider again some arbitrary time $t$. Call $s'$ the beginning of
the server busy period (note that $s' \leq s$). At time $s'$, the
backlogs for both flows are empty, namely, $R_H^*(s')=R_H(s')$ and
$R_L^*(s')=R_L(s')$. Over the interval $(s',t]$, the output is
$C(t-s')$. Thus
 $$R_L^*(t) - R_L^*(s') = C(t-s') - \left[R_H^*(t) -R_H^*(s') \right]
 $$
Now
$$ R_H^*(t) - R_H^*(s') = R_H^*(t) - R_H(s') \leq R_H(t) - R_H(s')
\leq \alpha_H(t-s')
$$
and obviously $R_H^*(t) - R_H^*(s') \geq 0$ thus
$$R_L^*(t) - R_L(s') = R_L^*(t) - R_L^*(s') \geq S(t-s')
$$
with $S(u) = \left(Cu -\alpha_H(u) \right)^+$. Thus, if $S$ is
wide-sense increasing, the low-priority flow receives a service
curve equal to function $S$. Assume further that $\alpha_H =
\gamma_{r,b}$, namely, the high priority flow is constrained by
one single leaky bucket or GCRA. In that case, the service curve
$S(t)$ offered to the low-priority flow is equal to the
rate-latency function $\beta_{R,T}(t)$, with $R=C-r$ and
$T=\frac{b}{C-r}$.

We have thus shown the following.
\begin{proposition}
Consider a constant bit rate server, with rate $C$, serving two
flows, $H$ and $L$, with non-preemptive priority given to flow
$H$. Then the high priority flow is guaranteed a rate-latency
service curve with rate $C$ and latency $\frac{l^L_{\max}}{C}$
where $l^L_{\max}$ is the maximum packet size for the low priority
flow.

If in addition the high priority flow is $\gamma_{r,b}$-smooth,
with $r<C$, then the low priority flow is guaranteed a
rate-latency service curve with rate $C-r$ and latency
$\frac{b}{C-r}$. \mylabel{prop-prionode}
\end{proposition}

This example justifies the importance of the rate-latency service
curve. We will also see in Chapter~\ref{L21}
(Theorem~\ref{theo-GR} on page~\pageref{theo-GR}) that all
practical implementations of GPS offer a service curve of the
rate-latency type.


%Let us introduce the following
%definition (Figure~\ref{fig-ratelat}).%
%%\index{rate latency function}%
%%\index{1S@$S_{R,T}$ (rate-latency function)}%
%\begin{definition}
%\mylabel{def-2dfjS0} The ``rate-latency'' function $S_{R,T}$ is
%defined by $$
%    \beta_{R,T}(t)=  R(t-T)^{+}= R (t-T) \mif t \geq T \mand 0 \mif  t < T
%$$
%\end{definition}
%\begin{figure}[!htbp]
%  \insfig{ratelat}{0.3}
%  \mylabel{fig-ratelat}
%  \mycaption{Definition of rate-latency function $\beta_{R,T}$
%  with rate $R$ and latency $T$.}
%\end{figure}

%in Chapter~\ref{L10}.

\paragraph{Strict service curve}
An important class of network nodes fits in the following
framework.
\begin{definition}[Strict Service Curve]
\mylabel{def-ssc} We say that system $\calS$ offers a strict
service curve $\beta$ to a flow if, during any backlogged period
of duration $u$, the output of the flow is at least equal to
$\beta(u)$.
\end{definition}%
\index{strict service curve} A GPS node is an example of node that
offers a strict service curve of the form $\beta(t)=rt$. Using the
same busy-period analysis as with the GPS example in the previous
section, we can easily prove the following.
\begin{proposition}\mylabel{prop-ssc}
If a node offers $\beta$ as a strict service curve to a flow, then
it also offers $\beta$ as a service curve to the flow.
\end{proposition}

The strict service curve property offers a convenient way of
visualizing the service curve concept: in that case, $\beta(u)$ is
the minimum amount of service guaranteed during a busy period.
Note however that the concept of service curve, as defined in
Definition~\ref{def-sercur} is more general. A greedy shaper
(\sref{ioshaper}) is an example of system that offers its shaping
curve as a service curve, without satisfying the strict service
curve property. In contrast, we will find later in the book some
properties that hold only if a strict service curve applies. The
framework for a general discussion of strict service curves is
given in \cref{L23}.

\paragraph{Variable Capacity Node}
\index{variable capacity node} Consider a network node that offers
a variable service capacity to a flow. In some cases, it is
possible to model the capacity by a cumulative function $M(t)$,
where $M(t)$ is the total service capacity available to the flow
between times $0$ and $t$. For example, for an ATM system, think
of $M(t)$ as the number of time slots between times $0$ and $t$
that are available for sending cells of the flow. Let us also
assume that the node buffer is large enough so that overflow is
not possible. The following proposition is obvious but important
in practice
\begin{proposition}
 If the variable capacity satisfies a minimum
guarantee of the form
\begin{equation}\mylabel{eq-mot-sc4}
 M(t) - M(s) \geq \beta(t-s)
\end{equation}
for some fixed function $\beta$ and for all $0\leq s \leq t$, then
$\beta$ is a strict service curve,
\end{proposition}
Thus $\beta$ is also a service curve for that particular flow. The
concept of variable capacity node is also a convenient way to
establish service curve properties. For an application to real
time systems (rather than communication networks) see
\cite{thiele98}.

We will show in \cref{L11} that the output of the variable
capacity node is given by
$$
R^*(t) = \inf_{0 \leq  s \leq t} \{M(t)-M(s) + R(s) \}
$$

Lastly, coming back to the priority node, we have:
\begin{proposition}
 The service curve property in \pref{prop-prionode} for the high-priority flow is strict.
 \mylabel{prop-priostric}
\end{proposition}
The proof is left to the reader. It relies on the fact that
constant rate server is a shaper.




%An important class of variable capacity nodes is the priority
%node. Consider a node that serves two flows, $R_H(t)$ and
%$R_L(t)$. The first flow has preemptive priority over the second
%one (Figure \ref{fig-HOL}). This example is simplified, but it
%explains the general framework used when some traffic classes have
%priority over some others, such as with the Internet
%differentiated services \cite{RFC2475}. A similar example with
%non-preemptive priority can be analyzed in the same manner. The
%rate of the server is constant, equal to $C$. Call $R_H^*(t)$ and
%$R_L^*(t)$ the outputs for the two flows. In order to assure that
%the high priority flow does not let the low priority flow starve,
%we assume in such situations that the high priority flow is
%constrained by an arrival curve $\alpha$.
%\begin{figure}[!htbp]
%    \insfig{sercurhol}{0.6}
%    \mycaption{Two priority flows (H and L) served with a preemptive head of
%    the line (HOL) service discipline. The high priority flow is constrained by
%    arrival curve $\alpha$.}
%\mylabel{fig-HOL}
%\end{figure}
%Consider some arbitrary time interval $]s,t]$. Over such an
%interval, the node offers to the low-priority flow a capacity
%equal to $C(t-s) - R_H^*(t) + R_H^*(s)$.  Now we will see in a
%following section that greedy shapers keep arrival constraints,
%and that our system is a greedy shaper for the flow with high
%priority. Thus $R_H^*(t)-R_H^*(s) \leq \alpha(t-s)$ and finally:
%$$R_H^*(t)-R_H^*(s) \leq C(t-s) \wedge \alpha(t-s)$$
%Thus the capacity node offered to the low priority flow is at
%least equal to $S(t-s)$, with $S$ defined by $S(u) = Cu - \left(
%\alpha(u) \wedge Cu \right)$. Thus the low-priority flow receives
%a service curve equal to function $S$.
%
%Assume further that $\alpha = \gamma_{r,b}$, namely the high
%priority flow is constrained by one single leaky bucket or GCRA.
%Let us introduce the following
%definition (Figure~\ref{fig-ratelat}).%
%%\index{rate latency function}%
%%\index{1S@$S_{R,T}$ (rate-latency function)}%
%\begin{definition}
%\mylabel{def-2dfjS0} The ``rate-latency'' function $S_{R,T}$ is
%defined by $$
%    \beta_{R,T}(t)=  R(t-T)^{+}= R (t-T) \mif t \geq T \mand 0 \mif  t < T
%$$
%\end{definition}
%\begin{figure}[!htbp]
%  \insfig{ratelat}{0.3}
%  \mylabel{fig-ratelat}
%  \mycaption{Definition of rate-latency function $\beta_{R,T}$
%  with rate $R$ and latency $T$.}
%\end{figure}
%In that case, the service curve $S(t)$ offered to the low-priority
%flow is equal to the rate-latency function $\beta_{R,T}(t)$, with
%$R=C-r$ and $T=\frac{b}{C-r}$. The rate-latency function
%\cite{SV96} is defined by $\beta_{R,T}(t)=  R(t-T)^+$, see also
%Figure~\ref{fig:examplesoffunctions} on
%page~\pageref{fig:examplesoffunctions}.
%%in Chapter~\ref{L10}.
%
%We will see in Chapter~\ref{L21} (Theorem~\ref{theo-GR} on
%page~\pageref{theo-GR}) that all practical implementations of GPS
%offer a service curve of the rate-latency type.

\section{Network Calculus Basics}
\mylabel{bounds} In this section we see the main simple network
calculus results. They are all bounds for lossless systems with
service guarantees.

\subsection{Three Bounds}
\mylabel{3bornes}
The first theorem says that the backlog is
bounded by the vertical deviation between the arrival and service
curves:
\begin{theorem}[Backlog Bound]
Assume a flow, constrained by arrival curve $\alpha$, traverses a
system that offers a service curve $\beta$.  The backlog
$R(t)-R^{*}(t)$ for all $t$ satisfies: $$R(t)-R^{*}(t) \leq
\sup_{s \geq 0} \{\alpha(s) - \beta(s) \}$$
    \mylabel{theo-backlog}
\end{theorem}
\pr The proof is a straightforward application of the definitions
of service and arrival curves:
$$R(t) - R^{*}(t)
\leq R(t) - \inf_{0 \leq s \leq t} [R(t-s) + \beta(s)]$$ Thus
$$R(t) - R^{*}(t) \leq \sup_{0 \leq s \leq t}[R(t) -
R(t-s) + \beta(s)] \leq \sup_{0 \leq s \leq t}[\alpha(s) +
\beta(t-s)]$$ \qed

We now use the concept of horizontal deviation, defined in
Chapter~\ref{L10}, Equation~(\ref{eq:maxhorizontal}). The
definition is a little complex, but is supported by the following
intuition. Call $$\delta(s)=\inf \left\{ \tau \geq 0:\alpha(s)
\leq \beta(s + \tau) \right\}$$ From Definition~\ref{L20-intro},
$\delta(s)$ is the virtual delay for a hypothetical system that
would have $\alpha$ as input and $\beta$ as output, assuming that
such a system exists (in other words, assuming that ($\alpha \leq
\beta$). Then,  $h(\alpha, \beta)$ is the supremum of all values
of $\delta(s)$. The second theorem gives a bound on delay for the
general case.
\begin{theorem}[Delay Bound]
Assume a flow, constrained by arrival curve $\alpha$, traverses a
system that offers a service curve of $\beta$.  The virtual delay
$d(t)$ for all $t$ satisfies: $d(t) \leq h( \alpha,\beta)$.
    \mylabel{theo-delay}
\end{theorem}
\pr Consider some fixed $t\geq 0$; for all $\tau < d(t)$, we have,
from the definition of  virtual delay, $R(t) > R^*(t+ \tau)$. Now
the service curve property at time $t+ \tau$ implies that there is
some $s_0$ such that
$$
R(t) > R(t + \tau - s_0) + \beta(s_0)
$$
It follows from this latter equation that $t + \tau - s_0 < t$.
Thus
$$
\alpha(\tau - s_0) \geq [R(t) - R(t + \tau - s_0)] > \beta(s_0)
$$
Thus $\tau \leq \delta(\tau - s_0)\leq h( \alpha,\beta)$. This is
true for all $\tau < d(t)$ thus $ d(t)\leq h( \alpha,\beta)$. \qed

\begin{theorem}[Output Flow]
Assume a flow, constrained by arrival curve $\alpha$, traverses a
system that offers a service curve of $\beta$.  The output flow is
constrained by the arrival curve $\alpha^{*}=\alpha \mpd \beta$.
\mylabel{theo-output}
\end{theorem}
The theorem uses min-plus deconvolution, introduced in
Chapter~\ref{L10}, which we have already used in
Theorem~\ref{theo-mac}.

\pr
With the same notation as above, consider $R^{*}(t) - R^{*}(t-s)$,
for $0 \leq t-s \leq t$. Consider the definition of the service
curve, applied at time $t-s$. Assume for a second that the $\inf$
in the definition of $R\mpc \beta$ is a $\min$, that is to say,
there is some $u \geq 0$ such that $0\leq t-s-u$ and
$$(R\mpc \beta )(t-s)=R(t-s-u) + \beta(u)
$$
Thus
$$ R^{*}(t-s) - R(t-s-u) \geq \beta(u)
$$ and thus
$$ R^{*}(t) - R^{*}(t-s) \leq R^{*}(t) - \beta(u) - R(t-s-u) $$
Now $ R^{*}(t) \leq R(t)$,  therefore $$ R^{*}(t) - R^{*}(t-s)
\leq R(t) - R(t-s-u) - \beta(u) \leq \alpha(s+u) -\beta(u) $$ and
the latter term is bounded by $(\alpha \mpd \beta) (s)$ by
definition of the $\mpd$ operator.

Now relax the assumption that the the $\inf$ in the definition of
$R\mpc \beta$ is a $\min$. In this case, the proof is essentially
the same with a minor complication. For all $\epsilon
>0$ there is some $u\geq 0$ such that $0\leq t-s-u$ and
$$(R\mpc \beta )(t-s)\geq R(t-s-u) + \beta(u)- \epsilon$$
and the proof continues along the same line, leading to:
$$ R^{*}(t) - R^{*}(t-s)
\leq (\alpha \mpd \beta) (s) + \epsilon$$ This is true for all
$\epsilon >0$, which proves the result. \qed

\paragraph{A simple Example and Interpretation of Leaky Bucket}
\begin{figure}[!htbp]
  \insfig{calcules}{0.5}
  \mycaption{Computation of buffer, delay and output bounds for an input flow
  constrained by one leaky bucket, served in one node offered a rate-latency
  service curve. If $r \leq R$, then the buffer bound is $x=b + rT$, the delay bound
  is $d= T + \frac{b}{R}$ and
  the burstiness of the flow is increased by $rT$. If $r>R$, the bounds are infinite.}
  \mylabel{fig-calcules}
\end{figure}
Consider a flow constrained by one leaky bucket, thus with an
arrival curve of the form $\alpha=\gamma_{r,b}$, served in a node
with the service curve guarantee $\beta_{R,T}$. The alert reader
will enjoy applying the three bounds and finding the results shown
in Figure~\ref{fig-calcules}.

Consider in particular the case $T=0$, thus a flow constrained by
one leaky bucket served at a constant rate $R$. If $R\geq r$ then
the buffer required to serve the flow is $b$, otherwise, it is
infinite. This gives us a common interpretation of the leaky
bucket parameters $r$ and $b$: $r$ is the minimum rate required to
serve the flow, and $b$ is the buffer required to serve the flow
at any constant rate $\geq r$.


\paragraph{Example: VBR flow with rate-latency service curve}
Consider a VBR flow, defined by T-SPEC $(M, p, r, b)$. This means
that the flow has $\alpha(t) = \min(M+pt, rt+b)$ as an arrival
curve (Section~\ref{con-flows}). Assume that the flow is served in
one node that guarantees a service curve equal to the rate-latency
function $\beta=\beta_{R,T}$. This example is the standard model
used in Intserv. Let us apply Theorems~\ref{theo-backlog}
and~\ref{theo-delay}.  Assume that $R \geq r$, that is, the
reserved rate is as large as the sustainable rate of the flow.

From the convexity of the region between $\alpha$ and $\beta$
(Figure~\ref{fig-calculei}), we see that the vertical deviation
$v=\sup_{s \geq 0}[\alpha(s)-\beta(s)]$ is reached for at an
angular point of either $\alpha$ or $\beta$. Thus
$$v = \max [\alpha(T), \alpha(\theta)-\beta(\theta)]
$$with $\theta =\frac{b-M}{p-r}$.
Similarly, the horizontal distance is reached an angular point. In
the figure, it is either the distance marked as $AA'$ or $BB'$.
Thus, the bound on delay $d$ is given by
$$d = \max \left( \frac{\alpha(\theta)}{R}+T-\theta, \frac{M}{R}+T \right)$$
After some max-plus algebra, we can re-arrange these results as
follows.
\begin{proposition}[Intserv model, buffer and delay bounds]
\mylabel{theo-vbr-ratlat}Consider a VBR flow, with TSPEC $(M, p,
r, b)$, served in a node that guarantees to the flow a service
curve equal to the rate-latency function $\beta=\beta_{R,T}$. The
buffer required for the flow is bounded by
$$v=b + r T + \left(\frac{b-M}{p-r} - T\right)^+ [(p-R)^+ -p + r]
$$ The maximum delay for the flow is bounded by
$$d=\frac{M + \frac{b-M}{p-r}(p-R)^+ }{R} +T$$
\end{proposition}
\begin{figure}[htbp]
  \insfig{calculeI}{0.5}
  \mylabel{fig-calculei}
  \mycaption{Computation of buffer and delay bound for one VBR flow
  served in one Intserv node.}
\end{figure}

We can also apply Theorem~\ref{theo-output} and find an arrival
curve $\alpha^{*}$ for the output flow. We have $\alpha^{*}=\alpha
\mpd (\lambda_{R}\mpc \delta_{T}) = (\alpha \mpd \lambda_{R}) \mpd
\delta_{T}$ from the properties of $\mpd$ (Chapter~\ref{L10}).
Note that
$$ (f \mpd \delta_{T}) (t) = f (t+T) $$ for all $f$
(shift to the left).

The computation of $\alpha \mpd \lambda_{R}$ is explained in
\thref{theo-rep} on \pgref{theo-rep}: it consists in inverting
time, and smoothing. Here, we give however a direct derivation,
which is possible since $\alpha$ is concave. Indeed, for a concave
$\alpha$, define $t_{0}$ as
$$ t_{0}=\inf \{t \geq 0: \alpha'(t) \leq R \} $$ where $\alpha'$
is the left-derivative, and assume that $t_{0} < +\infty$. A
concave function always has a left-derivative, except maybe at the
ends of the interval where it is defined. Then by studying the
variations of the function $u \rightarrow \alpha(t+u) -Ru$ we find
that $(\alpha \mpd \lambda_{R})(s) = \alpha(s) $ if $s \geq
t_{0}$, and $(\alpha \mpd \lambda_{R})(s) = \alpha(t_{0}) +
(s-t_{0})R$ if $s < t_{0}$.
\begin{figure}[!htbp]
    \insfig{F110}{0.7}
    \mycaption{Derivation of arrival curve for the output of a flow
  served in a node with rate-latency service curve $\beta_{R,T}$.}
    \mylabel{fig-outgqos1}
\end{figure}

Putting the pieces all together we see that the output function
$\alpha^{*}$ is obtained from $\alpha$ by
\begin{itemize}
    \item replacing $\alpha$ on $[0, t_{0}]$ by the linear function
    with slope $R$ that has the same value as $\alpha$ for $t=t_{0}$,
    keeping the same values as $\alpha$ on $[t_{0}, +\infty[$,
\item and shifting by $T$ to the left.
\end{itemize}
Figure \ref{fig-outgqos1} illustrates the operation. Note that the
two operations can be performed in any order since $\mpc$ is
commutative. Check that the operation is equivalent to the
construction in \thref{theo-rep} on \pgref{theo-rep}.

If we apply this to a VBR connection, we obtain the following
result.
\begin{proposition}[Intserv model, output bound]
\mylabel{theo-vbr-ratlat2}With the same assumption as in
Proposition~\ref{theo-vbr-ratlat}, the output flow has an arrival
curve $\alpha^*$ given by:
$$\bracket{
 \mif \frac{b-M}{p-r} \leq T \mthen \alpha^*(t) =
 b+r(T+t) \\
 \melse \alpha^*(t) = \min\left\{\ (t+T)(p\wedge R) + M +  \frac{b-M}{p-r}(p-R)^+ ,
 b+r(T+t)
  \right\}
 }
 $$
\end{proposition}

\paragraph{An ATM Example}

Consider the example illustrated in Figure~\ref{fig-atmnetcalex}.
The aggregate flow has as an arrival curve equal to the stair
function $10 v_{25,4}$. The figure illustrates that the required
buffer is $10$ ATM cells and the maximum delay is $18$ time slots.
\begin{figure}[!htbp]
  \insfig{atmnetcalex}{0.7}
  \mycaption{Computation of bounds for buffer $x$ and delay $d$ for an ATM
  example.
  An ATM node serves $10$ ATM connections, each
constrained with GCRA($25, 4$) (counted in time slots). The node
offers to the aggregate flow a service curve $\beta_{R,T}$ with
rate $R=1$ cell per time slot and latency $T=8$ time slots. The
figure shows that
  approximating the
  stair function $10 v_{25, 4}$ by an affine function
  $\gamma_{r,b}$ results into an overestimation of
  the bounds.}
  \mylabel{fig-atmnetcalex}
\end{figure}
We know from Corollary~\ref{coro-lb-eq-gcra} that a GCRA
constraint is equivalent to a leaky bucket. Thus, each of the 10
connections is constrained by an affine arrival curve
$\gamma_{r,b}$ with $r=\frac{1}{25}=0.04$ and
$b=1+\frac{4}{25}=1.16$. However, if we take as an arrival curve
for the aggregate flow the resulting affine function $10
\gamma_{r,b}$, then we obtain a buffer bound of $11.6$ and a delay
bound of $19.6$. The affine function overestimates the buffer and
delay bounds. Remember that the equivalence between stair function
and affine function is only for a flow where the packet size is
equal to the value of the step, which is clearly not the case for
an aggregate of several ATM connections.

A direct application of Theorem~\ref{theo-output} shows that an
arrival curve for the output flow is given by
 $\alpha^*_0(t) =  \alpha(t+T)= v_{25, 12}(t)$.

In Chapter~\ref{L21}, we give a slight improvement to the bounds
if we know that the service curve is a strict service curve.

\subsection{Are the Bounds Tight ?}
We now examine how good the three bounds are. For the backlog and
delay bounds, the answer is simple:
\begin{theorem}
 \mylabel{theo-bist}
Consider the backlog and delay bounds in
Theorems~\ref{theo-backlog} and ~\ref{theo-delay}. Assume that
\begin{itemize}
  \item $\alpha$ is a ``good" function (that is, namely,
  is wide-sense increasing, sub-additive and $\alpha(0)=0$)
  \item $\beta$ is wide-sense increasing and  $\beta(0)=0$
\end{itemize}
Then the bounds are tight. More precisely, there is one causal
system with input flow $R(t)$ and output flow $R^*(t)$, such that
the input is constrained by $\alpha$, offering to the flow a
service curve $\beta$, and which achieves both bounds.
\end{theorem}
A causal system means that $R(t) \leq R^*(t)$. The theorem means
that the backlog bound in Theorem~\ref{theo-backlog} is equal to
  $\sup_{t \geq 0} [R(t)-R^*(t)]$,
  and the delay bound in Theorem~\ref{theo-backlog} is equal to
  $\sup_{t \geq 0} d(t)$.
In the above, $d(t)$ is the virtual delay defined in
Definition~\ref{L20-intro}.
\pr
We build one such system $R, R^*$ by defining $R=\alpha,
R^*=\min(\alpha, \beta)$. The system is causal because $R^*\leq
\alpha=R$. Now consider some arbitrary time $t$. If
$\alpha(t)<\beta(t)$ then $$R^*(t) = R(t) = R(t) + \beta(0)$$
Otherwise, $$R^*(t) =\beta(t)=R(0)+ \beta(t)$$ In all cases, for
all $t$ there is some $s\leq t$ such that $R^*(t) \geq R(t-s) +
\beta(s)$, which shows the service curve property. \qed

Of course, the bounds are as tight as the arrival and service
curves are. We have seen that a source such that $R(t)=\alpha(t)$
is called \emph{greedy}. Thus, the backlog and delay bounds are
worst-case bounds that are achieved for greedy sources.

In practice, the output bound is also a worst-case bound, even
though the detailed result is somehow less elegant.
\begin{theorem} \mylabel{theo-outworst}
Assume that
\begin{enumerate}
  \item $\alpha$ is a ``good" function (that is,
  is wide-sense increasing, sub-additive and $\alpha(0)=0$)
  \item $\alpha$ is left-continuous
  \item $\beta$ is wide-sense increasing and  $\beta(0)=0$
  \item $\alpha \Mpd \alpha $ is not bounded from above.
\end{enumerate}
Then the output bound in Theorem~\ref{theo-output} is tight.  More
precisely, there is one causal system with input flow $R(t)$ and
output flow $R^*(t)$, such that the input is constrained by
$\alpha$, offering to the flow a service curve $\beta$, and
$\alpha^*$ (given by Theorem~\ref{theo-output}) is the
\emph{minimum} arrival curve for $R^*$.
\end{theorem}
We know in particular from Section ~\ref{con-flows} that the first
three conditions are not restrictive. Let us first discuss the
meaning of the last condition. By definition of max-plus
deconvolution:
$$(\alpha \Mpd \alpha)(t)= \inf_{s \geq 0} \{\alpha (t+s) - \alpha(s)  \}
$$
One interpretation of $\alpha \Mpd \alpha$ is as follows. Consider
a greedy source, with $R(t)=\alpha(t)$; then $(\alpha \Mpd
\alpha)(t)$ is the minimum number of bits arriving over an
interval of duration $t$. Given that the function is wide-sense
increasing, the last condition means that $\lim_{t \rightarrow +
\infty}(\alpha \Mpd \alpha)(t) = + \infty$. For example, for a VBR
source with T-SPEC $(p,M,r,b)$ (Figure~\ref{fig-vbrenv}), we have
$(\alpha \Mpd \alpha)(t)= rt$ and the condition is satisfied. The
alert reader will easily be convinced that the condition is also
true if the arrival curve is a stair function.

The proof of Theorem~\ref{theo-outworst} is a little technical and
is left at the end of this chapter.


We might wonder whether the output bound $\alpha^*$ is a ``good"
function. The answer is no, since $\alpha^*(0)$ is the backlog
bound and is positive in reasonable cases. However, $\alpha^*$ is
sub-additive (the proof is easy and left to the reader) thus the
modified function $\delta_0 \wedge \alpha^*$ defined as
$\alpha^*(t)$ for $t
>0$ and $0$ otherwise is a ``good" function. If $\alpha$ is
left-continuous, $\delta_0 \wedge \alpha^*$ is even a ``very good"
function since we know from the proof of
Theorem~\ref{theo-outworst} that it is left-continuous.

\subsection{Concatenation}
\mylabel{sec-concat}
So far we have considered elementary network
parts. We now come to the main result used in the concatenation of
network elements.

\begin{theorem}[Concatenation of Nodes]
    \mylabel{theo-netSerCur}
Assume a flow traverses systems $\mathcal{S}_{1}$ and
$\mathcal{S}_{2}$ in sequence.  Assume that $\mathcal{S}_{i}$
offers a service curve of $\beta_{i}$, $i=1,2$ to the flow. Then
the concatenation of the two systems offers a service curve of
$\beta_{1} \mpc \beta_{2}$ to the flow.
\end{theorem}

\pr Call $R_{1}$ the output of node 1, which is also the input to
node 2. The service curve property at node 1 gives $$ R_{1} \geq R
\mpc \beta_{1}$$ and at node 2 $$R^{*} \geq R_{1} \mpc \beta_{2}
\geq (R \mpc \beta_{1}) \mpc \beta_{2} = R \mpc (\beta_{1} \mpc
\beta_{2})$$ \qed
\paragraph{Examples: }
Consider two nodes offering each a rate-latency service curve
$\beta_{R_i, T_i}$, $i=1,2$, as is commonly assumed with Intserv.
A simple computation gives
 $$\beta_{R_1, T_1} \mpc  \beta_{R_1, T_1} = \beta_{\min(R_1, R_2), T_1+T_2}
 $$
Thus concatenating Intserv nodes amounts to adding the latency
components and taking the minimum of the rates.

We are now also able to give another interpretation of the
rate-latency service curve model. We know that $ \beta_{R,T} =
(\delta_{T} \mpc \lambda_{R})(t)$; thus we can view a node
offering a rate-latency service curve as the concatenation of a
guaranteed delay node, with delay $T$ and a constant bit rate or
GPS node with rate $R$.

\paragraph{Pay Bursts Only Once}

The concatenation theorem allows us to understand a phenomenon
known as ``Pay Bursts Only Once". Consider the concatenation of
two nodes offering each a rate-latency service curve $\beta_{R_i,
T_i}$, $i=1,2$, as is commonly assumed with Intserv. Assume the
fresh input is constrained by $\gamma_{r, b}$. Assume that $r<
R_{1}$ and $r< R_{2}$. We are interested in the delay bound, which
we know is a worst case. Let us compare the results obtained as
follows.
\begin{enumerate}
    \item  by applying the network service curve;
    \item  by iterative application of the individual bounds
    on every node
\end{enumerate}

The delay bound $D_{0}$ can be computed by applying Theorem
\ref{theo-delay}:
 $$ D_{0}=\frac{b}{R} + T_{0} $$
with $R=\min_{i}(R_{i})$ and $T_{0}= \sum_{i}T_{i}$ as seen above.

Now apply the second method. A bound on the delay at node 1 is
(Theorem \ref{theo-delay}):
 $$ D_{1}= \frac{b}{R_{1}}+T_{1} $$
The output of the first node is constrained by $\alpha^*$, given
by :
 $$ \alpha^*(t) = b + r \times (t+T_{1}) $$
A bound on the delay at the second buffer is:
 $$ D_{2}=\frac{b + r T_{1}}{R _{2}}+T_{2} $$
And thus
 $$D_1 + D_2 = \frac{b}{R_{1}}+ \frac{b + r T_{1}}{R _{2}}+ T_0
 $$
It is easy to see that $D_{0}< D_{1}+D_{2}$. In other words, the
bounds obtained by considering the global service curve are better
than the bounds obtained by considering every buffer in isolation.

Let us continue the comparison more closely. The delay through one
node has the form $\frac{b}{R_{1}}+T_{1}$ (for the first node).
The element $\frac{b}{R_1}$ is interpreted as the part of the
delay due to the burstiness of the input flow, whereas $T_1$ is
due to the delay component of the node. We see that $D_{1}+D_2$
contains twice an element of the form $\frac{b}{R_i}$, whereas
$D_0$ contains it only once. We sometimes say that ``we pay bursts
only once''. Another difference between $D_0$ and $D_1+D_2$ is the
element $\frac{r T_1}{R_2}$: it is due to the increase of
burstiness imposed by node 1. We see that this increase of
burstiness does not result into an increase of the overall delay.

A corollary of Theorem~\ref{theo-netSerCur} is also that the
end-to-end delay bound does not depend on the order in which nodes
are concatenated.

\subsection{Improvement of Backlog Bounds}
\mylabel{sec-bb-imp}

We give two cases where we can slightly improve the backlog
bounds. %We use these improvements in Chapters~\ref{L24}
%and~\ref{L30}.

\begin{theorem}%\cite{leb99}
\mylabel{fifo-theo3} Assume that a lossless node offers a
\emph{strict} service curve $\beta$ to a flow with arrival curve
$\alpha$. Assume that $\alpha(u_0) \leq \beta (u_0)$ for some $u_0
>0 $. Then the duration of the busy period is $\leq u_0$.
Furthermore, for any time $t$, the backlog $R(t)-R^*(t)$ satisfies
 $$
 R(t) -R^*(t) \leq \sup_{u: 0 \leq u <u_0} \left[R(t)-R(t-u) - \beta(u)\right]
 \leq \sup_{u: 0 \leq u <u_0} \left[\alpha(u)- \beta(u)\right]
 $$
 \end{theorem}
The theorem says that, for the computation of a buffer bound, it
is sufficient to consider time intervals less than $u_0$. The idea
is that the busy period duration is less than $u_0$.
\pr
Consider a given time $t$ at which the buffer is not empty, and
call $s$ the last time instant before $t$ at which the buffer was
empty. Then, from the strict service curve property, we have
$$R^*(t) \geq R^*(s) + \beta(t-s) = R(s) + \beta(t-s)$$
Thus the buffer size $b(t)=R(t)-R^*(t)$ at time $t$ satisfies
$$b(t)\leq R(t)-R(s) - \beta(t-s) \leq \alpha(t-s)-\beta(t-s)$$
Now if $t-s  \geq u_0$, then there is a time $t' = s + u_0 $, with
$s+1 \leq t' \leq  t$ such that $b(t')=0$. This contradicts the
definition of $s$. Thus we can assume that $t-s < u_0$. \qed


\begin{theorem}%\cite{clv00}
\mylabel{fifo-theo4} Assume that a lossless node offers a service
curve $\beta$ to a flow with sub-additive arrival curve $\alpha$.
Assume that $\beta$ is \emph{super-additive}, and that
$\alpha(u_0) \leq \beta (u_0)$ for some $u_0
>0 $. Then for any time $t$, the backlog $R(t)-R^*(t)$ satisfies
 $$
 R(t) -R^*(t) \leq \sup_{u: 0 \leq u <u_0} \left[R(t)-R(t-u) - \beta(u)\right]
 \leq \sup_{u: 0 \leq u <u_0} \left[\alpha(u)- \beta(u)\right]
 $$
 \end{theorem}
 Note that the condition that $\alpha$ is sub-additive is not a
 restriction. In contrast, the condition that $\beta$ is
 super-additive is a restriction. It applies in  particular to
rate-latency service curves. The theorem does not say anything
about the duration of the busy period, which is consistent with
the fact we do not assume here that the service curve is strict.

 \pr
 For an arbitrary time $t$ the backlog at time $t$ satisfies
 $$b(t) \leq \sup_{u \geq 0} \left[R(t)-R(t-u) - \beta(u)\right]
 $$
 For $s \leq t$ define $k=\lceil \frac{t-s}{u_0} \rceil$ and
 $s'=k u_0  + s$. We have $s\leq s' \leq t$ and
\begin{equation}\mylabel{eq-defu0}
t -u_0 < s'
\end{equation}
 Now from the super-additivity of $\beta$:
 $$R(t)-R(s) \leq \left[R(t)-R(s') - \beta(t-s')\right] +
 \left[R(s')-R(s) - \beta(s'-s) \right]
 $$
Note that for the second part we have
$$R(s')- R(s) - \beta(s'-s) \leq k\left[\alpha(u_0) -
\beta(u_0)\right] \leq 0
$$
thus
$$R(t)-R(s) \leq \left[R(t)-R(s') - \beta(t-s')\right]
$$
which shows the theorem. \qed


\section{Greedy Shapers}
\mylabel{shapers}
\subsection{Definitions}

We have seen with the definition of the leaky bucket and of the
GCRA two examples of devices that enforce a general arrival curve.
We call \emph{policer}%
\index{policer}%
 with curve $\sigma$ a device that counts
the bits arriving on an input flow and decides which bits conform
with an arrival curve of $\sigma$. We call \emph{shaper}%
\index{shaper}%
, with shaping curve%
\index{shaping curve}%
  $\sigma$, a bit processing device that forces
its output to have $\sigma$ as an arrival curve. We call
\emph{greedy
shaper}%
\index{greedy shaper} a shaper that delays the input bits in a
buffer, whenever sending a bit would violate the constraint
$\sigma$, but outputs them as soon as possible.

With ATM and sometimes with Intserv, traffic sent over one
connection, or flow, is policed at the network boundary.  Policing
is performed in order to guarantee that users do not send more
than specified by the contract of the connection.  Traffic in
excess is either discarded, or marked with a low priority for loss
in the case of ATM, or passed as best effort traffic in the case
of Intserv.  In the latter case, with IPv4, there is no marking
mechanism, so it is necessary for each router along the path of
the flow to perform the policing function again.

Policing devices inside the network are normally buffered, they
are thus shapers. Shaping is also often needed because the output
of a buffer normally does not conform any more with the traffic
contract specified at the input.
%
%In this Section we learn the main properties of greedy shapers.
%
\subsection{Input-Output Characterization of Greedy Shapers}
\mylabel{ioshaper}

The main result with greedy shapers is the following.

\begin{theorem}[Input-Output Characterization of Greedy Shapers]
    Consider a greedy shaper with shaping curve $\sigma$.  Assume that the
    shaper buffer is empty at time $0$, and that it is is large enough
    so that there is no data loss.  For an input flow $R$, the output
    $R^{*}$ is given by
    \begin{equation}
    R^{*}=R\mpc \bar{\sigma}
    \mylabel{eq-netcal-ndfdk8723}
\end{equation}
where $\bar{\sigma}$ is the sub-additive closure of $\sigma$.
\mylabel{theo-gen-shaper}
\end{theorem}

\pr Remember first that if $\sigma$ is sub-additive and
$\sigma(0)=0$, then $ \bar{\sigma} = \sigma$.  In general, we know
that we can replace $\sigma$ by $\bar{\sigma}$ without changing
the definition of the shaper.  We thus assume without loss of
generality that $\bar{\sigma} = \sigma$.

The proof of the theorem is an application of min-plus algebra.
First, let us consider a virtual system that would take $R$ as
input and have an output $S$ satisfying the constraints:
\begin{equation}
    \left \{
    \begin{array}{l}
         S  \leq  R \\
        S  \leq S \mpc \sigma
    \end{array}
    \right .
    \mylabel{eq-netcal-dkcnqlfenq9}
\end{equation}

Such a system would behave as a buffer (the first equation says
that the output is derived from the input) and its output would
satisfy the arrival curve constraint $\sigma$.  However, such a
system is not necessarily a greedy shaper; we could have for
example a lazy shaper with $S(t)=0$ for all $t\geq 0$~!  For this
system to be a greedy shaper, it has to output the bits as soon as
possible.  Now there is a general result about systems satisfying
conditions~\ref{eq-netcal-dkcnqlfenq9}.
 \begin{lemma}[A min-plus linear system]
 Assume that $\sigma$ is a ``good" function
 (that is, is sub-additive and $\sigma(0)=0$).
 Among all functions $S(t)$ satisfying
 conditions~\ref{eq-netcal-dkcnqlfenq9} for some fixed function $R$,
 there is one that is an
 upper bound for all. It is equal to $R \mpc \sigma$
 \mylabel{lem-gs}
 \end{lemma}
 \paragraph{Proof of the lemma:} The lemma is a special case of a
 general result in Chapter~\ref{L11}. However, it is also possible
 to give a very simple proof, as follows.

Define $S^*=R \mpc \sigma$. Since $\sigma$ is a ``good" function,
it follows immediately that $S^*$ is a solution to
System~(\ref{eq-netcal-dkcnqlfenq9}). Now, let $S'$ be some other
solution. By the first condition in~(\ref{eq-netcal-dkcnqlfenq9}), $S'\leq R$ and thus $S'\mpc \sigma\leq R\mpc \sigma=S^*$. By the second condition,
 $$S'\leq S' \mpc \sigma \leq S^{*}$$
This shows that $S^{*}$ is the maximal solution.\qed

Note that the lemma proves the existence of a maximal solution to
System~(\ref{eq-netcal-dkcnqlfenq9}). Note also that, in the
lemma, function $R$ need not be wide-sense increasing.

Now we can use the lemma by showing that $R^{*}= S^{*}$. Function
$R$ is wide-sense increasing, thus so is $S^*$. Obviously, $R^{*}$
is a solution of System~(\ref{eq-netcal-dkcnqlfenq9}), thus
$R^{*}(t) \leq S^{*}(t)$ for all $t$. Now if there would be some
$t$ such that $R^{*}(t) \neq S^{*}(t)$, then this would contradict
the condition that the greedy shaper attempts to send the bits out
as early as possible.\qed

The following corollary derives immediately.

\begin{corollary}[Service Curve offered by a Greedy Shaper]
Consider a greedy shaper with shaping curve $\sigma$. Assume that
$\sigma$ is sub-additive and $\sigma(0)=0$. This system offers to
the flow a service curve equal to $\sigma$. \mylabel{cor-0987}
\end{corollary}

\paragraph{Example: Buffer Sizing at a Re-shaper}
\begin{figure}[htbp]
  \insfig{reshap}{0.7}
  \mycaption{Reshaping example.}
  \mylabel{fig-reshap}
\end{figure}
Re-shaping is often introduced because the output of a buffer
normally does not conform any more with the traffic contract
specified at the input. For example, consider a flow with the
arrival curve $\sigma(t) = \min(pt+M,rt+b)\ind{t>0}$ that traverses a
sequence of nodes, which offer a service curve
$\beta_1=\beta_{R,T}$. A greedy shaper, with shaping curve
$\sigma$, is placed after the sequence of nodes
(Figure~\ref{fig-reshap}). The input to the shaper ($R$ in the
figure) has an arrival curve $\alpha^*$, given by
Proposition~\ref{theo-vbr-ratlat2}. Corollary~\ref{cor-0987} gives
a service curve property for the greedy shaper; observe that we need to make sure that $\sigma(t)=0$.
The buffer
$B$ required at the greedy shaper is then obtained as the vertical distance
$v(\alpha^*,\sigma)$. After some algebra, we obtain:
\begin{equation}
B = \left\{
\begin{array}{ll}
    \mif \frac{b-M}{p-r}< T & \mthen  b +T r  \\
    \mif \frac{b-M}{p-r}\geq T \mand p > R  & \mthen  M +
    \frac{(b-M)(p-R)}{p-r} +T R \\
    \melse &  M + T p
\end{array}
\right.
    \mylabel{eq-d4-shap-jkcnui}
\end{equation}

\begin{corollary}[Buffer Occupancy at a Greedy Shaper]
Consider a greedy shaper with shaping curve $\sigma$. Assume that
$\sigma$ is sub-additive and $\sigma(0)=0$. Call $R(t)$ the input
function. The buffer occupancy $x(t)$ at time $t$ is given by $$
x(t) = \sup_{0 \leq s \leq t} \{ R(t) - R(s) - \sigma(t-s) \} $$
\mylabel{coro-shaper-buffer-occupancy}
\end{corollary}
\pr The backlog is defined by $x(t)=R(t)-R^*(t)$, where $R^*$ is
the output. We apply Theorem~\ref{theo-gen-shaper} and get:
 $$
 x(t)=R(t)-\inf_{0 \leq s \leq t}\{ R(s)+\sigma(t-s)\}
 =R(t)+\sup_{0 \leq s \leq t}\{- R(s)-\sigma(t-s)\}
 $$
 \qed

Note that Lemma~\ref{lem-buf-simple} is a special case of this
corollary.

In min-plus algebraic terms, we say that a system is linear and
time invariant if its input-output characterization has the form
$R^{*}=R\mpc \beta$ (where $\beta$ is not necessarily
sub-additive).  We can thus say from the theorem that greedy
shapers are min-plus linear and time invariant systems. There are
min-plus linear and time invariant system that are not greedy
shapers. For example, a node imposing a \emph{constant} delay $T$
is characterized by the input-output relationship
 $$R^* = R \mpc \delta_T
 $$
Compare to the guaranteed delay node (namely, a node imposing a
variable delay bounded by $T$), for which the input-output
relationship is a service curve property :
 $$R^* \geq R \mpc \delta_T$$
The rest of this Section illustrates similarly that the
input-output characterization of greedy shapers $R^* = R \mpc
\sigma$ is much stronger than the service curve property described
in Corollary~\ref{cor-0987}.

\subsection{Properties of Greedy Shapers}
Consider again Figure~\ref{fig-reshap}. We have seen in the
previous section how we can compute the buffer size required at
the greedy shaper. Now if greedy shapers are introduced along a
path, then some bits may be delayed at the shaper, thus the
end-to-end delay might increase. However, this is not true, as the
following results state that, from a global viewpoint, ``greedy
shapers come for free''.

\begin{theorem}[Re-Shaping does not increase delay or buffer requirements]
Assume a flow, constrained by arrival curve $\alpha$, is input to
networks $\mathcal{S}_{1}$ and $\mathcal{S}_{2}$ in sequence.
Assume a greedy shaper, with curve $\sigma \geq \alpha$ is
inserted between $\mathcal{S}_{1}$ and $\mathcal{S}_{2}$. Then the
backlog and delay bounds given by Theorem \ref{theo-delay} for the
system without shaper are also valid for the system with shaper.
\mylabel{theo-shaping-for-free}
\end{theorem}
The condition $\sigma \geq \alpha$ means that re-shaping maybe
only partial.
 \pr Call $\beta_{i}$ the service curve of
$\mathcal{S}_{i}$. The backlog bound in Theorem \ref{theo-backlog}
is given by
\begin{equation}\mylabel{eq-0599-a}
  v(\alpha, \beta_{1} \mpc \sigma \mpc \beta_{2}) = v(\alpha,
 \sigma \mpc \beta_{1}  \mpc \beta_{2})
\end{equation}
Now the last expression is the backlog bound obtained if we put
the shaper immediately at the entrance of the network. Clearly,
this introduces no backlog, which shows that the overall backlog
is not influenced by the shaper. The same reasoning applies to the
delay bound. \qed

If you read carefully, you should not agree with the last
paragraph.  Indeed, there is a subtlety. The bounds in Section
\ref{bounds} are tight, but since we are using several bounds
together, there is no guarantee that the resulting bound is tight.
All we can say at this point is that the bound computed for the
system with shaper is the same if we put the shaper in front; we
still need to show that the bound for such a system is the same
bound as if there would be no shaper. This can be proven in a
number of ways. We give here a computational one. The proof relies
on Lemma~\ref{leCor}, given below. \qed

\begin{lemma}
\mylabel{leCor}
    Let $\alpha$ and $\sigma$ be ``good" functions. Assume $\alpha
    \leq \sigma$. Then for any
    function $\beta$,
    $v(\alpha, \sigma \mpc \beta)=v(\alpha,\beta)$
    and
    $h(\alpha, \sigma \mpc \beta)=h(\alpha,\beta)$.
\end{lemma}
\pr
We use the reduction to min-plus deconvolution explained in
\sref{sec:deviations}. We have:
$$v(\alpha, \sigma \mpc \beta)= [\alpha \mpd (\sigma \mpc
\beta)](0)
$$
Now from \thref{thm:rule11-14} on \pgref{thm:rule11-14}: $\alpha
\mpd (\sigma \mpc \beta)= (\alpha \mpd \sigma) \mpd \beta$. Also,
since $\sigma \geq \alpha$, we have $\alpha \mpd \sigma \leq
\alpha \mpd \alpha$. Now $\alpha \mpd \alpha=\alpha$ because
$\alpha$ is a ``good" function, thus
\begin{equation}\mylabel{eq-demolecor}
  \alpha \mpd (\sigma \mpc \beta)= \alpha \mpd \beta
\end{equation}
and finally $v(\alpha, \sigma \mpc \beta) = v(\alpha, \beta)$.

Similarly $h(\alpha, \beta)= \inf\{ d \mst (\alpha \mpd \beta)(-d)
\leq 0 \}$ which, combined with \eref{eq-demolecor} proves that
$h(\alpha, \sigma \mpc \beta)=h(\alpha, \beta)$.
\qed

Consider again Figure~\ref{fig-reshap}. Assume that the first
network element and the greedy shaper are placed in the same node.
Theorem~\ref{theo-shaping-for-free} says that the \emph{total}
buffer required for this combined node is the same as if there
would be no greedy shaper at the output. Thus, if you can
dynamically allocate buffer space from a common pool to the first
network element and the greedy shaper, then the greedy shaper
costs no memory. However, the greedy shaper does need some buffer
space, as given in Equation~(\ref{eq-d4-shap-jkcnui}). Similarly,
the theorem says that there is no penalty for the worst-case
delay.

In contrast, placing a greedy shaper has an obvious benefit. The
burstiness of the flow admitted in the next network element is
reduced, which also reduces the buffer required in that element.
To be more concrete, consider the example ``Pay Bursts Only Once"
in Section~\ref{sec-concat}. Assume that a re-shaper is introduced
at the output of the first node. Then the input to the second node
has the same arrival curve as the fresh input, namely,
$\gamma_{r,b}$ instead of $\gamma_{r,b+r T_1}$. The buffer
required for the flow at node 2 is then $b+rT_2$ instead of
$b+r(T_1+T_2)$.

The following result is another ``physical" property of greedy
shapers. It says that shaping cannot be undone by shaping.

\begin{theorem}[Shaping Conserves Arrival Constraints]
    Assume a flow with arrival curve $\alpha$ is input to a greedy shaper with
    shaping curve $\sigma$. Assume $\sigma$ is a ``good" function. Then the
    output flow is still constrained by the original arrival curve $\alpha$.
    \mylabel{theo-shape-keeps-arrival}
\end{theorem}

\pr
$$ R^{*}=R \mpc \sigma \leq (R \mpc \alpha) \mpc \sigma
$$ since the condition $R \leq R \mpc \alpha$ expresses that
$\alpha$ is an arrival curve. Thus $$ R^{*} \leq R \mpc \sigma
\mpc \alpha = R^{*} \mpc \alpha $$ \qed

The output of the greedy shaper has thus $\min(\alpha, \sigma)$ as
an arrival curve. If $\alpha$ is also a ``good" function, we know
(Lemma~\ref{lem-cmpsou}) that the sub-additive closure of
$\min(\alpha, \sigma)$ is $\alpha \mpc \sigma$.

\paragraph{Example (ATM Multiplexer): } Consider an ATM
switch that receives 3 ATM connections, each constrained by
GCRA(10, 0) (periodic connections). The switch serves the
connection in any work conserving manner and outputs them on a
link with rate 1 cell per time slot. What is a good arrival curve
for the aggregate output ?

The aggregate input has an arrival curve $\alpha=3 v_{10,0}$. Now
the server is a greedy shaper with shaping curve $\sigma=v_{1,0}$,
thus it keeps arrival constraints. Thus the output is constrained
by $3 v_{10,0} \mpc v_{1,0}$, which is a ``good" function. We have
already met this example in Figure~\ref{fig-alpha-sadd}.

\section{Maximum Service Curve, Variable and Fixed Delay}
\mylabel{sec-maxsercur}

\subsection{Maximum Service Curves}
\mylabel{sec-maxsercur-prop}

If we modify the sense of the inequation in the definition of
service curve in Section~\ref{sec-sercur}, then we obtain a new
concept, called \emph{maximum service curve}, which is useful to
(1) account for constant delays and (2) in some cases to establish
a relationship between delay and backlog.
\begin{definition}[Maximum Service Curve]
\mylabel{def-maxsercur}
 Consider a  system $\mathcal{S}$ and a flow
through $\mathcal{S}$ with input and output function $R$ and
$R^{*}$. We say that $\mathcal{S}$ offers to the flow a
\emph{maximum service curve} $\gamma$ if and only if $\gamma \in
\calF$ and $R^{*} \leq R \mpc \gamma$
\end{definition}
Note that the definition is equivalent to saying that $\gamma$ is
wide-sense increasing and that
 $$R^*(t) \leq R(s) + \gamma(t-s)
 $$
for all $t$ and all $s \leq t$, or equivalently
 $$R^*(t)-R^*(s) \leq B(s) + \gamma(t-s) $$
 where $B(s)$ is the backlog at time $s$.
A greedy shaper with shaping curve $\sigma$ offers $\sigma$ both
as a service curve and a maximum service curve.

In general, the concept of maximum service curve is not as
powerful as the concept of service curve. However, as we see
below, it can be useful to account for maximum rates and for
constant propagation delays. We also see in \cref{L24} that it
allows us to find good bounds for aggregate multiplexing.

The following propositions give two special cases of interest.
Their proof is easy and left to the reader.
\begin{proposition}[Minimum Delay]
A lossless node offers a maximum service curve equal to $\delta_T$
if and only if it imposes a minimum virtual delay equal to $T$.
\end{proposition}
\begin{proposition}[Arrival Constraint on Output]
Assume the output of a lossless node is constrained by some
arrival curve $\sigma$. Then the node offers $\sigma$ as a maximum
service curve.
\end{proposition}

Like minimum service curves, maximum service curves can be
concatenated:

\begin{theorem}[Concatenation of Nodes]
    \mylabel{theo-maxNetSerCur}
Assume a flow traverses systems $\mathcal{S}_{1}$ and
$\mathcal{S}_{2}$ in sequence.  Assume that $\mathcal{S}_{i}$
offers a maximum service curve of $\gamma_{i}$, $i=1,2$ to the
flow. 
\end{theorem}
\pr The proof mimics the proof of Theorem~\ref{theo-netSerCur}
\qed

\paragraph{Application: }
Consider a node with a maximum output rate equal to $c$ and with
internal propagation delay equal to $T$. It follows from
Theorem~\ref{theo-maxNetSerCur} and the two previous propositions
that this node offers to any flow a maximum service curve equal to
the rate-latency function $\beta_{c,T}(t)=\left[c(t-T) \right]^+$.

Maximum service curves do not allow us to derive as strong results
as (ordinary) service curves. However, they can be used to reduce
the output bound and, in some cases, to obtain a minimum delay
bound. Indeed, we have the following two results.

\begin{theorem}[Output Flow, generalization of Theorem ~\ref{theo-output} ]
    \mylabel{theo-output-maxsc}
Assume a flow, constrained by arrival curve $\alpha$, traverses a
system that offers a service curve $\beta$ and a maximum service
curve $\gamma$. The output flow is constrained by the arrival
curve $\alpha^{*}=(\alpha \mpc \gamma) \mpd \beta$.
\end{theorem}
\pr
Instead of a computational proof as with
Theorem~\ref{theo-output}, it is simpler at this stage to use
min-plus algebra. Call $R$ and $R^*$ the input and output
functions, and consider $R^* \mpd R^*$, the minimum arrival curve
for $R^*$. We have $R^* \leq R \mpc \gamma$ and $R^* \geq R \mpc
\beta$, thus
$$R^* \mpd R^* \leq (R \mpc \gamma) \mpd (R \mpc
\beta)
 $$
From Rule 12 in Chapter~\ref{L10}, Theorem~\ref{thm:rule11-14},
applied to $f=R \mpc \gamma$, $g=R$ and $h=\beta$, we derive
$$R^* \mpd R^* \leq \left\{(R \mpc \gamma) \mpd R \right\} \mpd \beta
 $$
Now from the commutativity of $\mpc$ and from Rule 13 in
Theorem~\ref{thm:rule11-14}:
$$ \left\{(R \mpc \gamma) \mpd R \right\} =
 \left\{(\gamma \mpc R) \mpd R \right\}
 \leq \left\{ \gamma \mpc (R \mpd R )\right\}
 $$
Thus
$$R^* \mpd R^* \leq \left\{ \gamma \mpc (R \mpd R )\right\} \mpd
\beta \leq ( \gamma \mpc \alpha ) \mpd \beta
 $$
 \qed


\begin{theorem}[Minimum Delay Bound]
Assume a flow, constrained by arrival curve $\alpha$, traverses a
system that offers a maximum service curve of $\gamma$ and is FIFO for this flow.  Assume
that $\gamma(D)=0$. %The virtual delay $d(t)$ satisfies $d(t) \geq
%D$ for all $t$.
The delay for any bit is $\geq D$.
    \mylabel{theo-delay-maxsc}
\end{theorem}
\pr We have $R^*(t+D) \leq R(t) + \gamma(D)$ thus $R^*(t+D) \leq
R(t)$ for any $t\geq 0$. Assume first that the input and output functions are left-continuous and consider a bit that arrives at some time, say $t$. It follows that for any $t_1>t$ we have $R(t)<R(t_1)$ and thus $R^*(t+D) <R(t_1)$. Assume further that $t_1\leq t+D$; the previous inequality can be re-written as $R^*(t_1+(D-(t_1-t))) <R(t_1)$ and thus $d(t_1)\geq D+(t-t_1)$. Take the limit when $t_1\to t$ and obtain $d_r(t)\geq D$ (where $d_r$ is the limit from the right of $d$). Now the delay for a bit that arrives at time $t$ is $d_r(t)$.

If the input and output functions are right-continuous instead of left-continuous (recall that we always assume either case), the proof is similar, by observing that $R(t_1)<R(t)$ for $t_1<t$ and establishing that $d(t)\geq D$.\qed

Note that the output bound is improved by the knowledge of the
maximum service curve since in general we expect $\alpha \mpc
\gamma$ to be less than $\alpha$. In contrast, the minimum delay
bound gives some new information only in the cases where there is
a latency part in the maximum service curve, which is the case for
the first example (Minimum Delay ), but not in general for the
second example (Arrival Constraint on Output).

\paragraph{Numerical Example: }

%Consider again the example illustrated in
%Figure~\ref{fig-atmnetcalex}. Let us first apply
%Theorem~\ref{theo-output} and compute an arrival curve
%$\alpha^*_0$ for the output. The details are as follows. We have
% $$\alpha^*_0 = 10 v_{25,4} \mpd \beta_{1,8} =  10 v_{25,4} \mpd
% \left(
% \lambda_{1} \mpc \delta_8 \right)$$
% Now from Rule 15 in Chapter~\ref{L10}, we have
% $$\alpha^*_0 =  \left( 10 v_{25,4} \mpd \delta_8 \right) \mpd
%\lambda_{1} $$ Now $(10 v_{25,4} \mpd \delta_8)(t)= 10
%v_{25,4}(t+8)= 10 v_{25,12}(t)$. To compute the deconvolution with $\lambda_{1}$ we can use its interpretation as a smoothing operation in \sref{sec:timeinversion}, or a direct computation, and finally obtain $\alpha^*_0$ as shown on \fref{fig-113again}.
%

Consider the following example, which is a variant of
Figure~\ref{fig-atmnetcalex}; time is discrete; a flow with arrival curve $\alpha_0=v_{25,4}$ is served in a system that guarantees a service curve $\beta_{1,8}$. Let us apply
Theorem~\ref{theo-output} and compute an arrival curve
$\alpha^*_0$ for the output. We have
 $$\alpha^*_0 =  v_{25,4} \mpd \beta_{1,8} =  v_{25,4} \mpd
 \left(
 \lambda_{1} \mpc \delta_8 \right)$$
 Now from Rule 15 in Chapter~\ref{L10}, we have
 $$\alpha^*_0 =  \left( v_{25,4} \mpd \delta_8 \right) \mpd
\lambda_{1} $$ Now $( v_{25,4} \mpd \delta_8)(t)=
v_{25,4}(t+8)= v_{25,12}(t)$. To compute the deconvolution with $\lambda_{1}$ we can use its interpretation as a smoothing operation in \sref{sec:timeinversion}, or a direct computation, and finally obtain $\alpha^*_0=v_{25,12}$ (recall that time is discrete; in continuous time, we would obtain $\alpha^*_0=v_{25,11}\mpc \lambda_1$, i.e. the jumps of the staircase function are smoothed with a slope of $1$).


Assume now that we have more information about the node, and that
we can model is as node $\calS_1$ defined as the concatenation of
two schedulers and a fixed delay element
(Figure~\ref{fig-atmnetcalex2}). Each scheduler offers to the
aggregate flow a service curve $\beta_{R_0,T_0}$ with rate $R_0=1$
cell per time slot and latency $T_0=2$ time slots. The delay
element is a link with maximum rate equal to $1$ cell per time
slot, and a fixed propagation and transmission delay equal to $4$
time slots. The delay element is thus the combination of a greedy
shaper with shaping curve $\lambda_1(t)=t$ and a fixed delay
element $\delta_4$. We can verify that the concatenation of the
three elements in node 1 offers a service curve equal to
$\beta_{1,2} \mpc \lambda_1 \mpc \delta_4 \mpc \beta_{1,2}=
\beta_{1,8}$. Now, from the delay element allows us to say that,
in addition, the node also offers to the aggregate flow a
\emph{maximum service curve} equal to $\beta_{1,4}$. We can apply
Theorem~\ref{theo-output-maxsc} and derive from that the output is
constrained by the arrival curve $\alpha^*_1$ given by
$$\alpha^*_1=\left( \alpha \mpc \beta_{1,4} \right) \mpd
\beta_{1,8}
$$
The computation is similar to that of $\alpha^*_0$ and involves
the computation of $10 v_{25, 4} \mpc \lambda_1$, which is similar
to the example illustrated in Figure~\ref{fig-alpha-sadd}.
Finally, we have:
 $$\alpha^*_1(t) = (10 v_{25, 4} \mpc \lambda_1)(t+4) $$
Figure~\ref{fig-atmnetcalex2} shows that $\alpha^*_1$ is a better
bound than the arrival curve $\alpha^*_0$ that we would obtain if
we did not know the maximum service curve property.
\begin{figure}[!htbp]
   \insfig{atmnetcalex2a}{0.6}
   \insfig{atmnetcalex2b}{0.6}
  \insfig{atmnetcalex2c}{0.6}
  \mycaption{Use of maximum service curve to improve output bound.
  The figure is for the same example as Figure~\ref{fig-atmnetcalex2}.
  Top: nodes $\calS_1$ and $\calS_2$, two possible
  implementations of a system offering the overall service curve
  $\beta_{1,8}$. Middle: arrival curve $\alpha$ and overall
  service curve $\beta_{1,8}$. Bottom: constraint for the output.
  $\alpha^*_0$ (top curve, thick, plain line) is obtained with the only
  knowledge that the service curve is $\beta_{1,8}$. $\alpha^*_1$
  (middle curve, thick, dashed line) is obtained assuming the
  system is $\calS_1$. $\alpha^*_2$
  (bottom curve, thin, plain line) is obtained assuming the
  system is $\calS_2$.
  }
  \mylabel{fig-atmnetcalex2}
\end{figure}

Assume next that we change the order of the delay element in node
$\calS1$ and place it as the last element of the node. Call
$\calS_2$ the resulting node. Then the conclusion of the previous
paragraph remains, since the bounds are insensitive to the order,
due to the commutativity of min-plus convolution. Thus the output
of system $\calS_2$ also has $\alpha^*_1$ as an arrival curve.
However, in that case, we can also model the delay element as the
combination of a shaper, with shaping curve $\lambda_1$
(corresponding to a fixed rate of $1$ cell per time slot),
followed by a fixed delay element, with constant delay equal to
$4$ time slots. The input to the shaper has an arrival curve equal
to $\alpha \mpd \beta_{1,4}$, where $\alpha = 10 v_{25,4}$ is the
fresh arrival curve. Thus, from the properties of shapers, the
output of the shaper is constrained by
 $$\alpha^*_2 = ( \alpha \mpd
\beta_{1,4}) \mpc \lambda_1= 10 v_{25,8}\mpc \lambda_1$$
 Since the
fixed delay component does not alter the flow, the output of
system $\calS_2$ has $\alpha^*_2$ as an arrival curve.
Figure~\ref{fig-atmnetcalex2} shows that $\alpha^*_2$ is a better
bound than $\alpha^*_1$.

This fact is true in general: whenever a network element can be
modeled as a shaper, then this model provides stronger bounds than
the maximum service.

\subsection{Delay from Backlog}
In general it is not possible to bound delay from backlog with the
framework of service curves, except in one particular but
important case.
\begin{theorem}
Assume a lossless node offers to a flow a minimum service curve
$\beta$ and a maximum service curve $\gamma$, such that
$\beta(t)=\gamma(t-v)$. Let $f$ be the max-plus deconvolution
$\gamma \Mpd \gamma$, that is,
$$f(t)=\inf_{s \geq 0} [\gamma(s+t)-\gamma(s)]$$
Then the backlog
$B(t)$ and the virtual delay $d(t)$ satisfy
 $$f(d(t)-v) \leq  B(t)$$
 If in addition $\gamma$ is super-additive, then
 $$\beta(d(t)) \leq  B(t)$$
\mylabel{theo-backlog-delay}
\end{theorem}
\pr
Fix some $t \geq 0$; we have$d(t)=\inf E_t$ where the set $E_t$ is
defined by
$$
 E_t= \{s\geq 0: R^*(t+s) \geq R(t)\}
$$
Since $R^*$ and $R$ are wide-sense increasing, $E_t$ is an
interval. Thus
 $$d(t)=\sup\{s\geq 0: R^*(t+s) < R(t)\}$$
We assume that $R$ and $R^*$ are left-continuous. It follows that
$$
 R^*(t+d(t)) \leq R(t)
$$
For some arbitrary $\epsilon$, we can find some $s$ such that
 $$
 R^*(t + d(t)) \geq R(s) + \beta(t-s+ d(t)) - \epsilon
 $$
 Now from the maximum service curve property
 $$
 R^*(t)-R(s) \leq \gamma (t-s)
 $$
Combining the three gives
$$
  B(t)= R(t) - R^*(t) \geq \beta(t-s+ d(t))  - \gamma (t-s) -
  \epsilon = \gamma(t-s+ d(t)-v)  - \gamma (t-s) -
  \epsilon
$$
and thus
\begin{equation}\mylabel{eq-minmaxetdel}
  B(t) \geq \inf_{u \geq 0} [ \gamma(d(t)-v+u)  - \gamma (u)]
\end{equation}
 From the definition of $f$, the latter term is $f(d(t)-v)$.
 Finally, if $\gamma$ is super-additive, then $\gamma \Mpd \gamma=
 \gamma$
 \qed

We can apply the theorem to a practical case:
\begin{corollary}
Assume a lossless node offers to a flow a minimum service curve
$\beta=\beta_{r,v}$ and a maximum service curve
$\gamma=\beta_{r,v'}$, with $v'\leq v$. The backlog $B(t)$ and the
virtual delay $d(t)$ satisfy
 $$d(t) \leq \frac{B(t)}{r} + v
 $$
\mylabel{coro-backlog-delay1}
\end{corollary}
\pr
We apply the theorem and note that $\gamma$ is super-additive,
because it is convex. \qed
%
%In general, the max-plus deconvolution of $\beta$ and $\gamma$ is
%equal to $-\infty$, which gives no useful result.

\subsection{Variable versus Fixed Delay}
\mylabel{sec-fixedDelay}


Some network elements impose fixed delays (propagation and
transmission), whereas some other network elements impose variable
delays (queueing). In a number of cases, it is important to
evaluate separately the total delay and the variable part of the
delay. The total delay is important, for example, for determining
throughput and response time; the variable part is important for
dimensioning playout buffers (see Section~\ref{sec-playoutBuf} for
a simple example, and chapter~\ref{L22} for a more general
discussion). We have seen at the end of end of
Section~\ref{ioshaper} that a node imposing a constant delay can
be modeled as a min-plus linear system. Beyond this, the concept
of maximum service curve is a tool for telling apart variable
delay from fixed delay, as follows.

Consider a network, made of a series of network elements $1,...,
I$, each element being the combination of a fixed delay $d_i$ and
a variable delay. Assume the variable delay component offers a
service curve $\beta_i$. A fixed delay component offers
$\delta_{d_i}$ both as a service curve and as a maximum service
curve. Define $\beta = \beta_1 \mpc ... \mpc \beta_I$; the network
offers as end-to-end service curve
 $\beta  \mpc \delta_{d_1+...+d_I}$,
and as end-to-end maximum service curve $\delta_{d_1+...+d_I}$.
Assume the input flow is constrained by some arrival curve
$\alpha$; from Theorems~\ref{theo-delay}
and~\ref{theo-delay-maxsc}, the end-to-delay $d(t)$ satisfies
 $$d_1+...+d_I \leq d(t) \leq h(\alpha, \beta \mpc
 \delta_{d_1+...+d_I})
 $$
By simple inspection, $h(\alpha, \beta \mpc \delta_{d_1+...+d_I})=
d_1+...+d_I + h(\alpha, \beta)$, thus the end-to-end delay
satisfies
$$0 \leq d(t)- \left[d_1+...+d_I \right] \leq h(\alpha, \beta)
 $$
In the formula, $d_1+...+d_I$ is the fixed part of the delay, and
$ h(\alpha, \beta)$ is the variable part. Thus, for the
computation of the variable part of the delay, we can simply
ignore fixed delay components.

Similarly, an arrival curve constraint for the output is
$$\alpha^*=\left(\alpha \mpc \delta_{d_1+...+d_I}\right) \mpd \left(\beta \mpc
\delta_{d_1+...+d_I} \right) = \alpha \mpd \beta
$$
thus the fixed delay can be ignored for the computation of the
output bound.

For the determination of backlog, the alert reader can easily be
convinced that fixed delays cannot be ignored. In summary:

\begin{proposition}
\begin{enumerate}
  \item For the computation of
backlog and fixed delay bounds, fixed or variable delay are
modeled by introducing $\delta_{T}$ functions in the service
curves. As a consequence of the commutativity of $\mpc$, such
delays can be inserted in any order along a sequence of buffers,
without altering the delay bounds.
  \item For the computation of variable delay bounds, or for an
  arrival constraint on the output, fixed delays can be ignored.
\end{enumerate}
\end{proposition}

\section{Handling Variable Length Packets}
\label{sec-vlp}

All results in this chapter apply directly to ATM systems, using
discrete time models. In contrast, for variable length packets (as
is usually the case with IP services), there are additional
subtleties, which we now study in detail. The main parts in this
section is the definition of a packetizer, and a study of its
effect on delay, burstiness and backlog bounds. We also revisit
the notion of shaper in a variable length context. For the rest of
this section, time is continuous.

Throughout the section, we will consider some wide sense
increasing sequences of packet arrival times $T_i \geq 0$. We
assume that for all $t$ the set $\{i: T_i\leq t\}$ is finite.

\subsection{An Example of Irregularity Introduced by Variable Length Packets}
\mylabel{sec-exvlp} The problem comes from the fact that real
packet switching systems normally output entire packets, rather
than a continuous data flow. Consider the example illustrated in
\fref{fig:vlp1}. It shows the output of a constant bit rate trunk,
with rate $c$, that receives as input a sequence of packets, of
different sizes. Call $l_i, T_i$ the size (in bits) and the
arrival epoch for the $i$th packet, $i=1,2,...$. The input
function is
\begin{equation}\mylabel{eq-rvlp}
  R(t)= \sum_i l_i 1_{\{T_i \leq t \}}
\end{equation}
In the formula, we used the indicator
 function
 $1_{\{\mbox{\emph{expr}}\}}$which
 is equal to $1$ if $\mbox{\emph{expr}}$ is
 true, and $0$ otherwise.%
\index{1@$1_{\{\mbox{\emph{expr}}\}}$(Indicator function)}%
%The stair function $v_{T_i}$ is defined in \cref{L10}.

We assume, as is usual in most systems, that we observe only
entire packets delivered by the trunk. This is shown as $R'(t)$ in
the figure, which results from the bit-by-bit output $R^*$ by a
packetization operation. The bit-by-bit output $R^*$ is well
understood; we know from \sref{shapers} that  $R^*=R \mpc
\lambda_c$. However, what is the effect of packetization ? Do the
results in Sections~\ref{bounds} and \ref{shapers} still hold ?


\begin{figure}[htbp]
  \insfig{vlp1a}{1.0}
  \mycaption{A real, variable length packet trunk of constant bit
  rate, viewed as the concatenation of a greedy shaper and a packetizer.
  The input is $R(t)$, the
  output of the greedy shaper is $R^*(t)$, the final output is the output of the packetizer is $R'(t)$.}
  \mylabel{fig:vlp1}
\end{figure}

Certainly, we should expect some modifications. For example, the
bit-by-bit output $R^*$ in the figure is the output of a greedy
shaper with curve $\lambda_c$, thus it has $\lambda_c$ as an
arrival curve, but this is certainly not true for $R'$. Worse, we
know that a greedy shaper keeps arrival constraints, thus if $R$
is $\sigma$-smooth for some $\sigma$, then so is $R^*$. However,
this is not true for $R'$. Consider the following example (which
is originally from \cite{guerin-00}). Assume that
$\sigma(t)=l_{\max}+rt$ with $r<c$. Assume that the input flow
$R(t)$ sends a first packet of size $l_1 = l_{\max}$ at time $T_1=
0$, and a second packet of size $l_2$ at time $T_2=\frac{l_2}{r}$.
Thus the flow $R$ is indeed $\sigma$-smooth. The departure time
for the first packet is $T'_1=\frac{l_{\max}}{c}$. Assume that the
second packet $l_2$ is small, specifically, $l_2 <
\frac{r}{c}l_{\max}$; then the two packets are sent back-to-back
and thus the departure time for the second packet is
$T'_2=T'_1+\frac{l_2}{c}$. Now the spacing $T'_2 - T'_1$ is less
than $\frac{l_2}{r}$, thus the second packet is not conformant, in
other words, $R'$ is not $\sigma$-smooth. Note that this example
is not possible if all packets are the same size.

We will see in this section that this example is quite general:
packetizing variable length packets does introduce some additional
irregularities. However, we are able to quantify them, and we will
see that the irregularities are small (but may be larger than the
order of a packet length). Most results are extracted from
\cite{leb01}

\subsection{The Packetizer}

We first need a few definitions.
\begin{definition}[cumulative packet lengths]
\mylabel{def-L} A sequence $L$ of cumulative packet lengths is a
wide sense increasing sequence $(L(0)=0, L(1), L(2), ...)$ such
that $\limit{n}{\infty}L(n)=+\infty$ and $$l_{\max}=\sup_n\{L(n+1)-L(n)\}$$ is finite
\index{Cumulative Packet Length}%
\end{definition}
In this chapter, we interpret $L(n)-L(n-1)$ as the length of the
$n$th packet. We now introduce a new building block, which was
introduced in \cite{Changbook}.

\begin{definition}[Function $P^L$~\cite{Changbook}]
Consider a sequence of cumulative packet lengths $L$ with
$L(0)=0$. For any real number $x$, define
\begin{equation}\mylabel{eq-vlp21}
P^L(x)= \sup_{n \in \Nats} \{ L(n) 1_{\{L(n) \leq  x\}}
 \}
\end{equation}%
\index{PL@$P^L$} \mylabel{def-pl}
\end{definition}
\begin{figure}[htbp]
  \insfig{pldef1}{0.7}
  \mycaption{Definition of function $P^L$.}
  \mylabel{fig:PL}
\end{figure}
\fref{fig:PL} illustrates the definition. Intuitively, $P^L(x)$ is
the largest cumulative packet length that is entirely contained in
$x$. Function $P^L$ is right-continuous; if $R$ is
right-continuous, then so is $P^L(R(t))$. For example, if all
packets have unit length, then $L(n)=n$ and for $x>0$:
$P^L(x)=\lfloor x \rfloor$. An equivalent characterization of
$P^L$ is
\begin{equation}\mylabel{eq-pl1}
  P^L(x) = L(n) \Longleftrightarrow L(n) \leq x < L(n+1)
\end{equation}

\begin{definition}[Packetizer~\cite{ggp-96,AR-96,cruzln98,Changbook}]
Consider a sequence $L$ of cumulative packet lengths. An
$L$-packetizer is the system that
transforms the input $R(t)$ into $P^L(R(t))$.%
\index{Packetizer}
\end{definition}

For the example in \fref{fig:vlp1}, we have $R'(t)=P^L(R^*(t))$
and the system can thus be interpreted as the concatenation of a
greedy shaper and a packetizer.

The following equation follows immediately:
\begin{equation}\mylabel{eq-plmax}
   x - l_{\max} < P^L(x) \leq x
\end{equation}
\begin{definition}
We say that a flow $R(t)$ is $L$-packetized if $P^L(R(t))=R(t)$
for all $t$.
\end{definition}

The following properties are easily proven and left to the reader.
\begin{itemize}
\item (The packetizer is isotone) If $x\leq y$
then $P^L(x) \leq P^L(y)$ for all $x,y \in
 \Reals$.
 \item ($P^L$ is idempotent) $P^L(P^L(x))=P^L(x)$ for all $x \in
 \Reals$
\item (Optimality of Packetizer) We can
characterize a packetizer in a similar way as we did for a greedy
shaper in \sref{shapers}. Among all flows $x(t)$ such that
\begin{equation}\mylabel{eq-shap-pack}
  \bracket {
x \mbox{ is } L \mbox{-packetized}\\
 x \leq R
  }
\end{equation}
there is one that upper-bounds all, and it is given by $x(t)=P^L(R(t))$.

The proof for this last item mimics that of \lref{lem-gs}; it
relies on the property that $P^L$ is idempotent.
\end{itemize}

We now study the effect of packetizers on the three bounds found
in \sref{bounds}. We first introduce a definition.
\begin{definition}[Per-packet delay]
Consider a system with $L$- packetized input and output. Call
$T_i, T'_i$ the arrival and departure time for the $i$th packet.
Assume there is no packet loss. The per-packet delay is $\sup_i
(T'_i -T_i)$
\end{definition}

Our main result in this section is the following theorem,
illustrated in \fref{fig-theo-pl}.
\begin{theorem}[Impact of packetizer]
Consider a system (\emph{bit-by-bit system}) with $L$-packetized
input $R$ and bit-by-bit output $R^*$, which is then
$L$-packetized to produce a final packetized output $R'$. We call
\emph{combined system} the system that maps $R$ into $R'$. Assume
both systems are first-in-first-out and lossless.
\begin{enumerate}
  \item The \emph{per-packet delay} for the combined system
   is the maximum virtual delay for the bit-by-bit system.
  \item Call $B^*$ the maximum backlog
for the bit-by-bit system and $B'$ the maximum backlog for the
combined system. We have $$B^*\leq B' \leq B^* + l_{\max}$$
  \item Assume that
the bit-by-bit system offers to the flow a maximum service curve
$\gamma$ and a minimum service curve $\beta$. The combined system
offers to the flow a maximum service curve $\gamma$ and a minimum
service curve $\beta'$ given by
$$\beta'(t) = [\beta(t)-l_{\max}]^+
$$
\item If some flow $S(t)$ has $\alpha(t)$ as an arrival curve, then
$P^L(S(t))$ has $\alpha(t) + l_{\max} 1_{\{t >0\}}$ as an arrival
curve.
\end{enumerate}
\mylabel{theo-delvlp}
\end{theorem}

The proof of the theorem is given later in this section. Before,
we discuss the implications.
\begin{figure}[!htbp]
  \insfig{theopl}{0.6}
  \mycaption{The scenario and notation in \thref{theo-delvlp}.}
  \mylabel{fig-theo-pl}
\end{figure}
Item 1 says that appending a packetizer to a node does not
increase the packet delay at this node. However, as we see later,
packetization does increase the end-to-end delay.

Consider again the example in \sref{sec-exvlp}. A simple look at
the figure shows that the backlog (or required buffer) is
increased by the packetization, as indicated by item 2. Item 4
tells us that the final output $R'$ has $\sigma'(t)= \sigma(t) +
l_{\max} 1_{t>0}$ as an arrival curve, which is consistent with
our observation in \sref{sec-exvlp} that $R'$ is not
$\sigma$-smooth, even though $R^*$ is. We will see in
\sref{sec-pgs} that there is a stronger result, in relation with
the concept of ``packetized greedy shaper".

Item 3 is the most important practical result in this section. It
shows that packetizing weakens the service curve guarantee by one
maximum packet length. For example, if a system offers a
rate-latency service curve with rate $R$, then appending a
packetizer to the system has the effect of increasing the latency
by $\frac{l_{\max}}{R}$.
  %reducing the service curve to
% $\beta'=\beta_{R, T+\frac{l_{\max}}{R}}$.

 Consider also the example in \fref{fig:vlp1}.
 The combination of the
 trunk and the packetizer can be modeled as a system offering
\begin{itemize}
  \item a minimum service curve $\beta_{c,\frac{l_{\max}}{c}}$
  \item a maximum service curve $\lambda_c$
\end{itemize}


\paragraph{Proof of \thref{theo-delvlp}}
\begin{enumerate}
\item
For some $t$ such that $T_i \leq t < T_{i+1}$ we have $R(t)=L(i)$
and thus
$$\sup_{t \in [T_i, T_{i+1})} d(t) =d(T_i)$$
now $$d(T_i)=T'_i -T_i$$ Combining the two shows that
$$\sup_t d(t) =\sup_i (T'_i -T_i)$$

\item The
proof is a direct consequence of \eref{eq-plmax}.

\item The result on maximum service curve $\gamma$ follows immediately
from \eref{eq-plmax}. Consider now the minimum service curve
property.

Fix some time $t$ and define $i_0$ by $T_{i_0} \leq t <T_{i_0
+1}$. For $1\leq i\leq i_0$ and for $T_{i-1} \leq s < T_{i}$
we have $R(s)=R(T_{i-1})$ and $\beta$ is wide-sense
increasing, thus
 $$
 \inf_{T_{i-1} \leq s < T_{i}}\left( R(s) + \beta(t-s) \right)
 =
 R(T_{i-1}) + \beta_r(t-T_{i})=R_l(T_{i})+ \beta_r(t-T_{i})
 $$
 where $\beta_r$ [resp. $R_l$] is the
 limit of $\beta$ from the right [resp. of $R$ from the left]. Similarly
 $$\inf_{s \in [T_{i_0},t]} \left( R(s) + \beta(t-s)
 \right)=R(t)
 $$
 since $\beta(0)=0$. Thus (case 1) either there is some $i
  \leq i_0$ such that $(R \mpc \beta )(t)=R_l(T_i)+
 \beta_r(t-T_i)$ or (case 2) $(R \mpc \beta )(t)=R(t)$.

 Consider case 1. By hypothesis, $R^*(t) \geq (R\mpc \beta) (t)$, thus
 $$R'(t) \geq R^*(t) - l_{\max}
 \geq  R_l(T_i) + \beta_r(t-T_i) -l_{\max}
 $$
On the other hand, $R^*(t) \geq R_l(T_i)=R(T_{i-1})$ and $R$
is $L$-packetized, thus
$$R'(t) \geq R_l(T_i)$$
Combining the two shows that
$$\begin{array}{rl}
   R'(t) \geq & \max\left[ R_l(T_i), R_l(T_i) + \beta_r(t-T_i) -l_{\max}
         \right] \\
     = & R_l(T_i) + \max\left[\beta_r(t-T_i) -l_{\max},0 \right]\\
     = & R_l(T_i) + \beta_r'(t-T_j)
  \end{array}
 $$
 Now fix some arbitrary $\epsilon >0$. By definition of the limit from the right,
we can find some $s \in (T_{i-1}, T_i)$ such that $\beta(t-s)
\leq \beta_r(t-T_i) +\epsilon$. Now $R(s)=R_l(T_i)$ thus
$$
 R'(t) \geq R(s) + \beta(t-s) - \epsilon \geq (R
 \mpc \beta')(t) -\epsilon
 $$
This is true for all $\epsilon >0$ thus $R'(t) \geq (R
 \mpc \beta')(t)$, which proves that the service curve
 property holds for case 1. The proof for case 2 is immediate.
\item The
proof is a direct consequence of \eref{eq-plmax}.
\end{enumerate}

\paragraph{Example: concatenation of GPS nodes}

Consider the concatenation of the theoretical GPS node, with
guaranteed rate $R$ (see \sref{sec-sercurdef} on
\pgref{sec-sercurdef}) and an $L$-packetizer. Assume this system
receives a flow of variable length packets. This models a
theoretical node that would work as a GPS node but is constrained
to deliver entire packets. This is not very realistic, and we will
see in \cref{L21} more realistic implementations of GPS, but this
example is sufficient to explain one important effect of
packetizers.

By applying \thref{theo-delvlp}, we find that this node offers a
rate-latency service curve $\beta_{R,\frac{l_{\max}}{R}}$.
\begin{figure}[htbp]
  \insfig{congpsvl}{0.8}
  \mycaption{The concatenation of several GPS fluid nodes with
  packetized outputs}
  \mylabel{fig-congpsvl}
\end{figure}
Now concatenate $m$ such identical nodes, as illustrated in
\fref{fig-congpsvl}. The end-to-end service curve is the rate
latency-function $\beta_{R, T}$ with
$$
 T = m \frac{l_{\max}}{R}
$$
We see on this example that the additional latency introduced by
one packetizer is indeed of the order of one packet length;
however, this effect is multiplied by the number of hops.

For the computation of the end-to-end delay bound, we need to take
into account \thref{theo-delvlp}, which tells us that we can
forget the last packetizer. Thus, a bound on end-to-end delay is
obtained by considering that the end-to-end path offers a service
curve equal to the latency-function $\beta_{R, T_0}$ with
$$
 T_0 = (m-1) \frac{l_{\max}}{R}
$$
For example, if the original input flow is constrained by one
leaky bucket of rate $r$ and bucket pool of size $b$, and if $r\leq R$, then an
end-to-end delay bound is
\begin{equation}\mylabel{eq-e2edelvlp}
\frac{b + (m-1) l_{\max}}{R}
\end{equation}
The alert reader will easily show that this bound is a worst case
bound. This illustrates that we should be careful in interpreting
\thref{theo-delvlp}. It is only at the last hop that the
packetizer implies no delay increase. The interpretation is as
follows. Packetization delays the first bits in a packet, which
delays the processing at downstream nodes. This effect is captured
in \eref{eq-e2edelvlp}. In summary:
\begin{remark}
Packetizers do not increase the maximum delay at the node where
they are appended. However, they generally increase the end-to-end
delay.
\end{remark}

We will see in \cref{L21} that many practical schedulers can be
modeled as the concatenation of a node offering a service curve
guarantee and a packetizer, and we will give a practical
generalization of \eref{eq-e2edelvlp}.

\subsection{A Relation between Greedy Shaper and Packetizer}
\mylabel{sec-psps}

We have seen previously that appending a packetizer to a greedy
shaper weakens the arrival curve property of the output. There is
however a case where this is not true. This case is important for
the results in \sref{sec-pgs}, but also has practical applications
of its own. \fref{fig-theo-psps} illustrates the theorem.
\begin{figure}[htbp]
  \insfig{theopsps}{0.7}
  \mycaption{\thref{theo-psps} says that $R^{(1)}$ is
  $\sigma$-smooth.}
  \mylabel{fig-theo-psps}
\end{figure}
\begin{theorem}
Consider a sequence $L$ of cumulative packet lengths and call
$\calP_L$ the $L$-packetizer. Consider a ``good" function $\sigma$
and assume that
\begin{equation}\mylabel{eq:condsigma}
\bracket{
  \mbox{There is a sub-additive function
$\sigma_0$ and a number $l \geq l_{\max}$ such that}\\
  \sigma(t)=\sigma_0(t) + l 1_{t>0}
 }
\end{equation}
Call $\calC_{\sigma}$ the greedy shaper with shaping curve
$\sigma$.
For any input, the output of the composition%
\footnote{We use the notation $\calP_L \circ \calC_{\sigma}$ to
denote the composition of the two operators, with $\calC_{\sigma}$
applied first;  see \sref{sec:catalogofoperators}.}%
$\calP_L \circ \calC_{\sigma} \circ \calP_L$ is $\sigma$-smooth.
 \mylabel{theo-psps}
\end{theorem}

In practical terms, the theorem is used as follows. Consider an
$L$-packetized flow, pass it through a greedy shaper with shaping
curve $\sigma$; and packetize the output; then the result is
$\sigma$-smooth (assuming that $\sigma$ satisfies condition in
\eref{eq:condsigma} in the theorem).

Note that in general the output of $\calC_{\sigma} \circ \calP_L $
is \emph{not} $L$-packetized, even if $\sigma$ satisfies the
condition in the theorem (finding a counter-example is simple and
is left to the reader for her enjoyment). Similarly, if the input
to $\calP_L \circ \calC_{\sigma}$ is not $L$-packetized, then the
output is not $\sigma$-smooth, in general.

The theorem could also be rephrased by saying that, under
condition in \eref{eq:condsigma}
$$ \calP_L \circ
\calC_{\sigma} \circ \calP_L = \calC_{\sigma} \circ \calP_L \circ
\calC_{\sigma} \circ \calP_L$$ since the two above operators
always produce the same output.

\paragraph{Discussion of Condition in \eref{eq:condsigma}}

Condition \eref{eq:condsigma} is satisfied in practice if $\sigma$
is concave and $\sigma_r(0) \geq l_{\max}$, where $\sigma_r(0)=
\inf_{t>0} \sigma(t)$ is the limit from the right of $\sigma$ at
$0$. This occurs for example if the shaping curve is defined by
the conjunction of leaky buckets, all with bucket size at least as
large as the maximum packet size.

This also sheds some light on the example in \fref{fig:vlp1}: the
problem occurs because the shaping curve $\lambda_C$ does not
satisfy the condition.

The alert reader will ask herself whether a sufficient condition
for \eref{eq:condsigma} to hold is that $\sigma$ is sub-additive
and $\sigma_r(0) \geq l_{\max}$. Unfortunately, the answer is no.
Consider for example the stair function $\sigma=l_{\max}v_{T}$. We
have $\sigma_r(0) = l_{\max}$ but if we try to rewrite $\sigma$
into $\sigma(t)=\sigma_0(t) + l 1_{t>0}$ we must have $l=l{\max}$
and $\sigma_0(t)=0$ for $t \in (0,T]$; if we impose that
$\sigma_0$ is sub-additive, the latter implies $\sigma_0=0$ which
is not compatible with \eref{eq:condsigma}.%
\footnote{The same conclusion unfortunately also holds if we
replace sub-additive by ``star-shaped" (\sref{sec:minplus}).}

\paragraph{Proof of \thref{theo-psps}: }
 We use the notation in
\fref{fig-theo-psps}. We want to show that $R^{(1)}$ is
$\sigma$-smooth. We have $R^* = R \mpc \sigma$. Consider now some
arbitrary $s$ and $t$ with $s<t$. From the definition of min-plus
convolution, for all $\epsilon
>0$, there is some $u \leq s$ such that
\begin{equation}\mylabel{eq-sladk8}
 (R \mpc \sigma)(s)  \geq R(u) + \sigma(s-u) - \epsilon
\end{equation}
Now consider the set $E$ of $\epsilon >0$ such that we can find
one $u < s$ satisfying the above equation. Two cases are possible:
either $0$ is an accumulation point for $E$%
\footnote{namely, there is a sequence of elements in $E$ which
converges to $0$} (case 1) , or not (case 2).

Consider case 1; there is a sequence $(\epsilon_n, s_n)$, with
$s_n < s$,
$$\lim_{n\rightarrow + \infty} \epsilon_n =0$$
and
$$
(R \mpc \sigma)(s)  \geq R(s_n) + \sigma(s-s_n) - \epsilon_n
$$
Now since $s_n \leq t$:
$$
(R \mpc \sigma)(t) \leq R(s_n) + \sigma(t-s_n)
$$
Combining the two:
$$
(R \mpc \sigma)(t) - (R \mpc \sigma)(s) \leq \sigma(t-s_n)-
\sigma(s-s_n) + \epsilon_n
$$
Now $t-s_n >0$ and $s-s_n >0$ thus
$$\sigma(t-s_n) - \sigma(s-s_n) = \sigma_0(t-s_n) -
\sigma_0(s-s_n)
$$
We have assumed that $\sigma_0$ is sub-additive. Now $t \geq s$
thus
$$
\sigma_0(t-s_n) - \sigma_0(s-s_n) \leq \sigma_0(t-s)
$$
we have thus shown that, for all $n$
$$
 (R \mpc \sigma)(t) - (R \mpc \sigma)(s)
 \leq \sigma_0(t-s) + \epsilon_n
$$ and thus
$$
 (R \mpc \sigma)(t) - (R \mpc \sigma)(s) \leq \sigma_0(t-s)
$$
Now from \eref{eq-plmax}, it follows that
$$
 R^{(1)}(t) -   R^{(1)}(s) \leq \sigma_0(t-s) + l_{\max}
 \leq \sigma(t-s)
$$
which ends the proof for case 1.

Now consider case 2. There is some $\epsilon_0$ such that for $0 <
\epsilon < \epsilon_0$, we have to take $u=s$ in \eref{eq-sladk8}, and therefore
$$
(R \mpc \sigma)(s)  \geq R(s) + \sigma(0) - \epsilon = R(s) - \epsilon
$$
Since this holds for every $\epsilon$ such that $0 <
\epsilon < \epsilon_0$, it comes that $(R \mpc \sigma)(s)\geq R(s)$, and thus $
(R \mpc \sigma)(s)  = R(s)
$.

Now $R$ is $L$-packetized by hypothesis. Thus
$$R^{(1)}(s)=P^L((R \mpc \sigma)(s))=
P^L(R (s)) =R(s) = (R \mpc \sigma)(s)$$ thus
$$
  \begin{array}{rl}
    R^{(1)}(t) - R^{(1)}(s)= & P^L((R \mpc \sigma)(t)) - (R \mpc
\sigma)(s) \\
     \leq & (R \mpc \sigma)(t) - (R \mpc \sigma)(s)
  \end{array}
$$
now $R \mpc \sigma$ has $\sigma$ as an arrival curve thus
$$
 R^{(1)}(t) - R^{(1)}(s) \leq \sigma(t-s)
$$
which ends the proof for case 2.
 \qed

\paragraph{Example: Buffered Leaky Bucket Controller based on
Virtual Finish Times} \thref{theo-psps} gives us a practical
implementation for a packet based shaper. Consider that we want to
build a device that ensures that a packet flow satisfies some
concave, piecewise linear arrival curve (and is of course $L$-
packetized). We can realize such a device  as the concatenation of
a buffered leaky bucket controller operating bit-by-bit and a
packetizer.  We compute the output time for the last bit of a
packet (= finish time) under the bit-by-bit leaky bucket
controller, and release the entire packet instantly at this finish
time. If each bucket pool is at least as large as the maximum
packet size then \thref{theo-psps} tells us that the final output
satisfies the leaky bucket constraints.

\paragraph{Counter-example}

If we consider non-concave arrival curves, then we can find an
arrival curve $\sigma$ that does satisfy $\sigma(t) \geq l_{\max}$
for $t>0$ but that does not satisfy \eref{eq:condsigma}. In such a
case, the conclusion of \thref{theo-psps} may not hold in general.
\fref{fig:ctxpsps} shows an example where the output $R^{(1)}$ is
not $\sigma$-smooth, when $\sigma$ is a stair function.
\begin{figure}[!htbp]
  \insfig{ctxpsps}{0.5}
  \mycaption{A counter example for \thref{theo-psps}. A burst of
  10 packets of size
  equal to 10 data units arrive at time $t=0$,
  and $\sigma= 25 v_{1}$. The greedy shaper emits $25$ data
  units at times $0$ and $1$, which forces the packetizer to
  create a
  burst of 3 packets at time 1, and thus $R^{(1)}$ is not $\sigma$-smooth.}
  \mylabel{fig:ctxpsps}
\end{figure}

\subsection{Packetized Greedy Shaper}
\mylabel{sec-pgs} We can come back to the questions raised by the
example in \fref{fig:vlp1} and give a more fundamental look at the
issue of packetized shaping. Instead of synthesizing the
concatenation of a greedy shaper and a packetizer as we did
earlier, we define the following, consistent with \sref{shapers}.
\begin{definition}
\mylabel{def-packshap} [Packetized Greedy Shaper] Consider an
input sequence of packets, represented by the function $R(t)$ as
in \eref{eq-rvlp}. Call $L$ the cumulative packet lengths. We call
\emph{packetized shaper}, with shaping curve $\sigma$, a system
that forces its output to have $\sigma$ as an arrival curve {\emph
and} be $L$-packetized. We call \emph{packetized greedy shaper} a
packetized shaper that delays the input packets in a buffer,
whenever sending a packet would violate the constraint $\sigma$,
but outputs them as soon as possible.
\end{definition}
\paragraph{Example: Buffered Leaky Bucket Controller based on
Bucket Replenishment}  The case $\sigma(t)= \min_{m=1,...,M}
(\gamma_{r_m, b_m}(t)$ can be implemented by a controller that
observes a set of $M$ fluid buckets, where the $m$th bucket is of
size $b_m$ and leaks at a constant rate $r_m$. Every bucket
receives $l_i$ units of fluid when packet $i$ is released ($l_i$
is the size of packet $i$). A packet is released as soon as the
level of fluid in bucket $m$ allows it, that is, has gone down
below $b_m - l_i$, for all $m$. We say that now we have defined a
buffered leaky bucket controller based on ``bucket replenishment".
It is clear that the output has $\sigma$ as an arrival curve, is
$L$-packetized and sends the packets as early as possible. Thus it
implements the packetized greedy shaper. Note that this
implementation differs from the buffered leaky bucket controller
based on virtual finish times introduced in \sref{sec-psps}. In
the latter, during a period where, say, bucket $m$ only is full,
fragments of a packet are virtually released at rate $r_m$, bucket
$m$ remains full, and the (virtual) fragments are then
re-assembled in the packetizer; in the former, if a bucket becomes
full, the controller waits until it empties by at least the size
of the current packet. Thus we expect that the level of fluid in
both systems is not the same, the former being an upper bound. We
will see however in Corollary~\ref{corol-234} that both
implementations are equivalent.

In this example, if a bucket size is less than the maximum packet
size, then it is never possible to output a packet: all packets
remain stuck in the packet buffer, and the output is
$\overline{R}(t)=0$. In general, we can say that
\begin{proposition}
If $\sigma_r(0) < l_{\max}$ then the the packetized greedy shaper
blocks all packets for ever (namely, $\overline{R}(t)=0$). Thus in
this section, we assume that $\sigma(t) \geq l_{\max}$ for $t>0$.
\mylabel{prop:cnvls}
\end{proposition}
Thus, for practical cases, we have to assume that the arrival
curve $\sigma$ has a discontinuity at the origin at least as large
as one maximum packet size.

How does the packetized greedy shaper compare with the
concatenation of a greedy shaper with shaping curve $\sigma$ and a
packetizer~? We know from the example in \fref{fig:vlp1} that the
output has $\sigma'(t)= \sigma(t) + l_{\max} 1_{t>0}$ as an
arrival curve, but not $\sigma$. Now, does the concatenation
implement a packetized greedy shaper with shaping curve
$\sigma'$~?  Before giving a general answer, we study a fairly
general consequence of \thref{theo-psps}.

\begin{theorem}[Realization of packetized Greedy Shaper]
Consider a sequence $L$ of cumulative packet lengths and a ``good"
function $\sigma$. Assume that $\sigma$ satisfies the condition in
\eref{eq:condsigma}. Consider only inputs that are $L$ packetized.
Then the packetized greedy shaper for $\sigma$ and $L$ can be
realized as the concatenation of the greedy shaper with shaping
curve $\sigma$ and the $L$-packetizer.
 \mylabel{theo-iopshprime}
\end{theorem}
\begin{figure}[htbp]
  \insfig{theorpgs}{0.5}
  \mycaption{The packetized greedy shaper can be realized as a (bit-by-bit
  fluid shaper followed by a packetizer, assuming \eref{eq:condsigma} holds.
  In practice, this means
  that we can realize packetized greedy shaping by computing finish times
  in the virtual fluid system and release packets at their finish times.}
  \mylabel{fig-theorpgs}
\end{figure}

\pr
Call $R(t)$ the packetized input; the output of the bit-by-bit
greedy shaper followed by a packetizer is $R^{(1)}(t)=P^L(R \mpc
\sigma)(t))$. Call $\overline{R}(t)$ the output of the packetized
greedy shaper. We have $\overline{R} \leq R$ thus
$\overline{R}\mpc \sigma \leq R\mpc \sigma$ and thus
$$P^L(\overline{R}\mpc \sigma) \leq P^L(R\mpc \sigma)$$
But $\overline{R}$ is $\sigma$-smooth, thus $\overline{R}\mpc
\sigma = \overline{R}$, and is $L$-packetized, thus
$P^L(\overline{R}\mpc \sigma)=\overline{R}$. Thus the former
inequality can be rewritten as
 $\overline{R} \leq R^{(1)}$.
 Conversely, from \thref{theo-psps}, $R^{(1)}$ is also
 $\sigma$-smooth and
 $L$-packetized. The definition of the packetized greedy shaper
 implies that $\overline{R} \geq R^{(1)}$ (for a formal proof, see
 \lref{lem:pakshap}) thus finally $\overline{R} = R^{(1)}$.
\qed

We have seen  that the condition in the theorem is satisfied in
particular if
 $\sigma$ is concave and $\sigma_r(0) \geq l_{\max}$, for
 example if the shaping curve is defined by the conjunction of
 leaky buckets, all with bucket size at least as large as the
 maximum packet size. This shows the following.
\begin{corollary}
 For $L$-packetized inputs, the implementations of buffered leaky
 bucket controllers based on bucket replenishment and virtual finish
 times are equivalent.
 \mylabel{corol-234}
 \end{corollary}


If we relax \eref{eq:condsigma} then the construction of the
packetized greedy shaper is more complex:
\begin{theorem}[I/O characterisation of packetized greedy shapers]
Consider a packetized greedy shaper with shaping curve $\sigma$
and cumulative packet length $L$. Assume that $\sigma$ is a
``good" function. The output $\overline{R}(t)$ of the packetized
greedy shaper is given by
\begin{equation}\mylabel{eq-sp3}
\overline{R} = \inf
 \left\{ R^{(1)},  R^{(2)}, R^{(3)}, ...
 \right\}
\end{equation}
with $R^{(1)}(t)=P^L((\sigma \mpc  R)(t))$ and $R^{(i)}(t)=P^L(
(\sigma \mpc R^{(i-1)})(t))$ for $i\geq 2$.
 \mylabel{theo:iopshapers}
\end{theorem}
\fref{fig:theopgsio} illustrates the theorem, and shows the
iterative construction of the output on one example. Note that
this example is for a shaping function that does not satisfy
\eref{eq:condsigma}. Indeed, otherwise, we know from
\thref{theo-iopshprime} that the iteration stops at the first
step, namely, $\overline{R}=R^{(1)}$ in that case. We can also
check for example that if $\sigma=\lambda_r$ (thus the condition
in \pref{prop:cnvls} is satisfied) then the result of
\eref{eq-sp3} is $0$.
\begin{figure}[htbp]
%  \insfig{theopgsio}{0.5}
  \insfig{theoiogps}{1.0}
  \mycaption{Representation of the output of the packetized greedy
  shaper (left) and example of output (right). The data are the same
  as with \fref{fig:ctxpsps}.}
  \mylabel{fig:theopgsio}
\end{figure}
\pr
The proof is a direct application of \lref{lem:pakshap} (which
itself is an application of the general method in
\sref{sec:fixedpointequation} on \pgref{sec:fixedpointequation}).
\qed

\begin{lemma}
Consider a sequence $L$ of cumulative packet lengths and a ``good"
function $\sigma$. Among all flows $x(t)$ such that
\begin{equation}\mylabel{eq-sp2}
  \bracket {
  x \leq R \\
  x \mbox{ is } L \mbox{-packetized}\\
  x \mbox{ has } \sigma \mbox{ as an arrival curve}
  }
\end{equation}
there is one flow $\overline{R}(t)$ that upper-bounds all. It is
given by \eref{eq-sp3}.
 \mylabel{lem:pakshap}
  \end{lemma}
  \pr
 The lemma is a direct application of \thref{thm:spacemethod}, as
 explained in \sref{sec:iopacketizedgreedyshaper}. However, in
 order to make this chapter self-contained, we give an alternative, direct
 proof, which is quite short.

If $x$ is a solution, then it is straightforward to show by
induction on $i$ that $x(t) \leq R^{(i)}(t)$ and thus $x \leq
\overline{R}$. The difficult part is now to show that
$\overline{R}$ is indeed a solution. We need to show that the
three conditions in \eref{eq-sp2} hold. Firstly, $R^{(1)}\leq
R(t)$ and by induction on $i$, $R^{(i)}\leq R$ for all $i$; thus
$\overline{R}\leq R$.

Secondly, consider some fixed $t$; $R^{(i)}(t)$ is $L$-packetized
for all $i \geq 1$. Let $L(n_0):= R^{(1)}(t)$. Since
$R^{(i)}(t)\leq R^{(1)}(t)$,  $R^{(i)}(t)$ is in the set
$$\{L(0),
L(1), L(2), ..., L(n_0)\}.$$ This set is finite, thus,
$\overline{R}(t)$, which is the infimum of elements in this set,
has to be one of the $L(k)$ for $k\leq n_0$. This shows that
$\overline{R}(t)$ is $L$-packetized, and this is true for any time
$t$.

Thirdly, we have, for all $i$
$$\overline{R}(t) \leq R^{(i+1)}(t) = P^L ((\sigma \mpc R^{(i)})(t)) \leq (\sigma \mpc
R^{(i)})(t)$$
 thus
 $$\overline{R} \leq \inf_{i} (\sigma \mpc
R^{(i)})$$

Now convolution by a fixed function is upper-semi-continuous,
which means that
$$\inf_{i} (\sigma \mpc
R^{(i)}) = \sigma \mpc \overline{R}$$ This is a general result in
\cref{L11} for any min-plus operator. An elementary proof is as
follows.
 $$
  \begin{array}{rl}
    \inf_i ( \sigma \mpc R^{(i)} )(t) = &
    \inf_{s \in [0,t], i \in \Nats}\left[\sigma(s) + R^{(i)}(t-s)\right] \\
    = &
    \inf_{s \in [0,t]} \left\{\inf_{i \in \Nats}
    \left[(\sigma(s) + R^{(i)}(t-s)\right] \right\}\\
    = &  \inf_{s \in [0,t]} \left\{\sigma(s) + \inf_{i \in \Nats}
    \left[R^{(i)}(t-s)\right] \right\}\\
    = & \inf_{s \in [0,t]} \left[\sigma(s) +  \overline{R}(t-s) \right]\\
    = & (\sigma \mpc \overline{R})(t)
  \end{array}
 $$


Thus
$$\overline{R} \leq \sigma \mpc \overline{R},$$
which shows the third condition. Note that $\overline{R}$ is
wide-sense increasing. \qed

%If $x$ is a solution, then it is straightforward to show by
%induction on $i$ that $x(t) \leq R^{(i)}(t)$ and thus $x \leq
%\overline{R}$. The difficult part is now to show that
%$\overline{R}$ is indeed a solution. We need to show that the
%three conditions in \eref{eq-sp2} hold. Firstly, $R^{(1)}\leq
%R(t)$ and by induction on $i$, $R^{(i)}\leq R$ for all $i$; thus
%$\overline{R}\leq R$.
%
%Secondly, $R^{(i)}$ is clearly $L$-packetized, thus takes values
%in the set $\calS=\{L(0), L(1), L(2), ...\}$. Now $\calS$ has no
%finite accumulation point, by hypothesis. Thus, for any $t$,
%$\overline{R}(t)$, which is the infimum of elements in $\calS$,
%has to be one of the $L(k)$ for some $k$, which shows that
%$\overline{R}$ is $L$-packetized.
%
%Thirdly, we have, for all $i$
%$$\overline{R}(t) \leq R^{(i+1)}(t) = P^L ((\sigma \mpc R^{(i)})(t)) \leq (\sigma \mpc
%R^{(i)})(t)$$
% thus
% $$\overline{R} \leq \inf_{i} (\sigma \mpc
%R^{(i)})$$



%\paragraph{Does packetized greedy shaping increase the delay bound ?}
%To be done. Je crois que la r\'{e}ponse est non, mais donner des
%bornes.

\paragraph{Does a packetized greedy shaper keep arrival constraints ?}

\fref{fig:ctxshk} shows a counter-example, namely, a variable
length packet flow that has lost its initial arrival curve
constraint after traversing a packetized greedy shaper.

\begin{figure}[!htbp]
  \insfig{ctxshk}{0.7}
  \mycaption{The input flow is shown above; it consists of 3 packets
  of size 10 data units and one of size 5 data units,
   spaced by one time unit. It is
  $\alpha$-smooth with $\alpha=10 v_{1,0}$. The bottom flow is the
  output of the packetized greedy shaper with $\sigma=25 v_{3,0}$.
  The output has a burst of $15$ data units packets at time $3$. It
  is $\sigma$-smooth but \emph{not} $\alpha$-smooth.}
  \mylabel{fig:ctxshk}
\end{figure}


However, if arrival curves are defined by leaky buckets, we have a
positive result.

\begin{theorem}[Conservation of concave arrival constraints]
Assume an $L$-packetized flow with arrival curve $\alpha$ is input
to a packetized greedy shaper with cumulative packet length $L$
and shaping curve $\sigma$. Assume that $\alpha$ and $\sigma$ are
concave with $\alpha_r(0) \geq l_{\max}$ and $\sigma_r(0) \geq
l_{\max}$. Then the output flow is still constrained by the
original arrival curve $\alpha$.
 \mylabel{theo-pshkparc}
\end{theorem}
 \pr
%We use the notation in \thref{theo:iopshapers}.
 Since $\sigma$
 satisfies \eref{eq:condsigma}, it follows from
 \thref{theo-iopshprime} that
 $
 \overline{R} = P^L(\sigma \mpc R)$.
Now $R$ is $\alpha$-smooth thus it is not modified by a bit-by-bit
greedy shaper with shaping curve $\alpha$, thus
    $R= \alpha \mpc R$.
Combining the two and using the associativity of $\mpc$ gives
    $
    \overline{R}=
     P^L [(\sigma \mpc \alpha) \mpc
    R]$.
From our hypothesis, $\sigma \mpc \alpha= \min(\sigma, \alpha) $
(see \thref{thm:rule8-9} on \pgref{thm:rule8-9})
    and thus $\sigma \mpc \alpha$ satisfies
    \eref{eq:condsigma}. Thus, by \thref{theo-psps}, $\overline{R}$ is
    $\sigma \mpc \alpha$-smooth, and thus $\alpha$-smooth.
    \qed

%    We use the notation in \thref{theo-psps}.
%    Call $R(t)$ the input to the packetized greedy shaper.
%    By applying \thref{theo-iopshprime}, since $\alpha$
%    satisfies \eref{eq:condsigma}, there is some $R_0$ such
%    that
%    $$R=
%      \calP_L \circ \calC_{\alpha} \circ \calP_L (R_0)
%    $$
%    By the same token, the output $\overline{R}$
%    of the
%    packetized greedy shaper satisfies
%    $$
%\overline{R}=   \calP_L \circ \calC_{\sigma} \circ \calP_L \circ
%\calP_L \circ \calC_{\alpha} \circ \calP_L
%    (R_0)
%    $$
%    Now $\calP_L$ is idempotent, thus
%    $$
%    \overline{R}= \calP_L \circ \calC_{\sigma} \circ \calP_L \circ \calC_{\alpha} \circ \calP_L (R_0)
%    $$
%    From \thref{theo-psps}, we have
%    $\calP_L \circ \calC_{\alpha} \circ \calP_L= \calC_{\alpha}\circ \calP_L \circ \calC_{\alpha} \circ \calP_L $, thus
%    $$
%\overline{R} = \calP_L \circ \calC_{\sigma} \circ \calC_{\alpha}
%\circ \calP_L \circ \calC_{\alpha} \circ \calP_L
%    (R_0)
%    $$
%    Note that $\calC_{\sigma} \circ \calC_{\alpha}= \calC_{\alpha \mpc \sigma}$, thus
%    $$
%\overline{R} =\calP_L \circ \calC_{\alpha \mpc \sigma} \circ
%\calP_L  \circ \calC_{\alpha} \circ \calP_L(R_0)
%    $$
%    Now from our hypothesis, $\alpha \mpc \sigma= \alpha \wedge \sigma$
%    and thus $\alpha \mpc \sigma$ satisfies
%    \eref{eq:condsigma}. Thus
%    $$
%   \overline{R}=
%     \calC_{\alpha \mpc \sigma}
%     \circ \calP_L \circ \calC_{\alpha \mpc \sigma} \circ \calP_L \circ \calC_{\alpha}
%     \circ \calP_L( R_0 )
%    $$
%    which shows that $\overline{R}$ is $\alpha \mpc
%    \sigma$-smooth, and thus $\alpha$-smooth.
%
% \qed
%

\paragraph{Series decomposition of shapers}
\begin{theorem}
Consider a tandem of $M$ packetized greedy shapers in series;
assume that the shaping curve $\sigma^m$ of the $m$th shaper is
concave with $\sigma^m_r(0) \geq l_{\max}$. For $L$-packetized
inputs, the tandem is equivalent to the packetized greedy shaper
with shaping curve $\sigma= \min_m \sigma^m$.
 \mylabel{theo-pshserdec}
\end{theorem}
\pr We do the proof for $M=2$ as it extends without difficulty to
larger values of $M$. Call $R(t)$ the packetized input, $R'(t)$
the output of the tandem of shapers, and $\overline{R}(t)$ the
output of the packetized greedy shaper with input $R(t)$ and shaping curve $\sigma$.

Firstly, by \thref{theo-iopshprime}
$$R' = P^L [\sigma^2 \mpc P^L (\sigma^1 \mpc R)]$$
Now $\sigma^m \geq \sigma$ for all $m$ thus
$$R' \geq  P^L [\sigma \mpc P^L (\sigma \mpc R)]$$
Again by \thref{theo-iopshprime}, we have $\overline{R} = P^L
(\sigma \mpc R)$. Moreover  $\overline{R}$ is $L$-packetized and
$\sigma$-smooth, thus $\overline{R}=P^L(\overline{R})$ and
$\overline{R}= \sigma \mpc \overline{R}$. Thus finally
\begin{equation}\mylabel{eq-prl-ks8}
 R' \geq  \overline{R}
\end{equation}

Secondly, $R'$ is $L$-packetized and by \thref{theo-pshkparc}, it
is $\sigma$-smooth. Thus the tandem is a packetized (possibly non
greedy) shaper. Since $\overline{R}(t)$ is the output of the
packetized greedy shaper, we must have $R' \leq \overline{R}$.
Combining with \eref{eq-prl-ks8} ends the proof. \qed

It follows that a shaper with shaping curve $\sigma(t)=
\min_{m=1,...,M} (r_m t+b_m)$, where $b_m \geq l_{\max}$ for all
$m$, can be implemented by a tandem of $M$ individual leaky
buckets, in any order. Furthermore, by \coref{corol-234}, every
individual leaky bucket may independently be based either on
virtual finish times or on bucket replenishment.

If the condition in the theorem is not satisfied, then the
conclusion may not hold. Indeed, for the example in
\fref{fig:ctxshk}, the tandem of packetized greedy shapers with
curves $\alpha$ and $\sigma$ does not have an $\alpha$-smooth
output, therefore it cannot be equivalent to the packetized greedy
shaper with curve $\min (\alpha, \sigma)$.



Unfortunately, the other shaper properties seen in \sref{shapers}
do not generally hold. For shaping curves that satisfy
\eref{eq:condsigma}, and when a packetized greedy shaper is
introduced, we need to compute the end-to-end service curve by applying \thref{theo-delvlp}.%
%
% the end-to-end service g a packettizer
%
%To do: show that reshaping adds $2 l_{\max}$ to arrival curve (is
%it true ?) .
%
%
%Comparer concatenation ($\sigma_0$ + $L$-packetizer) avec sigma
%prime packetized greedy shaper.
%
%Reponse: cela diff\`{e}re d'au plus $2 l_{\max}$ (voir
%propri\'{e}t\'{e} S2).


\section[Effective Bandwidth and Equivalent Capacity]{Lossless Effective Bandwidth and Equivalent Capacity}
\mylabel{sec-effbw}
\subsection{Effective Bandwidth of a Flow}

We can apply the results in this chapter to define a function of a
flow called the effective bandwidth. This function characterizes
the bit rate required for a given flow. More precisely, consider a
flow with cumulative function $R$; for a fixed, but arbitrary
delay $D$, we define the \emph{effective bandwidth} $e_D(R)$ of
the flow as the bit rate required to serve the flow in a work
conserving manner, with a virtual delay $\leq D$.
\begin{proposition}
\mylabel{prop-efb-def} The effective bandwidth of a flow is given
by
\begin{equation}
    e_{D}(R) = \sup_{0\leq s \leq t} \frac{R(t)-R(s)}{t-s+D}
    \mylabel{eq-effBw}
\end{equation}
\end{proposition}
For an arrival curve $\alpha$ we define the effective bandwidth
$e_D(\alpha)$ as the effective bandwidth of the greedy flow
$R=\alpha$. By a simple manipulation of Equation~\ref{eq-effBw},
the following comes.
\begin{proposition}
\mylabel{prop-efb-def2} The effective bandwidth of a ``good"
arrival curve is given by
\begin{equation}
    e_{D}(\alpha) = \sup_{0\leq s } \frac{\alpha(s)}{s+D}
    \mylabel{eq-effBw2}
\end{equation}
\end{proposition}
The alert reader will check that the effective bandwidth of a flow
$R$ is also the effective bandwidth of its minimum arrival curve
$R \mpd R$. For example, for a flow with T-SPEC $(p,M,r,b)$, the
effective bandwidth is the maximum of $r$ and the slopes of lines
$(QA_{0})$ and $(QA_{1})$ in Figure \ref{fig-2784}; it is thus
equal to:
\begin{equation}
    e_{D} = \max \left\{ \frac{M}{D}, r, p \left(1 -
\frac{D-\frac{M}{p}}{\frac{b-M}{p-r}+D}\right) \right\}
    \mylabel{eq-cac2-11}
\end{equation}
\begin{figure}[!htbp]
\insfig{vbrefb1}{0.8}
    \mycaption{Computation of Effective Bandwidth for a VBR flow (left); example
     for $r=20$ packets/second, $M=10$ packets, $p=200$ packets per second and
     $b=26$ packets (right). }
    \mylabel{fig-2784}
\end{figure}
Assume $\alpha$ is sub-additive. We define the sustainable rate
$m$ as $m=\liminf_{s\rightarrow +\infty} \frac{\alpha(s)}{s}$ and
the peak rate by $p = \sup_{s>0}\frac{\alpha(s)}{s}$.  Then $m
\leq e_{D}(\alpha) \leq p$ for all $D$.  Moreover, if $\alpha $ is
concave, then $\lim_{D \rightarrow + \infty} e_{D}(\alpha) = m$.If
$\alpha$ is differentiable, $e(D)$ is the slope of the tangent to
the arrival curve, drawn from the time axis at $t=-D$ (Figure
\ref{fig-effBw}).
\begin{figure}[!htbp]
\insfig{F12}{0.9}
    \mycaption{Effective Bandwidth for a delay constraint $D$ and
    Equivalent Capacity for a buffer size $B$}
    \mylabel{fig-effBw}
\end{figure}
It follows also directly from the definition in (\ref{eq-effBw})
that
\begin{equation}
    e_{D}(\sum_{i} \alpha_{i})  \leq \sum_{i}e_{D}(\alpha_{i})
    \mylabel{eq-netcal-y<nmjjkwe889237834hj}
\end{equation}
In other words, the effective bandwidth for an aggregate flow is
less than or equal to the sum of effective bandwidths.  If the
flows have all \emph{identical} arrival curves, then the aggregate
effective bandwidth is simply $ I\times e_{D}(\alpha_{1})$.  It is
this latter relation that is the origin of the term ``effective
bandwidth''. The difference $\sum_{i}e_{D}(\alpha_{i}) -
e_{D}(\sum_{i} \alpha_{i})$ is a buffering gain; it tells us how
much capacity is saved by sharing a buffer between the flows.

\subsection{Equivalent Capacity}
 Similar results hold if we replace delay constraints by the
requirement that a fixed buffer size is not exceeded. Indeed, the
queue with constant rate $C$, guarantees a maximum backlog of $B$
(in bits) for a flow $R$ if $C \geq f_{B}(R)$, with
\begin{equation}
f_{B}(R) = \sup_{0 \leq s < t} \frac{R(t)-R(s)-B}{t-s}
\mylabel{eq-eqCapa}
\end{equation}
Similarly, for a ``good" function $\alpha$, we have:
\begin{equation}
f_{B}(\alpha) = \sup_{s > 0} \frac{\alpha(s)-B}{s}
\mylabel{eq-eqCapa2}
\end{equation}
We call $f_{B}(\alpha)$ the \emph{equivalent capacity}, by analogy
to \cite{autobahn}.  Similar to effective bandwidth, the
equivalent capacity of a heterogeneous mix of flows is less than
or equal to the sum of equivalent capacities of the flows,
provided that the buffers are also added up; in other words,
$f_{B}(\alpha)  \leq \sum_{i}f_{B_{i}}(\alpha_{i})$, with $\alpha
= \sum_{i} \alpha_{i}$ and $B= \sum_{i}B_{i}$. Figure
\ref{fig-effBw} gives a graphical interpretation.

For example, for a flow with T-SPEC $(p,M,r,b)$, using the same
method as above, we find the following equivalent capacity:
\begin{equation}
    f_{B} = \bracket{
    \mif B < M \mthen +\infty\\
    \melse r + \frac{(p-r)(b-B)^+}{b-M}
    }
    \mylabel{eq-cac2-111}
\end{equation}

An immediate computation shows that $f_b(\gamma_{r,b})=r$. In
other words, if we allocate to a flow, constrained by an affine
function $\gamma_{r,b}$, a capacity equal to its sustainable rate
$r$, then a buffer equal to its burst tolerance $b$ is sufficient
to ensure loss-free operation.

Consider now a mixture of Intserv flows (or VBR connections), with
T-SPECs ($M_i, p_i, r_i, b_i$). If we allocate to this aggregate
of flows the sum of their sustainable rates $\sum_i r_i$, then the
buffer requirement is the sum of the burst tolerances $\sum_i
b_i$, regardless of other parameters such as peak rate.
Conversely, Equation~\ref{eq-cac2-111} also illustrates that there
is no point allocating more buffer than the burst tolerance: if
$B>b$, then the equivalent capacity is still $r$.

The above has illustrated that it is possible to reduce the
required buffer or delay by allocating a rate larger than the
sustainable rate. In Section~\ref{sec-L20-fcs}, we described how
this may be done with a protocol such as RSVP.

Note that formulas (\ref{eq-effBw}) or (\ref{eq-eqCapa}), or both,
can be used to estimate the capacity required for a flow, based on
a measured arrival curve. We can view them as low-pass filters on
the flow function $R$.

\subsection{Example: Acceptance Region for a FIFO Multiplexer}

Consider a node multiplexing $n_1$ flows of type 1 and $n_2$ flows
of type 2, where every flow is defined by a T-SPEC $(p_i, M_i,
r_i, b_i)$. The node has a constant output rate $C$. We wonder how
many flows the node can accept.


If the only condition for flow acceptance is that the delay for
all flows is bounded by some value $D$, then the set of acceptable
values of $(n_1, n_2)$ is defined by
 $$
 e_D(n_1 \alpha_1+ n_2 \alpha_2) \leq C
 $$
We can use the same convexity arguments as for the derivation of
formula~(\ref{eq-cac2-11}), applied to the function $n_1 \alpha_1+
n_2 \alpha_2$. Define $\theta_i=\frac{b_i-M}{p_i-r_i}$ and assume
$\theta_1 \leq \theta_2$. The result is:
 $$
   e_D(n_1 \alpha_1+ n_2 \alpha_2) = \max \bracket{
   \frac{n_1 M_1+n_2 M_2}{D},\\
   \frac{n_1 M_1+n_2 M_2 + (n_1 p_1 + n_2 p_2)\theta_1}{\theta_1 +
   D},\\
   \frac{n_1 b_1 +n_2 M_2 + (n_1 r_1 + n_2 p_2)\theta_2}{\theta_2 +
   D},\\
   n_1 r_1 +n_2 r_2
   }
 $$
The set of feasible $(n_1, n_2)$ derives directly from the
previous equation; it is the convex part shown in
Figure~\ref{fig-cacefb-0599}. The alert reader will enjoy
performing the computation of the equivalent capacity for the case
where the acceptance condition bears on a buffer size $B$.
\begin{figure}[!htbp]
\begin{center}
%\insfig{cacefb}{0.7}
 \begin{tabular}{|c|c|c|c|c|c|}
        \hline
        $i$ &  $p_{i}$ & $M_i$ & $r_{i}$ & $b_{i}$ & $\theta_{i}$  \\
        \hline
        \hline
        1 & 20'000 packets/s & 1 packet & 500 packets/s & 26 packets & 1.3 ms  \\
        \hline
         2 & 5'000 packets/s & 1 packet & 500 packets/s & 251 packets & 55.5 ms \\
        \hline
    \end{tabular}
\end{center}
 \mycaption{Acceptance region for a mix of type 1 and type 2 flows. Maximum delay
  $D=xx$.
  The parameters for types 1 and 2 are shown in the table, together with the resulting values of
  $\theta_i$.}
  \mylabel{fig-cacefb-0599}
\end{figure}

Coming back to equation \ref{eq-netcal-y<nmjjkwe889237834hj}, we
can state in more general terms that the effective bandwidth is a
convex function of function $\alpha$, namely:
$$ e_{D}(a\alpha_{1} + (1-a) \alpha_{2}) \leq
a e_{D}(\alpha_{1}) + (1-a) e_{D}(\alpha_{2}) $$ for all $a \in
[0,1]$. The same is true for the equivalent capacity function.


Consider now a call acceptance criterion based solely on a delay
bound, or based on a maximum buffer constraint, or both. Consider
further that there are $I$ types of connections, and define the
acceptance region $\cal{A}$ as the set of values
$(n_{1},\ldots,n_{I})$ that satisfy the call acceptance criterion,
where $n_{i}$ is the number of connections of class $i$. From the
convexity of the effective bandwidth and equivalent capacity
functions, it follows that the acceptance region $\cal{A}$ is
\emph{convex}. In chapter~\ref{L30} we compare this to acceptance
regions for systems with some positive loss probability.


\paragraph{Sustainable Rate Allocation}

If we are interested only in course results, then we can
reconsider the previous solution and take into account only the
sustainable rate of the connection mix. The aggregate flow is
constrained (among others) by $\alpha(s) = b + rs$, with
$b=\sum_{i}n_i b_{i}$ and $r=\sum_{i}n_{i}r_i$. Theorem
\ref{theo-backlog} shows that the maximum aggregate buffer
occupancy is bounded by $b$ as long as $C \geq r$. In other words,
allocating the sustainable rate guarantees a loss-free operation,
as long as the total buffer is equal to the burstiness.

In a more general setting, assume an aggregate flow has $\alpha$
as minimum arrival curve, and assume that some parameters $r$ and
$b$ are such that
    $$\lim_{s \rightarrow + \infty} \alpha(s) - rs - b = 0
    $$
so that the sustainable rate $r$ with burstiness $b$ is a tight
bound. It can easily be shown that if we allocate a rate $C=r$,
then the maximum buffer occupancy is $b$.


Consider now multiplexing a number of VBR connections. If no
buffer is available, then it is necessary for a loss-free
operation to allocate the sum of the peak rates. In contrast,
using a buffer of size $b$ makes it possible to allocate only the
sustainable rate. This is what we call the \emph{buffering gain},
namely, the gain on the peak rate obtained by adding some buffer.
The buffering gain comes at the expense of increased delay, as can
easily be seen from Theorem~\ref{theo-delay}.

\section{Proof of Theorem~\ref{theo-outworst}}
\paragraph{Step 1:} Consider a fixed time $t_0$ and assume, in this
step, that there is some time $u_0$ that achieves the supremum in
the definition of $\alpha \mpd \beta$. We construct some input and
output functions $R$ and $R^*$ such that $R$ is constrained by
$\alpha$, the system $(R, R^*)$ is causal, and $\alpha^*(t_0) =
(R^*\mpd R^*)(t_0)$. $R$ and $R^*$ are given by
(Figure~\ref{fig-stepq1})
\begin{figure}[htbp]
  \insfig{stepq1}{0.4}
  \mycaption{Step 1 of the proof of Theorem~\ref{theo-outworst}:
   a system that attains
   the output bound at one value $t_0$.}
   \mylabel{fig-stepq1}
\end{figure}
$$
\bracket{ R(t)=\alpha(t) \mif t < u_0 + t_0\\
 R(t)=\alpha(u_0 + t_0) \mif t \geq u_0 + t_0 \\
 R^*(t)=\inf[\alpha(t),\beta(t)] \mif t < u_0 + t_0  \\
 R^*(t)=R(t) \mif t \geq u_0 + t_0
 }
$$
It is easy to see, as in the proof of Theorem~\ref{theo-bist} that
$R$ and $R^*$ are wide-sense increasing, that $R^* \leq R$ and
that $\beta$ is a service curve for the flow. Now
 $$
 R^*(u_0+t_0)-R^*(u_0)=\alpha(u_0+t_0)-R^*(u_0)
  \geq \alpha(u_0+t_0)-\beta(u_0) = \alpha^*(t_0)
 $$

\paragraph{Step 2:} Consider now a sequence of times
$t_0, t_1, ..., t_n, ...$ (not necessarily increasing). Assume, in
this step, that for all $n$ there is a value $u_n$ that achieves
the supremum in the definition of $(\alpha \mpd \beta)(t_n)$. We
prove that there are some functions $R$ and $R^*$ such that $R$ is
constrained by $\alpha$, the system $(R, R^*)$ is causal, has
$\beta$ as a service curve, and $\alpha^*(t_n) = (R^*\mpd
R^*)(t_n)$ for all $n \geq 0$.

We build $R$ and $R^*$ by induction on a set of increasing
intervals $[0,s_0]$,  $[0,s_1]$,..., $[0,s_n] ...$. The induction
property is that the system restricted to time interval $[0,s_n]$
is causal, has $\alpha$ as an arrival curve for the input, has
$\beta$ as a service curve, and satisfies $\alpha^*(t_i) =
(R^*\mpd R^*)(t_i)$ for $i \leq n$.

The first interval is defined by $s_0= u_0+t_0$; $R$ and $R^*$ are
built on $[0,s_0]$ as in step 1 above. Clearly, the induction
property is true for $n=0$. Assume we have built the system on
interval $[0,s_n]$. Define now $s_{n+1}= s_n + u_n + t_n +
\delta_{n+1}$. We chose $\delta_{n+1}$ such that
\begin{equation}\mylabel{eq-delta}
  \alpha(s+\delta_{n+1}) - \alpha(s) \geq R(s_n) \mfa s \geq 0
\end{equation}
This is possible from the last condition in the Theorem. The
system is defined on $]s_n, s_{n+1}]$ by (Figure~\ref{fig-stepq2})
 $$
 \bracket{
 R(t) = R^*(t) = R(s_n) \mf s_n < t \leq s_n + \delta_{n+1}\\
 R(t) = R(s_n) + \alpha(t- s_n - \delta_{n+1}) \mf s_n + \delta_{n+1} < t \leq s_{n+1}\\
 R^*(t) = R(s_n) + (\alpha \wedge \beta) (t- s_n - \delta_{n+1}) \mf s_n + \delta_{n+1} < t <
 s_{n+1}\\
 R^*(s_{n+1}) = R(s_{n+1})
 }
 $$
\begin{figure}[htbp]
  \insfig{stepq2}{0.7}
  \mycaption{Step 2 of the proof of Theorem~\ref{theo-outworst}:
   a system that attains
   the output bound for all values $t_n$, $n \in \Nats$.}
   \mylabel{fig-stepq2}
\end{figure}
We show now that the arrival curve constraint is satisfied for the
system defined on $[0,s_{n+1}]$. Consider $R(t)-R(v)$ for $t$ and
$v$ in $[0,s_{n+1}]$. If both $t \leq s_n$ and $v \leq s_n$, or if
both $t
> s_n$ and $v > s_n$ then the arrival curve property holds from
our construction and the induction property. We can thus assume
that $t>s_n$ and $v\leq s_n$. Clearly, we can even assume that $t
\geq s_n + \delta_{n+1}$, otherwise the property is trivially
true. Let us rewrite $t=s_n + \delta_{n+1} + s$. We have, from our
construction:
 $$
 R(t) -R(v)= R(s_n + \delta_{n+1} + s) - R(v) = R(s_n) + \alpha(s) -R(v)
 \leq R(s_n) + \alpha(s)
 $$
Now from Equation~(\ref{eq-delta}), we have:
 $$
 R(s_n) + \alpha(s)\leq \alpha(s+\delta_{n+1}) \leq \alpha(s+\delta_{n+1} +
 s_n - v) = \alpha(t-v)
 $$
which shows the arrival curve property.

Using the same arguments as in step 1, it is simple to show that
the system is causal, has $\beta$ as a service curve, and that
 $$
 R^*(u_{n+1}+t_{n+1})-R^*(u_{n+1})= \alpha^*(t_{n+1})
 $$
which ends the proof that the induction property is also true for
$n+1$.


\paragraph{Step 3: }
Consider, as in step 2, a sequence of times $t_0, t_1, ..., t_n,
...$ (not necessarily increasing). We now extend the result in
step 2 to the case where the supremum in the definition of
$\alpha^*=(\alpha \mpd \beta)(t_n)$ is not necessarily attained.
Assume first that $\alpha^*(t_n)$ is finite for all $n$. For all
$n$ and all $m \in \Nats^*$ there is some $u_{m,n}$ such that
\begin{equation}\mylabel{eq-umn}
  \alpha(t_n + u_{m,n}) - \beta(u_{m,n}) \geq \alpha^*(t_n) -
 \frac{1}{m}
\end{equation}
Now the set of all couples $(m,n)$ is enumerable. Consider some
numbering $(M(i), N(i))$, $i \in \Nats$ for that set. Using the
same construction as in step 2, we can build by induction on $i$ a
sequence of increasing intervals $[0, s_i]$ and a system $(R,R^*)$
that is causal, has $\alpha$ as an arrival curve for the input,
has $\beta$ as a service curve, and such that
 $$
 R^*(s_i)- R^*(s_i- t_{N(i)}) \geq \alpha^*(t_{N(i)})- \frac{1}{M(i)}
 $$
Now consider an arbitrary, but fixed $n$. By applying the previous
equations to all $i$ such that $N(i)=n$, we obtain
$$
  \begin{array}{rl}
    (R^* \mpd R^*)(t_n)
     \geq & \sup_{i \mst N(i)=n} \left\{
    \alpha^*(t_{N(i)}) - \frac{1}{M(i)}\right\}\\
    =  & \alpha^*(t_n) - \inf_{i \mst N(i)=n}
  \left\{\frac{1}{M(i)}\right\}
  \end{array}
 $$
Now the set of all $\frac{1}{M(i)}$ for $i$ such that $N(i)=n$ is
$\Nats^*$, thus
 $$\inf_{i \mst N(i)=n}\left\{\frac{1}{M(i)}\right\}=0$$
and thus $(R^* \mpd R^*)(t_n)= \alpha^*(t_n)$, which ends the
proof of step 3 in the case where $\alpha^*(t_n)$ is finite for
all $n$.

A similar reasoning can be used if $\alpha^*(t_n)$ is infinite for
some $t_n$. In that case replace Equation~(\ref{eq-umn}) by
$\alpha(t_n + u_{m,n}) - \beta(u_{m,n}) \geq m $.

\paragraph{Step 4:} Now we conclude the proof. If time is
discrete, then step 3 proves the theorem. Otherwise we use a
density argument. The set of nonnegative rational numbers
$\Rats^+$ is enumerable; we can thus apply step 3 to the sequence
of all elements of $\Rats^+$, and obtain system $(R,R^*)$, with
 $$
 (R^* \mpd R^*)(q)= \alpha^*(q) \mfa q \in \Rats^+
 $$
Function $R^*$ is right-continuous, thus, from the discussion at
the end of Theorem~\ref{theo-mac}, it follows that $R^* \mpd R^*$
is left-continuous. We now show that $\alpha^*$ is also
left-continuous. For all $t \geq 0$ we have:
 $$
 \sup_{s < t} \alpha^*(s) =
 \sup_{(s,v) \mst s < t \mand v \geq 0} \{ \alpha(s+v) -\beta(v)\}
 =
 \sup_{v \geq 0} \{
   \sup_{s < t} [\alpha(s+v) -\beta(v)]\}
 $$
Now
 $$\sup_{s < t} \alpha(s+v)= \alpha(t+v)$$
because $\alpha$ is left-continuous. Thus
 $$
  \sup_{s < t} \alpha^*(s)= \sup_{v \geq 0} \{
   \alpha(t+v) -\beta(v)]\} =  \alpha^*(t)
 $$
which shows that $\alpha$ is left-continuous.

Back to the main argument of step 4, consider some arbitrary $t
\geq 0$. The set $\Rats^+$ is dense in the set of nonnegative real
numbers, thus there is a sequence of rational numbers $q_n \in
\Rats^+$, with $n \in \Nats$, such that $q_n \leq t$ and $\lim_{n
\rightarrow +\infty }q_n = t$. From the left-continuity of $R^*
\mpd R^*$ and $\alpha^*$ we have:
 $$
 (R^* \mpd R^*)(t)=
 \lim_{n\rightarrow +\infty }(R^* \mpd R^*)(q_n)=
 \lim_{n\rightarrow +\infty }\alpha^*(q_n)= \alpha^*(t)
 $$
 \qed

\section{Bibliographic Notes}

Network calculus as has been applied to dimensioning ATM switches
in \cite{naudts96}. A practical algorithm for the determination of
the minimum arrival curve for ATM system is described in
\cite{naudts99}. It uses the burstiness function of a flow,
defined in \cite{LV91} as  follows. For any $r$, $B(r)$ is the
minimum $b$ such that the flow is $\gamma_{r,b}$-smooth, and is
thus the required buffer if the flow is served at a constant rate
$r$. Note that $B(r)$ is the Legendre transform of the minimum
arrival curve $\sigma$ of the flow, namely, $B(r)= \sup_{t \geq
0}(\sigma(t)-rt)$ \cite{naudts99} gives a fast algorithm for
computing $B(r)$. Interestingly, the concept is applied also to
the distribution of symbols in a text.

In \cite{thiele98}, the concepts of arrival and service curve are
used to analyze real time processing systems. It is shown that the
service curve for a variable capacity node must be super-additive,
and conversely, any super-additive function is a service curve for
a variable capacity node. Compare to greedy shapers, which have a
sub-additive service curve. This shows that, except for constant
bit rate trunks, a greedy shaper cannot be modeled as a variable
capacity node, and conversely.

In \cite{chang99}, the authors consider a crossbar switch, and
call $r_{i, j}$ the rate assigned to the traffic from input port
$i$ to output port $j$. Assume that $\sum_i r_{i,j} \leq 1$ for
all $j$ and $\sum_j r_{i,j} \leq 1$ for all $i$. Using properties
of doubly-stochastic matrices (such as $(r_{i,j})$ is), they give
a simple scheduling algorithm that guarantees that the flow from
port $i$ to port $j$ is allocated a variable capacity $C$
satisfying $C_{i,j}(t) - C_{i,j}(s) \geq r_{i,j}(t-s) - s_{i,j}$
for some $s_{i,j}$ defined by the algorithm. Thus, the node offers
a service curve equal to the rate-latency function
$\beta_{r_{i,j}, s_{i,j}}$.

A dual approach to account for variable length packets is
introduced in \cite{Changbook}. It consists in replacing the
definition of arrival curve (or $\sigma$-smoothness) by the
concept of $g$-regularity. Consider a flow of variable length
packets, with cumulative packet length $L$ and call $T_i$ the
arrival epoch for the $i$th packet. The flow is said to be
$g$-regular if $T(j)-T(i) \geq g(L(j)-L(i))$ for all packet
numbers $i\leq j$. A theory is then developed with concepts
similar to the greedy shaper. The theory uses max-plus convolution
instead of min-plus convolution. The $(b, r)$ regulator originally
introduced by Cruz \cite{cru91a} is a shaper in this theory, whose
output is $g$-regular, with $g(x)=\frac{(x-b)}{r}^+$. This theory
does not exactly correspond to the usual concept of leaky bucket
controllers. More specifically, there is not an exact
correspondence between the set of flows that are $g$-regular on
one hand, and that are $\sigma$-smooth on the other. We explain
why with an example. Consider the set of flows that are
$g$-regular, with $g(x)=\frac{x}{r}$. The minimum arrival curve we
can put on this set of flows is $\sigma(t)= r t + l_{\max}$
\cite{Changbook}. But conversely, if a flow is $\sigma$-smooth, we
cannot guarantee that it is $g$-regular. Indeed, the following
sequence of packets is a flow that is $\sigma$-smooth but not
$g$-regular: the flow has a short packet (length $l_1<l_{\max}$)
at time $T_1=0$, followed by a packet of maximum size $l_{\max}$
at time $T_2=\frac{l_1}{r}$. In fact, if a flow is
$\sigma$-smooth, then it is $g'$-regular, with $g'(x)=
\frac{(x-l_{\max})}{r}^+$.

The strict service curve in \dref{def-ssc} is called ``strong"
service curve in \cite{CO96}.

\section{Exercises}
%
% put below the list of exercises you want to see in the document
% you must have imported the database in the preamble, with commands
% such as \problemfile{L20-1}
\begin{problem}
Compute the maximum buffer size $X$ for a system that is initially
empty, and where the input function is $R(t)=\int_0^t r(s)ds$, for
the following cases.
 \begin{enumerate}
        \item  if $r(t) = a$ (constant)
        \item  one on-off connection with peak rate 1 Mb/s, on period 1 sec,
        off period $\tau$ seconds, and trunk bit rate $c=0.5$ Mb/s.
        \item  if $r(t) = c + c \sin \omega t$, with trunk bit rate $c >0$.
 \end{enumerate}
\end{problem}
\end{filecontents}
%%%%%%%%%%%%%%%%%%%%%%%%%%%%%%%%%%%%%%%%%%%%%%%%%%%%%%%%%%%%%%%%%%%%%%%%%%%%%%%%%%%%%%%%%%%%%%%%%%%
\problemfile{L20-2}
\begin{problem}
You have a fixed buffer of size $X$, that receives a data input
$r(t)$. Determine the output rate $c$ that is required to avoid
buffer overflow given that the buffer is initially empty.
\end{problem}
\end{filecontents}
%%%%%%%%%%%%%%%%%%%%%%%%%%%%%%%%%%%%%%%%%%%%%%%%%%%%%%%%%%%%%%%%%%%%%%%%%%%%%%%%%%%%%%%%%%%%%%%%%%%
\problemfile{L20-3}
\begin{problem}
\begin{enumerate}
        \item  For a flow with constant bit rate $c$, give some
        possible arrival curves.
        \item  Consider a flow with an arrival curve given by: $\alpha(t)=B$,
        where $B$ is constant. What does this mean for the flow ?
\end{enumerate}
\end{problem}
\end{filecontents}
%%%%%%%%%%%%%%%%%%%%%%%%%%%%%%%%%%%%%%%%%%%%%%%%%%%%%%%%%%%%%%%%%%%%%%%%%%%%%%%%%%%%%%%%%%%%%%%%%%%%
%\problemfile{ex-flui-pb1}
%\begin{problem}
%\begin{enumerate}
%        \item  Draw the arrival curve for a linearly constrained flow with
%parameters $(m,b)$ that also has a
%limit on the peak rate. What is the arrival curve ?
%
%        \item  Show that a flow is $(P,B)$ constrained iff $W_{P}(t) \leq B$
%        for all $t$.
%\end{enumerate}
%\end{problem}
%\end{filecontents}
%%%%%%%%%%%%%%%%%%%%%%%%%%%%%%%%%%%%%%%%%%%%%%%%%%%%%%%%%%%%%%%%%%%%%%%%%%%%%%%%%%%%%%%%%%%%%%%%%%%
\problemfile{L20-4}
\begin{problem}
\begin{enumerate}
We say that a flow is $(P, B)$ constrained if it has
$\gamma_{P,B}$ as an arrival curve.
        \item  A trunk system has a buffer size of $B$ and a trunk bitrate of $P$.
Fill in the dots: (1) there is no loss if the input is $(., .)$
constrained (2) the output is $(.,.)$ constrained.

        \item  A $(P, B)$ constrained flow is fed into an infinite buffer served at a
        rate of $c$. What is the maximum delay~?
\end{enumerate}

\end{problem}
\end{filecontents}
%%%%%%%%%%%%%%%%%%%%%%%%%%%%%%%%%%%%%%%%%%%%%%%%%%%%%%%%%%%%%%%%%%%%%%%%%%%%%%%%%%%%%%%%%%%%%%%%%%%
\problemfile{L20-5}
\begin{problem}
[On-Off flows]
\begin{enumerate}
\item Assume a data flow is periodical, with period $T$, and satisfies the
following: $r(t) = p$ for $0 \leq t < T_{0}$, and
$r(t) = 0$ for $T_{0}\leq t <  T$.
\begin{enumerate}
\item Draw $R(t) = \int_{0}^{t} r(s) ds$

\item Find an arrival curve for the flow. Find the minimum arrival
curve for the flow.
\item Find the minimum $(r, b)$ such that
the flow is $(r, b)$ constrained.

\sol{An arrival curve could be $\alpha(t)=pt$\\
The minimal arrival curve is: \[ \alpha (t) = \left\{ \begin{array}{ll}
                R(t) & \mbox{if R(t) is sub-additive}\\
                inf\{\delta (t),\alpha (t),\alpha (t)\mpc\alpha (t),..\} & \mbox{otherwise}
                \end{array}
                \right. \]
When the flow is (r,b) contrained, the minimum (r,b) values are $r = \frac{p
T_{0}}{T}$, $b=T_{0} (p-r)$}

\end{enumerate}

\item A traffic flow uses a link with bitrate $P$ (bits/s). Data is
sent as packets of variable length. The flow is controlled by a leaky
bucket $(r, b)$. What is the maximum packet size ? What is the minimum
time interval between packets of maximum size ?

Application: P = 2 Mb/s, r = 0.2 Mb/s; what is the required burst
tolerance $b$
if the packet length is 2 Kbytes ? What is then the minimum
spacing between packets ?

\sol{
maximum packet size $L = p \frac{b}{p-r}$ \\
minimum time between packets $T_{m}=\frac{b}{r}$ \\
application: $b=L \frac{p-r}{p} = 1.8$Kbytes, $T_{m}=72$ ms.
}
\end{enumerate}
\end{problem}
\end{filecontents}
%%%%%%%%%%%%%%%%%%%%%%%%%%%%%%%%%%%%%%%%%%%%%%%%%%%%%%%%%%%%%%%%%%%%%%%%%%%%%%%%%%%%%%%%%%%
\problemfile{L20-6}
\begin{problem}
\nfs{Alternative definition of GCRA} Consider the following
alternative definition of the GCRA:
\begin{definition}
The GCRA ($T,\tau$) is a controller that takes as input a cell
arrival time {\tt t} and returns {\tt result}. It has internal
(static) variables {\tt X} (bucket level) and {\tt LCT} (last
conformance time).

\begin{itemize}
        \item  initially, {\tt X = 0} and {\tt LCT = 0}
        \item  when a cell arrives at time {\tt t}, then
        \begin{verbatim}
                if (X - t + LCT > tau)
                   result = NON-CONFORMANT;
                else {
                     X = max (X - t + LCT, 0) + T;
                     LCT = t;
                     result = CONFORMANT;
                     }
        \end{verbatim}
\end{itemize}
\end{definition}
Show that the two definitions of GCRA are equivalent.\\
\sol{
Comparing the two algorithms we see that, if equivalence holds, we
should have \texttt{tat = X + LCT}.\\ The proof is now as follows.
We show by recurrence that for every input sequence, the output of
both algorithms is the same and that \texttt{tat = X + LCT} at the
end of the processing for the last input.  This is true for the
first input, at $t=0$. Now let us assume that this is true for all
inputs up to time $t$. The test done by both algorithms is the
same since \texttt{tat = X + LCT} so the output is the same. Then
\texttt{max(t, tat) + T = t + max(X- t + LCT, 0) + T} thus the
relation \texttt{tat = X + LCT} still holds after the processing
of this input. }
\end{problem}
\end{filecontents}
%%%%%%%%%%%%%%%%%%%%%%%%%%%%%%%%%%%%%%%%%%%%%%%%%%%%%%%%%%%%%%%%%%%%%%%%%%%%%%55
\problemfile{L20-7}
\begin{problem}
\begin{enumerate}
\item For the following flows and a GCRA(10, 2), give the conformant and
non-conformant cells. Times are in cell slots at the link rate. Draw
the leaky bucket behaviour assuming instantaneous cell arrivals.
 \begin{enumerate}
        \item  0, 10, 18, 28, 38

        \item  0, 10, 15, 25, 35

        \item  0, 10, 18, 26, 36

        \item  0, 10, 11, 18, 28
\end{enumerate}
\ifsol
\begin{enumerate}
        \item  \texttt{+ + + + +}
        \item  \texttt{+ + - + +}
        \item  \texttt{+ + + - +}
        \item  \texttt{+ + - + +}
\end{enumerate}
\fi
\item What is the maximum number of cells that can flow back to back
with GCRA(T, CDVT) (maximum ``clump'' size) ?
\sol{maximum clump size = floor$(\frac{CDVT}{T-1} + 1)$}
\end{enumerate}
\end{problem}
\end{filecontents}
%%%%%%%%%%%%%%%%%%%%%%%%%%%%%%%%%%%%%%%%%%%%%%%%%%%%%%%%%%%%%%%%%%%%%%%%%%%%%%
\problemfile{L20-8}
\begin{problem}
\begin{enumerate}
        \item For the following flows and a GCRA(100, 500), give the conformant and
  non-conformant cells. Times are in cell slots at the link rate.
  \begin{enumerate}
        \item  0, 100, 110, 12, 130, 140, 150, 160, 170, 180, 1000, 1010
        \item  0, 100, 130, 160, 190, 220, 250, 280, 310, 1000, 1030
        \item  0, 10, 20, 300, 310, 320, 600, 610, 620, 800, 810, 820, 1000,
        1010, 1020, 1200, 1210, 1220, 1400, 1410, 1420, 1600, 1610, 1620
    \end{enumerate}

   \ifsol
        \begin{enumerate}
        \item  \texttt{+ + + + + + + - - - + +}
        \item  \texttt{+ + + + + + + + + + + +}
        \item  \texttt{20(+) - + + -}
        \end{enumerate}
\fi

        \item Assume that a cell flow has a minimum spacing of $\gamma$
        time units between cell emission times ($\gamma$ is the minimum
        time between the beginnings of two cell transmissions).  What is
        the maximum burst size for GCRA($T, \tau)$ ?  What is the minimum
        time between bursts of maximum size ?
\sol{
Considering $\gamma$ as the cell separation between cell beginnings,\\
Maximum burst size(MBS) = floor$\lfloor \frac{\tau}{T - \gamma}+1 \rfloor$\\
Minimum time between bursts = MBS$(T-\gamma)+\gamma$
}

        \item Assume that a cell flow has a minimum spacing between cells of
        $\gamma$ time units, and a minimum spacing between bursts of
        $T_{I}$. What is the maximum burst size ?
        \sol{Maximum burst size = floor$ \lfloor \frac{\min(T_{i}-T,\tau)}{T
-\gamma}+1 \rfloor $}

\end{enumerate}
\end{problem}
\end{filecontents}
%%%%%%%%%%%%%%%%%%%%%%%%%%%%%%%%%%%%%%%%%%%%%%%%%%%%%%%%%%%%%%%%%%%%%%%%%%%%%%
\problemfile{L20-9}
\begin{problem}
For a CBR connection, here are some values from an ATM operator:
        \begin{verbatim}
 peak cell rate (cells/s)     100   1000  10000 100000
 CDVT     (microseconds)     2900   1200    400    135
        \end{verbatim}
        \begin{enumerate}
                \item What are the $(P, B)$ parameters in b/s and bits for
                each case ?  How does $T$ compare to $\tau$ ?
\ifsol

\begin{verbatim}
P          (b/s)  42400  424000   4240000   42400000
B            (b)    547     933      2120       6148
T    (microsecs)  10000    1000       100         10
CDVT (microsecs)   2900    1200       400        135
\end{verbatim}
\fi
                \item If a connection requires a peak cell rate of 1000 cells
                per second and a cell delay variation of 1400 microseconds,
                what can be done ?
\sol{
We need to find an arrival curve in the family given by Swiss Telecom
that can bound from above your arrival curve. In the  absence of information
about the link rate, this  means choosing a peak rate $P'$ larger  than 1000,
such that the  tolerance in cells, $B'_{p}= \delta + CDVT' \times P'$ is as
large as the requested tolerance  $B_{p}= \delta + CDVT \times P$. In other
words, call $f$ the function mapping the peak rate to the tolerance, we need
to find the minimum $P'$ such that $P' f(P') \geq Pf(P)$. A plot shows that
$\log f$ is linear in $\log P$. This gives $P' =$ 1738 cells/sec.
\begin{figure}[ht]
\begin{center}
%\includegraphics[angle=-0,scale=.4]{cdvtswisscom.eps}
\insfig{cdvtswisscom}{0.4} \caption{relationship between tolerance
and peak rate} \label{first}
\end{center}
\end{figure}
}
                \item Assume the operator allocates the peak rate to every
                connection at one buffer.  What is the amount of buffer
                required to assure absence of loss ?  Numerical Application
                for each of the following cases, where a number $N$ of
                identical connections with peak cell rate $P$ is multiplexed.
                \begin{verbatim}
 case                       1      2      3      4
 nb of connnections      3000    300     30      3
 peak cell rate (c/s)     100   1000  10000 100000
            \end{verbatim}
\ifsol
From proposition 1.3.2, it is sufficient to allocate a buffer equal to
$B_{p}$. This gives:
\begin{verbatim}
case                           1      2      3      4
number of connnections      3000    300     30      3
peak cell rate (cells/s)     100   1000  10000 100000
required buffer (cells)     3870    660    150     44
\end{verbatim}
It is possible to reduce the buffer size if we can exploit some constraints on
the peak rate.
\fi
        \end{enumerate}

\end{problem}
\end{filecontents}
%%%%%%%%%%%%%%%%%%%%%%%%%%%%%%%%%%%%%%%%%%%%%%%%%%%%%%%%%%%%%%%%%%%%%%%%%%%%%%%
%\problemfile{L20ex-fluid-gal0}
%\begin{problem}
%Define in at most four lines each of the concepts given in the
%Introduction
%\end{problem}
%\end{filecontents}
%
%\problemfile{L20ex-fluid-delay}
%\begin{problem}
%\begin{enumerate}
%        \item Is is true that the virtual delay defined in Section
%        \ref{d4-fluid-def} satisfies the following equation~?
%        (Justify your answer)
%        $$
%        d(t) = \max \left\{
%               T: T\geq 0 \mand R(t) \geq R^*(t + T)
%              \right\}
%        $$
%
%        \item  Show that if $R^{*}$ is continuous, then
%        $R^{*}\left(t +d(t)\right) = R(t)$. How can you interprete the
%        assumption on continuity ?
%\end{enumerate}
%\end{problem}
%\end{filecontents}
%
%\problemfile{L20ex-buflb2}
%\begin{problem}
%\begin{enumerate}
%        \item  Consider again Proposition \ref{prop-lb1} but assume that the trunk
%system is not first-in first-out, but only work conserving. What is
%now the bound on the delay ?
%        \item  Same question if we assume in addition that the trunk system
%        is first-in first-out {\em per flow}.
%\end{enumerate}
%\end{problem}
%\end{filecontents}
%%%%%%%%%%%%%%%%%%%%%%%%%%%%%%%%%%%%%%%%%%%%%%%%%%%%%%%%%%%%%%%%%%%%%%%%%%%%
%% Sept 98
%% PROBLEM FILES: Module D21 = LB and GCRA
%% supp problems to be merged with the new file
%%%%%%%%%%%%%%%%%%%%%%%%%%%%%%%%%%%%%%%%%%%%%%%%%%%%%%%%%%%%%%%%%%%%%%%%%%%%
%%%%%%%%%%%%%% EXAM ED 97 %%%%%%%%%%%%%%%%%%%%%%%%%%%%%%%%%%%%%%%%%%%%%%%%%%%
\problemfile{L20-10}
\begin{problem}
\nfs{September 1998}

The two questions in this problem are independent.
\begin{enumerate}
        \item  An ATM source is constrained by GCRA($T=30$ slots, $\tau=60$ slots),
where time is counted in slots. One slot is the time it takes to
transmit one cell on the link. The source sends cells according to
the following algorithm.
\begin{itemize}
        \item In a first phase, cells are sent at times $t(1)=0$,
        $t(2)=15$, $t(3)=30, \ldots,t(n)= 15(n-1)$ as long as all cells
        are conformant.  In other words, the number $n$ is the largest
        integer such that all cells sent at times $t(i)=15(i-1)$, $i\leq
        n$ are conformant. The sending of  cell $n$ at time $t(n)$ ends
        the first phase.

        \item  Then the source enters the second phase. The subsequent cell
        $n+1$ is sent at the earliest time after
        $t(n)$ at which a conformant cell can be sent, and the same is
        repeated for ever. In other words, call $t(k)$ the sending time for
        cell $k$, with $k>n$; we have then: $t(k)$ is the earliest time after
        $t(k-1)$ at which a conformant cell can be sent.
\end{itemize}

How many cells were sent by the source in time interval $[0, 151]$~?


        \item A network node can be modeled as a single buffer with a
        constant output rate $c$ (in cells per second).  It receives $I$
        ATM connections labeled $1, \ldots, I$.  Each ATM connection has a
        peak cell rate $p_{i}$ (in cells per second) and a cell delay
        variation tolerance $\tau_{i}$ (in seconds) for $1 \leq i \leq I$.
        The total input rate into the buffer is at least as large as
        $\sum_{i=1}^I p_{i}$ (which is equivalent to saying that it is
        unlimited). What is the buffer size (in cells) required for a
        loss-free operation~?

\end{enumerate}
\end{problem}
\end{filecontents}
%%%%%% ED Exam 98%%%%%%%%%%%%%%%%%%%%%%%%%%%%%%%%%%%%%%%%%%%%%%%%%%%%%%
\problemfile{L20-11}
\begin{problem}
In this problem, time is counted in slots. One slot is the duration to transmit
one ATM cell on the link.

\begin{enumerate}
        \item An ATM source $S_{1}$ is constrained by GCRA($T=50$ slots,
        $\tau=500$ slots), The source sends cells according to the
        following algorithm.
\begin{itemize}
        \item In a first phase, cells are sent at times $t(1)=0$,
        $t(2)=10$, $t(3)=20, \ldots,t(n)= 10(n-1)$ as long as all cells
        are conformant.  In other words, the number $n$ is the largest
        integer such that all cells sent at times $t(i)=10(i-1)$, $i\leq
        n$ are conformant. The sending of  cell $n$ at time $t(n)$ ends
        the first phase.

        \item  Then the source enters the second phase. The subsequent cell
        $n+1$ is sent at the earliest time after
        $t(n)$ at which a conformant cell can be sent, and the same is
        repeated for ever. In other words, call $t(k)$ the sending time for
        cell $k$, with $k>n$; we have then: $t(k)$ is the earliest time after
        $t(k-1)$ at which a conformant cell can be sent.
\end{itemize}

How many cells were sent by the source in time interval $[0, 401]$~?


        \item An ATM source $S_{2}$ is constrained by \emph{both}
        GCRA($T=10$ slots, $\tau=2$ slots) and GCRA($T=50$ slots,
        $\tau=500$ slots).  The source starts at time $0$, and has an
        infinite supply of cells to send.  The source sends its cells as
        soon as it is permitted by the combination of the GCRAs.  We call
        $t(n)$ the time at which the source sends the $n$th cell, with
        $t(1)=0$. What is the value of $t(15)$~?

\end{enumerate}
\end{problem}
\end{filecontents}
%%%%%%%%%%%%%%%%%%%%%%%%%%%%%%%%%%%%%%%%%%%%%%%%%%%%%%%%%%%%%%%%%%%%%%%%%%%%
\problemfile{L20-11a}
 \begin{problem}
 Consider a flow $R(t)$ receiving a minimum service curve
 guarantee $\beta$. Assume that
\begin{itemize}
  \item $\beta$ is concave and wide-sense increasing
  \item the $\inf$ in $R \mpc \beta$ is a $\min$
 \end{itemize}
For all $t$, call $\tau(t)$ a number such that
$$
(R \mpc \beta)(t)= R(\tau(t)) + \beta(t-\tau(t))
$$
Show that it is possible to choose $\tau$ such that if $t_1 \leq
t_2$ then  $\tau(t_1) \leq \tau(t_2)$.
 \sol{~\\Call $\tau_1=\tau(t_1)$ and
consider any $t' \leq \tau_1$. From the definition of $\tau_1$, we
have
$$
S(t') + \beta(t_1 - t' ) \geq S(\tau_1) + \beta(t_1 - \tau_1 )
$$
and thus
$$
S(t') + \beta(t_2 - t' ) \geq S(\tau_1) + \beta(t_1 - \tau_1 ) -
\beta(t_1 - t' ) + \beta(t_2 - t' )
$$
Now $\beta$ is convex, thus for any four numbers $a, b, c, d$ such
that $a \leq c  \leq b$,  $a \leq d  \leq b$ and $a+b=c+d$, we
have
$$\beta(a) + \beta(b) \geq \beta(c) + \beta(d)$$
(the interested reader will be convinced by drawing a small
figure). Applying this to $a=t_1- \tau_1, b=t_2 - t', c=t_1-t',
d=t_2-\tau_1$ gives
$$
S(t')+ \beta(t_2-t') \geq S(\tau_1)+\beta(t_2 - \tau_1)
$$
and the above equation holds for all $t'\leq \tau_1$. Consider now
the minimum, for a fixed $t_2$, of $S(t') +\beta(t_2-t')$ over all
$t' \leq t_2$. The above equation shows that the minimum is
reached for some $t' \geq \tau_1$.}
 \end{problem}
 \end{filecontents}
%%%%%%%%%%%%%%%%%%%%%%%%%%%%%%%%%%%%%%%%%%%%%%%%%%%%%%%%%%%%%%%%%%%%%%%%%%%%
\problemfile{L20-12}
\begin{problem}
        \begin{enumerate}
                        \item Find the maximum backlog and maximum delay for an ATM
                CBR connection with peak rate $P$ and cell delay variation
                $\tau$, assuming the service curve is $c(t) = r(t-T_{0})^{+}$
\sol{The CBR connection is defined by
        GCRA($\frac{\delta}{P}$, $\tau$)\\
        if p<r then,
        BackLog = \(B+PT_0\)\\
        Maximum Delay = \( \frac{B}{r}+T_0 \)\\
        otherwise,  backlog and max, delay are infinity.}

                \item Find the maximum backlog and maximum delay for an ATM
                VBR connection with peak rate $P$, cell delay variation
                $\tau$, sustainable cell rate $M$ and burst tolerance
$\tau_{B}$ (in
                seconds), assuming the service curve is $c(t) = r(t-T_{0})^{+}$
\sol{ }
\end{enumerate}

\end{problem}
\end{filecontents}
%%%%%%%%%%%%%%%%%%%%%%%%%%%%%%%%%%%%%%%%%%%%%%%%%%%%%%%%%%%%%%%%%%%%%%%%%%%%
\problemfile{L20-13}
\begin{problem}
Show the following statements:
\begin{enumerate}
 \item  Consider a $(P,B)$ constrained flow, served at a rate $c\geq P$.
       The output is also $(P,B)$ constrained.
\sol{ We wish to compute the output $\alpha \mpd \beta$ where
$\alpha(t)=Pt+B$ and $\beta(t)=ct$.
$$
\alpha \mpd \beta(t) = Pt+B + \sup_{u\geq0} (P-c)u
$$
Since $P\leq c$, this is largest when $u=0$. Therefore $\alpha
\mpd \beta(t)=\alpha(t)$. }
 \item  Assume $a()$ has a bounded right-handside derivative.
        Then the output for a flow constrained by $a()$, served in a buffer
at a
        constant rate $c \geq \sup_{t \geq 0} a'(t)$, is also constrained by
        $a()$.
\sol{
$$
a \mpd \beta(t) = \sup_{u\geq0} a(t+u) -cu
$$
By the mean-value theorem there is a $v\in[t,t+u]$ such that
$$a(t+u) = a(t) + u a'(v).$$
Thus,
$$
a \mpd \beta(t) = a(t) + \sup_{u\geq0} (a'(v)-c)u
$$
Since all derivatives are smaller than $c$, this is at its
largest when $u=0$, giving $a \mpd \beta(t) = a(t)$. }
\end{enumerate}
\end{problem}
\end{filecontents}
%%%%%%%%%%%%%%%%%%%%%%%%%%%%%%%%%%%%%%%%%%%%%%%%%%%%%%%%%%%%%%%%%%%%%%%%%%%%
\problemfile{L20-14}
\begin{problem}
\begin{enumerate}
                \item Find the the arrival curve constraining the output for
                an ATM CBR connection with peak rate $P$ and cell delay
                variation $\tau$, assuming the service curve is $c(t) =
                r(t-T_{0})^{+}$

                \item Find the arrival curve constraining the output
                for an ATM VBR connection with peak rate $P$, cell delay
                variation $\tau$, sustainable cell rate $M$ and burst
                tolerance $\tau_{B}$ (in seconds), assuming the service curve
                is $c(t) = r(t-T_{0})^{+}$
\end{enumerate}

\end{problem}
\end{filecontents}
%%%%%%%%%%%%%%%%%%%%%%%%%%%%%%%%%%%%%%%%%%%%%%%%%%%%%%%%%%%%%%%%%%%%%%%%%%%%
\problemfile{L20-15}
\begin{problem}
Consider the figure ``Derivation of arrival curve for the output
of a flow
  served in a node with rate-latency service curve $\beta_{R,T}$".
What can be said if $t_{0}$ in the Figure is infinite, namely, if
$a'(t) > r$ for all $t$ ?
\end{problem}
\end{filecontents}
%%%%%%%%%%%%%%%%%%%%%%%%%%%%%%%%%%%%%%%%%%%%%%%%%%%%%%%%%%%%%%%%%%%%%%%%%%%%
\problemfile{L20-16}
\begin{problem}

Consider a series of guaranteed service nodes with service curves
$c_{i}(t) = r_{i}(t-T_{i})^{+}$. What is the maximum delay through
this system for a flow constrained by $(m,b)$ ? \sol{Defining
\(T_0 = \sum_{\forall i} T_i \) and \(R_0 = min_{\forall i}
\{R_i\}\)
                \[ Max. Delay = \left\{ \begin{array}{ll}
                \frac{b}{R_0}+T_0 & \mbox{if $R_0>m$}\\
                \infty & \mbox{otherwise}
                \end{array}
                \right. \]
}
\end{problem}
\end{filecontents}
%%%%%%%%%%%%%%%%%%%%%%%%%%%%%%%%%%%%%%%%%%%%%%%%%%%%%%%%%%%%%%%%%%%%%%%%%%%%
\problemfile{L20-17}
\begin{problem}

A flow with T-SPEC $(p,M,r,b)$ traverses nodes 1 and 2.  Node $i$
offers a service curve $c_{i}(t) = R_{i}(t-T_{i})^{+}$. What
buffer size is required for the flow at node 2~? \sol{ Do this
proof again using max instead.\\
 Define $t_i=\frac{B-M}{p-r}$
\\
The buffer required for the flow at node 2 is:\\ if $t_i<T_1$
BackLog = \(b+r(T_1+T_2)\)\\ if $t_i>T_1$ and $p>R_1>r$\\
\hspace{5mm}if $T_2>t_i-T_1$ BackLog = \(b+rT_2+rT_1\)\\
\hspace{5mm}if $T_2<t_i-T_1$ and $R_2>R_1$ BackLog =
\(b+rt_i+(T_1-t_i)R_1+R_1T_2\)\\ \hspace{5mm}if $T_2<t_i-T_1$ and
$R_2<R_1$ BackLog = \(b+rt_i-R_2(t_i-T_1-T_2)\)\\ if $t_i>T_1$ and
$p<R_1$\\ \hspace{5mm}if $T_2>t_i-T_1$ BackLog =
\(M+pt_i+r(T_2-t_i+T_1)\)\\ \hspace{5mm}if $T_2<t_i-T_1$ and
$R_2>p$ BackLog = \(M+pT_1+pT_2\)\\ \hspace{5mm}if $T_2<t_i-T_1$
and $R_2<p$ BackLog = \(M+pt_i-R_2(t_i-T_1-T_2)\)\\ else BackLog =
$\infty$ }
\end{problem}
\end{filecontents}
%%%%%%%%%%%%%%%%%%%%%%%%%%%%%%%%%%%%%%%%%%%%%%%%%%%%%%%%%%%%%%%%%%%%%%%%%%%%
\problemfile{L20-18}
\begin{problem}
A flow with T-SPEC $(p,M,r,b)$ traverses nodes 1 and 2.  Node $i$
offers a service curve $c_{i}(t) = R_{i}(t-T_{i})^{+}$. A shaper
is placed between nodes 1 and 2. The shaper forces the flow to the
arrival curve $z(t)=\min(R_{2}t, bt+m)$.
\begin{enumerate}

\item
What buffer size is required for the flow at the shaper~? \sol{
Let $\tau=\frac{b-M}{p-r}$. If $T_1 > \tau$ then the buffer size
is $B=rT_1+b$. Otherwise,
$$
B = \bracket{ M+(P-R_1)*\tau+R_1T_1 \gap,R_1\leq P, R_1\leq R_2 \\
M+(P-R_2)*\tau+R_2T_1 \gap,R_1\leq P, R_1>R_2 \\ M+PT_1 \gap\gap
,R_1 > P, R_2\geq P \\ M+\tau*(P-R_2)+R_2T_1 \gap ,R_1 > P, R_2<
P }
$$
}
\item
What buffer size is required at node 2~? What value do you find if
$T_{1}=T_{2}$~?
\sol{\\
Let $\tau_{2}=\frac{m}{R_{2}-a}$. The buffer size is:
$$
B_{2}=\bracket{ aT_{2} + m \gap, \tau_{2}<T_{2} \\ R_{2}T_{2} \gap, otherwise}.
$$
$B_{2}$ does not depend on $T_{1}$.
}
\item Compare the sum of the preceding buffer sizes to the size
that would be required if no re-shaping is performed.
\item Give an arrival curve for the output of node 2.
\sol{ Choose any arrival curve that satifies the
conditions}
\end{enumerate}

\end{problem}
\end{filecontents}
%%%%%%%%%%%%%%%%%%%%%%%%%%%%%%%%%%%%%%%%%%%%%%%%%%%%%%%%%%%%%%%%%%%%%%%%%%%%
\problemfile{L20-19}
\begin{problem}
        Prove the formula giving of paragraph ``Buffer Sizing at a
        Re-shaper"
\sol{Follows}
\end{problem}
\end{filecontents}
%%%%%%%%%%%%%%%%%%%%%%%%%%%%%%%%%%%%%%%%%%%%%%%%%%%%%%%%%%%%%%%%%%%%%%%%%%%%
\problemfile{L20-20}
\begin{problem}
        Is Theorem ``Input-Output Characterization of Greedy Shapers" a stronger result
        than Corollary ``Service Curve offered by a Greedy Shaper"~?
\sol{ No, the theorem is equivalent to the corollary. Indeed, if
corollary is true, then
$$
R^*\geq R \mpc \sigma
$$
Now by the arrival curve property
$$
R* \leq R^* \mpc \sigma \leq R \otimes \sigma
$$
since $R^*\leq R$. Thus $R*=R\otimes \sigma.$ }
\end{problem}
\end{filecontents}
%%%%%%%%%%%%%%%%%%%%%%%%%%%%%%%%%%%%%%%%%%%%%%%%%%%%%%%%%%%%%%%%%%%%%%%%%%%%%
%\problemfile{L20ex-fluid-gal}
%\begin{problem}
%
%Define in at most ten lines each of the concepts given in the
%Introduction
%\end{problem}
%\end{filecontents}
%%%%%%%%%%%%%%%%%%%%%%%%%%%%%%%%%%%%%%%%%%%%%%%%%%%%%%%%%%%%%%%%%%%%%%%%%%%%
\problemfile{L20-21}
\begin{problem}

\begin{enumerate}

    \item Explain what is meant by ``we pay bursts only once''.

        \item Give a summary in at most 15 lines of the main properties of
        shapers

        \item  Define the following concepts by using the $\otimes $ operator:
        Service Curve, Arrival Curve, Shaper

        \item  What is a greedy source~?
\end{enumerate}
\end{problem}
\end{filecontents}
%%%%%%%%%%%%%%%%%%%%%%%%%%%%%%%%%%%%%%%%%%%%%%%%%%%%%%%%%%%%%%%%%%%%%%%%%%%%
\problemfile{L20-22}
\begin{problem}

\begin{enumerate}
        \item  Show that for a constant bit rate trunk with rate $c$, the backlog at
time $t$ is given by
$$
W(t) = \sup_{s \leq t} \left \{ R(t) - R^{*}(s) - c(t-s) \right \}
$$

        \item  What does the formula become if we assume only that,
        instead a constant bit rate trunk, the node is a scheduler
offering $\beta$ as a service curve~?
\end{enumerate}

\end{problem}
\end{filecontents}
%%%%%%%%%%%%%%%%%%%%%%%%%%%%%%%%%%%%%%%%%%%%%%%%%%%%%%%%%%%%%%%%%%%%%%%%%%%%
\problemfile{L20-23}
\begin{problem}
Is it true that offering a service curve $\beta$ implies that,
during any busy period of length $t$, the amount of service
received rate is at least $\beta(t)$ ? \sol{No, the property means
that $\beta$ is a strict service curve. In general, a service
curve property is not strict. Consider for example a shaper. }
\end{problem}
\end{filecontents}
%%%%%%%%%%%%%%%%%%%%%%%%%%%%%%%%%%%%%%%%%%%%%%%%%%%%%%%%%%%%%%%%%%%%%%%%%%%%
%% Sept 98
%% PROBLEM FILES: Module D22 = network calculus
%% supp problems to be merged with the new filw
%%%%%%%%%%%%%%%%%%%%%%%%%%%%%%%%%%%%%%%%%%%%%%%%%%%%%%%%%%%%%%%%%%%%%%%%%%%%
%%%%%%%%%%%%%%%%%%%%%%%%%%%%%%%%%%%%%%%%%%%%%%%%%%%%%%%%%%%%%%%%%%%%%%%%%%%%
\problemfile{L20-24}
\begin{problem}
A flow $S(t)$ is constrained by an arrival curve $\alpha$. The
flow is fed into a shaper, with shaping curve $\sigma$. We assume
that
$$
\alpha(s) = \min (m + ps, b +rs)
$$
and
$$
\sigma (s) = \min ( Ps, B+Rs )
$$
We assume that $p>r$, $m \leq b$ and $P \geq R$.

The shaper has a fixed buffer size equal to $X \geq m$. We require
that the buffer never overflows.

\begin{enumerate}
        \item Assume that $B=+\infty$.  Find the smallest
        of $P$ which guarantees that there is no buffer overflow.
        Let $P_{0}$ be this value.

        \item We do not assume that $B=+ \infty$ any more, but we
        assume that $P$ is set to the value $P_{0}$ computed in the
        previous question.  Find the value $(B_{0}, R_{0})$ of $(B,R)$  which
        guarantees that there is no buffer overflow and minimizes the cost
        function $c(B,R)= aB + R$, where $a$ is a positive constant.

        What is the
        maximum virtual delay if $(P, B, R)=(P_{0},B_{0}, R_{0})$~?
\end{enumerate}
\end{problem}
\end{filecontents}
%%%%%%%%%%%%%%  %%%%%%%%%%%%%%%%%%%
\problemfile{L20-25}
\begin{problem}
\nfs{GCRA and Eff Bw, easy; GCRA and Eff Bw exam ED 99, modified
feb 99 by leb\\}
We consider a buffer of size $X$ cells, served at
a constant
        rate of $c$ cells per second.  We put $N$ identical connections into
        the buffer; each of the $N$ connections is constrained both by
        GCRA($T_{1}$, $\tau_{1}$) and GCRA($T_{2}$, $\tau_{2}$).  What is
        the maximum value of $N$ which is possible if we want to guarantee
        that there is no cell loss at all~?

        Give the numerical application for $T_{1}=0.5$~ms, $\tau_{1}=4.5$~ms,
        $T_{2}=5$~ms, $\tau_{2}=495$~ms, $c=10^6$~cells/second, $X=10^4$~cells
        \ifsol

        Call $e$ the effective bandwidth for a delay of $\frac{X}{c}$. The
        answer is the largest integer $\leq$ $\frac{c}{e}$.

        NA: $D=  \frac{X}{c} =10$ms. $e=11000/6$ cells/second. $N=6000/11 = 545$
        \fi
\end{problem}%
\end{filecontents}
%%%%%%%%%%%%%%%%%%%%%exam ED 99%%%%%%%%%%%%%%%%%%%%%%%%%%%%%%
\problemfile{L20-26}
\begin{problem}
\nfs{non initial bucket level, hard\\}
 We consider a flow defined by
its function $R(t)$, with
                $R(t)=$ the number of bits observed since time $t=0$.
 \begin{enumerate}
  \item The
   flow is fed into a buffer, served at a rate $r$.  Call $q(t)$
   the buffer content at time $t$.  We do the same assumptions as
   in the lecture, namely, the buffer is large enough, and is
   initially empty. What is the expression of $q(t)$ assuming we
   know $R(t)$~?

   We assume now that, unlike what we saw in the lecture, the initial
   buffer content (at time $t=0$) is \emph{not} $0$, but some value $q_{0}
   \geq 0$. What is now the expression for $q(t)$~?
\sol{\\
The expression of $q(t)$ when the  buffer is initially empty is:
$$
q(t) = \sup_{s: s \leq t} \left \{ R(t) - R(s) - r(t-s) \right \}
$$
If the initial buffer content is some value $q_0$ then we have:
$$
q(t) = \max { ( \sup_{s: s \leq t} \left \{ R(t) - R(s) - r(t-s) \right \}, R(t)- rt+q_0  ) }
$$
}
 \item The flow is put into a leaky bucket policer, with rate
 $r$ and bucket size $b$.  This is a policer, not a shaper, so
 nonconformant bits are discarded.  We assume that the bucket
 is large enough, and is initially empty.  What is the
 condition on $R$ which ensures that no bit is discarded by the
 policer (in other words, that the flow is conformant)~?

 We assume now that, unlike what we saw in the lecture, the
 initial \emph{bucket} content (at time $t=0$) is \emph{not}
 $0$, but some value $b_{0} \geq 0$.  What is now the condition
 on $R$ which ensures that no bit is discarded by the policer
 (in other words, that the flow is conformant)~?
\sol{\\
The condition on $R$ which ensures that the flow is conformant when the bucket is initially empty is :
$$
\sup_{s: s \leq t} \left \{ R(t) - R(s) - r(t-s) \right \} \leq b
$$
If the initial bucket content is some value $b_0$ then the condition becomes:
$$
\max { ( \sup_{s: s \leq t} \left \{ R(t) - R(s) - r(t-s) \right \}, R(t)- rt+b_0  ) } \leq b
$$
}
\end{enumerate}
\end{problem}%
\end{filecontents}
%%%%%% de Thiele %%%%%%%%%%%%%%%%%%%%%%%%%%%%%%%%%%%%%%%%%%%%%%%%%%%%%%
\problemfile{L20-27}
\begin{problem}
Consider a variable capacity network node, with capacity curve
$M(t)$. Show that there is one maximum function $S^*(t)$ such that
for all $0 \leq s \leq t$, we have
$$M(t)-M(s) \geq S^*(t-s)
$$
Show that $S^*$ is super-additive.

Conversely, if a function $\beta$ is super-additive, show that
there is a variable capacity network node, with capacity curve
$M(t)$, such that for all $0 \leq s \leq t$, we have $M(t)-M(s)
\geq S^*(t-s) $.

Show that, with a notable exception, a shaper cannot be modeled as
a variable capacity node.
\end{problem}
\end{filecontents}
%%%%%% packetized shaper %%%%%%%%%%%%%%%%%%%%%%%%%%%%%%%%%%%%%%%%%%%%%%%%%%%%
\problemfile{L20-28}
\begin{problem}
\begin{enumerate}
  \item Consider a packetized greedy shaper with shaping curve
  $\sigma(t)=rt$ for $t \geq 0$. Assume that $L(k)= k M$ where $M$ is fixed.
  Assume that the input is given by $R(t)=10 M$ for $t>0$ and $R(0)=0$. Compute
  the sequence $R^{(i)}(t)$ used in the representation of the output of the
  packetized greedy shaper, for $i=1,2,3,...$.
  \sol{
  The output $R^{(i)}(t)$ is a train of packets $m=1,...,10$, all of size $M$ at
  times $T^{(i)}_m = (m+i-1) \frac{M}{r}$. As $i$ goes to infinity,
  $R^{(i)}(t)$ goes to $0$, for any fixed $t$.}
  \item Same question if $\sigma(t)=(rt+ 2M) 1_\{t>0\}$.
  \sol{
  $T^{(1)}_1 = T^{(1)}_2 = 0$ and otherwise $T^{(1)}_m = (m-2) \frac{M}{r}$.
  For $i \geq 1$ we have
  $R^{(i)}=R^{(1)}$ since \thref{theo-psps} applies.}
\end{enumerate}

\end{problem}
\end{filecontents}
%%%%%%%%%%%%%% equivalence GCRA1 et GCRA 2 %%%%%%%%%%%%%%%%%%%%%%%%%%%%%%%%%
\problemfile{L20-gcra2}
\begin{problem}
\nfs{April 1999} Consider the discrete time leaky bucket
algorithm, defined as follows. It operates like the GCRA, with the
following code.
\begin{itemize}
        \item  initially, {\tt X = 0} and {\tt LCT = 0}
        \item  when a cell arrives at time {\tt t}, then
        \begin{verbatim}
                if (X - t + LCT > tau)
                   result = NON-CONFORMANT;
                else {
                     X = max (X - t + LCT, 0) + T;
                     LCT = t;
                     result = CONFORMANT;
                     }
        \end{verbatim}
\end{itemize}
Show now that for any input, both the discrete time leaky bucket
and the GCRA produce the same output.

\ifsol {\em We show by induction on time that, at every point in
time, the following relation holds
\begin{equation}\label{eq-gcra-33}
  {\tt tat} = {\tt X} + {\tt LCT}
\end{equation}
and that they produce the same output.

This  is true for $t=0$. Now assume that this is also true up to
some time $s$ and consider the first cell arrival time $t > s$. It
follows from the induction hypothesis that both algorithms perform
the same test and thus produce the same output. In addition, the
new value of ${\tt tat}$ with the GCRA is
$${\tt tat'} = {\tt max (t, tat) + T}$$
while the discrete time leaky bucket produces
$${\tt X'= max (X - t + LCT, 0) + T }
\gap \mand \gap {\tt LCT' = t }$$ from where it follows
immediately that Equation(~\ref{eq-gcra-33}) continues to hold. }
\fi

Interpret the discrete time leaky bucket algorithm as a leaky
bucket with appropriate units for time and data, and use this to
show that the GCRA is equivalent to a leaky bucket. What is the
mapping between GCRA parameters and leaky bucket parameters~?
\ifsol {\em We map the discrete time flow to a continuous time
flow, using Equation~(~\ref{eq-map-inter}). This is equivalent to
assuming a continuous time model, with instantaneous cell
arrivals. We can interprete the discrete time leaky bucket as
follows. Upon a cell arrival, the level of fluid is updated,
accounting for a leak rate of 1; then, if the updated level is
less than {\tt tau}, the arriving cell pours into the bucket an
amount of {\tt T} units of fluid. This is equivalent to requiring
that the bucket never exceeds {\tt T + tau}. Remember that we
think of an arriving cell stream as a fluid with an impulse at a
cell arrival. The algorithm is thus equivalent to a leaky bucket
controller, with bucket depth {\tt T + tau}, leak rate equal to 1
unit of fluid per time unit, and with every cell pouring into the
bucket an instantaneous amount of {\tt T} units of fluid.

Now we change the unit of data in order to let the cell size be
equal to $\delta$ units of data. The leak rate is now equal to $r
= \frac{\delta}{\tt T}$, and the bucket size to $b =
\frac{(T+\tau)\delta}{T}$. } \fi
\end{problem}
\end{filecontents}
%%%%%%%%%%%%%%%%%%%%%%%%%%%%%%%%%%%%%%%%%%%%%%%%%%%%%%%%%%%%%%%%%%%%%%%%%%%%
\problemfile{L20-29}
\begin{problem}
\nfs{exam ed fev 2000; `` une classe au mus\'{e}e"} Consider a
source given by the function
$$
\bracket{R(t) = B \mf t > 0\\R(t)= 0 \mf t \leq 0 }
$$
Thus the flow consists of an instantaneous burst of $B$ bits.
\begin{enumerate}
\item What is the minimum arrival curve for the flow~?
  \item Assume that the flow is served in one node that offers a
  minimum service curve of the rate latency type, with rate $r$
  and latency $\Delta$. What is the maximum delay for the last bit
  of the flow~?
  \item We assume now that the flow goes through a series of two
  nodes, $\calN_1$ and $\calN_2$, where $\calN_i$ offers to the
  flow a
  minimum service curve of the rate latency type, with rate $r_i$
  and latency $\Delta_i$, for $i=1,2$. What is the the maximum delay for the last bit
  of the flow through the series of two nodes~?
  \item With the same assumption as in the previous item, call $R_1(t)$ the function describing the
  flow at the output of node $\calN_1$ (thus at the input of node $\calN_2$). What is
  the worst case minimum arrival curve for $R_1$~?
  \item We assume that we insert between $\calN_1$ and $\calN_2$ a
  ``reformatter" $\calS$.
  The input to $\calS$ is $R_1(t)$. We call $R'_1(t)$ the output of $\calS$.
  Thus $R'_1(t)$ is now the input to $\calN_2$.
   The function of the ``reformatter"$\calS$ is to delay the
  flow $R_1$ in order to output a flow $R'_1$ that is a delayed version of
  $R$. In other words, we must have $R'_1(t)=R(t-d)$ for some
  $d$. We assume that the reformatter $\calS$ is optimal in the
  sense that it chooses the smallest possible $d$. In the worst
  case, what is this optimal value of $d$~?
  \item With the same assumptions as in the previous item, what is
  the worst case end-to-end delay through the series of nodes
  $\calN_1, \calS, \calN_2$~? Is the reformatter transparent~?
\end{enumerate}
\end{problem}
\end{filecontents}
%%%%%%%%%%%%%%%%%%%%%%%%%%%%%%%%%%%%%%%%%%%%%%%%%%%%%%%%%%%%%%%%%%%%%%%%%%%%
 \problemfile{L20-31}
 \begin{problem}
 \nfs{another vl shaper}
 Let $\sigma$ be a good function. Consider the concatenation of a
 bit-by-bit greedy shaper, with curve $\sigma$, and an
 $L$-packetizer. Assume that $\sigma(0^+)=0$.
 Consider only inputs that are $L$-packetized
\begin{enumerate}
  \item Is this system a packetized shaper for $\sigma$ ?\sol{~No,
  the condition on $\sigma$ implies that a packetized shaper for
  $\sigma$ is a degenerate system, and this one is not.}
  \item Is it a packetized shaper for $\sigma+l_{\max}$ ?
  \sol{~This question is more reasonable. Indeed, since $P^L(x)
  \geq x -l_{\max}$, the answer is yes.}
  \item Is it a packetized greedy shaper for $\sigma+l_{\max}$ ?
  \sol{~This question is more difficult. In general, the answer is no.
  To see why, consider $\sigma(t)=rt$ and draw any sample path. With this
  system, the first packet is delayed by $\frac{l_1}{r}$ whereas the
  packetized greedy shaper for $rt + l_{\max}$ would not delay the
  first packet.}
\end{enumerate}
 \end{problem}
 \end{filecontents}
%%%%%%%%%%%%%%%%%%%%%%%%%%%%%%%%%%%%%%%%%%%%%%%%%%%%%%%%%%%%%%%%%%%%%%%%%%%%
\problemfile{L20-32}
\begin{problem}
Assume that $\sigma$ is a good function and $\sigma= \sigma_0 + l
u_0$ where $u_0$ is the step function with a step at $t=0$. Can we
conclude that $\sigma_0$ is sub-additive ? \sol{ No, see textbook
for a counter-example. }
\end{problem}
\end{filecontents}
%%%%%%%%%%%%%%%%%%%%%%%%%%%%%%%%%%%%%%%%%%%%%%%%%%%%%%%%%%%%%%%%%%%%%%%%%%%%
\problemfile{L20-33}
\begin{problem}
Is the operator $(P^L)$ upper-semi-continuous ? \sol{No, consider
a sequence $x_n=l_1 + \frac{l_2}{n}$. We have $P^L(x_n)=l_1$, thus
$\inf_n [P^L(x_n)]=l_1$. Now $\inf_n x_n = l_1$ thus $P^L[\inf_n
(x_n)]=0$.}
\end{problem}
\end{filecontents}
%%%%%%%%%%%%%%%%%%%%%%%%%%%%%%%%%%%%%%%%%%%%%%%%%%%%%%%%%%%%%%%%%%%%%%%%%%%%
 \problemfile{L20-34}
 \begin{problem}
\begin{enumerate}
  \item
 Consider the concatenation of an $L$-packetizer and a network
 element with minimum service curve $\beta$ and maximum service
 curve $\gamma$. Can we say that the combined system offer a
 minimum service curve $(\beta(t)-l_{\max})^+$ and a maximum service
 curve $\gamma$, as in the case where the concatenation would be
 in the reverse order ? \sol{ Yes, and the proof is similar}.
  \item Consider the concatenation of a GPS node offering a guarantee
  $\lambda_{r_1}$, an $L$-packetizer, and a second GPS node offering a guarantee
  $\lambda_{r_2}$. Show that the combined system offers a rate-latency
  service curve with rate $R=\min(r_1, r_2)$ and
  latency$E=\frac{l_{\max}}{\max(r_1, r_2)}$.
  \sol{If $r_1< r_2$, consider the combined system made of the packetizer and
  the second node. From the previous question, it offers a rate latency
  service curve with rate $r_2$ and latency
  $E=\frac{l_{\max}}{r_2}=\frac{l_{\max}}{\max(r_1, r_2)}$. Thus
  the global system offers a rate latency service curve with rate
  $R=\min(r_1, r_2)$ and latency $E$.\\
  Otherwise, consider first as combined system the concatenation
  of the first GPS node and the packetizer, and apply the textbook
  result with the same reasoning. Thus the result is true in all cases.}
\end{enumerate}
 \end{problem}
 \end{filecontents}
%%%%%%%%%%%%%%%%%%%%%%%%%%%%%%%%%%%%%%%%%%%%%%%%%%%%%%%%%%%%%%%%%%%%%%%%%%%%
 \problemfile{L20-35}
 \begin{problem}
 Consider a node that offers to a flow $R(t)$ a rate-latency
 service curve $\beta= S_{R,L}$. Assume that $R(t)$ is $L$-packetized, with
 packet arrival times called $T_1, T_2,...$  (and
 is left-continuous,
 as usual)

 Show that
 $(R\mpc \beta)(t) = \min_{T_i \in [0, t]} [R(T_i) + \beta(t-T_i)]$
  (and thus, the $\inf$ is attained).
\sol{ For a fixed $t$, study the function of $s$ given by $R(s) +
\beta(t-s)$. For $T_{i-1} < s \leq T_i$ we have $R(s)=R(T_i)$ thus
the $\inf$ for $s \in (T_{i-1}, T_i]$ is attained at $T_i$.}
 \end{problem}
 \end{filecontents}
%%%%%%%%%%%%%%%%%%%%%%%%%%%%%%%%%%%%%%%%%%%%%%%%%%%%%%%%%%%%%%%%%%%%%%%%%%%%
% \problemfile{}
% \begin{problem}
% \end{problem}
% \end{filecontents}

%
% there is also an automatic mechanism to print all exercises, ask the author
%
%% LaTeX2e file `./temp/L20-1'
%% generated by the `filecontents' environment
%% from source `netCalBookv4' on 2019/02/19.
%%
\begin{problem}
Compute the maximum buffer size $X$ for a system that is initially
empty, and where the input function is $R(t)=\int_0^t r(s)ds$, for
the following cases.
 \begin{enumerate}
        \item  if $r(t) = a$ (constant)
        \item  one on-off connection with peak rate 1 Mb/s, on period 1 sec,
        off period $\tau$ seconds, and trunk bit rate $c=0.5$ Mb/s.
        \item  if $r(t) = c + c \sin \omega t$, with trunk bit rate $c >0$.
 \end{enumerate}
\end{problem}

%% LaTeX2e file `./temp/L20-2'
%% generated by the `filecontents' environment
%% from source `netCalBookv4' on 2019/02/19.
%%
\begin{problem}
You have a fixed buffer of size $X$, that receives a data input
$r(t)$. Determine the output rate $c$ that is required to avoid
buffer overflow given that the buffer is initially empty.
\end{problem}

%% LaTeX2e file `./temp/L20-3'
%% generated by the `filecontents' environment
%% from source `netCalBookv4' on 2019/02/19.
%%
\begin{problem}
\begin{enumerate}
        \item  For a flow with constant bit rate $c$, give some
        possible arrival curves.
        \item  Consider a flow with an arrival curve given by: $\alpha(t)=B$,
        where $B$ is constant. What does this mean for the flow ?
\end{enumerate}
\end{problem}

%% LaTeX2e file `./temp/L20-4'
%% generated by the `filecontents' environment
%% from source `netCalBookv4' on 2019/02/19.
%%
\begin{problem}
\begin{enumerate}
We say that a flow is $(P, B)$ constrained if it has
$\gamma_{P,B}$ as an arrival curve.
        \item  A trunk system has a buffer size of $B$ and a trunk bitrate of $P$.
Fill in the dots: (1) there is no loss if the input is $(., .)$
constrained (2) the output is $(.,.)$ constrained.

        \item  A $(P, B)$ constrained flow is fed into an infinite buffer served at a
        rate of $c$. What is the maximum delay~?
\end{enumerate}

\end{problem}

%% LaTeX2e file `./temp/L20-5'
%% generated by the `filecontents' environment
%% from source `netCalBookv4' on 2019/02/19.
%%
\begin{problem}
[On-Off flows]
\begin{enumerate}
\item Assume a data flow is periodical, with period $T$, and satisfies the
following: $r(t) = p$ for $0 \leq t < T_{0}$, and
$r(t) = 0$ for $T_{0}\leq t <  T$.
\begin{enumerate}
\item Draw $R(t) = \int_{0}^{t} r(s) ds$

\item Find an arrival curve for the flow. Find the minimum arrival
curve for the flow.
\item Find the minimum $(r, b)$ such that
the flow is $(r, b)$ constrained.

\sol{An arrival curve could be $\alpha(t)=pt$\\
The minimal arrival curve is: \[ \alpha (t) = \left\{ \begin{array}{ll}
                R(t) & \mbox{if R(t) is sub-additive}\\
                inf\{\delta (t),\alpha (t),\alpha (t)\mpc\alpha (t),..\} & \mbox{otherwise}
                \end{array}
                \right. \]
When the flow is (r,b) contrained, the minimum (r,b) values are $r = \frac{p
T_{0}}{T}$, $b=T_{0} (p-r)$}

\end{enumerate}

\item A traffic flow uses a link with bitrate $P$ (bits/s). Data is
sent as packets of variable length. The flow is controlled by a leaky
bucket $(r, b)$. What is the maximum packet size ? What is the minimum
time interval between packets of maximum size ?

Application: P = 2 Mb/s, r = 0.2 Mb/s; what is the required burst
tolerance $b$
if the packet length is 2 Kbytes ? What is then the minimum
spacing between packets ?

\sol{
maximum packet size $L = p \frac{b}{p-r}$ \\
minimum time between packets $T_{m}=\frac{b}{r}$ \\
application: $b=L \frac{p-r}{p} = 1.8$Kbytes, $T_{m}=72$ ms.
}
\end{enumerate}
\end{problem}

%% LaTeX2e file `./temp/L20-6'
%% generated by the `filecontents' environment
%% from source `netCalBookv4' on 2019/02/19.
%%
\begin{problem}
\nfs{Alternative definition of GCRA} Consider the following
alternative definition of the GCRA:
\begin{definition}
The GCRA ($T,\tau$) is a controller that takes as input a cell
arrival time {\tt t} and returns {\tt result}. It has internal
(static) variables {\tt X} (bucket level) and {\tt LCT} (last
conformance time).

\begin{itemize}
        \item  initially, {\tt X = 0} and {\tt LCT = 0}
        \item  when a cell arrives at time {\tt t}, then
        \begin{verbatim}
                if (X - t + LCT > tau)
                   result = NON-CONFORMANT;
                else {
                     X = max (X - t + LCT, 0) + T;
                     LCT = t;
                     result = CONFORMANT;
                     }
        \end{verbatim}
\end{itemize}
\end{definition}
Show that the two definitions of GCRA are equivalent.\\
\sol{
Comparing the two algorithms we see that, if equivalence holds, we
should have \texttt{tat = X + LCT}.\\ The proof is now as follows.
We show by recurrence that for every input sequence, the output of
both algorithms is the same and that \texttt{tat = X + LCT} at the
end of the processing for the last input.  This is true for the
first input, at $t=0$. Now let us assume that this is true for all
inputs up to time $t$. The test done by both algorithms is the
same since \texttt{tat = X + LCT} so the output is the same. Then
\texttt{max(t, tat) + T = t + max(X- t + LCT, 0) + T} thus the
relation \texttt{tat = X + LCT} still holds after the processing
of this input. }
\end{problem}

%% LaTeX2e file `./temp/L20-7'
%% generated by the `filecontents' environment
%% from source `netCalBookv4' on 2019/02/19.
%%
\begin{problem}
\begin{enumerate}
\item For the following flows and a GCRA(10, 2), give the conformant and
non-conformant cells. Times are in cell slots at the link rate. Draw
the leaky bucket behaviour assuming instantaneous cell arrivals.
 \begin{enumerate}
        \item  0, 10, 18, 28, 38

        \item  0, 10, 15, 25, 35

        \item  0, 10, 18, 26, 36

        \item  0, 10, 11, 18, 28
\end{enumerate}
\ifsol
\begin{enumerate}
        \item  \texttt{+ + + + +}
        \item  \texttt{+ + - + +}
        \item  \texttt{+ + + - +}
        \item  \texttt{+ + - + +}
\end{enumerate}
\fi
\item What is the maximum number of cells that can flow back to back
with GCRA(T, CDVT) (maximum ``clump'' size) ?
\sol{maximum clump size = floor$(\frac{CDVT}{T-1} + 1)$}
\end{enumerate}
\end{problem}

%% LaTeX2e file `./temp/L20-8'
%% generated by the `filecontents' environment
%% from source `netCalBookv4' on 2019/02/19.
%%
\begin{problem}
\begin{enumerate}
        \item For the following flows and a GCRA(100, 500), give the conformant and
  non-conformant cells. Times are in cell slots at the link rate.
  \begin{enumerate}
        \item  0, 100, 110, 12, 130, 140, 150, 160, 170, 180, 1000, 1010
        \item  0, 100, 130, 160, 190, 220, 250, 280, 310, 1000, 1030
        \item  0, 10, 20, 300, 310, 320, 600, 610, 620, 800, 810, 820, 1000,
        1010, 1020, 1200, 1210, 1220, 1400, 1410, 1420, 1600, 1610, 1620
    \end{enumerate}

   \ifsol
        \begin{enumerate}
        \item  \texttt{+ + + + + + + - - - + +}
        \item  \texttt{+ + + + + + + + + + + +}
        \item  \texttt{20(+) - + + -}
        \end{enumerate}
\fi

        \item Assume that a cell flow has a minimum spacing of $\gamma$
        time units between cell emission times ($\gamma$ is the minimum
        time between the beginnings of two cell transmissions).  What is
        the maximum burst size for GCRA($T, \tau)$ ?  What is the minimum
        time between bursts of maximum size ?
\sol{
Considering $\gamma$ as the cell separation between cell beginnings,\\
Maximum burst size(MBS) = floor$\lfloor \frac{\tau}{T - \gamma}+1 \rfloor$\\
Minimum time between bursts = MBS$(T-\gamma)+\gamma$
}

        \item Assume that a cell flow has a minimum spacing between cells of
        $\gamma$ time units, and a minimum spacing between bursts of
        $T_{I}$. What is the maximum burst size ?
        \sol{Maximum burst size = floor$ \lfloor \frac{\min(T_{i}-T,\tau)}{T
-\gamma}+1 \rfloor $}

\end{enumerate}
\end{problem}

%% LaTeX2e file `./temp/L20-9'
%% generated by the `filecontents' environment
%% from source `netCalBookv4' on 2019/02/19.
%%
\begin{problem}
For a CBR connection, here are some values from an ATM operator:
        \begin{verbatim}
 peak cell rate (cells/s)     100   1000  10000 100000
 CDVT     (microseconds)     2900   1200    400    135
        \end{verbatim}
        \begin{enumerate}
                \item What are the $(P, B)$ parameters in b/s and bits for
                each case ?  How does $T$ compare to $\tau$ ?
\ifsol

\begin{verbatim}
P          (b/s)  42400  424000   4240000   42400000
B            (b)    547     933      2120       6148
T    (microsecs)  10000    1000       100         10
CDVT (microsecs)   2900    1200       400        135
\end{verbatim}
\fi
                \item If a connection requires a peak cell rate of 1000 cells
                per second and a cell delay variation of 1400 microseconds,
                what can be done ?
\sol{
We need to find an arrival curve in the family given by Swiss Telecom
that can bound from above your arrival curve. In the  absence of information
about the link rate, this  means choosing a peak rate $P'$ larger  than 1000,
such that the  tolerance in cells, $B'_{p}= \delta + CDVT' \times P'$ is as
large as the requested tolerance  $B_{p}= \delta + CDVT \times P$. In other
words, call $f$ the function mapping the peak rate to the tolerance, we need
to find the minimum $P'$ such that $P' f(P') \geq Pf(P)$. A plot shows that
$\log f$ is linear in $\log P$. This gives $P' =$ 1738 cells/sec.
\begin{figure}[ht]
\begin{center}
%\includegraphics[angle=-0,scale=.4]{cdvtswisscom.eps}
\insfig{cdvtswisscom}{0.4} \caption{relationship between tolerance
and peak rate} \label{first}
\end{center}
\end{figure}
}
                \item Assume the operator allocates the peak rate to every
                connection at one buffer.  What is the amount of buffer
                required to assure absence of loss ?  Numerical Application
                for each of the following cases, where a number $N$ of
                identical connections with peak cell rate $P$ is multiplexed.
                \begin{verbatim}
 case                       1      2      3      4
 nb of connnections      3000    300     30      3
 peak cell rate (c/s)     100   1000  10000 100000
            \end{verbatim}
\ifsol
From proposition 1.3.2, it is sufficient to allocate a buffer equal to
$B_{p}$. This gives:
\begin{verbatim}
case                           1      2      3      4
number of connnections      3000    300     30      3
peak cell rate (cells/s)     100   1000  10000 100000
required buffer (cells)     3870    660    150     44
\end{verbatim}
It is possible to reduce the buffer size if we can exploit some constraints on
the peak rate.
\fi
        \end{enumerate}

\end{problem}

%% LaTeX2e file `./temp/L20-10'
%% generated by the `filecontents' environment
%% from source `netCalBookv4' on 2019/02/19.
%%
\begin{problem}
\nfs{September 1998}

The two questions in this problem are independent.
\begin{enumerate}
        \item  An ATM source is constrained by GCRA($T=30$ slots, $\tau=60$ slots),
where time is counted in slots. One slot is the time it takes to
transmit one cell on the link. The source sends cells according to
the following algorithm.
\begin{itemize}
        \item In a first phase, cells are sent at times $t(1)=0$,
        $t(2)=15$, $t(3)=30, \ldots,t(n)= 15(n-1)$ as long as all cells
        are conformant.  In other words, the number $n$ is the largest
        integer such that all cells sent at times $t(i)=15(i-1)$, $i\leq
        n$ are conformant. The sending of  cell $n$ at time $t(n)$ ends
        the first phase.

        \item  Then the source enters the second phase. The subsequent cell
        $n+1$ is sent at the earliest time after
        $t(n)$ at which a conformant cell can be sent, and the same is
        repeated for ever. In other words, call $t(k)$ the sending time for
        cell $k$, with $k>n$; we have then: $t(k)$ is the earliest time after
        $t(k-1)$ at which a conformant cell can be sent.
\end{itemize}

How many cells were sent by the source in time interval $[0, 151]$~?


        \item A network node can be modeled as a single buffer with a
        constant output rate $c$ (in cells per second).  It receives $I$
        ATM connections labeled $1, \ldots, I$.  Each ATM connection has a
        peak cell rate $p_{i}$ (in cells per second) and a cell delay
        variation tolerance $\tau_{i}$ (in seconds) for $1 \leq i \leq I$.
        The total input rate into the buffer is at least as large as
        $\sum_{i=1}^I p_{i}$ (which is equivalent to saying that it is
        unlimited). What is the buffer size (in cells) required for a
        loss-free operation~?

\end{enumerate}
\end{problem}

%% LaTeX2e file `./temp/L20-11'
%% generated by the `filecontents' environment
%% from source `netCalBookv4' on 2019/02/19.
%%
\begin{problem}
In this problem, time is counted in slots. One slot is the duration to transmit
one ATM cell on the link.

\begin{enumerate}
        \item An ATM source $S_{1}$ is constrained by GCRA($T=50$ slots,
        $\tau=500$ slots), The source sends cells according to the
        following algorithm.
\begin{itemize}
        \item In a first phase, cells are sent at times $t(1)=0$,
        $t(2)=10$, $t(3)=20, \ldots,t(n)= 10(n-1)$ as long as all cells
        are conformant.  In other words, the number $n$ is the largest
        integer such that all cells sent at times $t(i)=10(i-1)$, $i\leq
        n$ are conformant. The sending of  cell $n$ at time $t(n)$ ends
        the first phase.

        \item  Then the source enters the second phase. The subsequent cell
        $n+1$ is sent at the earliest time after
        $t(n)$ at which a conformant cell can be sent, and the same is
        repeated for ever. In other words, call $t(k)$ the sending time for
        cell $k$, with $k>n$; we have then: $t(k)$ is the earliest time after
        $t(k-1)$ at which a conformant cell can be sent.
\end{itemize}

How many cells were sent by the source in time interval $[0, 401]$~?


        \item An ATM source $S_{2}$ is constrained by \emph{both}
        GCRA($T=10$ slots, $\tau=2$ slots) and GCRA($T=50$ slots,
        $\tau=500$ slots).  The source starts at time $0$, and has an
        infinite supply of cells to send.  The source sends its cells as
        soon as it is permitted by the combination of the GCRAs.  We call
        $t(n)$ the time at which the source sends the $n$th cell, with
        $t(1)=0$. What is the value of $t(15)$~?

\end{enumerate}
\end{problem}

%% LaTeX2e file `./temp/L20-11a'
%% generated by the `filecontents' environment
%% from source `netCalBookv4' on 2019/02/19.
%%
 \begin{problem}
 Consider a flow $R(t)$ receiving a minimum service curve
 guarantee $\beta$. Assume that
\begin{itemize}
  \item $\beta$ is concave and wide-sense increasing
  \item the $\inf$ in $R \mpc \beta$ is a $\min$
 \end{itemize}
For all $t$, call $\tau(t)$ a number such that
$$
(R \mpc \beta)(t)= R(\tau(t)) + \beta(t-\tau(t))
$$
Show that it is possible to choose $\tau$ such that if $t_1 \leq
t_2$ then  $\tau(t_1) \leq \tau(t_2)$.
 \sol{~\\Call $\tau_1=\tau(t_1)$ and
consider any $t' \leq \tau_1$. From the definition of $\tau_1$, we
have
$$
S(t') + \beta(t_1 - t' ) \geq S(\tau_1) + \beta(t_1 - \tau_1 )
$$
and thus
$$
S(t') + \beta(t_2 - t' ) \geq S(\tau_1) + \beta(t_1 - \tau_1 ) -
\beta(t_1 - t' ) + \beta(t_2 - t' )
$$
Now $\beta$ is convex, thus for any four numbers $a, b, c, d$ such
that $a \leq c  \leq b$,  $a \leq d  \leq b$ and $a+b=c+d$, we
have
$$\beta(a) + \beta(b) \geq \beta(c) + \beta(d)$$
(the interested reader will be convinced by drawing a small
figure). Applying this to $a=t_1- \tau_1, b=t_2 - t', c=t_1-t',
d=t_2-\tau_1$ gives
$$
S(t')+ \beta(t_2-t') \geq S(\tau_1)+\beta(t_2 - \tau_1)
$$
and the above equation holds for all $t'\leq \tau_1$. Consider now
the minimum, for a fixed $t_2$, of $S(t') +\beta(t_2-t')$ over all
$t' \leq t_2$. The above equation shows that the minimum is
reached for some $t' \geq \tau_1$.}
 \end{problem}
 

%% LaTeX2e file `./temp/L20-12'
%% generated by the `filecontents' environment
%% from source `netCalBookv4' on 2019/02/19.
%%
\begin{problem}
        \begin{enumerate}
                        \item Find the maximum backlog and maximum delay for an ATM
                CBR connection with peak rate $P$ and cell delay variation
                $\tau$, assuming the service curve is $c(t) = r(t-T_{0})^{+}$
\sol{The CBR connection is defined by
        GCRA($\frac{\delta}{P}$, $\tau$)\\
        if p<r then,
        BackLog = \(B+PT_0\)\\
        Maximum Delay = \( \frac{B}{r}+T_0 \)\\
        otherwise,  backlog and max, delay are infinity.}

                \item Find the maximum backlog and maximum delay for an ATM
                VBR connection with peak rate $P$, cell delay variation
                $\tau$, sustainable cell rate $M$ and burst tolerance
$\tau_{B}$ (in
                seconds), assuming the service curve is $c(t) = r(t-T_{0})^{+}$
\sol{ }
\end{enumerate}

\end{problem}

%% LaTeX2e file `./temp/L20-13'
%% generated by the `filecontents' environment
%% from source `netCalBookv4' on 2019/02/19.
%%
\begin{problem}
Show the following statements:
\begin{enumerate}
 \item  Consider a $(P,B)$ constrained flow, served at a rate $c\geq P$.
       The output is also $(P,B)$ constrained.
\sol{ We wish to compute the output $\alpha \mpd \beta$ where
$\alpha(t)=Pt+B$ and $\beta(t)=ct$.
$$
\alpha \mpd \beta(t) = Pt+B + \sup_{u\geq0} (P-c)u
$$
Since $P\leq c$, this is largest when $u=0$. Therefore $\alpha
\mpd \beta(t)=\alpha(t)$. }
 \item  Assume $a()$ has a bounded right-handside derivative.
        Then the output for a flow constrained by $a()$, served in a buffer
at a
        constant rate $c \geq \sup_{t \geq 0} a'(t)$, is also constrained by
        $a()$.
\sol{
$$
a \mpd \beta(t) = \sup_{u\geq0} a(t+u) -cu
$$
By the mean-value theorem there is a $v\in[t,t+u]$ such that
$$a(t+u) = a(t) + u a'(v).$$
Thus,
$$
a \mpd \beta(t) = a(t) + \sup_{u\geq0} (a'(v)-c)u
$$
Since all derivatives are smaller than $c$, this is at its
largest when $u=0$, giving $a \mpd \beta(t) = a(t)$. }
\end{enumerate}
\end{problem}

%% LaTeX2e file `./temp/L20-14'
%% generated by the `filecontents' environment
%% from source `netCalBookv4' on 2019/02/19.
%%
\begin{problem}
\begin{enumerate}
                \item Find the the arrival curve constraining the output for
                an ATM CBR connection with peak rate $P$ and cell delay
                variation $\tau$, assuming the service curve is $c(t) =
                r(t-T_{0})^{+}$

                \item Find the arrival curve constraining the output
                for an ATM VBR connection with peak rate $P$, cell delay
                variation $\tau$, sustainable cell rate $M$ and burst
                tolerance $\tau_{B}$ (in seconds), assuming the service curve
                is $c(t) = r(t-T_{0})^{+}$
\end{enumerate}

\end{problem}

%% LaTeX2e file `./temp/L20-15'
%% generated by the `filecontents' environment
%% from source `netCalBookv4' on 2019/02/19.
%%
\begin{problem}
Consider the figure ``Derivation of arrival curve for the output
of a flow
  served in a node with rate-latency service curve $\beta_{R,T}$".
What can be said if $t_{0}$ in the Figure is infinite, namely, if
$a'(t) > r$ for all $t$ ?
\end{problem}

%% LaTeX2e file `./temp/L20-16'
%% generated by the `filecontents' environment
%% from source `netCalBookv4' on 2019/02/19.
%%
\begin{problem}

Consider a series of guaranteed service nodes with service curves
$c_{i}(t) = r_{i}(t-T_{i})^{+}$. What is the maximum delay through
this system for a flow constrained by $(m,b)$ ? \sol{Defining
\(T_0 = \sum_{\forall i} T_i \) and \(R_0 = min_{\forall i}
\{R_i\}\)
                \[ Max. Delay = \left\{ \begin{array}{ll}
                \frac{b}{R_0}+T_0 & \mbox{if $R_0>m$}\\
                \infty & \mbox{otherwise}
                \end{array}
                \right. \]
}
\end{problem}

%% LaTeX2e file `./temp/L20-17'
%% generated by the `filecontents' environment
%% from source `netCalBookv4' on 2019/02/19.
%%
\begin{problem}

A flow with T-SPEC $(p,M,r,b)$ traverses nodes 1 and 2.  Node $i$
offers a service curve $c_{i}(t) = R_{i}(t-T_{i})^{+}$. What
buffer size is required for the flow at node 2~? \sol{ Do this
proof again using max instead.\\
 Define $t_i=\frac{B-M}{p-r}$
\\
The buffer required for the flow at node 2 is:\\ if $t_i<T_1$
BackLog = \(b+r(T_1+T_2)\)\\ if $t_i>T_1$ and $p>R_1>r$\\
\hspace{5mm}if $T_2>t_i-T_1$ BackLog = \(b+rT_2+rT_1\)\\
\hspace{5mm}if $T_2<t_i-T_1$ and $R_2>R_1$ BackLog =
\(b+rt_i+(T_1-t_i)R_1+R_1T_2\)\\ \hspace{5mm}if $T_2<t_i-T_1$ and
$R_2<R_1$ BackLog = \(b+rt_i-R_2(t_i-T_1-T_2)\)\\ if $t_i>T_1$ and
$p<R_1$\\ \hspace{5mm}if $T_2>t_i-T_1$ BackLog =
\(M+pt_i+r(T_2-t_i+T_1)\)\\ \hspace{5mm}if $T_2<t_i-T_1$ and
$R_2>p$ BackLog = \(M+pT_1+pT_2\)\\ \hspace{5mm}if $T_2<t_i-T_1$
and $R_2<p$ BackLog = \(M+pt_i-R_2(t_i-T_1-T_2)\)\\ else BackLog =
$\infty$ }
\end{problem}

%% LaTeX2e file `./temp/L20-18'
%% generated by the `filecontents' environment
%% from source `netCalBookv4' on 2019/02/19.
%%
\begin{problem}
A flow with T-SPEC $(p,M,r,b)$ traverses nodes 1 and 2.  Node $i$
offers a service curve $c_{i}(t) = R_{i}(t-T_{i})^{+}$. A shaper
is placed between nodes 1 and 2. The shaper forces the flow to the
arrival curve $z(t)=\min(R_{2}t, bt+m)$.
\begin{enumerate}

\item
What buffer size is required for the flow at the shaper~? \sol{
Let $\tau=\frac{b-M}{p-r}$. If $T_1 > \tau$ then the buffer size
is $B=rT_1+b$. Otherwise,
$$
B = \bracket{ M+(P-R_1)*\tau+R_1T_1 \gap,R_1\leq P, R_1\leq R_2 \\
M+(P-R_2)*\tau+R_2T_1 \gap,R_1\leq P, R_1>R_2 \\ M+PT_1 \gap\gap
,R_1 > P, R_2\geq P \\ M+\tau*(P-R_2)+R_2T_1 \gap ,R_1 > P, R_2<
P }
$$
}
\item
What buffer size is required at node 2~? What value do you find if
$T_{1}=T_{2}$~?
\sol{\\
Let $\tau_{2}=\frac{m}{R_{2}-a}$. The buffer size is:
$$
B_{2}=\bracket{ aT_{2} + m \gap, \tau_{2}<T_{2} \\ R_{2}T_{2} \gap, otherwise}.
$$
$B_{2}$ does not depend on $T_{1}$.
}
\item Compare the sum of the preceding buffer sizes to the size
that would be required if no re-shaping is performed.
\item Give an arrival curve for the output of node 2.
\sol{ Choose any arrival curve that satifies the
conditions}
\end{enumerate}

\end{problem}

%% LaTeX2e file `./temp/L20-19'
%% generated by the `filecontents' environment
%% from source `netCalBookv4' on 2019/02/19.
%%
\begin{problem}
        Prove the formula giving of paragraph ``Buffer Sizing at a
        Re-shaper"
\sol{Follows}
\end{problem}

%% LaTeX2e file `./temp/L20-20'
%% generated by the `filecontents' environment
%% from source `netCalBookv4' on 2019/02/19.
%%
\begin{problem}
        Is Theorem ``Input-Output Characterization of Greedy Shapers" a stronger result
        than Corollary ``Service Curve offered by a Greedy Shaper"~?
\sol{ No, the theorem is equivalent to the corollary. Indeed, if
corollary is true, then
$$
R^*\geq R \mpc \sigma
$$
Now by the arrival curve property
$$
R* \leq R^* \mpc \sigma \leq R \otimes \sigma
$$
since $R^*\leq R$. Thus $R*=R\otimes \sigma.$ }
\end{problem}

%% LaTeX2e file `./temp/L20-21'
%% generated by the `filecontents' environment
%% from source `netCalBookv4' on 2019/02/19.
%%
\begin{problem}

\begin{enumerate}

    \item Explain what is meant by ``we pay bursts only once''.

        \item Give a summary in at most 15 lines of the main properties of
        shapers

        \item  Define the following concepts by using the $\otimes $ operator:
        Service Curve, Arrival Curve, Shaper

        \item  What is a greedy source~?
\end{enumerate}
\end{problem}

%% LaTeX2e file `./temp/L20-22'
%% generated by the `filecontents' environment
%% from source `netCalBookv4' on 2019/02/19.
%%
\begin{problem}

\begin{enumerate}
        \item  Show that for a constant bit rate trunk with rate $c$, the backlog at
time $t$ is given by
$$
W(t) = \sup_{s \leq t} \left \{ R(t) - R^{*}(s) - c(t-s) \right \}
$$

        \item  What does the formula become if we assume only that,
        instead a constant bit rate trunk, the node is a scheduler
offering $\beta$ as a service curve~?
\end{enumerate}

\end{problem}

%% LaTeX2e file `./temp/L20-23'
%% generated by the `filecontents' environment
%% from source `netCalBookv4' on 2019/02/19.
%%
\begin{problem}
Is it true that offering a service curve $\beta$ implies that,
during any busy period of length $t$, the amount of service
received rate is at least $\beta(t)$ ? \sol{No, the property means
that $\beta$ is a strict service curve. In general, a service
curve property is not strict. Consider for example a shaper. }
\end{problem}

%% LaTeX2e file `./temp/L20-24'
%% generated by the `filecontents' environment
%% from source `netCalBookv4' on 2019/02/19.
%%
\begin{problem}
A flow $S(t)$ is constrained by an arrival curve $\alpha$. The
flow is fed into a shaper, with shaping curve $\sigma$. We assume
that
$$
\alpha(s) = \min (m + ps, b +rs)
$$
and
$$
\sigma (s) = \min ( Ps, B+Rs )
$$
We assume that $p>r$, $m \leq b$ and $P \geq R$.

The shaper has a fixed buffer size equal to $X \geq m$. We require
that the buffer never overflows.

\begin{enumerate}
        \item Assume that $B=+\infty$.  Find the smallest
        of $P$ which guarantees that there is no buffer overflow.
        Let $P_{0}$ be this value.

        \item We do not assume that $B=+ \infty$ any more, but we
        assume that $P$ is set to the value $P_{0}$ computed in the
        previous question.  Find the value $(B_{0}, R_{0})$ of $(B,R)$  which
        guarantees that there is no buffer overflow and minimizes the cost
        function $c(B,R)= aB + R$, where $a$ is a positive constant.

        What is the
        maximum virtual delay if $(P, B, R)=(P_{0},B_{0}, R_{0})$~?
\end{enumerate}
\end{problem}

%% LaTeX2e file `./temp/L20-25'
%% generated by the `filecontents' environment
%% from source `netCalBookv4' on 2019/02/19.
%%
\begin{problem}
\nfs{GCRA and Eff Bw, easy; GCRA and Eff Bw exam ED 99, modified
feb 99 by leb\\}
We consider a buffer of size $X$ cells, served at
a constant
        rate of $c$ cells per second.  We put $N$ identical connections into
        the buffer; each of the $N$ connections is constrained both by
        GCRA($T_{1}$, $\tau_{1}$) and GCRA($T_{2}$, $\tau_{2}$).  What is
        the maximum value of $N$ which is possible if we want to guarantee
        that there is no cell loss at all~?

        Give the numerical application for $T_{1}=0.5$~ms, $\tau_{1}=4.5$~ms,
        $T_{2}=5$~ms, $\tau_{2}=495$~ms, $c=10^6$~cells/second, $X=10^4$~cells
        \ifsol

        Call $e$ the effective bandwidth for a delay of $\frac{X}{c}$. The
        answer is the largest integer $\leq$ $\frac{c}{e}$.

        NA: $D=  \frac{X}{c} =10$ms. $e=11000/6$ cells/second. $N=6000/11 = 545$
        \fi
\end{problem}%

%% LaTeX2e file `./temp/L20-26'
%% generated by the `filecontents' environment
%% from source `netCalBookv4' on 2019/02/19.
%%
\begin{problem}
\nfs{non initial bucket level, hard\\}
 We consider a flow defined by
its function $R(t)$, with
                $R(t)=$ the number of bits observed since time $t=0$.
 \begin{enumerate}
  \item The
   flow is fed into a buffer, served at a rate $r$.  Call $q(t)$
   the buffer content at time $t$.  We do the same assumptions as
   in the lecture, namely, the buffer is large enough, and is
   initially empty. What is the expression of $q(t)$ assuming we
   know $R(t)$~?

   We assume now that, unlike what we saw in the lecture, the initial
   buffer content (at time $t=0$) is \emph{not} $0$, but some value $q_{0}
   \geq 0$. What is now the expression for $q(t)$~?
\sol{\\
The expression of $q(t)$ when the  buffer is initially empty is:
$$
q(t) = \sup_{s: s \leq t} \left \{ R(t) - R(s) - r(t-s) \right \}
$$
If the initial buffer content is some value $q_0$ then we have:
$$
q(t) = \max { ( \sup_{s: s \leq t} \left \{ R(t) - R(s) - r(t-s) \right \}, R(t)- rt+q_0  ) }
$$
}
 \item The flow is put into a leaky bucket policer, with rate
 $r$ and bucket size $b$.  This is a policer, not a shaper, so
 nonconformant bits are discarded.  We assume that the bucket
 is large enough, and is initially empty.  What is the
 condition on $R$ which ensures that no bit is discarded by the
 policer (in other words, that the flow is conformant)~?

 We assume now that, unlike what we saw in the lecture, the
 initial \emph{bucket} content (at time $t=0$) is \emph{not}
 $0$, but some value $b_{0} \geq 0$.  What is now the condition
 on $R$ which ensures that no bit is discarded by the policer
 (in other words, that the flow is conformant)~?
\sol{\\
The condition on $R$ which ensures that the flow is conformant when the bucket is initially empty is :
$$
\sup_{s: s \leq t} \left \{ R(t) - R(s) - r(t-s) \right \} \leq b
$$
If the initial bucket content is some value $b_0$ then the condition becomes:
$$
\max { ( \sup_{s: s \leq t} \left \{ R(t) - R(s) - r(t-s) \right \}, R(t)- rt+b_0  ) } \leq b
$$
}
\end{enumerate}
\end{problem}%

%% LaTeX2e file `./temp/L20-27'
%% generated by the `filecontents' environment
%% from source `netCalBookv4' on 2019/02/19.
%%
\begin{problem}
Consider a variable capacity network node, with capacity curve
$M(t)$. Show that there is one maximum function $S^*(t)$ such that
for all $0 \leq s \leq t$, we have
$$M(t)-M(s) \geq S^*(t-s)
$$
Show that $S^*$ is super-additive.

Conversely, if a function $\beta$ is super-additive, show that
there is a variable capacity network node, with capacity curve
$M(t)$, such that for all $0 \leq s \leq t$, we have $M(t)-M(s)
\geq S^*(t-s) $.

Show that, with a notable exception, a shaper cannot be modeled as
a variable capacity node.
\end{problem}

%% LaTeX2e file `./temp/L20-28'
%% generated by the `filecontents' environment
%% from source `netCalBookv4' on 2019/02/19.
%%
\begin{problem}
\begin{enumerate}
  \item Consider a packetized greedy shaper with shaping curve
  $\sigma(t)=rt$ for $t \geq 0$. Assume that $L(k)= k M$ where $M$ is fixed.
  Assume that the input is given by $R(t)=10 M$ for $t>0$ and $R(0)=0$. Compute
  the sequence $R^{(i)}(t)$ used in the representation of the output of the
  packetized greedy shaper, for $i=1,2,3,...$.
  \sol{
  The output $R^{(i)}(t)$ is a train of packets $m=1,...,10$, all of size $M$ at
  times $T^{(i)}_m = (m+i-1) \frac{M}{r}$. As $i$ goes to infinity,
  $R^{(i)}(t)$ goes to $0$, for any fixed $t$.}
  \item Same question if $\sigma(t)=(rt+ 2M) 1_\{t>0\}$.
  \sol{
  $T^{(1)}_1 = T^{(1)}_2 = 0$ and otherwise $T^{(1)}_m = (m-2) \frac{M}{r}$.
  For $i \geq 1$ we have
  $R^{(i)}=R^{(1)}$ since \thref{theo-psps} applies.}
\end{enumerate}

\end{problem}

%% LaTeX2e file `./temp/L20-29'
%% generated by the `filecontents' environment
%% from source `netCalBookv4' on 2019/02/19.
%%
\begin{problem}
\nfs{exam ed fev 2000; `` une classe au mus\'{e}e"} Consider a
source given by the function
$$
\bracket{R(t) = B \mf t > 0\\R(t)= 0 \mf t \leq 0 }
$$
Thus the flow consists of an instantaneous burst of $B$ bits.
\begin{enumerate}
\item What is the minimum arrival curve for the flow~?
  \item Assume that the flow is served in one node that offers a
  minimum service curve of the rate latency type, with rate $r$
  and latency $\Delta$. What is the maximum delay for the last bit
  of the flow~?
  \item We assume now that the flow goes through a series of two
  nodes, $\calN_1$ and $\calN_2$, where $\calN_i$ offers to the
  flow a
  minimum service curve of the rate latency type, with rate $r_i$
  and latency $\Delta_i$, for $i=1,2$. What is the the maximum delay for the last bit
  of the flow through the series of two nodes~?
  \item With the same assumption as in the previous item, call $R_1(t)$ the function describing the
  flow at the output of node $\calN_1$ (thus at the input of node $\calN_2$). What is
  the worst case minimum arrival curve for $R_1$~?
  \item We assume that we insert between $\calN_1$ and $\calN_2$ a
  ``reformatter" $\calS$.
  The input to $\calS$ is $R_1(t)$. We call $R'_1(t)$ the output of $\calS$.
  Thus $R'_1(t)$ is now the input to $\calN_2$.
   The function of the ``reformatter"$\calS$ is to delay the
  flow $R_1$ in order to output a flow $R'_1$ that is a delayed version of
  $R$. In other words, we must have $R'_1(t)=R(t-d)$ for some
  $d$. We assume that the reformatter $\calS$ is optimal in the
  sense that it chooses the smallest possible $d$. In the worst
  case, what is this optimal value of $d$~?
  \item With the same assumptions as in the previous item, what is
  the worst case end-to-end delay through the series of nodes
  $\calN_1, \calS, \calN_2$~? Is the reformatter transparent~?
\end{enumerate}
\end{problem}

%\input{temp/L20-30} %replaced by L21-11 in chapter L21
%% LaTeX2e file `./temp/L20-31'
%% generated by the `filecontents' environment
%% from source `netCalBookv4' on 2019/02/19.
%%
 \begin{problem}
 \nfs{another vl shaper}
 Let $\sigma$ be a good function. Consider the concatenation of a
 bit-by-bit greedy shaper, with curve $\sigma$, and an
 $L$-packetizer. Assume that $\sigma(0^+)=0$.
 Consider only inputs that are $L$-packetized
\begin{enumerate}
  \item Is this system a packetized shaper for $\sigma$ ?\sol{~No,
  the condition on $\sigma$ implies that a packetized shaper for
  $\sigma$ is a degenerate system, and this one is not.}
  \item Is it a packetized shaper for $\sigma+l_{\max}$ ?
  \sol{~This question is more reasonable. Indeed, since $P^L(x)
  \geq x -l_{\max}$, the answer is yes.}
  \item Is it a packetized greedy shaper for $\sigma+l_{\max}$ ?
  \sol{~This question is more difficult. In general, the answer is no.
  To see why, consider $\sigma(t)=rt$ and draw any sample path. With this
  system, the first packet is delayed by $\frac{l_1}{r}$ whereas the
  packetized greedy shaper for $rt + l_{\max}$ would not delay the
  first packet.}
\end{enumerate}
 \end{problem}
 

%% LaTeX2e file `./temp/L20-32'
%% generated by the `filecontents' environment
%% from source `netCalBookv4' on 2019/02/19.
%%
\begin{problem}
Assume that $\sigma$ is a good function and $\sigma= \sigma_0 + l
u_0$ where $u_0$ is the step function with a step at $t=0$. Can we
conclude that $\sigma_0$ is sub-additive ? \sol{ No, see textbook
for a counter-example. }
\end{problem}

%% LaTeX2e file `./temp/L20-33'
%% generated by the `filecontents' environment
%% from source `netCalBookv4' on 2019/02/19.
%%
\begin{problem}
Is the operator $(P^L)$ upper-semi-continuous ? \sol{No, consider
a sequence $x_n=l_1 + \frac{l_2}{n}$. We have $P^L(x_n)=l_1$, thus
$\inf_n [P^L(x_n)]=l_1$. Now $\inf_n x_n = l_1$ thus $P^L[\inf_n
(x_n)]=0$.}
\end{problem}

%% LaTeX2e file `./temp/L20-34'
%% generated by the `filecontents' environment
%% from source `netCalBookv4' on 2019/02/19.
%%
 \begin{problem}
\begin{enumerate}
  \item
 Consider the concatenation of an $L$-packetizer and a network
 element with minimum service curve $\beta$ and maximum service
 curve $\gamma$. Can we say that the combined system offer a
 minimum service curve $(\beta(t)-l_{\max})^+$ and a maximum service
 curve $\gamma$, as in the case where the concatenation would be
 in the reverse order ? \sol{ Yes, and the proof is similar}.
  \item Consider the concatenation of a GPS node offering a guarantee
  $\lambda_{r_1}$, an $L$-packetizer, and a second GPS node offering a guarantee
  $\lambda_{r_2}$. Show that the combined system offers a rate-latency
  service curve with rate $R=\min(r_1, r_2)$ and
  latency$E=\frac{l_{\max}}{\max(r_1, r_2)}$.
  \sol{If $r_1< r_2$, consider the combined system made of the packetizer and
  the second node. From the previous question, it offers a rate latency
  service curve with rate $r_2$ and latency
  $E=\frac{l_{\max}}{r_2}=\frac{l_{\max}}{\max(r_1, r_2)}$. Thus
  the global system offers a rate latency service curve with rate
  $R=\min(r_1, r_2)$ and latency $E$.\\
  Otherwise, consider first as combined system the concatenation
  of the first GPS node and the packetizer, and apply the textbook
  result with the same reasoning. Thus the result is true in all cases.}
\end{enumerate}
 \end{problem}
 

%% LaTeX2e file `./temp/L20-35'
%% generated by the `filecontents' environment
%% from source `netCalBookv4' on 2019/02/19.
%%
 \begin{problem}
 Consider a node that offers to a flow $R(t)$ a rate-latency
 service curve $\beta= S_{R,L}$. Assume that $R(t)$ is $L$-packetized, with
 packet arrival times called $T_1, T_2,...$  (and
 is left-continuous,
 as usual)

 Show that
 $(R\mpc \beta)(t) = \min_{T_i \in [0, t]} [R(T_i) + \beta(t-T_i)]$
  (and thus, the $\inf$ is attained).
\sol{ For a fixed $t$, study the function of $s$ given by $R(s) +
\beta(t-s)$. For $T_{i-1} < s \leq T_i$ we have $R(s)=R(T_i)$ thus
the $\inf$ for $s \in (T_{i-1}, T_i]$ is attained at $T_i$.}
 \end{problem}
 

%% LaTeX2e file `./temp/L21-1'
%% generated by the `filecontents' environment
%% from source `netCalBookv4' on 2019/02/19.
%%
\begin{problem}
\begin{enumerate}
    \item Assume $K$ connections, each with peak rate $p$, sustainable rate
    $m$ and burst tolerance $b$, are offered to a trunk with constant
    service rate $P$ and FIFO buffer of capacity $X$. Find the
    conditions on $K$ for the system to be loss-free.
    \item If $Km =P$, what is the condition on $X$ for $K$
    connections to be accepted ?
    \item What is the maximum number of connection if $p$ = 2 Mb/s,
    $m$ = 0.2 Mb/s, $X= 10 $MBytes, $b=1 $Mbyte and $P=$ 0.1, 1, 2 or
    10 Mb/s ?
    \item For a fixed buffer size $X$, draw the acceptance
    region when $K$ and $P$ are the variables.
\end{enumerate}
\end{problem}

%% LaTeX2e file `./temp/L21-2'
%% generated by the `filecontents' environment
%% from source `netCalBookv4' on 2019/02/19.
%%
\begin{problem}
    Show the formulas giving the expressions for $f_B(R)$ and
    $f_B(\alpha)$.
\end{problem}

%% LaTeX2e file `./temp/L21-3'
%% generated by the `filecontents' environment
%% from source `netCalBookv4' on 2019/02/19.
%%
\begin{problem}
\begin{enumerate}
    \item What is the effective bandwith for a connection with $p$ = 2 Mb/s,
    $m$ = 0.2 Mb/s,  $b=$ 100 Kbytes when $D=$ 1msec, 10 msec, 100 msec,
    1s ?
    \item Plot the effective bandwidth $e$ as a function of the delay
    constraint in the general case of a connection with parameters $p,
    m, b$.
\end{enumerate}
\end{problem}

%% LaTeX2e file `./temp/L21-4'
%% generated by the `filecontents' environment
%% from source `netCalBookv4' on 2019/02/19.
%%
\begin{problem}
\begin{enumerate}
    \item Compute the effective bandwidth for a mix of VBR
    connections $1, \ldots, I$.

    \item Show how the homogeneous case can be derived from your formula

    \item Assume $K$ connections, each with peak rate $p$, sustainable rate
    $m$ and burst tolerance $b$, are offered to a trunk with constant
    service rate $P$ and FIFO buffer of capacity $X$. Find the
    conditions on $K$ for the system to be loss-free.

    \item  Assume that there are two classes of connections, with
    $K_{i}$ connections in class $i$, $i=1,2$, offered to a trunk with constant
    service rate $P$ and FIFO buffer of infinite capacity $X$. The
    connections are accepted as long as their queuing delay does
    not exceed some value $D$.
    Draw the acceptance region, that is, the set of $(K_{1}, K_{2})$
    that are accepted by CAC2. Is the
    acceptance region convex ? Is the complementary of the
    acceptance region in the positive orthant convex ? Does this
    generalize to more than two classes ?
\end{enumerate}
\end{problem}


%% LaTeX2e file `./temp/L20-29'
%% generated by the `filecontents' environment
%% from source `netCalBookv4' on 2019/02/19.
%%
\begin{problem}
\nfs{exam ed fev 2000; `` une classe au mus\'{e}e"} Consider a
source given by the function
$$
\bracket{R(t) = B \mf t > 0\\R(t)= 0 \mf t \leq 0 }
$$
Thus the flow consists of an instantaneous burst of $B$ bits.
\begin{enumerate}
\item What is the minimum arrival curve for the flow~?
  \item Assume that the flow is served in one node that offers a
  minimum service curve of the rate latency type, with rate $r$
  and latency $\Delta$. What is the maximum delay for the last bit
  of the flow~?
  \item We assume now that the flow goes through a series of two
  nodes, $\calN_1$ and $\calN_2$, where $\calN_i$ offers to the
  flow a
  minimum service curve of the rate latency type, with rate $r_i$
  and latency $\Delta_i$, for $i=1,2$. What is the the maximum delay for the last bit
  of the flow through the series of two nodes~?
  \item With the same assumption as in the previous item, call $R_1(t)$ the function describing the
  flow at the output of node $\calN_1$ (thus at the input of node $\calN_2$). What is
  the worst case minimum arrival curve for $R_1$~?
  \item We assume that we insert between $\calN_1$ and $\calN_2$ a
  ``reformatter" $\calS$.
  The input to $\calS$ is $R_1(t)$. We call $R'_1(t)$ the output of $\calS$.
  Thus $R'_1(t)$ is now the input to $\calN_2$.
   The function of the ``reformatter"$\calS$ is to delay the
  flow $R_1$ in order to output a flow $R'_1$ that is a delayed version of
  $R$. In other words, we must have $R'_1(t)=R(t-d)$ for some
  $d$. We assume that the reformatter $\calS$ is optimal in the
  sense that it chooses the smallest possible $d$. In the worst
  case, what is this optimal value of $d$~?
  \item With the same assumptions as in the previous item, what is
  the worst case end-to-end delay through the series of nodes
  $\calN_1, \calS, \calN_2$~? Is the reformatter transparent~?
\end{enumerate}
\end{problem}

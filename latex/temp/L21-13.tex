%% LaTeX2e file `./temp/L21-13'
%% generated by the `filecontents' environment
%% from source `netCalBookv4' on 2019/02/19.
%%
\begin{problem}
Assume all nodes in a network are made of a GR type with rate $R$ and latency $T$, before which
a re-shaper with
  shaping curve $\sigma=\gamma_{r,b}$ is inserted. A
  flow with T-SPEC $\alpha(t)=\min(rt+b, M+pt)$
  has performed a reservation with rate
  $R$ across a sequence of $H$ such nodes, with $p\geq R$.  What is a buffer requirement at the $h$th node along the path,
  for $h=1,...H$~?
\sol{~\\ An arrival curve for the input to the $h$th GR component is $\gamma_{r,b}$; thus the
buffer required at the GR component (excluding the shaper) is $B_1=v(\gamma_{r,b}, \beta)$,
where $\beta$ is the rate-latency service curve with with rate $R$ and latency
$T'=T+l_{\max}/R$. Thus $B_1=b+rT'$.\\ However, we also need to allow some buffer for the
shaper. It is equal to $v(\alpha_{h-1}, \gamma_{r,b})$, where $\alpha'$ is an arrival curve for
the output of the $(h-1)$th node. We have
$$
\alpha' = \gamma_{r,b} \mpd \beta = \gamma_{r, b +rT'}
$$
Thus a buffer bound at the shaper is $B_2=rT'$. A total buffer bound is $B=b+ 2rT'$. Note that
it is independent of $h$.\\ We might also assume that the node employs buffer sharing between
the shaper and the GR component. If so, a buffer bound is obtained by viewing the complete node
as a concatenation of two service curve components; the bound is then
$$v_h= v(\alpha', \gamma_{r,b} \mpc
\beta )$$ In this particular case however, this does not improve the bound.\\ A more
sophisticated bound can be found if we account for the original peak rate limitation $(p,M)$.}
\end{problem}

%% LaTeX2e file `./temp/L20-6'
%% generated by the `filecontents' environment
%% from source `netCalBookv4' on 2019/02/19.
%%
\begin{problem}
\nfs{Alternative definition of GCRA} Consider the following
alternative definition of the GCRA:
\begin{definition}
The GCRA ($T,\tau$) is a controller that takes as input a cell
arrival time {\tt t} and returns {\tt result}. It has internal
(static) variables {\tt X} (bucket level) and {\tt LCT} (last
conformance time).

\begin{itemize}
        \item  initially, {\tt X = 0} and {\tt LCT = 0}
        \item  when a cell arrives at time {\tt t}, then
        \begin{verbatim}
                if (X - t + LCT > tau)
                   result = NON-CONFORMANT;
                else {
                     X = max (X - t + LCT, 0) + T;
                     LCT = t;
                     result = CONFORMANT;
                     }
        \end{verbatim}
\end{itemize}
\end{definition}
Show that the two definitions of GCRA are equivalent.\\
\sol{
Comparing the two algorithms we see that, if equivalence holds, we
should have \texttt{tat = X + LCT}.\\ The proof is now as follows.
We show by recurrence that for every input sequence, the output of
both algorithms is the same and that \texttt{tat = X + LCT} at the
end of the processing for the last input.  This is true for the
first input, at $t=0$. Now let us assume that this is true for all
inputs up to time $t$. The test done by both algorithms is the
same since \texttt{tat = X + LCT} so the output is the same. Then
\texttt{max(t, tat) + T = t + max(X- t + LCT, 0) + T} thus the
relation \texttt{tat = X + LCT} still holds after the processing
of this input. }
\end{problem}

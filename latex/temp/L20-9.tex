%% LaTeX2e file `./temp/L20-9'
%% generated by the `filecontents' environment
%% from source `netCalBookv4' on 2019/02/19.
%%
\begin{problem}
For a CBR connection, here are some values from an ATM operator:
        \begin{verbatim}
 peak cell rate (cells/s)     100   1000  10000 100000
 CDVT     (microseconds)     2900   1200    400    135
        \end{verbatim}
        \begin{enumerate}
                \item What are the $(P, B)$ parameters in b/s and bits for
                each case ?  How does $T$ compare to $\tau$ ?
\ifsol

\begin{verbatim}
P          (b/s)  42400  424000   4240000   42400000
B            (b)    547     933      2120       6148
T    (microsecs)  10000    1000       100         10
CDVT (microsecs)   2900    1200       400        135
\end{verbatim}
\fi
                \item If a connection requires a peak cell rate of 1000 cells
                per second and a cell delay variation of 1400 microseconds,
                what can be done ?
\sol{
We need to find an arrival curve in the family given by Swiss Telecom
that can bound from above your arrival curve. In the  absence of information
about the link rate, this  means choosing a peak rate $P'$ larger  than 1000,
such that the  tolerance in cells, $B'_{p}= \delta + CDVT' \times P'$ is as
large as the requested tolerance  $B_{p}= \delta + CDVT \times P$. In other
words, call $f$ the function mapping the peak rate to the tolerance, we need
to find the minimum $P'$ such that $P' f(P') \geq Pf(P)$. A plot shows that
$\log f$ is linear in $\log P$. This gives $P' =$ 1738 cells/sec.
\begin{figure}[ht]
\begin{center}
%\includegraphics[angle=-0,scale=.4]{cdvtswisscom.eps}
\insfig{cdvtswisscom}{0.4} \caption{relationship between tolerance
and peak rate} \label{first}
\end{center}
\end{figure}
}
                \item Assume the operator allocates the peak rate to every
                connection at one buffer.  What is the amount of buffer
                required to assure absence of loss ?  Numerical Application
                for each of the following cases, where a number $N$ of
                identical connections with peak cell rate $P$ is multiplexed.
                \begin{verbatim}
 case                       1      2      3      4
 nb of connnections      3000    300     30      3
 peak cell rate (c/s)     100   1000  10000 100000
            \end{verbatim}
\ifsol
From proposition 1.3.2, it is sufficient to allocate a buffer equal to
$B_{p}$. This gives:
\begin{verbatim}
case                           1      2      3      4
number of connnections      3000    300     30      3
peak cell rate (cells/s)     100   1000  10000 100000
required buffer (cells)     3870    660    150     44
\end{verbatim}
It is possible to reduce the buffer size if we can exploit some constraints on
the peak rate.
\fi
        \end{enumerate}

\end{problem}

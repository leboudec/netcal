%% LaTeX2e file `./temp/L20-10'
%% generated by the `filecontents' environment
%% from source `netCalBookv4' on 2019/02/19.
%%
\begin{problem}
\nfs{September 1998}

The two questions in this problem are independent.
\begin{enumerate}
        \item  An ATM source is constrained by GCRA($T=30$ slots, $\tau=60$ slots),
where time is counted in slots. One slot is the time it takes to
transmit one cell on the link. The source sends cells according to
the following algorithm.
\begin{itemize}
        \item In a first phase, cells are sent at times $t(1)=0$,
        $t(2)=15$, $t(3)=30, \ldots,t(n)= 15(n-1)$ as long as all cells
        are conformant.  In other words, the number $n$ is the largest
        integer such that all cells sent at times $t(i)=15(i-1)$, $i\leq
        n$ are conformant. The sending of  cell $n$ at time $t(n)$ ends
        the first phase.

        \item  Then the source enters the second phase. The subsequent cell
        $n+1$ is sent at the earliest time after
        $t(n)$ at which a conformant cell can be sent, and the same is
        repeated for ever. In other words, call $t(k)$ the sending time for
        cell $k$, with $k>n$; we have then: $t(k)$ is the earliest time after
        $t(k-1)$ at which a conformant cell can be sent.
\end{itemize}

How many cells were sent by the source in time interval $[0, 151]$~?


        \item A network node can be modeled as a single buffer with a
        constant output rate $c$ (in cells per second).  It receives $I$
        ATM connections labeled $1, \ldots, I$.  Each ATM connection has a
        peak cell rate $p_{i}$ (in cells per second) and a cell delay
        variation tolerance $\tau_{i}$ (in seconds) for $1 \leq i \leq I$.
        The total input rate into the buffer is at least as large as
        $\sum_{i=1}^I p_{i}$ (which is equivalent to saying that it is
        unlimited). What is the buffer size (in cells) required for a
        loss-free operation~?

\end{enumerate}
\end{problem}

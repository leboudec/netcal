%% LaTeX2e file `./temp/L20-34'
%% generated by the `filecontents' environment
%% from source `netCalBookv4' on 2019/02/19.
%%
 \begin{problem}
\begin{enumerate}
  \item
 Consider the concatenation of an $L$-packetizer and a network
 element with minimum service curve $\beta$ and maximum service
 curve $\gamma$. Can we say that the combined system offer a
 minimum service curve $(\beta(t)-l_{\max})^+$ and a maximum service
 curve $\gamma$, as in the case where the concatenation would be
 in the reverse order ? \sol{ Yes, and the proof is similar}.
  \item Consider the concatenation of a GPS node offering a guarantee
  $\lambda_{r_1}$, an $L$-packetizer, and a second GPS node offering a guarantee
  $\lambda_{r_2}$. Show that the combined system offers a rate-latency
  service curve with rate $R=\min(r_1, r_2)$ and
  latency$E=\frac{l_{\max}}{\max(r_1, r_2)}$.
  \sol{If $r_1< r_2$, consider the combined system made of the packetizer and
  the second node. From the previous question, it offers a rate latency
  service curve with rate $r_2$ and latency
  $E=\frac{l_{\max}}{r_2}=\frac{l_{\max}}{\max(r_1, r_2)}$. Thus
  the global system offers a rate latency service curve with rate
  $R=\min(r_1, r_2)$ and latency $E$.\\
  Otherwise, consider first as combined system the concatenation
  of the first GPS node and the packetizer, and apply the textbook
  result with the same reasoning. Thus the result is true in all cases.}
\end{enumerate}
 \end{problem}
 

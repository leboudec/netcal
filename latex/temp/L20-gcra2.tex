%% LaTeX2e file `./temp/L20-gcra2'
%% generated by the `filecontents' environment
%% from source `netCalBookv4' on 2019/02/19.
%%
\begin{problem}
\nfs{April 1999} Consider the discrete time leaky bucket
algorithm, defined as follows. It operates like the GCRA, with the
following code.
\begin{itemize}
        \item  initially, {\tt X = 0} and {\tt LCT = 0}
        \item  when a cell arrives at time {\tt t}, then
        \begin{verbatim}
                if (X - t + LCT > tau)
                   result = NON-CONFORMANT;
                else {
                     X = max (X - t + LCT, 0) + T;
                     LCT = t;
                     result = CONFORMANT;
                     }
        \end{verbatim}
\end{itemize}
Show now that for any input, both the discrete time leaky bucket
and the GCRA produce the same output.

\ifsol {\em We show by induction on time that, at every point in
time, the following relation holds
\begin{equation}\label{eq-gcra-33}
  {\tt tat} = {\tt X} + {\tt LCT}
\end{equation}
and that they produce the same output.

This  is true for $t=0$. Now assume that this is also true up to
some time $s$ and consider the first cell arrival time $t > s$. It
follows from the induction hypothesis that both algorithms perform
the same test and thus produce the same output. In addition, the
new value of ${\tt tat}$ with the GCRA is
$${\tt tat'} = {\tt max (t, tat) + T}$$
while the discrete time leaky bucket produces
$${\tt X'= max (X - t + LCT, 0) + T }
\gap \mand \gap {\tt LCT' = t }$$ from where it follows
immediately that Equation(~\ref{eq-gcra-33}) continues to hold. }
\fi

Interpret the discrete time leaky bucket algorithm as a leaky
bucket with appropriate units for time and data, and use this to
show that the GCRA is equivalent to a leaky bucket. What is the
mapping between GCRA parameters and leaky bucket parameters~?
\ifsol {\em We map the discrete time flow to a continuous time
flow, using Equation~(~\ref{eq-map-inter}). This is equivalent to
assuming a continuous time model, with instantaneous cell
arrivals. We can interprete the discrete time leaky bucket as
follows. Upon a cell arrival, the level of fluid is updated,
accounting for a leak rate of 1; then, if the updated level is
less than {\tt tau}, the arriving cell pours into the bucket an
amount of {\tt T} units of fluid. This is equivalent to requiring
that the bucket never exceeds {\tt T + tau}. Remember that we
think of an arriving cell stream as a fluid with an impulse at a
cell arrival. The algorithm is thus equivalent to a leaky bucket
controller, with bucket depth {\tt T + tau}, leak rate equal to 1
unit of fluid per time unit, and with every cell pouring into the
bucket an instantaneous amount of {\tt T} units of fluid.

Now we change the unit of data in order to let the cell size be
equal to $\delta$ units of data. The leak rate is now equal to $r
= \frac{\delta}{\tt T}$, and the bucket size to $b =
\frac{(T+\tau)\delta}{T}$. } \fi
\end{problem}

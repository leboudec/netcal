%% LaTeX2e file `./temp/L20-11'
%% generated by the `filecontents' environment
%% from source `netCalBookv4' on 2019/02/19.
%%
\begin{problem}
In this problem, time is counted in slots. One slot is the duration to transmit
one ATM cell on the link.

\begin{enumerate}
        \item An ATM source $S_{1}$ is constrained by GCRA($T=50$ slots,
        $\tau=500$ slots), The source sends cells according to the
        following algorithm.
\begin{itemize}
        \item In a first phase, cells are sent at times $t(1)=0$,
        $t(2)=10$, $t(3)=20, \ldots,t(n)= 10(n-1)$ as long as all cells
        are conformant.  In other words, the number $n$ is the largest
        integer such that all cells sent at times $t(i)=10(i-1)$, $i\leq
        n$ are conformant. The sending of  cell $n$ at time $t(n)$ ends
        the first phase.

        \item  Then the source enters the second phase. The subsequent cell
        $n+1$ is sent at the earliest time after
        $t(n)$ at which a conformant cell can be sent, and the same is
        repeated for ever. In other words, call $t(k)$ the sending time for
        cell $k$, with $k>n$; we have then: $t(k)$ is the earliest time after
        $t(k-1)$ at which a conformant cell can be sent.
\end{itemize}

How many cells were sent by the source in time interval $[0, 401]$~?


        \item An ATM source $S_{2}$ is constrained by \emph{both}
        GCRA($T=10$ slots, $\tau=2$ slots) and GCRA($T=50$ slots,
        $\tau=500$ slots).  The source starts at time $0$, and has an
        infinite supply of cells to send.  The source sends its cells as
        soon as it is permitted by the combination of the GCRAs.  We call
        $t(n)$ the time at which the source sends the $n$th cell, with
        $t(1)=0$. What is the value of $t(15)$~?

\end{enumerate}
\end{problem}

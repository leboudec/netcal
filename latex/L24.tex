\chapter{Aggregate Scheduling}
\mylabel{L24}

\nfs{Adaptive service curve au lieu de strict service curve en
section~2.}

\section{Introduction}
\mylabel{sec-ringintro}

Aggregate scheduling arises naturally in many case. Let us just
mention here the differentiated services framework (\sref{sec-ds}
on \pgref{sec-ds}) and high speed switches with optical switching
matrix and FIFO outputs. The state of the art for aggregate
multiplexing is not very rich. In this chapter, we give a panorama
of results, a number of which are new.

In a first step (\sref{sec-arrcurtrans}), we evaluate how an
arrival curve is transformed through aggregate multiplexing; we
give a catalog of results, when the multiplexing node is either a
service curve element with FIFO scheduling, or a Guaranteed Rate
node (\sref{sec-gr}), or a service curve element with strict
service curve property. This provides many simple, explicit bounds
which can be used in practice.

In a second step (\sref{sec-l24-stability}), we consider a global
network using aggregate multiplexing (see assumptions below);
given constraints at the inputs of the network, can we obtain some
bounds for backlog and delay~? Here, the story is complex. The
question of delay bounds for a network with aggregate scheduling
was first raised by Chang \cite{cha91}. For a given family of
networks, we call \emph{critical load factor} $\nu_{cri}$ a value
of utilization factor below which finite bounds exist, and above
which there exist unstable networks, i.e., networks whose backlog
grow to infinity. For feed-forward networks with aggregate
multiplexing, an iterative application of \sref{sec-arrcurtrans}
easily shows that $\nu_{cri}=1$. However, many networks are not
feed-forward, and this result does not hold in general. Indeed,
and maybe contrary to intuition, Andrews \cite{andrews00} gave
some examples of FIFO networks with $\nu_{cri} < 1$. Still, the
iterative application of \sref{sec-arrcurtrans}, augmented with a
time-stopping argument, provides lower bounds of $\nu_{cri}$
(which are less than 1).

In a third step (\sref{sec-stability}), we give a number of cases
where we can say more. We recall the result in
\thref{theo-qofisbound} on \pgref{theo-qofisbound}, which says
that, for a general network with either FIFO service curve
elements, or with GR nodes, we have $\nu_{cri} \geq \frac{1}{h-1}$
where $h$ is a bound on the number of hops seen by any flow. Then,
in \sref{sec-l24-ring}, we show that the unidirectional ring
always always has $\nu_{cri} = 1$; thus, and this may be
considered a surprise, the ring is not representative of non
feed-forward topologies. This result is actually true under the
very general assumption that the nodes on the ring are service
curve elements, with any values of link speeds, and with any
scheduling policy (even non FIFO) that satisfies a service curve
property. As far as we know, we do not really understand why the
ring is always stable, and why other topologies may not be. Last,
and not least surprising, we present in \sref{sec-l24-cbr} a
particular case, originally found by Chlamtac, Farag\'o, Zhang,
and Fumagalli \cite{CFZF98}, and refined by Zhang \cite{zhang99}
and Le Boudec and H\'{e}buterne \cite{leb99} which shows that, for
a homogeneous network of FIFO nodes with constant size packets,
strong rate limitations at all sources have the effect of
providing simple, closed form bounds.

\section{Transformation of Arrival Curve through Aggregate Scheduling}
\mylabel{sec-arrcurtrans}

Consider a number of flows served as an aggregate in a common
node. Without loss of generality, we consider only the case of two
flows. Within an aggregate, packets are served according to some
unspecified arbitration policy. In the following sub-sections, we
consider three additional assumptions.

\subsection{Aggregate Multiplexing in a Strict Service Curve Element}
\mylabel{sec-l24-ssc} The strict service curve property is defined
in \dref{def-ssc} on \pgref{def-ssc}. It applies to some isolated
schedulers, but not to complex nodes with delay elements.
\begin{theorem}[Blind multiplexing]
Consider a node serving two flows, $1$ and $2$, with some unknown
arbitration between the two flows. Assume that the node guarantees
a \emph{strict} service curve $\beta$ to the aggregate of the two
flows. Assume that flow $2$ is $\alpha_2$-smooth. Define
$\beta_1(t):=[\beta(t) - \alpha_2(t)]^+$. If $\beta_1$ is
wide-sense increasing, then it is a service curve for flow $1$.
 \mylabel{theo-blindmux}
 \mylabel{theo-wcl}
 \end{theorem}
 \pr
 The proof is a straightforward extension of that of \pref{prop-prionode} on \pgref{prop-prionode}.
\qed

We have seen an example in \sref{sec-clasc}: if $\beta(t)=Ct$
(constant rate server or GPS node) and $\alpha_2= \gamma_{r,b}$
(constraint by one leaky bucket) then the service curve for flow
$1$ is the rate-latency service curve with rate $C-r$ and latency
$\frac{b}{C-r}$. Note that the bound in \thref {theo-blindmux} is
actually for a preemptive priority scheduler where flow 1 has low
priority. It turns out that if we have no other information about
the system, it is the only bound we can find. For completeness, we
give the following case.

\begin{corollary}[Non preemptive priority node]
Consider a node serving two flows, $H$ and $L$, with
non-preemptive priority given to flow $H$. Assume that the node
guarantees a \emph{strict} service curve $\beta$ to the aggregate
of the two flows. Then the high priority flow is guaranteed a
service curve $\beta_H(t)=[\beta(t) - l^L_{\max}]^+$ where
$l^L_{\max}$ is the maximum packet size for the low priority flow.

If in addition the high priority flow is $\alpha_H$-smooth, then
define $\beta_L$ by $\beta_L(t)=[\beta(t) - \alpha_H(t)]^+$. If
$\beta_L$ is wide-sense increasing, then it is a service curve for
the low priority flow.
 \index{Priority Node}
 \mylabel{cor-prionode}
 \end{corollary}
 \pr
 The first part is an immediate consequence of
 \thref{theo-blindmux}. The second part is proven in the same way
 as \pref{prop-prionode}. \qed

If the arrival curves are affine, then the following corollary of
\thref{theo-wcl} expresses the burstiness increase due to
multiplexing.
\begin{corollary}[Burstiness Increase due to Blind Multiplexing]
Consider a node serving two flows in an aggregate manner. Assume
the aggregate is guaranteed a \emph{strict} service curve
$\beta_{R,T}$. Assume also that flow $i$ is constrained by one
leaky bucket with parameters $(\rho_i,\sigma_i)$. If $\rho_1 +
\rho_2 \leq R$ the output of the first flow is constrained by a
leaky bucket with parameters $(\rho_1, b^*_1)$ with
$$
b^*_1 = \sigma_1 +\rho_1 T +\rho_1 \frac{\sigma_2+ \rho_2
T}{R-\rho_2}
$$
 \mylabel{coro-wcl}
\end{corollary}
Note that the burstiness increase contains a term $\rho_1 T$ that
is found even if there is no multiplexing; the second term $\rho_1
\frac{\sigma_2+ \rho_2 T}{R-\rho_2}$ comes from multiplexing with
flow 2. Note also that if we further assume that the node is FIFO,
then we have a better bound (\sref{sec-l24-fifosc}).
\pr
From \thref{theo-blindmux}, the first flow is guaranteed a service
curve $\beta_{R', T'}$ with $R'=R-\rho_2$ and $T'= T+\frac{\sigma_2
+T \rho_2}{R-\rho_2}$. The result follows from a direct
application of \thref{theo-output} on \pgref{theo-output}. \qed

\paragraph{Do we need that the service curve property be strict~?}
If we relax the assumption that the service curve property is
strict, then the above results do not hold. A counter-example can
be built as follows. All packets have the same size, 1 data unit,
and input flows have a peak rate equal to 1. Flow 1 sends one
packet at time $0$, and then stops. The node delays this packet
forever. With an obvious notation, we have, for $t \geq 0$:
$$R_1(t)= \min(t,1) \mand R'_1(t)=0$$
Flow 2 sends one packet every time unit, starting at time $t=1$.
The output is a continuous stream of packets, one per time unit,
starting from time $1$. Thus
 $$R_2(t)= (t-1)^+ \mand
R'_2(t)=R_2(t)$$ The aggregate flows are, for $t\geq 0$:
 $$R(t) =t \mand R'(t)=(t-1)^+$$
In other words, the node offers to the aggregate flow a service
curve $\delta_1$. Obviously, \thref{theo-blindmux} does not apply
to flow $1$: if it would, flow 1 would receive a service curve
$(\delta_1 - \lambda_1)^+=\delta_1$, which is not true since it
receives $0$ service. We can interpret this example in the light
of \sref{sec-bb-imp} on \pgref{sec-bb-imp}: if the service curve
property would be strict, then we could bound the duration of the
busy period, which would give a minimum service guarantee to low
priority traffic. We do not have such a bound on this example. In
\sref{sec-l24-fifosc} we see that if we assume FIFO scheduling,
then we do have a service curve guarantee.

\subsection{Aggregate Multiplexing in a FIFO Service Curve Element}
\mylabel{sec-l24-fifosc}

\nfs{Je refias cette section: d'abord le th\'{e}or\`{e}me le plus
g\'{e}n\'{e}ral; puis application \`{a} TONFIFO. Puis le cas
s\'{e}par\'{e}; puis le cas simple single leaky bucket ; donner la
service curve en remarque \`{a} la fin.}

Now we relax the strict service curve property; we assume that the
node guarantees to the aggregate flow a minimum service curve, and
in addition assume that it handles packets in order of arrival at
the node. We find some explicit closed forms bounds for some
simple cases.

\begin{proposition}[FIFO Minimum Service Curves~\cite{CruzSCED}]
Consider a lossless node serving two flows, $1$ and $2$, in FIFO
order. Assume that packet arrivals are instantaneous. Assume that
the node guarantees a minimum service curve $\beta$ to the
aggregate of the two flows. Assume that flow $2$ is
$\alpha_2$-smooth. Define the family of functions
$\beta^1_{\theta}$ by
$$
\beta^1_{\theta}(t)=[\beta(t) - \alpha_2(t-\theta)]^+ 1_{\{t >
\theta\}}$$

Call $R_1(t), R'_1(t)$ the input and output for flow $1$. Then for
any $\theta \geq 0$
\begin{equation}\mylabel{eq-fifomux0}
 R'_1 \geq R_1 \mpc \beta^1_{\theta}
\end{equation}
If $\beta^1_{\theta}$ is wide-sense increasing, flow $1$ is
guaranteed the service curve $\beta^1_{\theta}$
 \mylabel{prop-fifomux}
\end{proposition}
The assumption that packet arrivals are instantaneous means that
we are either in a fluid system (one packet is one bit or one
cell), or that the input to the node is packetized prior to being
handled in FIFO order.
 \pr
 We give the proof for continuous time and assume that flow
functions are left-continuous. All we need to show is
\eref{eq-fifomux0}. Call $R_i$ the flow $i$ input, $R=R_1 + R_2$,
and similarly $R'_i, R'$ the output flows.

Fix some arbitrary parameter $\theta$ and time $t$. Define
$$u:= \sup\{v : \; R(v) \leq R'(t)\}$$
Note that $u \leq t$ and that
\begin{equation}\mylabel{eq-fifosc1}
 R(u) \leq R'(t) \mand R_r(u) \geq R'(t)
\end{equation}
where $R_r(u)=\inf_{v >u}[R(v)]$ is the limit from the right of $R$
at $u$.

(Case 1) consider the case where $u=t$. It follows from the above
and from $R' \leq R$ that $R'_1(t)=R_1(t)$. Thus for any $\theta$,
we have $R'_1(t) = R_1(t) + \beta^1_{\theta}(0)$ which shows that
$R'_1(t) \geq (R_1 \mpc \beta^1_{\theta})(t)$ in that case.

(Case 2), assume now that $u< t$. We claim that
\begin{equation}\mylabel{eq-fifosc2}
  R_1(u) \leq R'_1(t)
\end{equation}
Indeed, if this is not true, namely, $R_1(u) > R'_1(t)$, it
follows from the first part of \eref{eq-fifosc1} that $R_2(u)
<R'_2(t)$. Thus some bits from flow $2$ arrived after time $u$ and
departed by time $t$, whereas all bits of flow $1$ arrived up to
time $u$ have not yet departed at time $t$. This contradicts our
assumption that the node is FIFO and that packets arrive
instantaneously.

Similarly, we claim that
\begin{equation}\mylabel{eq-fifosc3}
  (R_2)_r(u) \geq R'_2(t)
\end{equation}
Indeed, otherwise $x:=R'_2(t)- (R_2)_r(u) >0$ and there is some
$v_0 \in (u,t]$ such that for any $v \in (u, v_0]$ we have $R_2(v)
< R'_2(t) - \frac{x}{2}$. From \eref{eq-fifosc1}, we can find some
$v_1 \in (u, v_0]$ such that if $v \in (u, v_1]$ then $R_1(v) +
R_2(v) \geq R'(t) - \frac{x}{4}$. It follows that
$$
R_1(v) \geq R'_1(t) + \frac{x}{4}
$$Thus we can find some $v$ with $R_1(v) > R'_1(t)$ whereas $R_2(v)
<R'_2(t)$, which contradicts the FIFO assumption.

Call $s$ a time such that $R'(t) \geq R(s) + \beta(t-s) $. We have
$R(s) \leq R'(t)$ thus $s\leq u$.

(Case 2a) Assume that $u< t-\theta$ thus also $t -s > \theta$.
From \eref{eq-fifosc3} we derive
$$
R'_1(t) \geq R_1(s) + \beta(t-s) + R_2(s) -R'_2(t) \geq R_1(s) +
\beta(t-s) + R_2(s) -(R_2)_r(u)
$$
%Define $\vec{x}:=\vec{R}(u^+)-\vec{R}(s)$. Thus
Now there exist some $\epsilon >0$ such that $u+\epsilon \leq
t-\theta$, thus $(R_2)_r(u) \leq R_2(t-\theta)$ and
$$
R'_1(t) \geq R_1(s) + \beta(t-s) -\alpha_2(t-s -\theta)
$$
It follows from \eref{eq-fifosc2} that
$$
  R'_1(t) \geq R_1(s)
$$
which shows that
$$ R'_1(t) \geq R_1(s) + \beta^1_{\theta}(t-s)
  $$
%Now $u <t- \theta$ thus $\vec{x} \in \calX(t-s-\theta)$ which
%shows the theorem in that case.

 (Case 2b)  Assume that $u \geq t-\theta$. By
\eref{eq-fifosc2}:
$$
R'_1(t) \geq R_1(u) = R_1(u) + \beta^1_{\theta}(t-u)
$$
 \qed

 We cannot conclude from
\pref{prop-fifomux} that $\inf_{\theta} \beta^1_{\theta}$ is a
service curve. However, we can conclude something for the output.

\begin{proposition}[Bound for Output with FIFO]
Consider a lossless node serving two flows, $1$ and $2$, in FIFO
order. Assume that packet arrivals are instantaneous. Assume that
the node guarantees to the aggregate of the two flows a minimum
service curve $\beta$.
% and a maximum service curve
%$\gamma$.
Assume that flow $2$ is $\alpha_2$-smooth. Define the family of
functions as in \pref{prop-fifomux}. Then the output of flow $1$
is $\alpha_1^*$-smooth, with
$$\alpha_1^* (t) = \inf_{\theta \geq 0 }\left(
 \alpha_1 \mpd \beta^1_{\theta}
 \right)(t)
$$
 \mylabel{prop-fifomuxoutput}
\end{proposition}

\pr
Observe first that the network calculus output bound holds even if
$\beta$ is not wide-sense increasing. Thus, from
\pref{prop-fifomux}, we can conclude that
 $\alpha_1 \mpd \beta^1_{\theta}$
 is an arrival curve for the output of flow $1$. This is true for
 any $\theta$.
 \qed

We can apply the last proposition and obtain the following
practical result.

\begin{theorem}[Burstiness Increase due to FIFO, General Case]
Consider a node serving two flows, $1$ and $2$, in FIFO order.
Assume that flow $1$ is constrained by one leaky bucket with rate
$\rho_1$ and burstiness $\sigma_1$, and flow 2 is constrained by a
sub-additive arrival curve $\alpha_2$. Assume that the node
guarantees to the aggregate of the two flows a rate latency
service curve $\beta_{R,T}$. Call $\rho_2:=\inf_{t>0}
\frac{1}{t}\alpha_2(t)$ the maximum sustainable rate for flow $2$.

If $\rho_1 + \rho_2 < R$, then at the output, flow $1$ is
constrained by one leaky bucket with rate $\rho_1$ and burstiness
$b^*_1$ with
$$
b^*_1= \sigma_1 +\rho_1 \left(T+\frac{\hat{B}}{R}\right)
$$
and
$$
 \hat{B}= \sup_{t \geq 0} \left[
  \alpha_2(t) +\rho_1 t - R t
 \right]
$$

The bound is a worst case bound. \mylabel{theo-burincfifo}
\end{theorem}

\pr
(Step 1) Define $\beta^1_{\theta}$ as in \pref{prop-fifomux}.
Define $B_2=\sup_{t \geq 0} \left[\alpha_2(t)-R t\right]$. Thus
$B_2$ is the buffer that would be required if the latency $T$
would be $0$. We first show the following
\begin{equation}\mylabel{eq-fifomux37}
\mif \theta \geq \frac{B_2}{R}+T \mthen \mfor t \geq \theta: \;
\beta^1_{\theta}(t)
=
Rt-RT - \alpha_2(t-\theta)
\end{equation}

To prove this, call $\phi(t)$ the right hand-side in
\eref{eq-fifomux37}, namely, for $t \geq \theta$ define
$\phi(t)=Rt-\alpha_2(t- \theta)-RT$. We have
$$\inf_{t > \theta} \phi(t)=  \inf_{v > 0} \left[Rv -\alpha_2(v)
-RT + R\theta
 \right]$$
 From the definition of $B_2$:
$$\inf_{t > \theta} \phi(t) = -B_2 + R \theta -RT $$
If $\theta \geq \frac{B_2}{R}+T$ then  $ \phi(t) \geq 0$ for all
$t> \theta$. The rest follows from the definition of
$\beta^1_{\theta}$.

(Step 2) We apply the second part of \pref{prop-fifomux} with
$\theta=\frac{\hat{B}}{R}+T$. An arrival curve for the output of
flow $1$ is given by
$$\alpha_1^* = \lambda_{\rho_1, \sigma_1} \mpd \beta^1_{\theta}$$
We now compute $\alpha_1^*$. First note that, obviously, $\hat{B}
\geq B_2 $,  and therefore $\beta^1_{\theta}(t)= Rt-RT -
\alpha_2(t-\theta)$ for $t \geq \theta$. $\alpha_1^*$ is thus
defined for $t>0$ by
$$\alpha_1^*(t) = \sup_{s \geq 0} \left[
\rho_1 t + \sigma_1 +\rho_1 s - \beta^1_{\theta}(s)
 \right]
 =\rho_1 t + \sigma_1 + \sup_{s \geq 0}\left[
\rho_1 s - \beta^1_{\theta}(s) \right]
$$
Define $\psi(s):=\rho_1 s - \beta^1_{\theta}(s)$. Obviously:
$$\sup_{s \in [0, \theta]} [\psi(s)]=\rho_1 \theta$$
Now from Step 1, we have
\begin{eqnarray*}
\sup_{s > \theta}[\psi(s)]&=& \sup_{s > \theta} \left[\rho_1 s -
Rs+RT + \alpha_2(s-\theta) \right]\\ &=& \sup_{v >0} \left[\rho_1
v - Rv +\alpha_2(v)\right] + (\rho_1-R) \theta +R T
\end{eqnarray*}
From the definition of $\hat{B}$, the former is equal to
$$
 \sup_{s > \theta}[\psi(s)] = \hat{B} + (\rho_1-R) \theta +R T =\rho_1
 \theta
$$
which shows the burstiness bound in the theorem.

(Step 3) We show that the bound is attained. There is a time a
$\hat{\theta}$ such that $\hat{B}=
(\alpha_2)_r(\hat{\theta})-(R-\rho_1)\hat{\theta}$. Define flow
$2$ to be greedy up to time $\hat{\theta}$ and stop from there on:
$$ \bracket{
 R_2(t) = \alpha_2(t) \mfor t \leq \hat{\theta} \\
 R_2(t) = (R_2)_r(\hat{\theta})  \mfor t > \hat{\theta}
 }
 $$
Flow $2$ is $\alpha_2$-smooth because $\alpha_2$ is sub-additive.
Define flow $1$ by
$$
\bracket{
 R_1(t) =\rho_1 t \mfor t \leq \hat{\theta} \\
 R_1(t) =\rho_1 t + \sigma_1 \mfor t > \hat{\theta}
 }
$$
Flow $1$ is $\lambda_{\rho_1, \sigma_1}$-smooth as required.
Assume the server delays all bits by $T$ at time $0$, then after
time $T$ operates with a constant rate $R$, until time
$\hat{\theta}+ \theta$, when it becomes infinitely fast. Thus the
server satisfies the required service curve property. The backlog
just after time $\hat{\theta}$ is precisely $\hat{B}+RT$. Thus all
flow-$2$ bits that arrive just after time $\hat{\theta}$ are
delayed by $\frac{\hat{B}}{R}+T=\theta$. The output for flow $1$
during the time interval $(\hat{\theta}+ \theta, \hat{\theta}+
\theta + t]$ is made of the bits that have arrived in
$(\hat{\theta}, \hat{\theta}+t]$, thus there are $\rho_1 t +
b^*_1$ such bits, for any $t$. \qed

The following corollary is an immediate consequence.
\begin{corollary}[Burstiness Increase due to FIFO]
Consider a node serving two flows, $1$ and $2$, in FIFO order.
Assume that flow $i$ is constrained by one leaky bucket with rate
$\rho_i$ and burstiness $\sigma_i$. Assume that the node
guarantees to the aggregate of the two flows a rate latency
service curve $\beta_{R,T}$. If $\rho_1 + \rho_2 < R$,  then flow
$1$ has a service curve equal to the rate latency function with
rate $R-\rho_2$ and latency $T+\frac{\sigma_2}{R}$ and at the
output, flow $1$ is constrained by one leaky bucket with rate
$\rho_1$ and burstiness $b^*_1$ with
$$
b^*_1= \sigma_1 +\rho_1 \left(T+\frac{\sigma_2}{R}\right)
$$
\mylabel{coro-burincfifo}
\end{corollary}


Note that this bound is better than the one we used in
\coref{coro-wcl} (but the assumptions are slightly different).
Indeed, in that case, we would obtain the rate-latency service
curve with the same rate $R-\rho_2$ but with a larger latency:
$T+\frac{\sigma_2+\rho_2 T}{R-\rho_2}$ instead of
$T+\frac{\sigma_2}{R}$. The gain is due to the FIFO assumption.

%\paragraph{Example: }appliquer au ring: on trouve que
%le ring est stable pour toute valeur sous critique pour n=3 mais
%pas au-dessus. We will see in the next section an example where we
%can say more.


\subsection{Aggregate Multiplexing in a GR Node}
We assume now that the node is of the Guaranteed Rate type
(\sref{sec-gr} on \pgref{sec-gr}). A FIFO service curve element
with rate-latency service curve satisfies this assumption, but the
converse is not true (\thref{theo-grcrep} on \pgref{theo-grcrep}).

\begin{theorem}
\mylabel{theo-QoFISmethod} Consider a node serving two flows, $1$
and $2$ in some aggregate manner. Arbitration between flows is
unspecified, but the node serves the aggregrate as a GR node with
rate $R$ and latency $T$. Assume that flow $1$ is constrained by
one leaky bucket with rate $\rho_1$ and burstiness $\sigma_1$, and
flow 2 is constrained by a sub-additive arrival curve $\alpha_2$.
Call $\rho_2:=\inf_{t>0} \frac{1}{t}\alpha_2(t)$ the maximum
sustainable rate for flow $2$.

If $\rho_1 + \rho_2 < R$, then at the output, flow $1$ is
constrained by one leaky bucket with rate $\rho_1$ and burstiness
$b^*_1$ with
$$
b^*_1= \sigma_1 +\rho_1 \left(T+\hat{D}\right)
$$
and
$$
 \hat{D}=
 \sup_{t > 0} [\frac{\alpha_2(t)+\rho_1 t + \sigma_1}{R} -  t]
$$
\end{theorem}
\pr
From \thref{theo-grcdelaybound} on \pgref{theo-grcdelaybound}, the
delay for any packet is bounded by $\hat{D} +T$. Thus an arrival
curve at the output of flow 1 is $\alpha_1(t+\hat{D})$. \qed
\begin{corollary}
\mylabel{coro-QoFISmethod2} Consider a node serving two flows, $1$
and $2$ in some aggregate manner. Arbitration between flows is
unspecified, but the node serves the aggregrate as a GR node with
rate $R$ and latency $T$. Assume that flow $i$ is constrained by
one leaky bucket with rate $\rho_i$ and burstiness $\sigma_i$. If
$\rho_1 + \rho_2 < R$,  then, at the output, flow $1$ is
constrained by one leaky bucket with rate $\rho_1$ and burstiness
$b^*_1$ with
$$
b^*_1= \sigma_1 +\rho_1 \left(T+\frac{\sigma_1+\sigma_2}{R}\right)
$$
\end{corollary}
We see that the bound in this section is less good than
\coref{coro-burincfifo} (but the assumptions are more general).

\section{Stability and Bounds for a Network with Aggregate Scheduling}
\mylabel{sec-l24-stability}

\subsection{The Issue of Stability}
In this section we consider the following global problem: Given a network with aggregate
scheduling and arrival curve constraints at the input (as defined in the introduction) can we
find good bounds for delay and backlog ? Alternatively, when is a network with aggregate
scheduling stable (i.e., the backlog remains bounded)~? As it turns out today, this problem is
open in many cases. In the rest of the chapter, we make the following assumptions.
\paragraph{Assumption and Notation}
\begin{itemize}
  \item Consider a network with a fixed number $I$ of flows, following
fixed paths. The collection of paths is called the topology of the network. A network node is
modeled as a collection of output buffers, with no contention other than at the output buffers.
Every buffer is associated with one unidirectional link that it feeds.

\item Flow $i$ is constrained by one leaky bucket of rate $\rho_i$
and burstiness $\sigma_i$ at the input.
\item Inside the network, flows are treated as an aggregate by the
network; within an aggregate, packets are served according to some unspecified arbitration
policy. We assume that the node is such that the aggregate of all flows receives a service curve
at node $m$ equal to the rate-latency function with rate $r_m$ and latency $e_m$. This does not
imply that the node is work-conserving. Also note that we do not require, unless otherwise
specified, that the service curve property be strict. In some parts of the chapter, we make
additional assumptions, as explained later.

$e_m$ accounts for the latency on the link that exits node $m$; it also account for delays due
to the scheduler at node $m$.

  \item We write $i \ni m$ to express that node $m$ is on the route of
flow $i$. For any node  $m$, define $\rho^{(m)} = \sum_{i \ni m} \rho_i$. The utilization factor
of link $m$ is $\frac{\rho^{(m)}}{r_m}$ and the load factor of the network is $\nu = \max_m
\frac{\rho^{(m)}}{r_m}$.

\item  The bit rate of the link feeding node $m$ is $C_{m} < + \infty$, with $C_m \geq r_m$.
\end{itemize}


In the context of the following definition, we call ``network" $\calN$ a system satisfying the
assumptions above, where all parameters except $\rho_i, \sigma_i, r_m, e_m$ are fixed. In some
cases (\sref{sec-timestoppingm}), we may add additional constraints on these parameters.
\begin{definition}[Critical Load Factor]
We say that $\nu_{cri}$ is the critical load factor for a network
$\calN$ if\index{Critical Load Factor}\index{1nucri@$\nu_{cri}$}
\begin{itemize}
  \item for all values of $\rho_i, \sigma_i, r_m, e_m$ such that
  $\nu < \nu_{cri}$, $\calN$ is stable
  \item there exists values of $\rho_i, \sigma_i, r_m, e_m$ with $\nu > \nu_{cri}$ such
  that $\calN$ is unstable.
\end{itemize}
\end{definition}
It can easily be checked that $\nu_{cri}$ is unique for a given
network $\calN$.

It is also easy to see that for all well defined networks, the
critical load factor is $\leq 1$. However, Andrews gave in
\cite{andrews00} an example of a FIFO network with $\nu_{cri} <
1$. The problem of finding the critical load factor, even for the
simple case of a FIFO network of constant rate servers, seems to
remain open. Hajek \cite{hajek2000} shows that, in this last case,
the problem can be reduced to that where every source $i$ sends a
burst $\sigma_i$ instantly at time $0$, then sends at a rate
limited by $\rho_i$.

In the rest of this section and in \sref{sec-stability}, we give
lower bounds on $\nu_{cri}$ for some well defined sub-classes.

\paragraph{Feed-Forward Networks}
A feed-forward network is one in which the graph of unidirectional
links has no cycle. Examples are interconnection networks used
inside routers or multiprocessor machines. For a feed-forward
network made of \emph{strict} service curve element or GR nodes,
$\nu_{cri}=1$. This derives from applying the burstiness increase
bounds given in \sref{sec-arrcurtrans} repeatedly, starting from
network access points. Indeed, since there is no loop in the
topology, the process stops and all input flows have finite
burstiness.

\paragraph{A Lower Bound on the Critical Load Factor}
It follows immediately from \thref{theo-qofisbound} on
\pgref{theo-qofisbound} that for a network of GR nodes (or FIFO
service curve elements), we have $\nu_{cri} \geq \frac{1}{h-1}$,
where $h$ is the maximum hop count for any flow. A slightly better
bound can be found if we exploit the values of the peak rates
$C_m$ (\thref{theo-qofisbound2}).


\subsection{The Time Stopping Method}
\mylabel{sec-timestoppingm} \mylabel{sec-l24-prio}

For a non feed-forward network made of \emph{strict} service curve
element or GR nodes, we can find a lower bound on $\nu_{cri}$
(together with bounds on backlog or delay), using the time
stopping method. It was introduced by Cruz in \cite{cru91b}
together with bounds on backlog or delay. We illustrate the method
on a specific example, shown on \fref{fig-dsnet1}. All nodes are
constant rate servers, with unspecified arbitration between the
flows. Thus we are in the case where all nodes are strict service
curve elements, with service curves of the form $\beta_m=
\lambda_{C_m}$.

The method has two steps. First, we assume that there is a finite
burstiness bound for all flows;  using \sref{sec-arrcurtrans} we
obtain some equations for computing these bounds. Second, we use
the same equations to show that, under some conditions, finite
bounds exist.

\begin{figure}[!htbp]
  \insfig{dsnet1}{0.7}
  \mycaption{A simple example with aggregate scheduling, used to
   illustrate the bounding method. There are three nodes numbered
   $0,1,2$ and six flows, numbered $0,...,5$.
   For $i=0,1,2$, the path of flow $i$ is $i, (i+1) \mod 3, (i+2) \mod
   3$ and the path of flow $i+3$ is $i, (i+2) \mod 3, (i+1) \mod
   3$. The fresh arrival curve is the same
for all flows, and is given by $\alpha_i=\gamma_{\rho,\sigma}$.
All nodes are constant rate, work conserving servers, with rate
$C$. The utilization factor at all nodes is $6\frac{\rho}{C}$.}
  \mylabel{fig-dsnet1}
\end{figure}

\paragraph{First step: inequations for the bounds}


For any flow $i$ and any node $m \in i$, define $\sigma_i^m$ as
the maximum backlog that this flow would generate in a constant
rate server with rate $\rho_i$. By convention, the fresh inputs
are considered as the outputs of a virtual node numbered $-1$. In
this first step, we assume that $\sigma_i^m$ is finite for all $i$
and $m \in i$.

By applying \coref{coro-wcl} we find that for all $i$ and $m \in
i$:
\begin{equation}\mylabel{eq-ds-lin}
\bracket{
 \sigma_i^0 \leq \sigma_i \\
\sigma_i^m = \sigma_i^{\mbox{pred}_i(m)} + \rho_i \frac{
 \sum_{j \ni m, j \neq i}\sigma_j^{\mbox{pred}_j(m)}
 }
 {C-\sum_{j \ni m, j \neq i} \rho_j}
 }
\end{equation}
where $\mbox{pred}_i(m)$ is the predecessor of node $m$. If $m$ is
the first node on the path of flow $i$, we set by convention
$\mbox{pred}_i(m)=-1$ and $\sigma_i^{-1}=\sigma_i$.

Now put all the $\sigma_i^m$, for all $(i,m)$ such that $m \in i$,
into a vector $\vec{x}$ with one column and $n$ rows, for some
appropriate $n$. We can re-write \eref{eq-ds-lin} as
\begin{equation}\mylabel{eq-ds-lin2}
\vec{x} \leq A \vec{x} + \vec{a}
\end{equation}
where $A$ is an $n \times n$, non-negative matrix and $\vec{a}$ is
a non-negative vector depending only on the known quantities
$\sigma_i$. The method now consists in assuming that the spectral
radius of matrix $A$ is less than $1$. In that case the power
series $I + A + A^2 + A^3 + ...$ converges and is equal to
$(I-A)^{-1}$, where $I$ is the $n \times n$ identity matrix. Since
$A$ is non-negative, $(I-A)^{-1}$ is also non-negative; we can
thus multiply \eref{eq-ds-lin2} to the left by $(I-A)^{-1}$ and
obtain:
\begin{equation}\mylabel{eq-ds-lin3}
\vec{x} \leq (I-A)^{-1} \vec{a}
\end{equation}
which is the required result, since $\vec{x}$ describes the
burstiness of all flows at all nodes. From there we can obtain
bounds on delays and backlogs.

Let us apply this step to our network example. By symmetry, we
have only two unknowns $x$ and $y$, defined as the burstiness
after one and two hops:
$$\bracket{
x=\sigma_0^0=\sigma_1^1=\sigma_2^2=\sigma_3^0=\sigma_4^1=\sigma_5^2\\
y=\sigma_0^1=\sigma_1^2=\sigma_2^0=\sigma_3^2=\sigma_4^0=\sigma_5^1
 }
$$
\eref{eq-ds-lin} becomes
$$
\bracket{
 x \leq \sigma + \frac{\rho}{C -5 \rho} ( \sigma + 2x + 2y)\\
 y \leq x + \frac{\rho}{C -5 \rho} ( 2 \sigma +  x + 2y)
  }
$$
Define $\eta=\frac{\rho}{C-5 \rho}$; we assume that the
utilization factor is less than $1$, thus $0 \leq \eta <1$. We can
now write \eref{eq-ds-lin2} with
$$
 \vec{x}=\left(
  \begin{array}{c}
    x\\y
  \end{array}\right)
  ,\;
 A= \left(\begin{array}{c c}
    2 \eta &  2 \eta\\1 + \eta & 2 \eta
  \end{array}\right)
  ,\;
   \vec{a}=\left(
  \begin{array}{c}
    \sigma (1 + \eta)\\ 2 \sigma \eta
  \end{array}\right)
$$
Some remnant from linear algebra, or a symbolic computation
software, tells us that
$$
(I-A)^{-1}=\left(\begin{array}{c c}
    \frac{1-2  \eta}{1-6  \eta+2
{\eta^2}} & \frac{2  \eta}{1-6  \eta+2  {\eta^2}}  \\
   \frac{1+\eta}{1-6  \eta+2
{\eta^2}}& \frac{1-2  \eta}{1-6  \eta+2  {\eta^2}}
  \end{array}\right)
$$
If $\eta < \frac{1}{2} (3 -\sqrt{7}) \approx 0.177$ then
$(I-A)^{-1}$ is positive. This is the condition for the spectral
radius of $A$ to be less than 1. The corresponding condition on
the utilization factor $\nu= \frac{6\rho}{C}$ is
\begin{equation}\mylabel{eq-example-ds-1}
 \nu <2 \frac{8-\sqrt{7}}{19} \approx 0.564
\end{equation}
Thus, for this specific example, if \eref{eq-example-ds-1} holds,
and if the burstiness terms $x$ and $y$ are finite, then they are
bounded as given in \eref{eq-ds-lin3}, with $(I-A)^{-1}$ and
$\vec{a}$ given above.

\paragraph{Second Step: time stopping}

We now prove that there is a finite bound if the spectral radius
of $A$ is less than 1. For any time $\tau>0$, consider the virtual
system made of the original network, where all sources are stopped
at time $\tau$. For this network the total number of bits in
finite, thus we can apply the conclusion of step 1, and the
burstiness terms are bounded by \eref{eq-ds-lin3}. Since the
right-handside \eref{eq-ds-lin3} is independent of $\tau$, letting
$\tau$ tend to $+\infty $ shows the following.
\begin{proposition}
With the notation in this section, if the spectral radius of $A$
is less than $1$, then the burstiness terms $b^m_i$ are bounded by
the corresponding terms in \eref{eq-ds-lin3}.
\mylabel{prop-stoppingTimeMethod}
\end{proposition}
Back to the example of \fref{fig-dsnet1}, we find that if the
utilization factor $\nu$ is less than $0.564$, then the burstiness
terms $x$ and $y$ are bounded by
$$
\bracket{
 x \leq 2 \sigma \frac{18 -33 \nu + 16 \nu^2}
 {36 -96 \nu + 57 \nu^2 }\\
 y \leq  2 \sigma \frac{18 - 18  \nu + \nu^2}
 {36 -96 \nu + 57 \nu^2}
 }
$$
The aggregate traffic at any of the three nodes is $\gamma_{6\rho,
b}$-smooth with $b=2(\sigma + x + y)$. Thus a bound on delay is
(see also \fref{fig-L21-ds}):
$$
 d=\frac{b}{C}= 2 \frac{\sigma}{C}
 \frac{108 - 198 \nu + 91 \nu^2}{36 - 96 \nu + 57 \nu^2}
$$

\begin{figure}[!htbp]
  \insfig{L21-ds}{0.5}
  \mycaption{The bound $d$ on delay at any node obtained by the method
  presented here for the network of
  \fref{fig-dsnet1}  (thin line).
  The graph shows $d$ normalized by $\frac{\sigma}{C}$
  (namely, $\frac{d C}{\sigma}$),
  plotted as a function of
  the utilization factor. The thick line is a delay bound
  obtained if every flow is re-shaped at every output.}
  \mylabel{fig-L21-ds}
\end{figure}

\paragraph{The critical load factor for this example}
For the network in this example, where we impose the constraint
that all $\rho_i$ are equal, we find $\nu_{cri} \geq \nu_0 \approx
0.564$, which is much less than $1$. Does it mean that no finite
bound exists for  $\nu_0 \leq \nu <1$ ? The answer to this
question is not clear.

First, the $\nu_0$ found with the method can be improved if we
express more arrival constraints. Consider our particular example:
we have not exploited the fact that the fraction of input traffic
to node $i$ that originates from another node has to be
$\lambda_C$-smooth. If we do so, we will obtain better bounds.
Second, if we know that nodes have additional properties, such as
FIFO, then we may be able to find better bounds. However, even so,
the value of $\nu_{cri}$ seems to be unknown.

%In general, the issue of stability, or existence of finite bounds,
%for a given network with aggregate multiplexing is still
%unresolved. The question of delay bounds for a network with
%aggregate scheduling was first raised in \cite{cha91}. We will see
%in \cref{L24} that if the network is a ring of constant bit rate
%servers, then $\nu_0=1$, in other words, we can find a bound for
%any $\nu<1$ \cite{tg96}. However, the proof does not extend to
%general service curve guarantees, even strict service curves. In
%\cite{andrews00}, the author exhibits an unstable network similar
%to ours, with $\nu=1$ (a ``critical'' network), and claims that
%this also holds for some sub-critical networks, but the proof is
%not conclusive. In addition, a startling fact referred to as a
%``non-monotone property'' of FIFO networks is described in
%\cite{andrews00}, where it is shown that a network that is
%\emph{stable} with a set of sessions with given rates may become
%\emph{unstable} if the rates of some of these sessions are
%\emph{reduced} for a period of time. On the other side of the
%spectrum, we will also see in \cref{L24} strong and explicit
%results \cite{leb99,CFZF98} with FIFO multiplexing.

\paragraph{The price for aggregate scheduling}
Consider again the example on \fref{fig-dsnet1}, but assume now
that every flow is reshaped at every output. This is not possible
with differentiated services, since there is no per-flow
information at nodes other than access nodes. However, we use this
scenario as a benchmark that illustrates the price we pay for
aggregate scheduling.

With this assumption, every flow has the same arrival curve at
every node. Thus we can compute a service curve $\beta_1$ for flow
$1$ (and thus for any flow) at every node, using \thref{theo-wcl};
we find that $\beta_1$ is the rate-latency function with rate $(C-
5 \rho)$ and latency $\frac{5 \sigma}{C-5 \rho}$. Thus a delay
bound for flow at any node, including the re-shaper, is
$h(\alpha_1, \alpha_1 \mpc \beta_1)= h(\alpha_1,
\beta_1)=\frac{6C}{C- 5 \rho}$ for $\rho \leq \frac{C}{6}$.
\fref{fig-L21-ds} shows this delay bound, compared to the delay
bound we found if no reshaper is used. As we already know, we see
that with per-flow information, we are able to guarantee a delay
bound for any utilization factor $\leq 1$. However, note also that
for relatively small utilization factors, the bounds are very
close.

\section{Stability Results and Explicit Bounds}
\mylabel{sec-stability}

In this section we give strong results for two specific case. The
former is for a unidirectional ring of aggregate servers (of any
type, not necessarily FIFO or strict service curve). We show that
for all rings, $\nu_{cri}=1$. The latter is for any topology, but
with restrictions on the network type: packets are of fixed size
and all links have the same bit rate.

\subsection{The Ring is Stable}
\mylabel{sec-l24-ring} \mylabel{sec-ringisstable}  The result was
initially obtained in \cite{tg96} for the case of a ring of
constant rate servers, with all servers having the same rate. We
give here a more general, but simpler form.

\paragraph{Assumption and Notation}
We take the same assumptions as in the beginning of \sref{sec-l24-stability} and assume in
addition that the network topology is a unidirectional ring. More precisely:
\begin{itemize}
\item
The network is a unidirectional ring of $M$ nodes, labelled
$1,..., M$.
  We use the notation $m \oplus k= (m + k -1) \mod M + 1$
  and $m \ominus k= (m-k-1) \mod M + 1$,
  so that the successor of node $m$ on the ring is node $m \oplus 1$ and its
  predecessor is node $m \ominus 1$.

\item  The route of flow $i$ is $(0, i.\mathrm{first},
i.\mathrm{first}\oplus 1, ..., i.\mathrm{first}\oplus (h_i-1))$
where $0$ is a virtual node representing the source of flow $i$,
$i.\mathrm{first}$ is the first hop of flow $i$, and $h_i$ is the
number of hops of flow $i$.  At its last hop, flow $i$ exits the
network. We assume that a flow does not wrap, namely, $h_i \leq
M$. If $h_i=M$, then the flow goes around the all ring, exiting at
the same node it has entered.
\item Let $b_m=e_m r_m$ and let $b= \sum_m b_m$ reflect the total latency of the ring.

\item  For any node  $m$ let $\sigma^{(m)} = \sum_{i \ni m} \sigma_i$.

Let $\sigma_{\max}= \max_{m=1}^M \sigma^{(m)}$ and $\sigma= \sum_i
\sigma_i$. Note that $\sigma_{\max} \leq \sigma \leq M
\sigma_{\max}$.

\item  Define
$\eta= \min_m(r_m - \rho^{(m)})$.

\item Let $\rho^{(m)}_0=\sum_{i.\mathrm{first}=m}\rho_i$ and
$\mu= \max_{m=0}^M \left[ C_m - r_m + \rho^{(m)}_0\right]^+$. $\mu$
reflects the sum of the peak rate of transit links and the rates
of fresh sources, minus the rate guaranteed to the aggregate of
microflows. We expect high values of $\mu$ to give higher bounds.
\end{itemize}

\begin{theorem}
If $\eta >0$ (i.e. if the utilization factor is $<1$) the backlog
at any node of the unidirectional ring is bounded by
$$M \frac{\mu}{\eta} \left( M \sigma_{\max} + b \right) + \sigma +
b
$$
 \mylabel{theo-ring}
\end{theorem}

\pr
The proof relies on the concept of chain of busy periods, combined
with the time stopping method in \sref{sec-timestoppingm}.

For a node $m$ and a flow $i$, define $R^m_i(t)$ as the cumulative
amount of data of flow $i$ at the output of node $m$. For $m=0$,
this defines the input function. Also define
\begin{equation}\mylabel{eq-lemfifodefx}
 x_m(t) = \sum_{i \ni m} \left( R^0_i(t) -R^m_i(t)\right)
\end{equation}
thus $x_m(t)$ is the total amount of data that is present in the
network at time $t$ and will go through node $m$ at some time $>
t$.

We also define the backlog at node $m$ by
 $$
 q_m(t)=\sum_{i \ni m, i. \mathrm{first}\neq m} R^{m \ominus 1}_i(t)
 +
 \sum_{i. \mathrm{first}= m} R^{0}_i(t)
 -
 \sum_{i \ni m} R^{m}_i(t)
 $$
Now obviously, for all time $t$ and node $m$:
\begin{equation}\mylabel{eq-ringisstable2}
  q_m(t) \leq x_m(t)
\end{equation}
and
\begin{equation}\mylabel{eq-ringisstable}
  x_m(t) \leq \sum_{n=1}^M q_n(t)
\end{equation}

(Step 1) Assume that a finite bound $X$ exists. Consider a time
$t$ and a node $m$ that achieves the bound: $x_m(t)=X$. We fix $m$
and apply \lref{lem-cbp} to all nodes $n$. Call $s_n$ the time
called $s$ in the lemma. Since $x_n(s_n) \leq X$, it follows from
the first formula in the lemma that
\begin{equation}\mylabel{eq-ring1}
(t-s_n) \eta \leq M \sigma_{\max}  + b
\end{equation}
By combining this with the second formula in the lemma we obtain
 $$q_n(t) \leq  \mu \frac{M \sigma_{\max}  + b}{\eta} + b_n + \sigma^{(n)}_0
 $$
Now we apply \eref{eq-ringisstable} and note that $\sum_{n=1}^M
\sigma^{(n)}_0=\sigma$, from which we derive
\begin{equation}\mylabel{eq-ring3}
X \leq M \frac{\mu}{\eta} \left( M \sigma_{\max} + b \right) +
\sigma + b
\end{equation}

(Step 2) By applying the same reasoning as in
\sref{sec-timestoppingm}, we find that \eref{eq-ring3} is always
true. The theorem follows from \eref{eq-ringisstable2}. \qed
\begin{lemma}
 \mylabel{lem-cbp}
 For any nodes $m, n$ (possibly with $m = n$), and for any time $t$ there is
 some $s$ such that
 $$
 \bracket{
 x_m(t) \leq  x_n (s)-  (t-s) \eta + M \sigma_{\max} + b\\
 q_n(t) \leq  (t-s)\mu  + b_n + \sigma^{(n)}_0
 }
 $$
 with $\sigma^{(n)}_0=\sum_{i.\mathrm{first}=n}\sigma_i$.
\end{lemma}
\pr By definition of the service curve property at node $m$, there
is some $s_1$ such that
$$
   \sum_{i \ni m} R^m_i(t) \geq
   \sum_{i \ni m, i.\mathrm{first} \neq
m} R^{m \ominus 1}_i(s_1) + \sum_{i.\mathrm{first} = m} R^0_i(s_1)
 + r_m(t-s_1) - b_m $$
 which we can rewrite as
 $$
\sum_{i \ni m} R^m_i(t) \geq -A
    + \sum_{i \ni m} R^0_i(s_1)
+ r_m(t-s_1) - b_m
$$
with
  $$
   A = \sum_{i \ni m, i.\mathrm{first} \neq m}
  \left(R^0_i(s_1)-R^{m-1}_i(s_1)\right)
  $$
%  and
%  $$
% B =\sum_{i \ni m} \left(R^0_i(t)-R^0_i(s_1)\right)
% - r_m(t-s_1) + r_m E
%  $$
Now the condition $\left\{i \ni m, i.\mathrm{first} \neq
m\right\}$
 implies that flow $i$ passes through node $m-1$, namely, $\left\{i \ni
 (m-1)\right\}$. Furthermore, each element in the summation that
 constitutes
 $A$ is nonnegative. Thus
 $$A \leq \sum_{i \ni
 (m-1)}  \left(R^0_i(s_1)-R^{m-1}_i(s_1)\right) = x_{m \ominus 1} (s_1)
 $$
Thus
\begin{equation}\mylabel{eq-fifolem23}
  \sum_{i \ni m} R^m_i(t) \geq -x_{m \ominus 1} (s_1) + \sum_{i \ni m} R^0_i(s_1)
+ r_m(t-s_1) - b_m
\end{equation}
Now combining this with the definition of $x_m(t)$ in
\eref{eq-lemfifodefx} gives:
$$
x_m(t) \leq  x_{m \ominus 1} (s_1) + \sum_{i \ni m}
\left(R^0_i(t)-R^0_i(s_1)\right) - r_m(t-s_1) + b_m
$$
From the arrival curve property applied to all micro-flows $i$ in
the summation, we derive:
$$
x_m(t) \leq  x_{m \ominus 1} (s_1)- (r_m - \rho^{(m)}) (t-s_1) +
\sigma^{(m)} + b_m
$$
and since $r_m - \rho^{(m)} \geq \eta$ and $\sigma^{(m)} \leq
\sigma_{\max}$ by definition of $\eta$ and $\sigma_{\max}$, we
have
 $$
x_m(t) \leq  x_{m \ominus 1} (s_1)-  (t-s_1) \eta + \sigma_{\max}
+ b_m
 $$
 We apply the same reasoning to node $m \ominus 1$ and
time $s_1$, and so on iteratively until we reach node $n$
backwards from $m$. We thus build a sequence of times $s_0=t,s_1,
s_2, ..., s_j,  ..., s_k$ such that
\begin{equation}\mylabel{eq-fifolem893}
  x_{m \ominus j}(s_j) \leq  x_{m \ominus (j+1)} (s_{j+1})-
(t-s_{j+1}) \eta + \sigma_{\max} + b_{m \ominus j}
\end{equation}
until we have $m \ominus k= n$. If $n=m$ we reach the same node
again by a complete backwards rotation and $k=M$. In all cases, we
have $k \leq M$. By summing \eref{eq-fifolem893} for $j=0$ to
$k-1$ we find the first part of the lemma.

Now we prove the second part. $s=s_k$ is obtained by applying the
service curve property to node $n$ and time $s_{k-1}$. Apply the
service curve property to node $n$ and time $t$. Since $t \geq
s_{k-1}$, we know from \pref{prop-t0augmente} on
\pgref{prop-t0augmente} that we can find some $s' \geq s$ such
that
$$\sum_{i \ni n} R^n_i(t) \geq \sum_{i \ni
n, i.\mathrm{first} \neq n} R^{n-1}_i(s') + \sum_{i.\mathrm{first}
= n} R^0_i(s') + r_n(t-s') - b_n
$$
Thus
\begin{eqnarray*}
\lefteqn{ q_n(t) \leq
  \sum_{i \ni n, i.\mathrm{first} \neq n}
\left(R^{n\ominus 1}_i(t)-R^{n \ominus 1}_i(s')\right) +}
 \\ & &
\sum_{i.\mathrm{first} = n} (R^0_i(t)-R^0_i(s')) - r_n(t-s') + b_n
\\
 & & \leq (C_n-r_n + \rho^{(n)}_0)(t-s') + b_n + \sigma^{(n)}_0
\leq(t-s') \mu  + b_n + \sigma^{(n)}_0
\end{eqnarray*} the second
part of the formula follows from $s \leq s'$. \qed

\paragraph{Remark: }
A simpler, but weaker bound, is
$$M \frac{\mu}{\eta} \left( M \sigma + b \right) + \sigma +b
$$
or
\begin{equation}\mylabel{eq-tg33}
 M \frac{\mu}{\eta} \left( M \sigma_{\max} + b \right) + M \sigma_{\max} +b
\end{equation}
\paragraph{The special case in \cite{tg96}: }
Under the assumption that all nodes are constant rate servers of
rate equal to $1$ (thus $C_m=r_m=1$ and $b_m$ is the latency of
the link $m$), the following bound is found in \cite{tg96}:
\begin{equation}\mylabel{eq-tg96}
  B_1= \frac{M b + M^2 \sigma_{\max}}{\eta}+ b
\end{equation}
In that case, we have $\mu \leq 1 - \eta$. By applying
\eref{eq-tg33}, we obtain the bound
%$$
%B_2= \frac{M(1 -\eta)b + \left[ M^2 (1- \eta) \right]\sigma_{\max} +
%\eta \sigma }{\eta} + b
%$$
%Now
%\begin{equation}\mylabel{eq-ringsig}
% \sigma \leq M \sigma_{\max}
%\end{equation}
%thus
$$B_2 = \frac{M \mu b + \left[ M^2 \mu + M \eta
\right]\sigma_{\max}}{\eta} + b
$$
since
\begin{equation}\mylabel{eq-ringsig}
  \mu \leq 1-\eta
\end{equation}
and $0< \eta \leq 1, \; M \leq M^2$, we have $B_2 <B_1$, namely,
our bound is better than that in \cite{tg96}. If there is equality
in \eref{eq-ringsig} (namely, if there is a node that receives no
transit traffic), then both bounds are equivalent when $\eta
\rightarrow 0$. \nfs{However, in that case, there is no feedback
in the ring and even better bounds can be found by applying
\thref{theo-wcl}.}
%
%\section{A General Theory for Aggregate Multiplexing}
%Je refais cette section sur la base de la d\'{e}couverte que
%TONFIFO est valable pour autre que FIFO.
%\subsection{A General Theorem}
%\paragraph{Assumptions and Notation}
%Consider a lossless node serving a set of flows, $R_1(t), ...,
%R_I(t)$ with some unknown arbitration between flows. Call
%$R'_1(t), ..., R'_I(t)$ the output flows. We are interested in
%finding an arrival curve for the output $R'_A(t)=\sum_{i \in A}
%R'_i(t)$ of the sub-aggregate of all flows in some subset $A$ of
%$\{1, ..., I\}$.
%
%Assume that the aggregate of all flows receives a minimum service
%curve $\beta$ and a maximum service curve $\gamma$. For example, a
%work conserving link with constant bit rate $R$ corresponds to
%$\beta=\gamma=\lambda_R$.
%
%Assume that packet arrivals are instantaneous. We assume that
%functions $R$ and $R'$ are left-continuous.
%
%Assume that the flows are subject to $J$ joint \emph{combined}
%arrival curve constraints, defined by
%\begin{equation}\mylabel{eq-defjcs}
%  \sum_{i \in S_j} \left( R_i(t) -R_i(s) \right) \leq
%  \alpha_j(t-s)
%\end{equation}
%for $j=1$ to $J$, where $S_j$ is the set of flows that are
%constrained by the $j$th arrival curve $\alpha_j$.
%
%Put an example here.
%
%Define
%$$\calX(t):=\{\vec{x} \in (\Reals^+)^I | \mfa j \; :
%\sum_{i \in S_j} x_i \leq
%  \alpha_j(t) \}
%$$
%so that the arrival curve constraints can be rewritten
%$\vec{R}(t)-\vec{R}(t) \in \calX(t-s)$.
%
%Let $B$ be the complement of $A$. We denote with $\vec{x}_A$ the
%vector equal to the restriction of $\vec{x}$ to the set of indices
%$A$. We also use the notation $\vec{x}=(\vec{x}_A, \vec{x}_B)$.
%
%For any vector $\vec{x}$ we denote by $|\vec{x}|$ the sum of its
%coordinates. The flow function for the input of sub-aggregate $A$
%is thus $R_A(t):= |\vec{R}_A(t)|=\sum_{i \in A}R_i(t)$.
%
%From \pref{prop-sercurt0} we can find a function $\tau(t)$ such
%that the service curve property can be expressed by
%$$
%R'(t) \geq R(\tau(t)) + \beta(t- \tau(t))
%$$
%Call $\tau_{\max}:=\sup_{t \geq 0} \tau(t)$. $\tau_{\max}$ depends
%on the arrival curve constraints in \eref{eq-defjcs}. In some
%cases (``critical networks") it may be that $\tau_{\max}= +
%\infty$.
%
%Define
%$$\alpha_A(t):=\max_{\vec{x} \in \calX(t)}|\vec{x}_A(t)|
%$$ so that $\alpha_A$ is an arrival curve for the input aggregate
%$R_A(t)$.
%\begin{lemma}
%\mylabel{lem-fifo-alphA} $$\alpha_A(t)= \max_{(\vec{x}_A, \vec{0})
%\in \calX(t)} |\vec{x}_A(t)|$$
%\end{lemma}
%\pr
%to be done \qed
%
%\begin{theorem}
%\mylabel{theo-aggmux} An arrival curve for the output of aggregate
%$A$ is
% $$\alpha^*(t):=\sup_{s \in [0, \tau_{\max}]}\inf_{t' \in [0, t]}
% a(s, t, t')$$
% where
% $$
% a(s, t,t'):=\max_{(\vec{x}_A, \vec{x}_B) \in \calX(s), (\vec{x}_A+ \vec{y}_A, \vec{x}_B) \in \calX(s+t')}
% \left[|\vec{x}_A+\vec{x}_B+\vec{y}_A| - \beta(s) +
% \gamma(t-t')\right]
% $$
%\end{theorem}
%\paragraph{Remark: }
%In the separable case, a direct application of
%\thref{theo-blindmux} and \thref{theo-output-maxsc} gives the same
%result. In the non-separable case, the bad news is that we do not
%have a concept of service curve for flow aggregate $A$.
%
%\pr
%To be done
%
%\qed

\subsection{Explicit Bounds for a Homogeneous ATM Network with Strong
Source Rate Conditions} \mylabel{sec-l24-cbr}

When analyzing a global network, we can use the bounds in
\sref{sec-l24-fifosc}, using the same method as in
\sref{sec-ds}. However, as illustrated in
\cite{LeBoudec2000mMay}, the bounds so obtained are not
optimal: indeed, even for a FIFO ring, the method does
\emph{not} find a finite bound for all utilization factors
less than 1 (although we know from \sref{sec-ringisstable}
that such finite bounds exist).

In this section we show in \thref{theo-tonfifo} some partial
result that goes beyond the per-node bounds in
\sref{sec-l24-fifosc}. The result was originally found in
\cite{CFZF98,leb99,zhang99}.

Consider an ATM network with the assumptions as in \sref{sec-l24-stability}, with the following
differences
\begin{itemize}

\item  Every link has one origin node and one end node. We say
that a link $f$ is incident to link $e$ if the origin node of link
$e$ is the destination node of link $f$. In general, a link has
several incident links.
  \item All packets have the same size (called cell). All arrivals
  and departures occur at integer times (synchronized model).
All links have the same bit rate, equal to $1$ cell per
  time unit. The service time for one cell is $1$ time unit. The propagation times are
  constant per link and integer.
  \item All links are FIFO.
\end{itemize}

\begin{proposition}
\mylabel{prop-tonfifodm} For a network with the above assumption,
the delay for a cell $c$ arriving at node $e$ over incident link
$i$ is bounded by the number of cells arriving on incident links
$j \neq i$ during the busy period, and that will depart before
$c$.
\end{proposition}
\pr
Call $R'(t)$ (resp. $R_j(t), R(t)$)the output flow (resp. input
arriving on link $j$, total input flow). Call $d$ the delay for a
tagged cell arriving at time $t$ on link $i$. Call $A_j$ the
number of cells arriving on link $j$ up to time $t$ that will
depart before the tagged cell, and let $A=\sum_j A_j$. We have
$$
 d = A - R'(t) \leq A - R(s) - (t-s)
$$
where $s$ is the last time instant before the busy period at $t$.
We can rewrite the previous equation as
$$d\leq \sum_{j \neq i}[A_j-R_j(s)] +  [A_i(t)-R_i(s)] - (t-s)
$$
Now the link rates are all equal to $1$, thus $A_i -R_i(s) \leq
t-s$ and
$$
 d \leq \sum_{j \neq i}[A_j-R_j(s)]
$$
 \qed

An ``Interference Unit" is defined as a set $(e, \{j,k\})$ where
$e$ is a link, $\{j,k\}$ is a set of two distinct flows that each
have $e$ on their paths, and that arrive at $e$ over two different
incident links (\fref{fig-tonfifo1}). The Route Interference
Number (RIN) of flow $j$ is the number of interference units that
contain $j$. It is thus the number of other flows that share a
common sub-path, counted with multiplicity if some flows share
several distinct sub-paths along the same path. The RIN is used to
define a sufficient condition, under which we prove a strong
bound.

\begin{figure}[!htbp]
  \insfig{tonfifo1}{1.0}
  \mycaption{The network model and definition of an interference unit.
  Flows $j$ and $i_2$ have an interference unit at node $f$.
  Flows $j$ and $i_1$ have an interference unit at node $l$ and one at node $g$.}
  \mylabel{fig-tonfifo1}
\end{figure}

\begin{definition}[Source Rate Condition]
The fresh arrival curve constraint (at network boundary) for flow
$j$ is the stair function $v_{R+1, R+1}$, where $R$ is the RIN of
flow $j$.\mylabel{def-sourceratecond}
\end{definition}
The source rate condition is equivalent to saying that a flow
generates at most one cell in any time interval of duration
$\mathrm{RIN}+1$.

\begin{theorem}
\mylabel{theo-tonfifo} If the source rate condition holds at all
sources, then
\begin{enumerate}
  \item The backlog at any node is bounded by $N - \max_i N_i$,
  where $N_i$ is the number of flows entering the node via
  input link $i$, and $N=\sum_i N_i$.
   \item The end-to-end queuing delay for a given flow is bounded by its
   RIN.
  \item There is at most one cell per flow present during any
  busy period.
\end{enumerate}
\end{theorem}

The proof of item 3 involves a complex analysis of chained busy
periods, as does the proof of \thref{theo-ring}. It is given in a
separate section. Item 3 gives an intuitive explanation of what
happens: the source rate condition forces sources to leave enough
spacing between cells, so that two cells of the same flow do not
interfere, in some sense. The precise meaning of this is given in
the proof. Items 1 and 2 derive from item 3 by a classical network
calculus method (\fref{fig-tonfifoheb}).

\paragraph{Proof of \thref{theo-tonfifo}} \mylabel{sec-tonfifoproof} As a
simplification, we call ``path of a cell``  the path of the flow
of the cell. Similarly, we use the phrase ``interference unit of
$c$" with the meaning of interference unit of the flow of $c$.
%For a path $P$, a node $e$ and two flows $j,k$ we say that
%$(e,\{j,k\})$ is a route interference unit of $P$ if $P$ is a
%sub-path of the path of $j$, $e$ is on $P$ and $(e,\{j,k\})$ is an
%interference unit.

We define a busy period as a time interval during which the
backlog for the flow at the node is always positive. We now
introduce a definition (super-chain) that will be central in the
proof. First we use the following relation:
%\begin{definition}[A relation]
%\mylabel{def-relordretonfifo} Consider some fixed link $e$ and
%some fixed subset $A$ of flows using link $e$.
%
%For any pair of flows, there may be several interference units (if
%they interfere several times). We call ``merging point" of two
%flows $i$ and $j$ in $A$ the last node before or at $e$ at which
%they interfere.
%
%For two cells $c$ and $d$ in $A$ we say that $c \preccurlyeq d$ if
%$c$ and $d$ are in the same busy period at their merging point,
%and if $d$ joins the buffer of the merging point no earlier than
%$c$.
%\end{definition}
\begin{definition}[``Delay Chain" \cite{CFZF98}]
\mylabel{def-relordretonfifo} For two cells $c$ and $d$, and for
some link $e$, we say that $c \preccurlyeq_e d$ if $c$ and $d$ are
in the same busy period at $e$ and $c$ leaves $e$ before $d$.
\end{definition}
Figure~\ref{fig-fifof2} illustrates the definition.
\begin{figure}
\insfig{tonfifo2}{0.7}
  \mycaption{A time-space diagram illustrating the definitions of $d \preccurlyeq_g c_1$ and $c_1
   \preccurlyeq_f c_2$. Time flows
   downwards.
 Rectangles illustrate busy periods.}
 \mylabel{fig-fifof2}
\end{figure}
\begin{definition}[Super-Chain \cite{CFZF98}]
Consider a sequence of cells $\underline{c}=(c_0, ...,$ $c_i, ...,
c_k)$ and a sequence of nodes $\underline{f}=(f_1, ...,f_k)$. We
say that $(\underline{c}, \underline{f})$ is a super-chain if
% either $k=0$ and $\underline{f}$ is empty or
\begin{itemize}
  \item $f_1, ..., f_k$ are all on the
  path $P$
   of cell $c_0$ (but not necessarily consecutive)
   \item
 $c_{i-1}\preccurlyeq_{f_i} c_i$ for $i=1$ to $k$.
  \item the path of cell $c_i$ from $f_i$ to $f_{i+1}$ is a sub-path of $P$
\end{itemize}
\end{definition}
We say that the sub-path of $c_0$ that spans from node $f_1$ to
node $f_k$ is the path of the super-chain.

\begin{definition}[Segment Interfering with a Super-Chain]
For a given super-chain, we call ``segment" a couple $(d, P)$
where $P$ is a sub-path of the path of the super-chain, $d$ is a
cell whose path also has $P$ as a sub-path, and $P$ is maximal
(namely, we cannot extend $P$ to be a common sub-path of both $d$
and the super-chain). We say that the segment $(d,P)$ is
interfering with super-chain $(\underline{c},\underline{f})$ if
there is some $i$ on $P$ such that $d \preccurlyeq_{f_i} c_i$.
\end{definition}
\begin{lemma}
\mylabel{lem-superchain1}  Let $(\underline{c}, \underline{f})$ be
a super-chain. Let $s_0$ be the arrival time of cell $c_0$ at link
$f_1$ and  $s'_k$ the departure time of cell $c_k$ from link
$f_k$. Then $s'_k-s_0 \leq R_{1,k}+T_{1, k}$, where $R_{1,k}$ is
the total number of segments interfering with $(\underline{c},
\underline{f})$ and $T_{1, k}$ is the total transmission and
propagation time on the path of the super-chain.
\end{lemma}
\pr
Consider first some node $f_j$ on the super-chain. Let $s_{j-1}$
(resp. $t_j$) be the arrival time of cell $c_{j-1}$  (resp. $c_j$)
at the node. Let $t'_{j-1}$ (resp. $s'_j$) be the departure time
of cell $c_{j-1}$ (resp. $c_j$) (\fref{fig-lemtfsc1}). Let $v_j$
be the last time slot before the busy period that $t_j$ is in. By
hypothesis, $v_j + 1 \leq s_{j-1}$.
\begin{figure}[!htbp]
  \insfig{lemtfsc1}{0.7}
  \mycaption{The notation used in the proof of \lref{lem-superchain1}.}
  \mylabel{fig-lemtfsc1}
\end{figure}
Also define $\calB_j$ (resp. $\calB^0_j$) as the set of segments
$(d,P)$ where $d$ is a cell arriving at the node after time $v_j$
on a link incident to the path of the super-chain (resp. on the
path of the super-chain) and that will depart no later than cell
$c_{j}$, and where $P$ is the maximal common sub-path for $d$ and
the super-chain that $f_j$ is in. Also define $\calA^0_j$ as the
subset of those segments in $\calB^0_j$ for which the cell departs
after $c_{j-1}$. Let $B_j$ (resp. $B^0_j, A^0_j$) be the number of
elements in $\calB_j$ (resp. $\calB^0_j, \calA^0_j$), see
\fref{fig-lemtfsc1}.

Since the rate of all incident links is $1$, we have
$$
B^0_j-A^0_j \leq s_{j-1}- v_j
$$
Also, since the rate of the node is $1$, we have:
$$
s'_j - v_j = B_j + B^0_j
$$
Combining the two, we derive
\begin{equation}\mylabel{eq-lemsc73}
 s'_j - s_{j-1} = B_j + B^0_j -(s_{j-1} - v_j) \leq B_j + A^0_j
\end{equation}
By iterative application of \eref{eq-lemsc73} from $j=1$ to $k$,
we obtain
$$
s'_k -s_0 \leq \sum_{j=1}^k (B_j + A^0_j) + T_{1, k}
$$
Now we show that all sets in the collection $\{\calB_j, \calA^0_j,
\; j=1 \mbox{ to } k\}$ are two-by-two disjoint. Firstly, if
$(d,P) \in \calB_j$ then $f_j$ is the first node of $P$ thus
$(d,P)$ cannot be in some other $ \calB_{j'}$ with $j \neq j'$.
Thus the $\calB_j$ are two-by-two disjoint. Second, assume
$(d,P)\in \calB_j$ and $(d,P)\in \calA^0_{j'}$. It is obvious from
their definitions that, for a fixed $j$, $\calB_j$ and $\calA^0_j$
are disjoint; thus $j \neq j'$. Since $f_j$ is the first node of
$P$ and $j'$ is on $P$, it follows that $j < j'$. Now $d$ leaves
$f_j$ before $c_j$ and leaves $f_{j'}$ after $c_{j'-1}$, which
contradicts the FIFO assumption. Thus the $\calB_j$ and
$\calA^0_{j'}$ are two-by-two disjoint. The same reasoning shows
that it is not possible that $(d,P) \in \calA_j \bigcap
\calA_{j'}$ with $j < j'$.

Now, by definition, every segment in either $\calB_j$ or
$\calA^0_j$ is an interfering segment. Thus
$$\sum_{j=1}^k (B_j + A^0_j) \leq R_{1,k}$$.
 \qed

\begin{proposition}
\mylabel{prop-superchain2} Assume the source rate condition holds.
Let $(\underline{c}, \underline{f})$ be a super-chain.
\begin{enumerate}
  \item  For every interference unit of $c_0$
  there is at most one cell interfering with the super-chain.
  \item $c_k$ does not belong to the same flow as $c_0$.
\end{enumerate}
\end{proposition}
\pr
Define the time of a super-chain as the exit time for the last
cell $c_k$ on the last node $f_k$.  We use a recursion on the time
$t$ of the super-chain.
%For a super-chain as in the theorem,
%call $t$ the exit time for cell $c_k$ out  the time stopping
%method \sref{sec-timestoppingm} and do a recursion on time. Call
%$\calN_t$ the network derived from the original one, where all
%sources are stopped after time $t$. We show that the RIN property
%holds for all networks $\calN_t$, which is sufficient to show the
%theorem.

If $t=1$, the proposition is true because any flow has at most one
cell on a link in one time slot. Assume now that the proposition
holds for any super-chain with time $\leq t-1$ and consider a
super-chain with time $t$.

First, we associate an interference unit to any segment $(d, P)$
interfering with the sub-chain, as follows. The paths of $d$ and
$c_0$ may share several non contiguous sub-paths, and $P$ is one
of them. Call $f$ the first node of $P$. To $d$ we associate the
interference unit $(f, \{j_0, j\})$, where $j_0$ (resp. $j$) is
the flow of $c_0$ (resp. $d$).

We now show that this mapping is injective. Assume that another
segment $(d', P') \neq (d, P)$ is associated with the same
interference unit $(f, \{j_0, j\})$. Without loss of generality,
we can assume that $d$ was emitted before $d'$. $d$ and $d'$
belong to the same flow $j$, thus, since $P$ and $P'$ are maximal,
we must have $P=P'$. By hypothesis, have an interference with the
super-chain at a node on $P$. Let $f_l$ be a node on the
super-chain and on $P$ such that $d\preccurlyeq_{f_l} c_l$. If
$d'$ leaves node $f_l$ before $c_l$, then $d\preccurlyeq_{f_l}
d'$, and thus $((d, d'),(f_l))$ is a super-chain. Since $d'$ is an
interfering cell, necessarily, it must leave node $f_l$ before
$t$, thus the proposition is true for super-chain $((d,
d'),(f_l))$, which contradicts item 2. Thus $d'$ must leave node
$f_l$ after cell $c_l$. But there is some other index $m \leq k$
such that $d\preccurlyeq_{f_m} c_m$, thus cell $d'$ leaves node
$f_m$ before cell $c_m$. Define $l'$ as the smallest index with $l
< l' \leq m$ such that $d'$ leaves node $f_{l'}$ after cell
$c_{l'-1}$ and before $c_{l'}$. Then $((d, c_l, ..., c_{l'-1},
d'),(f_l, ..,f_{l'}))$ is a super-chain with time $\leq t-1$ which
would again contradict item 2 in the proposition. Thus, in all
cases we have a contradiction, the mapping is injective, and item
1 is shown for the super-chain.

Second, let us count a bound on the maximum queuing delay of cell
$c_0$. Call $u_0$ its emission time, $P_0$ the sub-path of $c_0$
from its source up to, but excluding, node $f_1$, and $T$ the
total transmission and propagation time for the flow of $c_0$. The
transmission and propagation time along $P_0$ is thus $T-T_{1,k}$.
By \pref{prop-tonfifodm}, the queuing delay of $c_0$ at a node $f$
on $P_0$ is bounded by the number of cells $d \preccurlyeq_{f}
c_0$ that arrive on a link not on $P_0$. By the same reasoning as
in the previous paragraph, there is at most one such cell $d$ per
interference unit of $c_0$ at $f$. Define $R$ as the number of
interference units of the flow of $c_0$ on $P_1$. We have thus
\begin{equation}\mylabel{eq-tonfifopr332}
 s_0 \leq u_0 + R + T-T_{1,k}
\end{equation}

Similarly, from \lref{lem-superchain1}, we have
$$s'_k \leq s_0 + R_{1,k}+T_{1,k}$$
Call $R'$ the number of interference units of the flow of $c_0$ on
the path of the super-chain. It follows from the first part of the
proof that $R_{1,k} \leq R'$, thus
$$
s'_k \leq s_0 + R'+T_{1,k}
$$
Combining with \eref{eq-tonfifopr332} gives
\begin{equation}\mylabel{eq-tonfifo238197}
 s'_k \leq u_0 + R + R' +T
\end{equation}
Now by the source condition, if $c_k$  belongs to the flow of
$c_0$, its emission time $u'$ must satisfy
$$
u' \geq u_0 + R + R' + 1
$$
and thus
$$
s'_k \geq u_0 + R + R' + 1 + T
$$
which contradicts \eref{eq-tonfifo238197}. This shows that the
second item of the proposition must hold for the super-chain.
 \qed

\paragraph{Proof of \thref{theo-tonfifo}: }
Item 3 follows from \pref{prop-superchain2}, since if there would
be two cells $d, d'$ of the same flow in the same busy period,
then $((d, d'),(e))$ would be a super-chain.

Now we show how items 1 and 2 derive from item 3. Call
$\alpha^*_i(t)$ the maximum number of cells that may ever arrive
on incident link $i$ during $t$ time units inside a busy period.
Since $\lambda_1$ is a service curve for node $e$, the backlog $B$
at node $e$ is bounded by
$$
B \leq \sup_{t \geq 0} \left[\sum_{i=1}^{I} \alpha^*_i(t) \;  - t
\right]
$$
Now by item 3, $\alpha^*_i(t) \leq N_i$ and thus
$$\alpha^*_i (t) \leq \alpha_i(t):= \min [N_i, t ]
$$
Thus
$$
B \leq \sup_{t \geq 0} \left[\sum_{i=1}^{I} \alpha_i(t) \;  - t
\right]
$$
Now define a renumbering of the $N_i$'s such that $N_{(1)} \leq
N_{(2)} \leq ...\leq N_{(I)}$. The function $\sum_i \alpha_i(t)
-t$ is continuous and has a derivative at all points except the
$N_{(i)}$'s (\fref{fig-tonfifoheb}). The derivative changes its
sign at $N_{(I)}$ (=$\max_{1\leq i\leq I}(N_i)$) thus the maximum
is at $N_{(I)}$ and its value is $N-N_{(I)}$, which shows item 1.
\begin{figure}[!htbp]
  \insfig{tonfifoheb}{0.8}
  \mycaption{Derivation of a backlog bound.}
  \mylabel{fig-tonfifoheb}
\end{figure}

From Item 1, the delay at a node is bounded by the number of
interference units of the flow at this node. This shows item 2.
 \qed

\section{Bibliographic Notes}

\nfs{<courbes de service inadequate pour FIFO>}

In \cite{leb99}, a stronger property is shown than
\thref{theo-tonfifo}: Consider a given link $e$ and a subset $A$
of $m$ connections that use that link. Let $n$ be a lower bound on
the number of route interferences that any connection in the
subset will encounter after this link. Then over any time interval
of duration $m+n$, the number of cells belonging to $A$ that leave
link $e$ is bounded by $m$.

It follows from item 1 in \thref{theo-tonfifo} that a better
queuing delay bound for flow $j$ is:
$$\delta(j) = \sum_{e \mst e \in j}
 \left\{\min_{i \mst 1 \leq i\leq I(e)}(N(e)-N_i(e)) \right\}$$
 where $I(e)$ is the number of incident links at node $e$,
 $N_i(e)$ is the number of flows entering node $e$ on link i, and
 $N=\sum{i=1}^{I(e)}N_i(e)$.
In other words, the end-to-end queuing delay is bounded by the sum
of the minimum numbers of route interference units for all flows
at all nodes along the path of a flow. For asymmetric cases, this
is less than the RIN of the flow.

\section{Exercises}
%% LaTeX2e file `./temp/L24-1'
%% generated by the `filecontents' environment
%% from source `netCalBookv4' on 2019/02/19.
%%
\begin{problem}
Consider the same assumptions as in \sref{sec-ringisstable} but
with a linear network instead of a ring. Thus node $m$ feeds node
$m+1$ for $m=1,..., M-1$; node $1$ receives only fresh traffic,
whereas all traffic exiting node $M$ leaves the network. Assume
that all service curves are strict. Find a bound which is finite
for $\nu \leq 1$. Compare to \thref{theo-ring}. \sol{Apply
\thref{theo-blindmux}}
\end{problem}

%% LaTeX2e file `./temp/L24-2'
%% generated by the `filecontents' environment
%% from source `netCalBookv4' on 2019/02/19.
%%
\begin{problem}
Consider the same assumptions as in \thref{theo-tonfifo}. Show
that the busy period duration is bounded by $N$.\sol{There is at
most one cell per flow in the busy period.}
\end{problem}

%% LaTeX2e file `./temp/L21-18'
%% generated by the `filecontents' environment
%% from source `netCalBookv4' on 2019/02/19.
%%
\begin{problem}
Consider the example of \fref{fig-dsnet1}. Apply the method of
\sref{sec-timestoppingm} but express now that the fraction of
input traffic to node $i$ that originates from another node must
have $\lambda_C$ as an arrival curve . What is the upper-bound on
utilization factors for which a bound is obtained ?
\end{problem}

%% LaTeX2e file `./temp/L24-3'
%% generated by the `filecontents' environment
%% from source `netCalBookv4' on 2019/02/19.
%%
\begin{problem}
Can you conclude anything on $\nu_{cri}$ from \pref{prop-grand} on
\pgref{prop-grand}~?
\end{problem}


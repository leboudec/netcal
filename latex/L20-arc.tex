\begin{definition} [Arrival Curve]
Given a wide-sense increasing function $\alpha$ defined for $t
\geq 0$ we say that a flow $R$ is constrained by $\alpha$ if and
only if for all $s \leq t$:
$$R(t)-R(s) \leq \alpha(t-s)$$
We say that $R$ has $\alpha$ as an arrival curve, or also that $R$
is $\alpha$-smooth.%
\index{arrival curve}%
\index{smooth ($\alpha$-smooth for some function $\alpha(t)$}
\end{definition}
Note that the condition is over a set of overlapping intervals, as
Figure~\ref{affArrCur} illustrates.
\begin{figure}[!htbp]
   \insfig{arrcurdef}{0.8}
  \mycaption{Example of Constraint by arrival curve,
  showing a cumulative function
  $R(t)$ constrained by the arrival curve $\alpha(t)$.}
   \mylabel{affArrCur}
\end{figure}

\paragraph{Affine Arrival Curves: }
For example, if $\alpha(t)=rt$, then the constraint means that, on
any time window of width $\tau$, the number of bits for the flow
is limited by $r \tau$. We say in that case that the flow is peak
rate limited. This occurs if we know that the flow is arriving on
a link whose physical bit rate is limited by $r$ b/s. A flow where
the only constraint is a limit on the peak rate is often
(improperly) called a ``constant bit rate" (CBR) flow, or
``deterministic bit rate" (DBR) flow.

Having $\alpha(t)=b$, with $b$ a constant, as an arrival curve
means that the maximum number of bits that may ever be sent on the
flow is at most $b$.

More generally, because of their relationship with leaky buckets,
we will often use \emph{affine} arrival curves $\gamma_{r,b}$,
defined by:
%\index{1gamma@$\gamma_{r,b}$ (affine arrival curve)} \index{affine
%arrival curve}
$\gamma_{r,b}(t)=rt +b$ for $t>0$ and $0$ otherwise. Having
$\gamma_{r,b}$ as an arrival curve allows a source to send $b$
bits at once, but not more than $r$ b/s over the long run.
Parameters $b$ and $r$ are called the burst tolerance (in units of
data) and the rate (in units of data per time unit).
Figure~\ref{affArrCur} illustrates such a constraint.

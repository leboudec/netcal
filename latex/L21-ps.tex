\subsection{GPS and a Practical Implementation (PGPS)}
\mylabel{sec-gpsdet}

A GPS node serves several flows in parallel, and has a total
output rate equal to $c$ b/s. A flow $i$ is allocated a given
weight, say $\phi_i$. Call $R_i(t), R^*_i(t)$ the input and output
functions for flow $i$. The guarantee is that at any time $t$, the
service rate offered to flow $i$ is 0 is flow $i$ has no backlog
(namely, if $R_i(t)=R_i^*(t)$), and otherwise is equal to
$\frac{\phi_i}{\sum_{j \in B(t)} \phi_{j}}c$, where $B(t)$ is the
set of backlogged flows at time $t$. Thus
 $$R^*_i(t) = \int_0^t \frac{\phi_i c}{\sum_{j \in B(s)} \phi_{j}} 1_{\{i \in B(s)\}}ds
 $$
 In the formula, we used the indicator function
 $1_{\{\mbox{\emph{expr}}\}}$, which is equal to $1$ if $\mbox{\emph{expr}}$ is
 true, and $0$ otherwise.

It follows immediately that the GPS node offers to flow $i$ a
service curve equal to $\lambda_{r_i}$, with $r_i=\frac{\phi_i
c}{\sum_{j} \phi_{j}}$. It is shown in \cite{pg94} that a better
service curve can be obtained for every flow if we know some
arrival curve properties for all flows; however the simple
property is sufficient to understand the integrated service model.

GPS satisfies the requirement of isolating flows and providing
differentiated guarantees. We can compute the delay bound and
buffer requirements for every flow if we know its arrival curve,
using the results of Chapter~\ref{L20}. However, a GPS node is a
theoretical concept, which is not really implementable, because it
relies on a fluid model, and assumes that packets are infinitely
divisible. How can we make a practical implementation of GPS ? One
simple solution would be to use the virtual finish times as we did
for the buffered leaky bucket controller in \sref{sec-psps}: for
every packet we would compute its finish time $\theta$ under GPS,
then at time $\theta$ present the packet to a multiplexer that
serves packets at a rate $c$. \fref{fig-pgps} (left) shows the
finish times on an example. It also illustrates the main drawback
that this method would have: at times 3 and 5, the multiplexer
would be idle, whereas at time 6 it would have a burst of 5
packets to serve. In particular, such a scheduler would not be
work conserving.

This is what motivated researchers to find other practical
implementations of GPS. We study here one such implementation of
GPS, called
packet by packet generalized processor sharing (PGPS)~\cite{pg93}. %
\index{PGPS: packet generalized processor sharing}%
Other implementations of GPS are discussed in
Section~\ref{sec-gr}.

PGPS emulates GPS as follows. There is one FIFO queue per flow.
The scheduler handles packets one at a time, until it is fully
transmitted, at the system rate $c$. For every packet, we compute
the finish time that it would have under GPS (we call this the
``GPS-finish-time"). Then, whenever a packet is finished
transmitting, the next packet selected for transmission is the one
with the earliest GPS-finish-time, among all packets present.
Figure~\ref{fig-pgps} shows one example. We see that, unlike the
simple solution discussed earlier, PGPS is work conserving, but
does so at the expense of maybe scheduling a packet \emph{before}
its finish time under GPS.
\begin{figure}[!htbp]
  \insfig{pgps}{0.7}
  \mycaption{Scheduling with GPS (left) and PGPS (right).
  Flow 0 has weight 0.5, flows 1 to 5 have weight 0.1. All packets
  have the same transmission time equal to 1 time unit.}
  \mylabel{fig-pgps}
\end{figure}

We can quantify the difference between PGPS and GPS in the
following proposition. In \sref{sec-gr}, we will see how to derive
a service curve property.
\begin{proposition}[\cite{pg93}]
The finish time for PGPS is at most the finish time of GPS plus
$\frac{L}{c}$, where $c$ is the total rate and $L$ is the maximum
packet size.
 \mylabel{prop-pgps}
\end{proposition}
\pr Call $D(n)$ the finish time of the $n$th packet for the
aggregate
 input flow under PGPS, in the order of departure, and $\theta(n)$ under GPS.
 Call $n_0$ the
 number of the packet that started the busy period in which packet
 $n$ departs. Note that PGPS and GPS have the same busy periods,
 since if we observe only the aggregate flows, there is no
 difference between PGPS and GPS.

 There may be some packets that
 depart before packet $n$ in PGPS, but that nonetheless have a
 later departure time under GPS.
 Call $m_0 \geq n_0$ the largest packet
 number for which this occurs, if any; otherwise let $m_0=n_0-1$.
 In this proposition, we call $l(m)$ the length in bits of packet $m$.
 Under PGPS, packet $m_0$ started service at
 $D(m_0)- \frac{l(m_0)}{c}$, which must be earlier than the
 arrival times of packets $m= m_0+1, ..., n$.
 Indeed, otherwise, by definition of PGPS, the
 PGPS scheduler would have scheduled packets $m= m_0+1, ..., n$
 before packet $m_0$. Now let us observe the GPS system. Packets $m= m_0+1, ...,
 n$ depart no later than packet $n$, by definition of $m_0$; they
 have arrived after $D(m_0)- \frac{l(m_0)}{c}$. By expressing the amount of
 service in the interval $[D(m_0)- \frac{l(m_0)}{c}, \theta(n)]$
 we find thus
 $$ \sum_{m=m_0 +1}^n l(m) \leq c \left(\theta(n) - D(m_0)+ \frac{l(m_0)}{c}\right)
 $$
Now since packets $m_0, ..., n$ are in the same busy period, we
have
 $$D(n) = D(m_0) + \frac{\sum_{m=m_0 +1}^n l(m)}{c}$$
By combining the two equations above we find $D(n) \leq \theta(n)
+ \frac{l(m_0)}{c}$, which shows the proposition in the case where
$m_0 \geq n_0$.

 If $m_0 = n_0-1$, then all packets $n_0,...,n$ depart before packet $n$
 under GPS and thus the same reasoning shows that
 $$ \sum_{m=n_0}^n l(m) \leq c \left(\theta(n) - t_0\right) $$
 where $t_0$ is the beginning of the busy period, and that
 $$D(n) = t_0 + \frac{\sum_{m=n_0 }^n l(m)}{c}
 $$
 Thus $D(n) \leq \theta(n)$ in that case. \qed
